\begin{center}
  {\Large \bf Executive Summary}
\end{center}

More than 80 years after its astrophysical discovery, the fundamental nature of dark matter remains one of the foremost open questions in physics.
Over the last several decades, an extensive experimental program has sought to determine the cosmological origin, fundamental constituents, and interactions mechanisms of dark matter. 
While the existing experimental program has largely focused on weakly-interacting massive particles, there is strong theoretical motivation to explore a broader set of dark matter candidates.
As the high-energy physics program expands to ``search for dark matter along every feasible avenue'' \citep{P5:2014}, it is essential to keep in mind that the only direct, empirical measurement of dark matter properties comes from astrophysical and cosmological observations.

The Large Synoptic Survey Telescope (LSST), a major joint experimental effort between NSF and DOE, provides a unique and impressive platform to pursue the study of fundamental dark matter physics.
While LSST was originally envisioned as the ``Dark Matter Telescope'' \citep{Tyson:2001}, in recent years studies of fundamental physics with LSST have been more focused on dark energy.
In this document, we reaffirm LSST's ability to test well-motivated theoretical models of dark matter: \ie, self-interacting dark matter, warm dark matter, baryon-scattering dark matter, ultra-light/axion-like dark matter, and primordial black holes. 

LSST will use Milky Way satellite galaxies and stellar streams detect and study the smallest dark matter halos, thereby probing the minimum mass of ultra-light dark matter and thermal warm dark matter.
Precise measurements of the density and shapes of dark matter halos in dwarf galaxies and galaxy clusters will be sensitive to self-interacting dark matter with a cross section to probe hidden sector and dark photon models.
Microlensing measurement will directly probe primordial black holes and the compact object fraction of dark matter at the sub-percent level over a wide range of masses.
Precise measurements of stellar populations will be sensitive anomalous energy loss mechanisms and constrain couplings between photons and electrons to axion-like particles.
Unprecedented measurements of large scale structure will spatially resolve the influence of both dark matter and dark energy, enabling searches for correlations between the only empirically confirmed components of the dark sector.
In addition, complementarity between LSST, direct, and indirect dark matter experiments will help constrain dark matter-baryon scattering, dark matter self-annihilation, and dark matter decay.

The study of dark matter with LSST presents a small experimental program with a short timescale and low cost that is guaranteed to provide critical information about the fundamental nature of dark matter over the next decade.
LSST will rapidly produce high-impact science on fundamental dark matter physics by exploiting an existing US facility. 
The study of dark matter with LSST explores dark matter parameter space beyond the high-energy physics program's current sensitivity, while being highly complementary to other experimental searches. % BRN text.
This has been recognized during the Snowmass Cosmic Frontier planning process \citep[\eg,][]{1305.1605, 1310.8642, 1310.5662} and in a series of more recent Cosmic Visions reports \citep[\eg,][]{1604.07626,1802.07216}, including the ``New Ideas in Dark Matter 2017:\ Community Report'' \citep{1707.04591}.
It is worth remembering that astrophysical probes provide the only constraints on the minimum and maximum mass-scale of dark matter, and it is likely that astrophysical observations will continue to guide the  experimental particle physics program for years to come.

\clearpage

\begin{center}
  {\Large \bf Preface}
\end{center}

This white paper is the product of a large, global community of scientists who are united in support of probing the fundamental nature of dark matter with LSST.
Although LSST was originally proposed as the ``Dark Matter Telescope'' \citep{Tyson:2001}, none of the eight existing LSST Science Collaborations is specifically focused on exploring the microscopic identity of dark matter.
Rather, the study of dark matter is distributed across several disparate groups, making it difficult to combine results and build a cohesive physical picture of dark matter.
It was recognized that the current situation would severely hamper the progress of dark matter physics with LSST, and this effort was started to coordinate across disparate groups, to enlarge the dark matter community, and to strengthen connections between theory and experiment.

The concept for this white paper emerged from a series of meetings and regular telecons held in 2017--2018 around the topic of astrophysical probes of dark matter in the era of LSST.
Sessions were held at the LSST Project and Community Workshop in 2017 and 2018, multiple LSST Dark Energy Science Collaboration (DESC) meetings, and two dedicated multi-day workshops at the University of Pittsburgh and Lawrence Livermore National Laboratory.
Funding was provided by individual institutions and through a grant from the LSST Corporation (LSSTC) Enabling Science Program.
Through participation in the workshops, numerous telecons, sensitivity analyses, writing, editing, and reviewing, more than \CHECK{100} have directly contributed to this edition of the LSST dark matter white paper.
We encourage interested scientist to join this effort at: \textcolor{blue}{\url{https://lsstdarkmatter.github.io/}}.

This document is currently under internal review within the LSST Dark Matter Group and is provided as an \emph{advanced draft}.

