\begin{center}
  {\Large \bf Executive Summary}
\end{center}

\ADW{To be expanded upon... Tie in with P5, Cosmic Visions, Basic Research Needs, etc.}

We present a summary of the astrophysical techniques that can be used to probe the fundamental nature of dark matter with the Large Synoptic Survey Telescope (LSST). 
LSST will inform our understanding of the fundamental properties of dark matter, such as dark matter particle mass, dark matter self-interaction cross section, compact macroscopic dark matter, and axion-like particles.
Astrophysical observations are currently the only robust, empirical measurements of dark matter, and observations with LSST will provide necessary guidance for the experimental dark matter program.
Observations with LSST will critically complement and guide other experimental efforts to discovery the fundamental nature of dark matter.
