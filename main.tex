  % URL for sharing the EDITABLE version :
% https://v2.overleaf.com/10858269czpqjyyjnbyx

% Yao has written a nice package to automate building the bib file from the citations that are present in the tex.
% https://github.com/yymao/adstex
% Valid citation formats are:
% \citep{1705.03888} % arXiv number
% \citet{Mao:2018} % first author's last name and year
% \citealt{10.1093/mnras/stx3111,2017arXiv170909665M} % DOI

\documentclass[modern,linenumbers]{aastex62}
\usepackage{newtxtext,newtxmath} % this is causing the etoolbox warning
\usepackage[USenglish]{babel}
\usepackage[utf8]{inputenc}
\usepackage[T1]{fontenc}
\usepackage{comment}

\usepackage{enumitem}
\setlist{noitemsep}

% Custom commands from LSST DESC, see texmf/styles/lsstdesc_macros.sty
\usepackage{lsstdesc_macros}
\graphicspath{{./}{./figures/}}

% Add your own macros here:
\newcommand{\Contributors}[1]{ {\footnotesize [\textit{#1}]}}
\newcommand{\Contact}[1]{ {\footnotesize [\textbf{#1}]}}
\newcommand{\Comment}[3]{\textcolor{#1}{(#2: #3)}}

\newcommand{\ADW}[1]{\Comment{blue}{ADW}{#1}} % Alex Drlica-Wagner
\newcommand{\YYM}[1]{\Comment{blue}{YYM}{#1}} % Yao-Yuan Mao
\newcommand{\WAD}[1]{\Comment{blue}{WAD}{#1}} % Will Dawson
\newcommand{\MSM}[1]{\Comment{green}{MSM}{#1}} % Manuel Meyer
\newcommand{\CPW}[1]{\Comment{blue}{CPW}{#1}} % Chanda Prescod-Weinstein
\newcommand{\RHW}[1]{\Comment{blue}{RHW}{#1}} % Risa Wechsler 
\newcommand{\EON}[1]{\Comment{purple}{EON}{#1}} % Ethan Nadler
\newcommand{\TT}[1]{\Comment{orange}{TT}{#1}} % Tony Tyson
\newcommand{\cora}[1]{\Comment{blue}{Cora}{#1}} %Cora Dvorkin
\newcommand{\CDF}[1]{\Comment{red}{CDF}{#1}} 
\newcommand{\CRK}[1]{\Comment{purple}{CRK}{#1}} % Chuck Keeton 
\newcommand{\Yashar}[1]{\Comment{red}{Yashar}{#1}}  % Yashar Hezaveh
\newcommand{\DAH}[1]{\Comment{blue}{DH}{#1}} %David Hendel
\newcommand{\AHGP}[1]{\Comment{cyan}{AHGP}{#1}} %Annika Peter
\newcommand{\FYCR}[1]{\Comment{magenta}{FYCR}{#1}} % Francis-Yan Cyr-Racine
\newcommand{\MG}[1]{\Comment{red}{MG}{#1}} % Maurizio Giannotti
\newcommand{\OS}[1]{\Comment{green}{OS}{#1}} % Oscar Straniero
\newcommand{\DE}[1]{\Comment{orange}{DE}{#1}} % Denis Erkal
\newcommand{\SDM}[1]{\Comment{blue}{SDM}{#1}} % Sam McDermott
\newcommand{\JGB}[1]{\Comment{red}{JGB}{#1}} % Juan Garcia-Bellido
\newcommand{\KB}[1]{\Comment{orange}{KB}{#1}} % Keith Bechtol

% ======================================================================

\begin{document}

\title{\Large Probing the Fundamental Nature of Dark Matter \\
with the Large Synoptic Survey Telescope}

\author{LSST Dark Matter Study Group}
%\collaboration{LSST DESC}

\begin{abstract}
We present a summary of the astrophysical techniques that can be used to probe the fundamental nature of dark matter with the Large Synoptic Survey Telescope (LSST). 
LSST will inform our understanding of the fundamental properties of dark matter, such as dark matter particle mass, dark matter self-interaction cross section, compact macroscopic dark matter, and axion-like particles.
Astrophysical observations are currently the only robust, empirical measurements of dark matter, and observations with LSST will provide necessary guidance for the experimental dark matter program.
Observations with LSST will critically complement and guide other experimental efforts to discovery the fundamental nature of dark matter.
\end{abstract}
\keywords{dark matter -- cosmology: observations -- elementary particles}

\tableofcontents 

%%%%%%%%%%%%%%%%%%%%%%%%%%%%%%%%%%%%%%%%%%%%%%%%%%%%%%%%%%%%%%%%%%%%%%%%%%%%%%%%%%%%%
% Introduction
%%%%%%%%%%%%%%%%%%%%%%%%%%%%%%%%%%%%%%%%%%%%%%%%%%%%%%%%%%%%%%%%%%%%%%%%%%%%%%%%%%%%%

\section{Introduction \Contact{Alex}}
\Contributors{Alex, Annika, Keith ...}
\label{sec:intro}

\ADW{Do we need an executive summary? Should we rename ``Introduction'' to ``Executive Summary''?}

The fundamental nature of dark matter, which constitutes $\roughly 85\%$ of the matter density and 26\% of the energy density of the Universe, represents a critical gap in our understanding of fundamental physics.
Over the past several decades, experimental searches for particle dark matter have proceeded along several complementary avenues.
Collider experiments (e.g., ATLAS and CMS at the Large Hadron Collider), attempt to produce and detect dark matter particles \citep{Boveia:2018yeb,Ehret:2010mh,Battaglieri:2017aum}.  
Direct detection experiments (e.g., LZ, CDMS, ADMX, PICO, DAMiC, SENSEI, etc.) attempt to directly detect energy deposition from very rare scattering between dark matter and standard model particles \citep{1509:02910, 1804.10697, Du:2018uak}. 
%\AHGP{Cite only US experiments?} 
%\ADW{It was intentional, but perhaps ill advised...}
In parallel, indirect dark matter searches (e.g., {\it Fermi}-LAT, AMS-02, {\it Chandra}, XMM, etc.) seek to detect the energetic Standard Model products from the annihilation or decay of dark matter particles {\it in situ} \citep[\eg][]{1503.02641,1402.2301, 1605.01043, 1603.06978}. 
%\AHGP{Get an axion indirect paper rec from Chanda}
Despite these extensive efforts, the only robust, positive empirical measurement of dark matter continues to come from astrophysical and cosmological observations. 
%\AHGP{This paragraph is fairly WIMP-centric.  I added some axion- and hidden-photon-related refs.}

Astrophysics and cosmology offer a complementary technique to study the fundamental properties of dark matter. 
They probe dark matter directly through gravity, the only force that dark matter is known to couple to, and have established the standard non-relativistic collisionless, cold dark matter (CDM) paradigm.
However, many viable theoretical models of dark matter predict observable deviations from CDM, which are testable with current and future experimental programs.
Fundamental properties of dark matter---e.g., particle mass, self-interaction cross section, coupling to the Standard Model, and time-evolution of dark matter properties---can imprint themselves on the macroscopic distribution of dark matter in a detectable manner.

In addition, astrophysical observations complement particle physics searches by measuring the local distribution of dark matter, targeting indirect searches, and constraining the viable range of dark matter particle mass and electric charge.
As the WIMP paradigm becomes more and more tightly constrained,  astrophysical observations will provide critical information to help direct the evolution of particle physics searches over the coming decade.  
In many cases, observations with telescopes provide \emph{the only} robust, empirical constraints on the viable range of dark matter models.

%\AHGP{I have no idea where these last two sentences are going, especially in light of the next paragraph.  Is the point you're trying to make that knowing the astrophysical distribution of DM can help target astroparticle searches?  Or that these are additional things we can learn with astrophysics?  How much of this paragraph is supposed to focus on constraining the microphysical properties of DM that don't involve standard model interactions, vs. helping sharpen constraints on standard model interactions?}
%\ADW{I've tried to break these into two paragraphs.}

%\ADW{Adapted from Buckley and Peter} 
At the same time, there is immense dark matter discovery potential at the intersection of particle physics and astrophysics.
Detecting a deviation from the gravitational predictions of CDM would provide much-needed experimental guidance on parameters that are not easily measured in particle physics experiments (\eg, dark matter self-interaction cross sections). 
If, on the other hand, all astrophysical studies of dark matter are found to agree with the CDM predictions, the improved knowledge of dark matter distributions will reduce major sources of theoretical uncertainties in the particle physics experiments. 
%ADW: I don't know what is being referred to here.
Likewise, results from particle experiments can either restrict the possible model space relevant to novel astrophysical signals, or suggest specific deviations from the CDM paradigm which can be investigated observationally.

The Large Synoptic Survey Telescope (LSST) is a next-generation wide-area optical survey instrument that will enable high-precision cosmological measurements to probe the fundamental physics of dark matter and dark energy \citep{0805.2366}. Following on predecessors such as the Sloan Digital Sky Survey (SDSS), Pan-STARRS, and the Dark Energy Survey (DES), LSST promises to greatly enhance our knowledge of the dark sector of the Universe. 
LSST will measure the properties of dark matter over a wide range of astrophysical scales. 
At the largest scales, LSST will use gravitational weak lensing and the large scale clustering of galaxies to trace the distribution of dark matter.
At the smallest scales, LSST will trace the distribution of dark matter with the faintest galaxies, gravitational perturbations from dark matter substructure detectable through gravitational strong lensing and tracer stellar populations.
In addition, the temporal component of the LSST ``wide, fast, deep'' survey will open a new window on the search for compact dark matter such as primordial black holes (PBHs).
LSST will provide a rich scientific data set that can be used to develop novel and unanticipated constraints on dark matter properties through precise measurements of physical processes, such as anomalous energy loss in stars that could be produced by axions or axion-like particles.

In this white paper, we present several techniques that LSST will employ to probe the fundamental properties of dark matter. 
Rather than presenting a comprehensive review of astrophysical probes of dark matter  \citep[e.g.,][]{BuckleyPeter:2017} or an extensive discussion of any particular dark matter model \citep[e.g.,][]{Jain:2019}, we choose to focus on what we believe are some of the most exciting opportunities to study dark matter physics with LSST. 
Many astrophysical measurements require collaborative observations between several instruments, and the study of dark matter with LSST is no exception. 
Throughout this paper we describe situations where LSST will complement other astrophysical and particle investigations of dark matter.
Our over-arching goal in this paper is to convince our particle physics colleagues that not only will LSST provide exciting results on the nature of dark matter, but that observations from LSST are {\it essential} for guiding future particle physics searches.

\begin{figure}[t]
\centering
\includegraphics[width=0.85\textwidth]{figures/interactions.pdf}
\caption{
\label{fig:interactions}
Dark matter may have non-gravitational interactions that may be probed by four complementary approaches: 
direct detection, indirect detection, particle colliders, and astrophysical probes.
The lines connect the experimental approaches with the categories of particles that they most stringently probe (additional lines can be drawn in specific model scenarios). 
Figure taken from \citet{1305.1605} produced as part of Snowmass 2013.
\KB{Is it worth including a figure here to replace the ``conventional'' collider-direct-indirect dark matter search triangle? I'm thinking of something that people can easily re-use in talks and serves as a very simple visual to convey the importance of astrophysical probes.}
\ADW{I've added this figure, but I'm not sure it's exactly what we want (we talk about non-gravitational interactions . It'd be good to hear Annika and Matt B.'s input, since I think they've thought about this a lot.}
}
\end{figure}


\begin{deluxetable*}{l c c c}
\tabletypesize{\footnotesize}
\tablecaption{\label{tab:models} 
Probes of fundamental dark matter physics with LSST.}
\tablehead{
 \colhead{Model} & \colhead{Probe} & \colhead{Parameter} & \colhead{Value}
}
\startdata 
Warm Dark Matter (WDM) & Halo Mass & Particle Mass & \FIXME{$m_{\rm WDM} \sim 30 \keV$} \\
Self-Interacting Dark Matter (SIDM) & Halo Profile & Cross Section & \FIXME{$\sigma/m \sim 1 \cm^2/\g$} \\
Baryon-Scattering Dark Matter (BSDM) & Halo Mass & Cross Section & \FIXME{$\sigma \sim 10^{-26} \cm^2$} \\
Axion-Like Particles (ALPs) & Energy Loss & Coupling Strength & \FIXME{$g_{\phi e} \sim 10^{-13} \GeV^{-1}$} \\
Fuzzy Dark Matter (FDM) & Halo Mass & Particle Mass & \FIXME{$m \sim 10^{-21} \eV$}  \\
Primordial Black Holes (PBHs) & Compact Objects & Mass & \FIXME{$m \grsim \sim 10^{-12} \Msun$} \\
Weakly Interacting Massive Particles (WIMPs) & Indirect Detection & Cross Section & \FIXME{$\sigmav \sim 10^{-27} \cm^3/\second$} \\[+0.5em]
\enddata
\tablecomments{\ADW{There are clearly a lot of caveats on each line, but I think that a table like this would be very useful. I could use help here!}}
\end{deluxetable*}




%%%%%%%%%%%%%%%%%%%%%%%%%%%%%%%%%%%%%%%%%%%%%%%%%%%%%%%%%%%%%%%%%%%%%%%%%%%%%%%%%%%%%
% Theory and Dark Matter Models
%%%%%%%%%%%%%%%%%%%%%%%%%%%%%%%%%%%%%%%%%%%%%%%%%%%%%%%%%%%%%%%%%%%%%%%%%%%%%%%%%%%%%

\section{Dark Matter Models}
\Contributors{Haibo, Chanda, Matt B., Simeon, Francis-Yan, Manoj?, Annika?}
\label{sec:theory}
\chapter{Dark Matter Models}
\label{sec:theory}
\bigskip
\Contributors{Simeon Bird, Kimberly Boddy, Matthew Buckley, Francis-Yan Cyr-Racine, Will Dawson, Alex Drlica-Wagner, Juan Garc\'ia-Bellido, Maurizio Giannotti, Vera Gluscevic, Nathan Golovich, Manoj Kaplinghat, Sam McDermott, Michael Medford, Manuel Meyer, Annika Peter, Chanda Prescod-Weinstein, Oscar Straniero, Hai-Bo Yu}

In this section, we provide a brief review of several theoretical models of dark matter with a specific focus on the properties of these models that can be explored by LSST. 
We divide the domain of models into three different categories. 
We first discuss reasonably minimal extensions of the popular cold, collisionless particle dark matter paradigm (\secref{particles}). 
We then extend our discussion to much lighter axion-like particle and wave-like dark matter (\secref{axions}). 
Finally, we discuss the potential to constrain alternative compact dark matter models, with a focus on primordial black holes (\secref{machos}).
We stress that exploring a broad theoretical landscape with LSST is strongly motivated by the lack of an experimental discovery of a conventional CDM particle candidate. 

\section{Particle Dark Matter \Contact{Haibo}}
\Contributors{Haibo, Alex, Francis-Yan}
\label{sec:particles}

%\footnote{WIMPs and axions are not truly ``collisionless" because they require a coupling with other particles during their birth epoch in the early universe.  However, their interactions are so small that they are effectively collisionless particles for cosmology applications after that epoch.  Hence, WIMPs and axions are often considered to be the canonical examples of particle models for the CDM paradigm even though they are technically not completely collisionless.\AHGP{I added this.  Not sure this is the best place for it, but it would be good if this idea went somewhere.}}  

The standard \LCDM cosmological model assumes that dark matter is fully nonrelativistic and interacts purely via gravitational interactions during the process of structure formation. However, a significant dark matter thermal velocity dispersion or the presence of large non-gravitational interactions in the dark sector, such as dark matter self-interactions, couplings to other dark sector particles, or couplings to Standard Model particles, can alter the distribution of dark matter in ways that are observable with LSST. Here we focus on three representive minimal extensions to CDM -- warm dark matter (WDM), self-interacting dark matter (SIDM), and baryon-scattering dark matter (BSDM) --  to demonstrate how measurements of the distribution of dark matter can be used to constrain its micro-physical particle properties. We leave an in-depth discussion of the particle physics responsible for producing these models to the literature; however, we do attempt to connect astrophysical observables to specific terms in the interaction Lagrangian for each model.

\subsection{Warm Dark Matter (WDM)}
\label{sec:wdm}

In the standard thermal dark matter paradigm, primordial inhomogeneities in the matter density field are washed out by collisional damping and free streaming of particle dark matter \citep{Hofmann:2001,Green:2003un, Bertschinger:2006nq, Loeb:2005pm}.  
For a canonical 100-GeV thermal relic dark matter particle \citep[\eg, the WIMP;][]{Jungman:1995df}, these processes erase cosmological perturbations with $M \leq 10^{-6} \Msun$ \citep[i.e., Earth mass;][]{Green:2003un, 2005Natur.433..389D}. 
Lighter particles continue to free stream until later times, thus suppressing the formation of structure at higher mass scales (\eg, structure formation occurs bottom-up for scales larger than the free-streaming scale and top-down for scales smaller than the free-streaming scale). Because these particles are created while they are semi-relativistic, they are conventionally referred to as warm dark matter (WDM) \citep{Bond:1983hb,Bode:2000gq,Dalcanton:2000hn}. 
WDM constitutes a subclass of sub-GeV dark matter candidates.

One well-motivated WDM candidate is a sterile neutrino, $\nu_{\rm s}$, with a mass in the keV range \citep[\eg][]{Abazajian:2017tcc,Adhikari:2017}. The most relevant Lagrangian term in this case is simply the Majorana mass term,
\begin{equation}
    \mathcal{L} \supset -\frac{1}{2}M_{\rm s}\bar{\nu}_{\rm s} \nu_{\rm s}.
\end{equation}
Interestingly, such a sterile neutrino typically mixes with active Standard Model neutrinos \citep[\eg,][]{Asaka:2005an}, allowing the former to decay and leading to a potentially observable X-ray signal \citep[\eg,][]{Abazajian:2001vt}. A possible hint of such a signal has been found in deep X-ray data in the form of a narrow 3.5 keV line \citep{Boyarsky:2014, Bulbul:2014, Boyarsky:2015, Iakubovskyi:2015}, which has prompted renewed interest in understanding structure formation in WDM cosmologies \citep[\eg,][]{Lovell:2013ola,Bose:2016irl,Bozek:2018ekc}. Generally speaking, both the sterile neutrino mass and its thermal history play an important role in determining the small-scale dark matter distribution within any given particle model. For instance, at a fixed particle mass, a species created in the early universe with a velocity distribution that is skewed towards low-momentum particles \citep[\eg,][]{Shi:1998km,Venumadhav:2015pla} will display less free-streaming damping of cosmological structure than a species with a thermal (i.e., Fermi-Dirac) distribution. To avoid ambiguity, it is customary to quote WDM constraints simply in terms of a particle mass $m_{\rm WDM}$, assuming that the DM followed a thermal distribution at early times.

The free-streaming scale can be approximated by the (comoving) size of the horizon when the WDM particles become nonrelativistic.
%The comoving horizon size at $z = 10^7$ corresponds to $m = 2.5 \keV$, and is approximately $50 \kpc$, which is significantly smaller than the scale derived above for $\Lstar$ galaxies \citep{Adhikari:2017}
Astrophysical constraints on WDM are generally placed by observing the smallest gravitationally bound dark matter halos.  
In particular, the half-mode mass (the scale at which the dark matter transfer function is reduced by half) represents a characteristic halo mass scale below which halo abundances are suppressed sufficiently to yield observable consequences. 
The half-mode halo mass, \Mhm, is related to the WDM thermal relic particle mass, \mWDM, by \citep[\eg][]{schneider2012,Bullock:2017xww}
\begin{equation} \label{eqn:Mhm}
    M_{\rm hm} = 5.5 \times 10^{10} \left( \frac{m_{\rm WDM}}{1 {\rm keV}} \right)^{-3.33} \Msun.
\end{equation}
Thus, an observed suppression in the abundance of dark matter halos smaller than \Mhm could signify the existence of a thermal dark matter particle with mass
\begin{equation} \label{eqn:mWDM}
    m_{\rm WDM} =  3.33 \left(\frac{M_{hm}}{10^{9} {\rm M_\odot}} \right)^{-0.3} \keV.
\end{equation}
It is important to remember that $m_{\rm WDM}$ is the thermal-relic-equivalent particle mass. Translating measurements of the halo mass function to constraints on the particle mass for a specific WDM model depends on the specific mapping between particle mass and the early-time momentum distribution.

Measurements of the Lyman-$\alpha$ forest \citep[\eg][]{Viel:2013,2017PhRvD..96b3522I} and ultra-faint satellite galaxies \citep[\eg][]{Jethwa:2018,Kim:2017iwr} place lower bounds on the mass of thermally produced WDM particles at $\roughly 3$--$5\keV$, corresponding to a halo mass scale of $\roughly 10^8-10^9 \Msun$.
The sensitivity and wide-area coverage of LSST has the potential to extend measurements of the dark matter halo mass function by three orders of magnitude, down to $\roughly 10^6 \Msun$ (\secref{halo_mass}). 
These observations have the potential to constrain WDM particle masses \CHECK{$m_{\rm WDM} \gtrsim 18\keV$}, thus effectively testing putative signatures of keV-mass sterile neutrinos.


\subsection{Self-Interacting Dark Matter (SIDM)}
\Contributors{Haibo, Francis-Yan, Manoj?,...}
\label{sec:sidm}

The self-interacting dark matter (SIDM) paradigm posits additional interactions in the dark sector \citep[\eg,][]{1992ApJ...398...43C,Spergel:1999mh,Dave:2000ar,Firmani:2000ce}, which allow energy and momentum exchange between particles within dark matter halos \citep[see][for a recent review]{Tulin:2017ara}. The figure of merit for dark matter halo structure is the cross section per dark matter particle mass, $\sigma/m_\chi$.
%where $\sigma$ is a transfer cross-section,
%\begin{equation}
%    \sigma = \int d\Omega (1-\cos\theta) \frac{d\sigma}{d\Omega}.
%\end{equation}
%\AHGP{For identical particles, isn't this the wrong cross section?  Kim Boddy and Hai-Bo Yu advocate for the viscosity cross section.}
Dark matter self-interactions with cross sections per mass roughly equivalent to the strong nuclear force ($\sigma/m_\chi \sim 1 \cmg$) would imply ${\cal O}(1)$ energy exchange in the central regions of halos within the age of the universe \citep{2012MNRAS.423.3740V,2013MNRAS.431L..20Z,2013MNRAS.430..105P,Rocha:2012jg}. This  would thermalize the inner regions of dark matter halos --- where visible baryonic matter resides --- with observational consequences \citep[\eg][]{Kaplinghat:2013xca}. For low-surface brightness galaxies, SIDM thermalization leads to a cored inner density profile, in contrast to the cupsy profiles predicted in CDM. For high-surface brightness galaxies, thermalization leads to a small core and more concentrated SIDM distribution because of the presence of the baryonic potential \citep{Kaplinghat:2015aga}. It has been shown that SIDM can explain both the diversity and uniformity of galaxy rotation curves, for $\sigma/m_\chi \gtrsim 1\cmg$ on galaxy scales \citep{Kamada:2016euw,Creasey:2016jaq,Ren:2018jpt}. This diversity of properties within SIDM halos also extends to galaxy cluster scales \citep{Robertson:2017mgj}.

%\ADW{This is coming from Section VI.C in 1705.02358.}
Large self-interaction cross sections are required to modify galactic structure, and such cross sections suggest either strongly-coupled systems \citep[\eg,][]{Frandsen:2011kt,Hochberg:2014dra,Hochberg:2014kqa} or a light mediator with perturbative couplings \citep[\eg,][]{Feng:2009mn,Ackerman:2008gi,Kaplan:2009de,Feng:2009hw,Buckley:2009in,Loeb:2010gj,Tulin:2012wi,Tulin:2013teo,Schutz:2014nka,Blennow:2016gde}. An interesting example of the latter type of model would be to charge dark matter under a $U(1)$ gauge symmetry. %which may be either broken or unbroken. 
Exchange of the gauge boson (a ``dark photon'') then mediates self-interactions, analogous to Rutherford scattering. A phenomenologically similar model replaces the vector mediator with a light scalar. The interaction Lagrangian is then described by %\citep{1210.0900,1302.3898}:
\begin{equation}
\label{eq:sidm}
{\cal L_{\rm int}}=\bigg\{
\begin{array}{c l}
g_\chi\bar{\chi}\gamma^\mu\chi\phi_\mu & \text{(vector mediator)}\\
g_\chi\bar{\chi}\chi\phi & \text{(scalar mediator)} \, ,
\end{array}
\end{equation}
where $\chi$ is the dark matter particle (which we assume to be a fermion for concreteness), $\phi$ is the mediator, and $g_\chi$ is the coupling constant. In the non-relativistic limit, self-interactions are described by the Yukawa potential
\begin{equation}
V(r)=\pm\frac{\alpha_\chi}{r}e^{-m_\phi r},
\label{eq:yukawa}
\end{equation}
where $\alpha_\chi = g_\chi^2/4\pi$. In order for annihilation through the mediator to not deplete the dark matter relic abundance during the early universe, it may be necessary to assume asymmetric dark matter (that is, dark matter which is composed mainly of $\chi$, with a minimal admixture of $\bar\chi$). In that case, the vector mediator would provide only a repulsive potential (``$+$'' in Eq.~\ref{eq:yukawa}), while the scalar mediator would have an attractive potential (``$-$'').

%(a $+$ in Eq.~\eqref{eq:yukawa}) (a $-$ sign)

In these light mediator models, the self-scattering cross section generally depends on the relative velocity of colliding dark matter particles, $v_{\rm rel}$, and scattering is not isotropic. In practice, we often consider the transfer (viscosity) cross section, defined as $\sigma_T = \int d\Omega(1-\cos\theta)d\sigma/d\Omega$ ($\sigma_V = \int d\Omega\sin^2\theta d\sigma/d\Omega$), to regulate small-angle scatterings and use them as a proxy to match to SIDM N-body simulations with a constant cross section for a given halo-mass scale \citep[][]{Tulin:2013teo,Kahlhoefer:2013dca}. The overall feature of the velocity dependence predicted in the models can be summarized as follows. When the momentum transfer is much larger than the mediator mass, the scattering is in the Rutherford limit, i.e., $\sigma/m_\chi\propto v^{-4}_{\rm rel}$. While in the opposite limit, $m_\chi v_{\rm rel}\ll m_\phi$, $\sigma/m_\chi$ is nearly a constant. If the scattering is in the quantum resonant regime for $\chi\textup{-}\bar{\chi}$ collisions, $m_\chi v_{\rm rel}\sim m_\phi$, $\sigma/m_\chi\propto v^{-2}_{\rm rel}$. Since large dark matter halos have much larger dark matter velocities compared to smaller halos, observations from different scales, ranging from dwarf galaxies to galaxy clusters, provide important tests of these models.

%($m_\chi$, $m_\phi$, $\alpha_\chi$)

There are numerous observations that are sensitive to dark matter self scattering \citep[\eg,  Table 1 in][]{Tulin:2017ara}. Notably, merging galaxy clusters, such as the Bullet cluster  \citep{Randall:2007ph,2017MNRAS.465..569R}, have been used to put an upper bound on the self-interaction cross section at large particle velocities \citep[\eg,][]{Kahlhoefer:2013dca,Kahlhoefer:2015vua,Kim:2016ujt,Harvey:2016bqd,Robertson:2016qef,Wittman:2017gxn}, yielding $\sigma/m_\chi \lesssim 2 \cmg$ for $v_{\rm rel} \sim 1000\textup{--}4000 \kms$. Moreover, observations from well-relaxed galaxy clusters \citep{Newman++11,Newman:2013,Newman++13b} show $\sigma/m_\chi \sim 0.1 \cmg$ for $v_{\rm rel} \sim 1500 \kms$ to be consistent with their inferred core sizes~\citep{Kaplinghat:2015aga,Andrade:2019wzn}. The diversity of rotation curves observed in spiral galaxies can be explained by dark matter scattering with $\sigma/m_\chi \gtrsim 1 \cmg$ in the range of $v_{\rm rel}\sim50\textup{--}200 \kms$. For these spiral galaxies, the large cross section is driven by galaxies with a high density core. In contrast, high surface brightness galaxies are baryon-dominated in their central regions and are thus effectively insensitive to the value of $\sigma/m_\chi$ \citep{Kamada:2016euw,Ren:2018jpt}. 

Velocity-dependent SIDM models with $\sigma/m_\chi \gtrsim 1 \cmg$ in dwarf galaxies and $\sigma/m_\chi \sim 0.1 \cmg$ in galaxy clusters are able to fit existing observational data (\figref{sidm_sigma}). This result has important implications for the particle properties of SIDM. For instance, consider the dark photon model given in Eq.~(\ref{eq:sidm}) and assume $\alpha_\chi=1/137$ to match the fine structure constant in the visible sector, we can determine $m_\chi\approx 15 \GeV$ and $m_\phi \approx 17 \MeV$ \citep{Kaplinghat:2015aga} and even infer the production mechanism of SIDM in the early universe \citep{Huo:2017vef}. Since LSST will probe scales ranging from the largest galaxy clusters to the smallest dwarf galaxy, it will be able to detect the influence of scattering cross sections at the level of $\sigma/m_\chi \sim 0.1$--$1 \cmg$ over a wide range of velocities. Thus, LSST will significantly improve our understanding of the self-interacting nature of dark matter. 

\begin{figure}
\centering
\includegraphics[width=0.6\columnwidth]{figures/sigmav.pdf}
\caption{Velocity-weighted self-interaction cross section per unit mass as a function of average relative particle velocity in a halo. Data points from astrophysical observations correspond to dwarf galaxies (red), low-surface-brightness galaxies (blue), and galaxy clusters (green). 
Diagonal lines show constant values of $\sigma/m_\chi$. 
Gray points are fits to mock data from SIDM simulations, with fixed $\sigma/m_\chi = 1 \cmg$.
Figure taken from \citet{Kaplinghat:2015aga}.
}
\label{fig:sidm_sigma}
\end{figure}

It is also natural to expect that SIDM has a modified matter power spectrum compared to CDM. For instance, in SIDM models where the dark matter particle couples to a massless particle in the early universe, either directly or through a light mediator, the tight coupling between dark matter and dark radiation can lead to dark acoustic oscillations \citep{Cyr-Racine:2013ab,Cyr-Racine:2013fsa}, resulting in a suppressed and oscillatory power spectrum \citep[\eg][]{1992ApJ...398...43C,Boehm:2001hm,Boehm:2004th,Feng:2009mn,Aarssen:2012fx}. It has been shown that realistic realizations of SIDM strongly prefer such a scenario \citep{ Huo:2017vef}.

To see the reach of LSST on the SIDM damping effect, we estimate the cut-off scale on the field halo mass due to the dark acoustic oscillations as $M_{\rm cut}\approx0.7\times10^{8}({\rm keV/T_{\rm kd}})^3 \Msun$~\citep{1512.05349}, where $T_{\rm kd}$ is the kinetic decoupling temperature. For SIDM models, where a dark matter particle ($\chi$) couples to a massless fermion ($f$) via a light mediator ($\phi$), $T_{\rm kd}$ is given by~\citep{Aarssen:2012fx,Cyr-Racine:2015ihg}
\begin{equation}
T_{\rm kd}\approx\frac{1.38~{\rm keV}}{\sqrt{g_\chi g_f}}\left(\frac{m_\chi}{100 \GeV}\right)^{\frac{1}{4}}\left(\frac{m_\phi}{10 \MeV}\right)\left(\frac{g_\star}{3.38}\right)^{\frac{1}{8}}\left(\frac{0.5}{\xi}\right)^{\frac{3}{2}},
\label{eq:tkd}
\end{equation}
where $g_f$ is the $\xi\textup{--}f$ coupling constant, $g_\star$ the is the number of massless degrees of freedom at decoupling and $\xi$ parameterizes the ratio of dark-to-visible temperatures. \citet{Huo:2017vef} recasts the Lyman-$\alpha$ bound on WDM to set upper limit on the decoupling temperature $T_{\rm kd}\gtrsim 1 \keV$, corresponding to the minimal halo mass $\roughly 10^8 \Msun$. Since LSST has the potential to extend measurements of the dark matter halo mass function by three orders of magnitude (\secref{halo_mass}), the expected sensitivity on the decoupling temperature is $T_{\rm kd}\sim 10 \keV$. If LSST detects a cutoff on the halo mass function, we can determine the corresponding $T_{\rm kd}$ and further narrow down the particle parameters contained in the Lagrangian via Eq.~(\ref{eq:tkd}) after combining with the measurements of $\sigma/m_\chi$ discussed above. Moreover, since the damping effect can also suppress the number of subhalos in the Milky Way, we expect LSST to provide another constraint on $T_{\rm kd}$ by providing a more complete census of ultra-faint satellites. In addition, although the acoustic damping effect may look similar to the free-streaming one \citep[\eg][]{1512.05349}, distinct signatures can be imprinted on the halo mass function \citep{Buckley:2014ab,Sameie:2018juk} or the Lyman-$\alpha$ forest spectrum \citep{Krall:2017xcw,Bose:2018juc}. By combining observables, including those from LSST, it might thus be possible to distinguish between WDM and SIDM with a damped matter power spectrum due to early-universe interactions (\secref{combine_probes}).

     
\subsection{Baryon-Scattering Dark Matter (BSDM) \Contact{Vera}}
\Contributors{Vera, Kim, ...}
\label{sec:bsdm}

In the standard WIMP scenario, dark matter may be directly observable through its scattering with Standard Model particles.
These models are conventionally probed by direct detection experiments that search for scattering between dark matter particles (from the local Galactic halo) and nuclei in their detectors.
These experiments are placed deep underground to provide shielding from cosmic-ray backgrounds and achieve exquisite sensitivity for low scattering cross sections. 
These experiments are largely insensitive to dark matter with very large scattering cross section because such particles would scatter many times before reaching the experiment, thus losing most of their kinetic energy \citep[\eg][]{Zaharijas:2004jv}.
%However, given the current null results from these experiments, 
Thus, it is important to broadly explore parameter space outside the standard WIMP region of interest.

Cosmological and astrophysical observables are unique and complementary probes of baryon-scattering dark matter (BSDM) models.
In particular, they are sensitive to very large (closer to nuclear-scale rather than weak-scale) scattering cross sections and sub-GeV dark matter masses, both of which are inaccessible to direct searches.
Such large cross sections may arise in a number of models.
One such model posits that dark matter is a flavor singlet sexaquark composed of $uuddss$ quarks \citep{Farrar:2017eqq}.
In this case, the scattering cross section with nucleons is expected to be geometric, though velocity-dependent enhancements may exist at very low energies, depending on the form of the sexaquark-nucleon potential.
For the sexaquark to be a viable dark matter candidate, it must be stable or have a sufficiently long lifetime; this criterion sets the sexaquark mass to be below a few GeV.
Alternatively, dark matter may be charged under a dark version of electromagnetism with field strength $\tilde{F}_{\mu\nu}$, which may kinetically mix with ordinary electromagnetism~\citep{Holdom:1985ag}:
\begin{equation}
    \mathcal{L} \supset \frac{\kappa}{2} F^{\mu\nu} \tilde{F}_{\mu\nu} ,
\end{equation}
where $\kappa$ parameterizes the strength of the mixing.
In this scenario, dark matter acquires a fractional amount of electric charge (proportional to $\kappa$ and its dark charge), allowing it to scatter with electrons and protons via Coulomb interactions that have a velocity dependence of $v^{-4}$.
This interaction is significant at late cosmological times, as the universe expands and the momentum of matter redshifts away.

Instead of focusing on particular theories, it is possible to describe the low-energy scattering processes of BSDM models with a nonrelativistic effective field theory~\citep{Fan:2010gt,Fitzpatrick:2012ix,Anand:2013yka}.
The effective Lagrangian has the form
\begin{equation}
    \mathcal{L}_\textrm{eff}(\vec{x})
    = c \Psi_\chi^\ast (\vec{x}) \mathcal{O}_\chi \Psi_\chi (\vec{x})
    \Psi_N^\ast (\vec{x}) \mathcal{O}_N \Psi_N (\vec{x}) ,
\end{equation}
where $\Psi (\vec{x})$ are the nonrelativistic fields for the dark matter, $\chi$, and nucleon, $N$.
Dark matter experiments seek to constrain and measure the coefficient $c$ for a variety of possible operators $\mathcal{O}_\chi$ and $\mathcal{O}_N$ that encode the BSDM physics.
However, regardless of the specific underlying BSDM model, cosmological observables are sensitive only to the magnitude (which scales as $c^2$) and velocity dependence of the cross section.
Thus, while laboratory searches for dark matter typically rely on assumptions about the detailed form of the interaction, cosmology offers very broad and generic probes of dark matter physics.

In a cosmological setting, scattering results in the exchange of momentum between the dark matter and the baryon fluids.
The momentum transfer induces a drag force, which suppresses structure increasingly at smaller scales. 
The effect of scattering is qualitatively similar to a cutoff in the matter power spectrum arising in the WDM and SIDM scenarios (\figref{dmbaryon_pk}).
This feature can be sought with tracers of matter on all observable scales. 
The best cosmological and astrophysical limits so far come from CMB temperature, polarization, and lensing anisotropy measurements~\citep{Xu:2018efh,Boddy:2018kfv,Gluscevic:2017ywp,Boddy:2018wzy,Slatyer:2018aqg}, cosmic-ray observations \citep{Cappiello:2018hsu}, and Lyman-$\alpha$-forest measurements~\citep{Dvorkin:2013cea,Xu:2018efh}. 
LSST will probe the matter power spectrum on even smaller scales, through substructure measurements from dwarf galaxies in the Local Volume, gaps in stellar streams, galaxy strong lensing, and galaxy-galaxy weak lensing; such observations will substantially extend current experimental sensitivity to BSDM models.

\begin{figure}
\centering
\includegraphics[width=0.6\columnwidth]{figures/dmbaryon_pk2.png}
\caption{Residuals in the linear matter power spectrum between the CDM case and a case where dark matter has a velocity-independent, spin-independent scattering with protons. The dark matter particle mass is set to $1\MeV$, and all other cosmological parameters are set to their best-fit Planck 2015 values \citep{Ade:2015xua}. Different residual curves display cutoffs at different angular scales, controlled by the magnitude of the interaction cross section. The highest cross section shown corresponds to the current 95\% confidence-level upper limit inferred from analyses of CMB data \citep{Gluscevic:2017ywp,Boddy:2018kfv}.
}
\label{fig:dmbaryon_pk}
\end{figure}

As an example, a measurement of the minimum halo mass translates into an upper limit on the dark matter-proton interaction cross section, based on the corresponding cutoff in the matter power spectrum $P(k)$. \figref{dmbaryon_pk} shows how the position of the cutoff in the linear $P(k)$ varies as a function of the interaction cross section. For example, a lower limit on the cutoff of $k_\text{cutoff} \sim 10/\Mpc$ roughly corresponds to an upper limit on the cross section which is 100 times more stringent than the current limit from CMB searches. Using limits on WDM as a proxy for a suppressed $P(k)$, a minimum halo mass of $\roughly 10^6 \Msun$ would imply an improvement of roughly five orders of magnitude compared to the best current cosmological limits on the interaction cross section.

\section{Field Dark Matter \Contact{Chanda}}
\label{sec:axions}
\Contributors{Chanda Prescod-Weinstein, Samuel D.\  McDermott, Oscar Straniero,  Maurizio Giannotti, Alex Drlica-Wagner, Manuel Meyer}

While current observations of the matter power spectrum constrain the minimum mass of thermally produced dark matter, other mechanisms can  produce dark matter with significantly lower masses. The landscape of light dark matter candidates is vast, and in this section, we specifically focus on the class of axion-like particle (ALP) dark matter candidates.
ALP models span a wide range of viable parameter space (both in coupling strength and mass), and many of the observables described in this section can be generically applied to a broader class of light scalar particles.

The ALP paradigm was inspired by the QCD axion, which arises as a by-product of the most successful solution to the Strong CP Problem in the Standard Model \citep{PecceiQuinn:1977}. 
The cosmological abundance of axions is set by the Peccei-Quinn symmetry breaking scale, $f_\phi$, with a value
\begin{equation}
\Omega_\phi\sim\left(\frac{f_\phi}{10^{11-12}\,{\rm GeV}}\right)^{7/6}.
\end{equation}
This expression may be altered due to the temperature-dependence of the axion mass and ignorance about whether the Peccei-Quinn symmetry breaks before or after inflation. 
QCD theory gives no {\it a priori} prediction for the axion mass; however, in the context of dark matter composed of QCD axions, the axion mass is considered to be $m_\phi< 10^{-3} \eV$. %($10^{-39}$ kg) 
If the initial misalignment angle is order unity, this yields a QCD axion mass of $m_\phi \sim 10^{-5} \eV$. %($10^{-41}$ kg).
The broader category of ALPs possess QCD-axion-like potentials producing light scalar particles that obey a shift symmetry ($\phi \rightarrow \phi + 2\pi n$), but do not obey the same coupling between particle mass and symmetry breaking scale. 
ALPs can be motivated by string theory, where there are many moduli with axion-like potentials, and can produce a range of astrophysical phenomenology.
%ALPs can be sufficiently different from the QCD axion so as to produce notably different astrophysical phenomenology. 
ALPs may be non-thermally produced in the early universe and survive as a cold dark matter population until today \citep[\eg][]{Arias:2012az}.



There has been significant debate in the literature about the astrophysical phenomenology of the QCD axion and ALPs.
\citet{Sikivie:2009} noted that because the axion is a scalar with high abundance in the early universe (circa matter-radiation equality), the axion could potentially settle into a Bose-Einstein condensate (BEC) state, whereby all particles can be described using one coherent ground state wave function. 
Furthermore, \citet{Sikivie:2009} argue that during the radiation-dominated era, axions will rethermalize into BECs with a Hubble-scale correlation length.
This could produce significant observational implications, such as several-kpc-scale caustic structures observable in the stellar distributions of the Milky Way and other low-redshift galaxies \citep[\eg,][]{Natarajan:2006,0805.4556,Rindler-Daller:2013zxa}.

On the other hand, \citet{1412.5930} argue that a particle such as the QCD axion, which has an attractive self-interaction, will not sustain Hubble-scale correlations in an attractive potential.
Instead, \citet{1412.5930} predict that axions will form coherent clumps that have been called ``Axion stars'' or ``Bose stars'' \citep[\eg][]{Kolb:1993}.\footnote{For the QCD axion it would be more appropriate to call these ``axteroids'' (a term coined by Anna Watts) due to their mass of $\roughly 10^{-11} \Msun$ \citep{Tkachev:1991ka,Braaten:2018nag}.} 
Looking beyond the QCD axion, some ALP models suggest that compact BEC ``miniclusters'' could form and grow to $\gtrsim 1\Msun$, at which point they may be detectable by LSST through mergers with other compact objects \citep{1808.04746} or through microlensing \citep{1707.03310}, as discussed in \secref{compact_objects}.

Additional astrophysical constraints on ALPs generally come from proposed couplings with photons and/or electrons. 
For example, the Lagrangian can be expressed as
\begin{equation}
    \mathcal{L} = -\frac{1}{2} \partial_\mu\phi\partial^\mu\phi + \frac{1}{2}m_\phi^2 \phi^2 - \frac{1}{4}g_{\phi\gamma}F_{\mu\nu}\tilde{F}^{\mu\nu}\phi - g_{\phi e}\frac{\partial_\mu\phi}{2m_e}\bar{\psi}_e \gamma^\mu\gamma_5\psi_e,
\end{equation}
where $g_{\phi\gamma}$ is the photon-axion coupling, $g_{\phi e}$ is the axion-electron coupling, $F^{\mu\nu}$ is the electromagnetic field tensor (and $\tilde{F}$ its dual), and $\psi_e$ is the electron field \citep[\eg][]{1302.6283,Redondo:2013wwa}.\footnote{Additional couplings to nucleons are allowed, but are not relevant for the LSST observations discussed here.}
For sufficiently large couplings to photons or electrons, the ALP can manifest as an additional anomalous energy loss mechanism, transporting energy out of the interiors of stars \citep[\eg,][]{Raffelt:1990}.
This energy loss could affect the evolution of stars, for example altering the lifetimes of giant stars \citep{Ayala:2014,Viaux:2013hca,Viaux:2013lha} or the cooling rate of white dwarf stars \citep{Isern:2008nt}.
The precise photometry of LSST will provide sensitive measurements of stellar populations to search for deviations from the predictions of standard stellar evolutionary models (\secref{cooling}).

Astrophysical observations place the only known lower bound on the mass of ALPs and other non-thermally produced ultra-light particles, commonly described as ``fuzzy'' dark matter \citep[FDM; \eg,][]{Hu:2000,Hui:2017}. 
The de Broglie wavelength of these particles is constrained to be smaller than the size of the smallest galaxy, $\mathcal{O}(1\kpc)$, setting a lower limit on particle mass at $m_\phi \gtrsim 10^{-21}\eV$ \citep{1703.04683}. 
In addition, FDM is predicted to produce solitonic cores in the centers of halos, which would measurably affect the velocity profiles of dark-matter dominated galaxies \citep{Robles:2012uy,Robles:2018fur,Schive:2014hza,Du:2016aik}. 
Dark matter substructure is predicted to be less abundant in FDM than its CDM counterpart due to quantum interference effects.
This is similar to the case of WDM, and again dark matter properties can be constrained through mesurements of the least massive dark matter halos.
Combining Equation 8 of \citet[][]{1703.09126} with \eqnref{Mhm} in \secref{wdm}, we can express constraints on the minimum FDM mass, $m_\phi$, as a function of the half-mode halo mass, $M_{\rm hm}$:
\begin{equation}
M_{\rm hm} = 1.2 \times 10^{11} \left( \frac{m_\phi}{10^{-22}\eV} \right)^{-1.4} \Msun,
\end{equation}
or expressed in terms of $m_\phi$, 
\begin{equation}
m_\phi = 3.1 \times 10^{-21} \left( \frac{M_{\rm hm}}{10^{9}\Msun} \right)^{-0.71} \eV.
\end{equation}
LSST will be sensitive to light bosonic dark matter with mass $m_\phi \sim 10^{-20} \eV$ by probing the power spectrum of dark matter halos with half-mode mass of $\Mhm \sim 10^{8} \Msun$. 
Sensitivity to heavier bosonic particles ($m_\phi > 10^{-19} \eV$) would be possible through the detection of even smaller halos ($\roughly 10^{6} \Msun$).





\section{Compact Objects \Contact{Simeon}}
\label{sec:machos}
\Contributors{Simeon Bird, Juan Garc\'ia-Bellido, George Chapline, William A.\ Dawson, Nathan Golovich, Michael Medford}

Compact objects, particularly black holes, represent one of the oldest and most venerable models of dark matter. 
%Primordial black holes could originate from small-scale density fluctuations during the era of inflation. 
\NEW{Primordial black holes could form at early times from the direct gravitational collapse of large density perturbations that originated during inflation.}
The same fluctuations that lay down the seeds of galaxies, if boosted on small scales, can lead to some small areas having a Schwarzschild mass within the horizon, which spontaneously collapse to form black holes. \NEW{Alternatively, some particle dark matter models may allow dark matter cooling and collapse, providing another mechanism for black hole formation \citep[\eg][]{1705.10341,1707.03419,1707.03829,1812.07000}.} 
\NEW{Such black holes are unlikely to radiate strongly enough from accretion to leave a detectable signature and are thus a natural candidate for dark matter
%Because these black holes do not accrete or radiate strongly (at the time of formation there is no gas to form an accretion disc), they are a natural candidate for dark matter 
\citep{Carr:1974nx,Meszaros:1974,1975Natur.253..251C,Bellido:1996,2016PhRvD..94h3504C,1802.08206}. \NEW{The abundance of compact objects tests dark matter through a purely gravitational channel and is thus sensitive to dark matter models that cannot be probed in the laboratory.}

Compact object dark matter is fundamentally different from particle models; primordial black holes cannot be studied in an accelerator and can only be detected through their gravitational force. Current constraints suggest that primordial black holes do not make up all of dark matter \citep[\eg][]{Sasaki:2018}. However, these constraints may be evaded if \NEW{primordial black holes} are spatially clustered \citep{Clesse:2015,Clesse:2017}. Moreover, primordial black holes are one possible source of the merging $30 \Msun$ black holes recently detected by LIGO \citep{Bird:2016,Clesse:2016}. This possibility has rekindled interest in these objects, both as a source of dark matter and in their own right.

Limits on the abundance and mass range of primordial black holes are wholly observational. The black hole mass is set by the mass enclosed within the horizon at the time of black hole collapse and thus ranges between $10^{-18} \Msun$ ($10^{15}\g$), below which the black hole would evaporate, and $10^9 \Msun$ ($10^{42}\g$), above which structure formation, Big Bang Nucleosynthesis and the formation of the microwave background would be severely affected \citep{Sasaki:2018}. 
For stellar mass black holes, the gold standard for detecting compact objects is microlensing. Current microlensing constraints set limits on the black hole abundance at the level of $10\%$ for black holes $0.01 - 10 \Msun$ \citep[however, see][]{Calcino:2018}. LSST will revolutionize the astrometric microlensing technique,  constraining the abundance of primordial black holes to a level of $10^{-4}$ of the dark matter over a wide range of masses (\secref{compact_objects}).

As primordial black holes form directly from the primordial density fluctuations, a measurement of their abundance would directly constrain the amplitude of density fluctuations \citep{Carr:1974nx, Meszaros:1974}. %1203.2681 
Although these constraints are several orders of magnitude weaker than, for example, those from the microwave background, they probe small scales between $k = 10^{7} - 10^{19}$ $h$/Mpc, much smaller than those measured by other current and future probes \citep{Bringmann:2012}. Because these scales are highly non-linear in the late-time universe, there is no other possible constraint; the information present at early times has been washed away by gravitational evolution. Primordial black holes are thus a probe of primordial density fluctuations in a range that is inaccessible to other techniques~\citep{Josan:2009,Bellido:2017,Bellido:2018}. These curvature fluctuations are imprinted on space-time hypersurfaces during inflation, at extremely high energies, beyond those currently accessible by terrestrial and cosmic accelerators. 
Our understanding of the universe at these high energies, of order $10^{15} \GeV$ and above, comes predominantly from extrapolations of known physics at the electroweak scale.
Measurements of the primordial density fluctuations via the abundance of primordial black holes would provide unique insights into physics at these ultra-high energies.

\NEW{In additon, dark matter interactions with the Standard Model may also generate new channels for black hole formation, by triggering collapse of astrophysical objects \citep[\eg][]{1989PhRvD..40.3221G,1004.0629,1012.2039,1405.1031,1804.06740}. LSST could be sensitive to transient events that could be triggered by these formation scenarios \citep[\eg][]{1706.00001} or as part of a muli-messanger campaign to measure sub-solar mass black hole mergers \citep[\eg][]{1808.04771,1808.04772}.}


Furthermore, it may be possible for LSST to constrain the existence of ultra-compact mini-halos using correlated microlensing signals \citep{erickcek2011,li2012}. These objects arise from initial overdensities that are too small to collapse into primordial black holes. These overdensities still collapse at high redshift to form low-mass halos; thus, since these objects form early and have few mergers \citep{Bringmann:2012,Delos:2018}, they have a high concentration and a steeper internal density profile than the standard Navarro-Frenk-White shape. In turn, this makes them easier to detect via lensing and harder to disrupt than standard CDM subhalos. Current constraints on these objects are highly model-dependent. In particular, they largely come from counting gamma-ray photons from astrophysical sources under the assumption of a WIMP dark matter annihilation cross-section. LSST will place new constraints on the existence of small halos via micro-lensing and thereby constrain the physics of the inflaton on scales of $k = 10 \textup{--} 10^7 h$/Mpc for the first time in a model-independent way.




%%%%%%%%%%%%%%%%%%%%%%%%%%%%%%%%%%%%%%%%%%%%%%%%%%%%%%%%%%%%%%%%%%%%%%%%%%%%%%%%%%%%%
% Dark Matter Probes
%%%%%%%%%%%%%%%%%%%%%%%%%%%%%%%%%%%%%%%%%%%%%%%%%%%%%%%%%%%%%%%%%%%%%%%%%%%%%%%%%%%%%

\section{Dark Matter Probes}
\label{sec:probes}

The theoretical models described in \secref{theory} each produce one or more deviations from the predictions of cold, collisionless, non-interacting dark matter.
These ``probes'' of dark matter physics include: a minimum dark matter halo mass, alterations to the halo density profile, over-abundance of compact objects, anomalous energy loss, and unexpected correlations in large-scale structure.
In some cases, several distinct physical models of dark matter can be probed by the same (or very similar) observable (\eg, a cutoff in the dark matter halo mass spectrum could be produced by both keV-mass thermal dark matter and $10^{-22}\eV$-mass fuzzy dark matter). 
On the other hand, a single probe can manifest itself in a wide range of astrophysical systems (\eg, changes to dark matter profile shape can produce observable effects in systems ranging from clusters of galaxies to dwarf galaxies).
In this section we do not attempt to provide a comprehensive discussion of all possible astrophysical probes of dark matter physics.
Rather, we focus on specific probes and observables where LSST will have a major impact.
%ADW: should we note something like, "major impact alone or in combination with other observation"?

\subsection{Minimum Halo Mass}
\label{sec:halo_mass}

The cold, collisionless model of dark matter makes the strong prediction that dark matter halos should exist down to Earth-mass scales (or below) in WIMP and non-thermal axion models \citep{Green:2003un,1412.5930}.
Several modifications to the cold, collisionless dark matter paradigm can suppress the formation of dark matter halos on these small scales.
Current observations provide a robust measurement of the dark matter halo mass spectrum for halos with mass $> 10^{10}\Msun$, and the smallest known galaxies provide an existence proof for halos of mass $\roughly 10^8 \Msun - 10^9 \Msun$ \citep{behroozi2018,Jethwa:2018,Kim:2017iwr,Nadler:2018}. 
Extending below this halo mass threshold is challenging due to our limited observational sensitivity to the faintest galaxies.
In addition, halos with mass $\lesssim 10^8 \Msun$ are expected to be devoid of galaxies, requiring novel detection techniques that do not rely on the baryonic content of halos.
Here we explore improvements that LSST will make in measuring the faintest galaxies, and in probing dark matter halos below the threshold of galaxy formation with stellar streams and strongly lensed systems.

\subsubsection{Constraints from the Smallest Galaxies \Contact{Ethan}} 
\Contributors{Alex, Keith, Mitch, Andrew Pace, Ethan, Yao, Arka, Risa, Mei-Yu, Francis-Yan, Kim,...}
\label{sec:smallest_galaxies}

% Some slides from KITP Dark Matter Program in May 2018 with potential ideas:
% https://drive.google.com/open?id=1BhuwyNE7vClIeVV6FhM99QtictTvpQ0p

%\YYM{maybe we need to be specific about dwarfs within the Milky and outside}
\vspace{1em} \noindent \textbf{The Threshold of Galaxy Formation}

Galaxies are born and grow within dark matter halos.
%(Willman and Strader 2012). \RHW{would use another ref for this} \KB{Agreed}
To first approximation, galaxies with the largest stellar masses reside within the highest-mass dark mark matter halos, while fainter galaxies---which are much more numerous---occupy dark matter halos with progressively smaller typical masses; however, the scatter between stellar mass and halo mass is likely large in the low-mass regime (see \citealt{Wechsler:2018} for a recent review).
The smallest and faintest galaxies offer a natural place to search for extremely low-mass dark matter halos, which are in turn sensitive probes of dark matter microphysics. Another advantage of probing low-mass dark matter halos using faint galaxies is that we can study their properties in detail, e.g.\ via follow-up spectroscopy (\secref{complementarity}). The challenge in interpreting these observations in terms of dark matter properties is the complex relationship between baryons and halos at this extreme mass scale and the effects of baryonic physics both within subhalos and on subhalo populations as a whole \citep[e.g.,][]{DOnghia:2009xhq,Brooks:2012ah,errani2017,Garrison-Kimmel:2017zes,Fitts:2018ycl,brooks2018}.

The least luminous galaxies currently known contain only a few hundred stars and have been found exclusively in the inner regions of the Milky Way due to observational selection effects. Although the census of Milky Way dwarf galaxies has grown from $\sim25$ to more than 50 in recent years \citep[e.g., with DES;][]{Bechtol:2015, Koposov:2015, Drlica-Wagner:2015}, our current census is certainly incomplete.
For example, the HSC-SSP collaboration has detected two ultra-faint galaxy candidates in the first 300 square deg of the survey \citep{1609.04346,1704.05977}; these galaxies are faint and distant enough to have been undetectable in previous optical imaging surveys. HSC is representative of the depth that will be achieved by LSST over half the sky---an area 60 times larger than the current HSC-SSP footprint. Thus, based on the results of SDSS, HSC, DES, etc., several groups have predicted that LSST could detect %at least 20 --- and as many as 50 \EON{check numbers} \ADW{Seems a bit small. I thought Hargis predicted $\roughly250$.} \AHGP{Check out the numbers from Table 1 of Kim, Peter, \& Hargis: we predict 50-600 depending mostly on the radial distribution of satellites in the halo.  The lower number is the most conservative choice, for a population of UFDs concentrated at the center of the halo, and is consistent with Newton et al.  The largest number is what we predict if the hydro sims of satellite disruption are correct.}--- 
tens to hundreds of new low-luminosity Milky Way satellites, mainly at larger distances and fainter luminosities than those accessible with current-generation surveys \citep{Hargis:2014,Newton:2018,Jethwa:2018,Nadler:2018,Kim:2017iwr}. LSST observations of Milky Way satellites therefore offer an exciting testing ground for dark matter models; for example, the measured abundance, luminosity function, and radial distribution of Milky Way satellites \emph{already} place competitive constraints on warm dark matter particle mass at the level of 3--4\keV \citep[e.g.,][]{Jethwa:2018,Kim:2017iwr}.%, and these constraints will improve as observations continue to probe the low-mass end of the subhalo mass function.

\vspace{1em} \noindent {\bf Minimum Subhalo Mass Inferred from Milky Way Satellites}

A driving question for near-field cosmology with LSST is how well we can use the population of Milky Way satellites to constrain the minimum dark matter halo mass necessary for galaxy formation. % by mapping out the extreme faint end of the stellar-mass to halo-mass relation. 
This ``minimum halo mass" depends on the details of reionization and other forms of baryonic feedback which prevent gas from accreting and cooling in low-mass subhalos; however, it might also reflect a cutoff in the subhalo mass function determined by the particle nature of dark matter (i.e., WDM or FDM). In particular, models that produce a cutoff in the matter power spectrum generally suppress the number of subhalos below a characteristic mass threshold (\eqnref{Mhm}). Thus, the existence, abundance, and properties of the smallest galaxies generically lead to constraints on dark matter models that reduce small-scale power.

To relate these questions to LSST observations, we have analyzed simulated ultra-faint galaxies as they would appear in LSST WFD coadd object catalogs to quantify LSST's ability to detect nearby satellite galaxies. %, and we have developed a theoretical framework to connect observations of the Milky Way satellite population to the underlying dark matter subhalo population. 
We detect ultra-faint galaxies as arcminute-scale statistical overdensities of individually resolved stars; in ground-based optical imaging surveys, it is often challenging to classify low signal-to-noise catalog objects near the detection threshold as either foreground stars or unresolved background galaxies. LSST will reach depths at which the galaxy counts far outnumber stellar counts, so the search sensitivity for ultra-faint galaxies will largely be determined by our ability to accurately perform star-galaxy separation at magnitudes $24 < r < 27.5$; importantly, our sensitivity analyses include these effects. 
In detail, we inject many simulated stellar populations into the center of the LSST DESC DC2 simulated data set and encapsulate the results of a satellite search algorithm run on these stellar populations using a surface brightness detection threshold of $\mu = 32\ \rm{mag\ arcsec}^{-2}$ and an absolute magnitude cutoff of $M_V = 0\ \rm{Mag}$ for all satellites within $300\ \rm{kpc}$.
%or just quote the results? ADW: probably need some detail because we haven't published the DES search yet.

Figure \ref{fig:satellite_mmin} shows the minimum subhalo mass that LSST can probe via observations of Milky Way satellites, obtained by folding our search sensitivity estimates through a cosmological model of the MW satellite population that predicts satellite luminosity functions, radial distributions, and size distributions that agree well with current observations. In particular, we generate many mock MW satellite populations using the model presented in \cite{Nadler:2018} given a ``true" value of the minimum subhalo mass necessary for galaxy formation, $\mathcal{M}_{\rm{min,true}}$, marginalizing over the relevant galaxy--halo connection parameters. We then perform mock observations of these generated satellite populations using the LSST selection function, and we compare these to the true satellite populations by MCMC sampling $\mathcal{M}_{\rm{min}}$ and the remaining galaxy--halo connection parameters assuming that satellite number counts are Poisson distributed in bins of absolute magnitude (see \citealt{Nadler:2018} for details on the fitting procedure). For each value of $\mathcal{M}_{\rm{min,true}}$, this procedure yields a posterior distribution over the minimum halo mass inferred by LSST observations. The red band in Figure \ref{fig:satellite_mmin} illustrates the recovered $95\%$ confidence interval as a function of $\mathcal{M}_{\rm{min,true}}$, and the blue dot-dashed line indicates the minimum halo mass inferred from known classical and SDSS-detected MW satellites. %LSST observations recover the true minimum halo mass at large values of $\mathcal{M}_{\rm{min,true}}$, since all of the predicted satellites are observable in this regime, while smaller values of $\mathcal{M}_{\rm{min,true}}$ yield satellites that do not pass our detection criteria, which prevents the lowest-mass subhalos that host satellites to be detected. 
For small $\mathcal{M}_{\rm{min,true}}$, the $95\%$ confidence level upper bound on the lowest detectable subhalo mass improves by more than a factor of $4$ with LSST, from $\sim 5 \times 10^{8}\ \Msun$ to $\sim 1.2 \times 10^{8}\ \Msun$.

Although we have presented a ``population-based" forecast for dark matter constraints from LSST-detected ultra-faint satellites, we note that kinematic data obtained by follow-up spectroscopy of newly discovered satellites also offers a powerful probe of dark matter microphysics. We estimate the number of LSST-detected MW satellites that can be spectroscopically confirmed in Section \ref{sec:spectroscopy}, and we forecast the constraints offered by these stellar velocity dispersion measurements for WDM and SIDM in the following sub-section.

\begin{figure}
\centering
\includegraphics[width=0.75\textwidth]{figures/LSST_Mmin.pdf}
\caption{Forecast for the minimum dark matter subhalo mass probed by LSST via observations of Milky Way satellites. The red band shows the $95\%$ confidence interval from our MCMC fits to mock satellite populations as a function of the true minimum subhalo mass necessary for galaxy formation. Note that we marginalize over the relevant nuisance parameters associated with the galaxy--halo connection---including the effects of baryons using a model calibrated on subhalo disruption in hydrodynamic simulations---in our sampling.}\label{fig:satellite_mmin}
\end{figure}

\vspace{1em} \noindent {\bf Discovery space for WDM and SIDM}

%The number of satellites of the Milky Way and Andromeda is getting close to 100, and corresponding constraints on the theory space is already impressive \FIXME{(refs)}. The discovery of ultra-faint satellites and a measurement of their mean dark matter densities opens up the possibility of discovering the particle physics of dark matter. 

As stated above, the cut-off in the matter power spectrum is related to early universe kinematics of the dark matter particle or the interactions of dark matter particles with light particles. This cut-off directly feeds into the subhalo mass function and hence the luminosity function of the satellites. Moreover, the central densities of these satellites are related to both the cut-off in the matter power spectrum and to dark matter self-interactions. The cut-off in the power spectrum delays halo formation and hence lowers the concentration at fixed halo mass. The self-interaction creates cores in the subhalos, which also lowers the central density. In the limit of large cross sections or significant mass loss, the self-interactions could lead to core collapse and an increase in the subhalo density, which is an exciting possibility that LSST would be sensitive to.

In order to demonstrate the power of LSST to probe dark matter physics, we focus on two cumulative number counts. The first is the number of satellites above a given luminosity threshold, $N(>L_{lim})$. The second is the number of satellites (above a luminosity threshold) that have densities above a given threshold, $N(L_\star>L_{lim}, \sigma_\star> \sigma_{\star,lim})$. For the density measurement, we use the luminosity-averaged stellar velocity dispersion as the proxy. Specifically, we adopt the following simple model for relating the stellar dispersion $\bar{\sigma}_\star$ to density $\rho_0$:
\begin{equation}
\bar{\sigma}_\star = 1 {\rm km}/{\rm s} \sqrt{\frac{\rho_0}{0.1 {\rm M}_\odot/{\rm pc}^3}} \frac{R_h}{50 {\rm pc}}\,,
\end{equation}
which is valid as long as the core radius is much larger than the project half-light radius $R_h$. 

The connection between the central density ($\rho_0$) and the self-interaction cross section or cut-off in the power spectrum can be modeled in the following way to isolate some key physics. In the limit where the self-interaction cross section is small, we assume that the core size is negligible. Note that stellar feedback can change this statement, so there is the possibility of degeneracy between feedback and self-interactions. In the limit that the core size is much smaller than the half-light radius $R_h$, we assume that the density profile is of the NFW form with a concentration that only depends on the half-mode mass (given by the power spectrum):
\begin{equation}
c(M;M_{\rm cut}) = c_{\rm cdm}(M)\left(\frac{M_{200}}{10^{12}{\rm M}_\odot}\right)^{\Delta\alpha(M_{\rm WDM})}\,,
\end{equation}
where $M_{200}$ is the mass of the subhalo at a radius where the density of the subhalo is 200 times the critical density of the universe today. 

In the limit of large cores created by self-interactions, the central density of the subhalo can be written as
\begin{equation}
\rho_0=\rho_s f(t/t_0)\,
\end{equation}
where $t_0 = ab(\sigma/m)\rho_s v_0$ with $v_0^2=4\pi G \rho_s r_s^2$, $a=\sqrt{16/\pi}$ for a hard-sphere interaction~\citep{Balberg:2002ue}, and $b=25\sqrt{\pi}/32$ calculated from Chapman-Enskog theory~\citep{1970mtnu.book.....C}. The function $f$ encodes the evolution of the core density with time in the presence of self-interactions. It depends on the orbital history such that highly stripped halos will evolve faster for the same $\rho_s$ and $r_s$, the NFW density and radial scale parameters. 

Let us define a core radius by the equation $\rho_{\rm NFW}(r_c) = \rho_0$ where $r_c$ is the core radius. If $r_c < R_h$, then we should modify the relation between $\sigma_\star$ and density. We use the simple modification that $f(t/t_0)$ has a maximum value given by $1/(x_h(1+x_h)^2)$, where $x_h = R_h/r_s$. 

To incorporate these results we make the further assumption that the tidal evolution of subhalos is not highly sensitive to dark matter particle physics. Current simulations tend to support this point of view, although it is likely that the stripped mass fraction will be somewhat larger for subhalos that are puffier in the center due to either a cut-off in the power spectrum or self-interactions \citep{Lovell:2013ola, 1603.08919}. 

To map the present-day mass of our subhalos to their luminosities, we combine the zoom-in simulations presented in \cite{Mao2015} with the subhalo--satellite galaxy model presented in \cite{Nadler:2018} to obtain the probability that a subhalo of present-day mass $M_{\rm{vir}}$ hosts a satellite of luminosity $L_\star$. The mapping from subhalos to satellites includes a prescription for hydrodynamic effects such as enhanced subhalo disruption due to a central galactic disk, the galaxy formation threshold due to reionization, and a flexible model for the relationship between luminosity and subhalo peak circular velocity. To characterize $P(L_\star|M_{\rm{vir}})$, we sample from the posterior distribution of model parameters from the fit to classical and SDSS-detected satellites in \cite{Nadler:2018}, generate a large number of satellite population realizations for each Milky Way host halo, and fit the resulting $(L_\star|M_{\rm{vir}})$ relation with a normal distribution.

Using this mapping from $L_\star$ and halo mass to $M_{200}$, we can write down the observables in the following form \EON{should $M_{200}$ here be $M_{\rm{vir}}$?}:
\begin{equation}
    \begin{aligned}
    N_1 = N(L_\star>L_{lim})= \int_{L_{lim}} dL_\star \int dM_{200} \frac{dn}{d M_{200}} P(L_\star|M_{200}) \\
    \end{aligned}
\end{equation}

\begin{equation}
    \begin{aligned}
N_2 =& N(L_\star>L_{lim}, \sigma_\star> \sigma_{\star,lim}) \\ 
    =& \int_{L_{lim}} dL_\star \int_{\sigma_{\star,lim}} d\sigma_\star \int dM_{200} \int dc_{200} \frac{dn}{d M_{200}} P(L_\star|M_{200})\\
    & \times\ P(c_{200}|M_{200}) \delta(\sigma_\star-\bar{\sigma}_\star(\rho_s,r_s))
    \end{aligned}
\end{equation}

We require information of this likelihood function:
\begin{eqnarray}
L(N_1,N_2|\sigma/m,m_{\rm wdm}) \propto & {\rm Possion}(L_\star>L, \sigma_\star < \sigma_{\star,lim}) \\
& \times\, {\rm Poisson}(L_\star>L, \sigma_\star> \sigma_{\star,lim})
\end{eqnarray}
For any LSST-detected satellite dataset, i.e., values of $N(L_\star>L, \sigma_\star < \sigma_{\star,lim})$, $N(L_\star>L,\sigma_\star> \sigma_{\star,lim})$, $L$ and $\sigma_{\star,lim}$, we can draw detection limit contours at fixed likelihood. 
We choose $L_{lim}=100 {\rm M}_\odot$ and $\sigma_{\star,lim}=3 {\rm km}/{\rm s}$. 
The $3 {\rm km/s}$ dispersion is chosen because we have order ten satellites that have velocity dispersion values larger than this. 
The lower limit on $L_\star$ is chosen with the view that LSST will detect many more systems with luminosity similar to Segue 1 \citep[\eg][]{Hargis:2014}. 
%EON: if we need to use a specific number for the total number of satellites (or the number of new satellites discovered by LSST), it should be consistent with what Josh cites in the complementarity section.
The resulting projected joint constraints on the WDM particle mass and the SIDM cross section are shown in \figref{sidm_wdm}.

To detect even lower-mass subhalos, it will likely be necessary to use gravitational probes. Two examples of such probes are described in the following subsections. \ADW{Can we also make a version of the plot where we make projections for stream/lensing constraints?}

\begin{figure}
\centering
\includegraphics[width=0.6\columnwidth]{figures/SIDM_WDM_fig.pdf}
\caption{\label{fig:sidm_wdm} Warm dark matter mass vs self-interacting dark matter cross section. Two fundamental characteristics of dark matter that are best probed with astrophysics. \EON{N.B. even though there is a degeneracy between SIDM/WDM when considering subhalo counts and profiles, it might be possible to break this degeneracy using the fact that halos form later in WDM -- see \citet{Bozek:2018ekc} for hydro simulation results.}}
\end{figure}


\subsubsection{Gaps in Stellar Streams \Contact{David H.}}
\Contributors{Nora Shipp, Ting Li, David Hendel, Ana Bonaca, Andrew Pace, Jo Bovy, Sergey Koposov, Nilanjan Banik}
\label{sec:stream_gaps}


Stellar streams, in particular the tidally disrupting remnants of globular clusters, are fragile, dynamically cold systems and are sensitive tracers of gravitational perturbations \citep[][]{Carlberg:2012}.
The main track of a stream in 6D phase space is shaped primarily by the Milky Way's global matter distribution while the detailed structure of the stream contains information about small-scale perturbations. 
In particular, a dark matter subhalo passing by the stream will provide a net velocity kick, altering the orbits of the closest stream stars.
The main observable consequence of this interaction is the formation of a gap in the density of stars along the stream; the relative depth and size of the underdensity can be used to infer the time since the encounter and the properties of the perturber \citep{Carlberg:2012, Erkal:2015}. The mass required to produce an observable gap \citep[$10^5-10^6 M_\odot$,][]{erkal2016,bovy:2017} is well below the limit where dark matter subhalos are expected to host galaxies. Thus, stellar streams provide one of the most exciting near-field tests of the minimum subhalo mass.

Current constraints on the minimum subhalo mass from stream gaps are limited by the small number of streams that are bright enough that observations can detect density variations at a useful signal-to-noise ratio. Deep and precise LSST photometry is expected to increase the contrast between streams and the contaminating Milky Way field stars, to have improved star-galaxy separation, and to extend much farther down the color-magnitude diagram for known streams, dramatically increasing our ability to detect density variations and thus leading to the identification of less prominent gaps created by low-mass perturbers. Critically, with LSST we move from examining individual gaps into the regime where we can ask questions about the subhalo population statistics and their (in)consistency with cold dark matter.
Here we estimate the least massive subhalo that can be detected with LSST observations of gaps in stellar streams.

We consider a mock-stream observed at a Galactic latitude of $b=-60^\circ$ in the $g$- and $r$-band. We assume the stream is old (12\,Gyr), metal-poor ($Z = 0.0002$), thin (1$\sigma$ stream width of 20\,pc), and cold (velocity dispersion of 1\,km\,s$^{-1}$) We generate synthetic photometry of the stream at a given mean surface brightness (within the $1\sigma$ width) and over a range of heliocentric distances from 10 to $40\kpc$.  Simulated stream stars are drawn from a Chabrier IMF \citep{2003PASP..115..763C}, while a synthetic background of Milky Way stars is generated from the Galaxia model \citep{sharma2011}. To mimic the observations, we add noise into the photometry for both the stream and Milky Way stars using the LSST error model \DAH{reference the LSST stack repository somehow?}. We then select stars in the color-magnitude diagram that are within $2\,\sigma$ of the theoretical isochrone of the stream's age and metallicity, where $\sigma$ is the magnitude-dependent photometric uncertainty using the same error model. 
% we assume $\sigma$ is no less then 0.02 mag (i.e. set to 0.02 if it's smaller).
We also assume a limiting magnitude to set the depth of the survey, choosing the point where the photometric uncertainty in either band exceeds 0.1 mag. We apply this color-magnitude selection to determine the density of stream stars and background stars.
The depth of the gap from a given subhalo mass is calculated using the theoretical relation derived by \citet{erkal2016}, assuming that the subhalo passed by within the past 0.5\,Gyr, moving at 150\,km\,s$^{-1}$, with an impact parameter equal to the perturber's scale radius. Finally, we define a detection as a gap depth that is $5\,\sigma$ above the noise background (the effects of star-galaxy separation are not considered in this calculation).

\begin{figure}
\centering
\includegraphics[width=0.85\textwidth]{figures/streamgap_constraint_2.png}
\caption{Detection limits for gaps formed from subhalos of different masses using photometry from SDSS (blue) or the 10-year LSST stack (green) as a function of the stream surface brightness. Shaded regions correspond to a 10-40 kpc distance range, with the lines representing 20 kpc. For streams with surface brightnesses similar to those found in the Dark Energy Survey, 32-33 $\mathrm{mag}\ \mathrm{arcsec}^{-2}$, LSST is expected to probe halo masses two to three orders of magnitude smaller than SDSS and substantially improve the current constraints from Milky Way satellites \citep{Nadler:2018, Jethwa:2018} and the Lyman $\alpha$ forest \citep{2017PhRvD..96b3522I}. We connect the detected halos to the mass of the warm dark matter particle that would produce a minimum halo of that mass using the relationship determined by \cite{Bullock:2017xww}. The observed surface brighnesses of two DES streams, Indus and ATLAS, are also shown for comparison \citep{2018ApJ...862..114S}. \label{fig:streamsurveys}}
\end{figure}


% In a nutshell: for a stream of a given surface brightness and heliocentric distance, we calculate the density level $5\,\sigma$ above the noise as the threshold for gap detection, and infer the corresponding subhalo mass under sensible assumptions regarding the encounter age and impact parameter.
% We consider a stream that is old (12\,Gyr), metal-poor ($Z = 0.0002$), thin (20\,pc) and cold (velocity dispersion 1\,km\,s$^{-1}$).
% In the fiducial case, this stream encountered a dark matter subhalo 0.5\,Gyr ago moving at 150\,km\,s$^{-1}$, which triggered gap formation.
% The final depth and size of the gap are a function of the perturber's mass, impact parameter (we assume it equals the perturber's scale radius), and time after the impact \citep{erkal2016}.
% The stream is mock-observed in a stellar field based on the Galaxia model of the Milky Way \citep{sharma2011}, assuming a galactic latitude of $b=-60^\circ$.
% Both the stream synthetic photometry (after assuming a distance to the stream) and the contamination from Galaxia have been convolved with the LSST error model.
% To estimate the density of stream stars, we sample the theoretical luminosity function to match the assumed surface brightness, and then select stars in the color-magnitude diagram that are within $2\,\sigma$ of the theoretical isochrone of the stream's age and metallicity ($\sigma$ are the color- and magnitude-dependent photometric uncertainties).
% Similarly, we estimate the background density by applying the same isochrone selection to the mock catalog from Galaxia.
% Finally, for a given subhalo mass, the relative density between the stream and the gap is calculated using relations from \citet{erkal2016}.

In Figure~\ref{fig:streamsurveys} we show how the lowest-mass dark matter subhalo detectable using the 10 year LSST stack depends on the stream's surface brightness and heliocentric distance. For a stream with a surface brightness of 33 (31.5) $\mathrm{mag}\,\mathrm{arcsec}^{-2}$, LSST is able to detect subhalos down to $2 \times 10^7$ ($1 \times10^6$) $M_\odot$ at 20 kpc. As a comparison, we used the same model to calculate the gap detectability using SDSS DR9 photometry. LSST provides $\sim 3$ orders of magnitude improvement at low surface brightnesses, where most known (and anticipated) streams lie. Crucially, this pushes the minimum detectable halo mass below current constraints from Milky Way satellites \citep[\eg,][]{Jethwa:2018} or the Lyman $\alpha$ forest \citep[\eg,][]{2017PhRvD..96b3522I}.

Given a gap density detection threshold and a subhalo population, this formalism can be used to predict the number of gaps in a given stream as in \citet{erkal2016}. Typically this is $\sim 1$ gap  so the well--studied individual streams (i.e. Palomar 5 and GD-1) are challenging to interpret. LSST is expected to measure dozens of streams as precisely as Palomar 5 and GD-1 have currently been mapped and will therefore provide a much stronger constraint: at the 10 year LSST depth, \LCDM predicts we should observe 17 gaps total in the 13 DES streams reported by \cite{2018ApJ...862..114S}. Observing fewer than 6 gaps would be inconsistent with \LCDM at a 99.9\,\% level. \ADW{Add a few words about power spectra analyses?}

%Depending on the surface brightness and distance of the streams detected with LSST, the corresponding bound on the warm dark matter particle mass may be significantly stronger than those obtained from Milky Way satellites \citep[$\sim 2.95$ keV,][]{2018MNRAS.473.2060J} or the Lyman $\alpha$ forest \citep[5.3 keV,][]{2017PhRvD..96b3522I}. Stream surface brightnesses from \cite{2018ApJ...862..114S}

%\EON{Might consider adding a band for the satellites constraint between $3 \times 10^{8}$ at the low end (the best-case result from Nadler+18, which also makes the constraint shown here consistent with Fig.\ 5) and the Jethwa+18 result at the high end. I would prefer not to include the satellite bounds in the sentence about WDM, since the theoretical uncertainties underlying the WDM constraint for satellites are large. After quoting the minimum mass streams can probe, I'd compare to the satellite bound in terms of Mmin.} \DAH{I believe I have dealt with this}.


%\begin{figure}
%\centering
%\includegraphics[width=\textwidth]{figures/lcd_limits_DES_LSST10.pdf}
%\caption{At the 10 year LSST depth, $\Lambda$CDM predicts we should observe 17 gaps in the 13 DES streams reported by \cite{2018ApJ...862..114S}. Observing fewer than 6 gaps would be inconsistent with $\Lambda$CDM at a 99.9\,\% level. %\EON{I'm curious how the disk model for subhalo disruption was implemented -- if the amount of subhalo disruption is radially dependent (which is the case in hydro simulations, e.g. Garrison-Kimmel+2017), the lower end of the constraint here might be weaker.} \DAH{An important point but there are a huge number of worse assumptions than that already, I think this is infeasible until we write a proper paper.}
%\label{fig:gapcounts}}
%\end{figure}

Depending on their orbit, stellar streams can also be perturbed by the baryonic structures such as the Galactic bar \citep[e.g.][]{erkal2017,pearson2017}, spiral arms \citep{Banik2018}, or giant molecular clouds \citep{amorisco2016}. The resulting gaps in may be indistinguishable from gap induced by dark matter subhalos and can therefore wrongly lead us in overestimating the number of subhalo impacts on a stellar stream. The only recourse is to carefully examine the streams' orbits to assess these possible confounding factors. Streams with pericenters of $\gtrsim 14\kpc$ should be relatively unaffected by these baryonic factors, and streams on retrograde orbits even less so. In addition, subhalos may also experience extra tidal shocks from the disk, which can alter the number of expected impacts in a given cosmological model \citep[e.g.][]{DOnghia2010,Garrison-Kimmel2017}. LSST will mitigate both of these issues by examining streams farther out into the Galactic halo where the effects are lessened.

In this summary we have only considered the density structure of the stream. However, the perturbation that creates the gap necessarily affects the other phase space dimensions as well. The inclusion of these phase space dimensions allows for an almost unique determination of both the subhalo's internal and impact properties for each gap \citep{erkal2015b}. Furthermore, the perturber's effect produces a correlated signal across observables, improving the precision with which the statistical properties of the stream (e.g. power spectrum and cross-correlation of observables) can be used to measure subhalo properties \citep{bovy:2017}. This provides an exciting opportunity for synergy with current and future spectroscopic and astrometric surveys in addition to precise photometric distances and proper motions from LSST itself. Such efforts will greatly aid in the removal of foreground and background contamination as well as tightening constraints on the stream progenitor's orbit and providing a better measurement of the perturbers mass and size. See Section~4 for a discussion of some complementary science programs.

% - minimum subhalo mass that can be detected by a stream w lsst 10yr depth photometry
% - assume surface brightness of a stream, long enough to detect a gap, assume distance
% - perturber creates an underdensity using erkal equation for gap width, depth as a function of time, for a given mass
% - fixed encounter to 0.5Gyr ago, fixed b, dispersion 1km/s (refills stream easier if larger)
% - based on the surface brightness, populate cmd from metal-poor gc -- assign stars w colors, get number of stars per deg 
% - there is a contamination galaxia model at a fixed location on the sky (b=-60), so there is the noise model at different magnitudes
% - add lsst projected uncertainties for magnitudes
% - define detection as snr=5 (delta density in gap >=5 noise in the)
% - find min perturber mass for snr=5 at a given surface brightness, distance to the stream

% * caveats:
% \begin{itemize}
% \item perfect star galaxy separation
% \item assumption on the gap age
% \item gap growth equation -- theoretical
% \item encounter parameters fixed
% \item stream parameters fixed
% \end{itemize}

% * limitations of the method
% \begin{itemize}
% \item wouldn't identify large gaps, instead see 2 streams
% \item only variations due to subhalos; GMCs, disk, spiral arms, etc. all known to cause similar gaps
% \end{itemize}

% * possibly could do better
% \begin{itemize}
% \item now selection done in 2 photometric bands (also assuming gr same for all of the surveys)
% \item try different sigma for isochrone match - trade for s/n?
% \item only considering density variation, not changes in the track
% \item proper motion selection should improve contrast too
% \end{itemize}

% * future:
% - better selection of members w additional bands, say u
% - astrometric selection
% - can we move to a more empirical framework of creating a forward model of the stream (perturbed w some powerspectrum of perturbers), analyze the power-spectrum
% -- observe jo's stream

% * david's test:
% - change noise model to sdss
% - isochrones w lsst filters (before used des gr)

% * to do:
% - vary magnitude limit to emulate star-galaxy separation issues
% - vary magnitude uncertainties: 1 epoch \& full lsst stack
% - emulate sdss, cfht, des, ...
% - vary time, impact parameter (lookup cosmologically sensible values)
% - vary progenitor parameters
% - vary different b
% - vary distance from the isochrone
% - add surface brightness for existing streams. e.g. ATLAS = 33, Phoenix = 32.6

% * assignments:
% - david: different latitudes in galaxies
% - nora: distribution of impact parameters, times
% - ting: validation w other surveys
% - ana: write + comb the code

% CFHT and SDSS survey condition added:

% For CFHT, we take the Pal 5 data from Ibata+2016 paper and calculate the median uncertainty in g and r band as a function of g-/r-band magnitude.

% For SDSS, we take SDSS DR9 data in the area of Pal 5 (1 deg radius) and calculate the median uncertainty in g and r band as a function of g-/r-band magnitude.

% For LSST, we take the estimation from LSST DM (?) and we set two cases, one for 1 year of LSST coadd and one for 10 year of LSST coadd.

%We then determine the survey to be at uncertainty  = 0.1 mag for both g and r band. This corresponds to g = 22.4, 24.X, 25.3, 26.6 for SDSS, CFHT, LSST-1yr, LSST-10yr, respectively.

%TODO:
%As we did not include any contamination from the background galaxies, we assess how the star galaxy separate would change the minimal detectable halo mass by changing the survey depth. Based on the error model, the limiting magnitude at $\sigma$ = 0.1 mag is at $g=26.6$ for LSST-10yr stack. We change to depth at $g = XX$ and $g = XX$ and ....

%Also, we did not consider the power of the astrometry from LSST. For example, proper motion info from LSST can remove part of the background star and improve the detectability.

%We should also mention that the track of stream will also get purtubed by the subhalo flyby. We did  not do a projection here, but in principle that is another component that LSST could use to detect the minimal subhalo mass.

%(Similarly, RV perturbation could be measured by  spectroscopic follow-up. Details in Sec 4.1.2.


\subsubsection{Strong Lensing: Substructure and Line-of-Sight Halos\Contact{Chris F.} 
\label{sec:stronglens}} 
\Contributors{Cora, Chris, Francis-Yan}

Strong gravitational lensing is one of the most powerful probes of dark matter halos beyond the Local Goup. 
Gravitational lensing directly probes the total mass distribution that a light ray encounters and does not require that mass to be luminous or baryonic.
Therefore, an analysis of lensing signals can be used to measure the presence, quantity, and mass of subhalos in massive galaxies and small isolated halos along the line-of-sight.  
The discovery of low-mass dark matter halos is possible even at cosmological distances, where the flux of any luminous material associated with the halos would fall below the detection limits of typical observations.  
Thus, the gravitational lensing approach is highly complementary to Local Group observations.

\begin{figure}
    \centering
    \includegraphics[width=0.4\textwidth]{figures/2045_vla_dec96_x.png}    \includegraphics[width=0.44\textwidth]{figures/Clone_labeled.png}
    \caption{Examples of two gravitational lens systems that exhibit perturbations due to (potentially unseen) halos.  {\bf (a)} Radio-wavelength imaging of a quasar lens system, B2045, that has one of the strongest flux-ratio anomalies known.  Component B should be the brightest of the three close images and instead it is the faintest. Figure from \citet{Fassnacht++99}
    {\bf (b)} HST imaging of the ``Clone'' showing that the long lensed arc is split by the presence of a perturber, in this case galaxy G4.  Note that the location and mass of G4 could have been determined {\em even if G4 were purely dark}.  Figure from \citet{Vegetti_2010_1}.}
    \label{fig:stronglens_examples}
\end{figure}

The (sub)halo-detection techniques described below utilize strong gravitational lensing, in which a massive foreground object bends the light from a background galaxy to produce multiple images of the background object.  
If the emission from the background object is dominated by a single point-like component, such as a quasar or other AGN, the lens system will contain multiple images of that component (e.g., Figure~\ref{fig:stronglens_examples}a).
Typically these quasar lens systems consist of two or four images, creating ``doubles'' and ''quads'' respectively. 
If, on the other hand, the background object is dominated by stellar emission, then the lensed emission is in the form of tangentially stretched arcs or a full Einstein ring that surrounds the lensing galaxy (e.g., Figure~\ref{fig:stronglens_examples}b).  
In both cases, substructure in the main lensing galaxy and small line-of-sight halos create small perturbations to the lensed images.

As will be described in detail below, there are three main techniques for detecting the presence of dark (sub)halos using strongly lensed systems: analysis of flux-ratio anomalies in lensed quasar system,
gravitational imaging for lensed galaxy systems, and power spectrum approaches. 
Improved constraints on dark matter properties via these measurements will require: (1) a much larger samples of lens systems, and (2) follow-up observations with high-resolution imaging and spectroscopy.
LSST will play a critical role by increasing the number of lensed systems from the current sample of hundreds to an expected samples of thousands of lensed quasars \citep{O+M10} and tens of thousands of lensed galaxies \citep{Collett2015}.
The vast increases in sample sizes will provide much stronger statistical constraints on dark matter models than are currently possible (\eg, \figref{lensing_wdmlim_vs_nlens}).
The study of lensed systems will also require coordination with other facilities, namely space-based observatories, large ground-based telescopes with adaptive optics systems, ALMA, and very-long-baseline radio interferometry (see \secref{SLcomplement}). These facilities provide the milliarcsecond-scale angular resolution that is required to push the (sub)halo detection sensitivity into unexplored mass regimes.

\begin{figure}
    \centering
    \includegraphics[width=0.75\textwidth]{figures/wdm_constraints_yh.jpg}
    \caption{Constraints on the WDM particle mass as a function of the number of strong lens systems that achieve a given (sub)halo mass detection threshold, under the assumption that CDM is correct.  Figure courtesy of Y. Hezaveh. \ADW{These LyA constraints are different from the ones in the dwarf and stream section.}}
    \label{fig:lensing_wdmlim_vs_nlens}
\end{figure}


\vspace{1em} \noindent {\bf Flux-ratio Anomalies}

The presence of clumpy (dark) matter, whether within the main halo of the primary lens or along the line of sight, will perturb the gravitational potential of a strong lens system.
One of the effects of these perturbations is to change the magnification of the lensed images of a background AGN.
The angular scales over which the perturbations are important depend on the mass the perturber, so the presence of a small (sub)halo will typically affect only one of the lensed images and, thus, will change the relative fluxes of the images.
Furthermore, because the image magnification depends on the second derivatives of the gravitational potential, this method is, in theory, sensitive to smaller-mass structures than the gravitational imaging approach described below.\footnote{Indeed, flux-ratio anomalies can be produced by stars in the lens galaxy, through the microlensing phenomenon discussed below.}

The utility of this effect was first presented in \citet{Mao:1998aa}, which considered the effects of spiral arms in the lensing galaxy on the flux ratios of the lensed images, and for many years this was the only lensing technique used to investigate the presence of substructure in massive galaxies.
The approach is to describe the lensing galaxy with a relatively simple smooth single halo model.
These simple models are nearly always capable of fitting the observed positions of the lensed images to within the observational errors.
At that point, any deviation between the model-predicted image fluxes and the observed fluxes could be ascribed to some type of non-smooth / clumpy mass, either in the lensing galaxy or along the line of sight.
At optical and near-IR wavelengths, there are often significant differences between the predicted and observed image fluxes.
However, these perturbations are most likely to be produced by stars in the lensing galaxies, a process known as microlensing, and thus optical and near-IR fluxes are not informative in terms of the statistics of dark matter halos.
What is required is to observe at wavelengths at which the angular size of the emitting region in the background source is large compared to the micro-arcsecond scales at which stars produce their effects.  
Until recently, this meant observing lensed quasar systems at radio or mid-IR wavelengths, which vastly reduced the available sample sizes.

A seminal paper by \cite{Dalal:2002aa} used the statistics of observed flux-ratios in a sample of seven lens systems to place limits on the substructure fraction in the lensing galaxies, i.e., the percentage of the lens mass that is composed of clumpy structures, in the $10^6 - 10^9 M_\odot$ range.
The small sample size was set by the number of radio-loud systems that was known at the time and the one lens system with a usable mid-IR data set.  
Because lensed radio-loud AGN are rare, and ground-based high-resolution mid-IR observations are extremely difficult, the sample size only increased by a few lenses over the next decade.  
In contrast, forecasts based on forward modeling simulations indicate that $\gtrsim$100 well-constrained flux-ratio systems are needed to provide 2$\sigma$ constraints of $10^{7.2} - 10^{7.5}~M_\odot$ for the half-mode mass in a WDM scenario, corresponding to a $\sim$5--6~keV thermal relic mass \citep{Gilman++18}.
Therefore, large increases in sample sizes are required.
The two most promising paths forward are to obtain large lensed quasar samples with LSST and then follow up with either high-resolution mid-IR imaging with JWST, or IFU spectrographs on ELTs or JWST.  The second technique takes advantage of the fact that in lensed AGN, the narrow-line region surrounding the central AGN is larger than the microlensing scale, even though the broad-line region and the source of the continuum emission are not.  Therefore, with high-resolution IFU observations, the narrow-line emission from each lensed image can be spatially resolved, thus providing the required microlensing-free flux ratios \citep{MoustakasMetcalf03, Nierenberg++14, Nierenberg:2017vlg}.

Deep high-resolution imaging in the optical or infrared is also necessary to address possible systematics in the flux-ratio technique.  
Investigations using Keck adaptive optics imaging of radio loud lenses have shown that, in some cases, the observed flux-ratio anomalies can be explained by baryonic structures in the lensing galaxy, namely edge-on stellar disks rather than dark matter halos \citep{Hsueh++2016, Hsueh++2017}.
These baryonic effects were also seen in simulated data \citep{Gilman++2017, Hsueh++2018}.
These studies indicate that a lack of knowledge about the baryonic structure of the lensing galaxy may lead to an overestimate of the amount of clumpy dark matter in the lens or along the line of sight.
With a sample of thousands of quasar lenses expected from LSST, it will be possible to select systems where baryonic effects are minimized.

\vspace{1em} \noindent {\bf Gravitational Imaging}

The presence of a massive peturber along the line of sight can change the shape of lensed emission. 
This effect can be utilized in strong lens system in which the background object is a galaxy that is lensed into long arcs or a complete Einstein ring.
Small (sub)halos that are close in projection to the lensed emission can distort arc shape to a degree that can be detected by high-resolution imaging observations.
This ``gravitational imaging'' technique was proposed by \cite{Koopmans:aa} and further refined by \citet{Vegetti:2008aa,Vegetti:2009aa}.  The size of this effect depends on the mass of the perturber and its projected distance from the lensed arcs, with more massive and closer perturbers having larger effects.
 
The first application of the gravitational imaging technique to real data was for the ``Clone'', a system for which the primary lensing halo is a compact galaxy group \citep[\figref{stronglens_examples},][]{Vegetti_2010_1}.
 In this system, the long lensed arc is broken and split at the location of the peturber, which in this case is a satellite galaxy in the group with a mass of $\roughly 10^{10} \Msun$ \citep[][]{Vegetti_2010_1}.  This massive galaxy located right on the arc produced an effect that could be seen by eye in high-resolution HST imaging.  Lower-mass detections were subsequently made using HST \citep[$\roughly 10^9 \Msun$;][]{Vegetti_2010_2}, Keck adaptive optics \citep[$\roughly 10^8 \Msun$][]{Vegetti_2012}, and ALMA mm-wave interferometry \citep[$\roughly 10^8 \Msun$][]{Hezaveh_2016ltk}.  
 Note that the masses reported in these papers usually assume a truncated mass distribution (e.g., a pseudo-Jaffe) or are explicitly given as mass contained within radii of, e.g., 600\pc, to better match dwarf galaxy measurements made within the Local Group.  Multiplying these values by a factor of 10 gives roughly the expected virial mass of their host halos.
 
The implications for the nature of dark matter from the gravitational imaging technique come from comparing the number of detected halos to those predicted by various dark-matter models.  
For this reason, one of the strengths of the technique is that {\em non-detections} are as valuable as detections, and can be especially powerful at low masses where CDM models predict a large number of halos.
This analysis relies on an understanding of the lowest mass that can be detected at each location in the lens system \citep[e.g.,][]{Vegetti2014, Hezaveh_2016ltk, Ritondale++18}.
 
Nearly all previous inferences on dark matter from gravitational imaging have considered solely the expected and measured effects of subhalos within the main halo of the primary lensing galaxy \citep[e.g.,][]{Vegetti:2009aa, Vegetti_2012, Vegetti2014, Hezaveh_2016ltk}.
However, an additional perturbation signal is provided by the presence of halos along the line of sight.
An analysis of simulated data has shown that the signal from line-of-sight structures is significant even for lower redshift lenses and is the dominant contribution to any lensing signal for higher redshifts \citep{Despali++18}.
\CRK{Cite any of the older literature?}
The line-of-sight structures may very well be a cleaner probe of dark-matter properties than substructures in the lensing galaxies.
This is because the line-of-sight halos are unlikely to have been tidally stripped and thus their measured masses reflect their true masses.
The techniques for including the line-of-sight signal have been developed and applied to recent analyses \citep{Ritondale++18}. 
 
The relatively high (sub)halo masses that have been probed so far, $\gtrsim 10^9\Msun$, there is little difference between the predictions of CDM and models with a mass cutoff (\eg, WDM).
Therefore, even analyses of $\roughly10$-lens samples have not achieved the statistical precision to distinguish between dark matter models \citep{Vegetti2014, Ritondale++18}.
What is urgently needed is both to increase the sample sizes and, more importantly, to probe further down the mass function.
The mass-detection limit for gravitational imaging is set by three properties of the observations: (1) the signal-to-noise ratio, (2) the angular resolution of the imaging data, and (3) the surface-brightness structure of the lensed background galaxy.  This last point arises because it is easier to detect small astrometric shifts if there are strong gradients in the surface brightness, as opposed to a smooth light distribution.
These properties lead to the need for sensitive high resolution observations of the large samples of appropriate lenses that LSST will provide.
The high-resolutions observations can come from ELTs, which should provide milliarcsecond-scale angular resolution currently only available from VLBI radio observations.
For the subset of LSST lenses that are radio loud, VLBI and ALMA observations will provide excellent complementarity.

\vspace{1em} \noindent {\bf Small-scale Structure Power Spectrum}

While gravitational imaging can detect highly significant and well-localized perturbers along lensed arcs and Einstein rings, less massive perturbers or those located further away from lensed images typically lead to observational signatures that are too subtle to be detected individually. However, the large number of such perturbers, both as subhalos within the lens galaxy and as field halos along the line of sight, means that their collective effect might be detectable at the statistical level \citep[\eg,][]{Birrer2017}. The power spectrum of the lense deflection field is a particularly powerful quantity for capturing the aggregate behavior of lensing perturbers. This approach was proposed in \cite{Hezaveh_2014}, and further expanded in \cite{Rivero:2017mao}, \cite{Chatterjee_2017}, and \cite{Cyr-Racine:2018htu}. 

A key advantage of this power spectrum approach is that it describe the effect of perturbers in terms of a \emph{spatial fluctuation} basis instead of the more traditionally used \emph{mass} basis. The power spectrum directly captures the spatial scales on which perturbers influence the lensed images without having to invoke the notion of (sub)halo density profile, the latter of which is usually required to map from the perturber mass space to the resulting spatial deflection field. As such, the power spectrum is a natural language to describe the collective effects of small lensing perturbers. 

To develop intuition about which dark matter properties could be probed from measurement of this new lensing statistic, \cite{Rivero:2017mao} developed a general formalism to compute from first principles the convergence power spectrum for different populations of subhalos (not yet including line-of-sight perturbers). The authors pointed out that this power spectrum can be mainly described by three quantities: a low-wavenumber amplitude, that depends on the subhalo abundance and on specific statistical moments of the subhalo mass function; on a turnover scale, that probes the truncation radius of the largest subhalos in the system; and on a higher-wavenumber ($k\gtrsim 1 \kpc^{-1}$) slope, that probes a combination of the subhalo inner density profiles and of a possible cutoff in the primordial matter power spectrum. These theoretical findings were then confirmed numerically in \cite{Brennan:2018jhq} using a semi-analytic galaxy formation model, and in \cite{Rivero:2018bcd} using high-resolution $N$-body simulations. It is now apparent that measurements of the power spectrum could provide a wealth of information about the behavior of dark matter on small scales.

Several challenges need to be addressed to fully enable the constraining abilities of power spectrum measurements. Most importantly, the degeneracy between the possibly complex brightness profile of the source and the statistical effects of the lensing perturbers needs to be accurately explored. Also, the importance of line-of-sight structure remains to be properly quantified, and the effect of lens galaxy light and other luminous foregrounds on the power spectrum inference needs to be better understood. Finally, since instrumental artifacts such as a mismodeled point-spread function or camera sensitivity could potentially mimic a power spectrum signal, it is likely that these effects would have to be reconstructed at a higher precision than what is typically done for gravitational imaging. 

\begin{figure}
\centering
\includegraphics[width=0.75\textwidth]{figures/Fisher_space_Pk_SIDM_rev.pdf}
\caption{Fisher forecast for the substructure convergence power spectrum in three logarithmic wavenumber bins. We consider here observations with the wide-field camera 3 (WFC3) aboard the Hubble Space Telescope (HST) using the F555W filter, resulting in a point-spread function FWHM of $0.07$ arcsec. The source is placed at $z_{\rm src}=0.6$ with an unlensed magnitude $m_{\rm AB}=24$. The error bars show the $1$-$\sigma$ regions, while the green rectangles display the sample variance contribution within each bin. We conservatively assume that only half of each orbit is available for observation. The blue solid line shows the fiducial substructure power spectrum model used in the forecast, which corresponds to a CDM population of subhalos modeled with truncated NFW profiles. The dotted magenta line shows the power spectrum for SIDM, assuming a subhalo core size equals to $70\%$ of the scale radius. For comparison, the orange dashed line shows the substructure power spectrum for a thermal relic warm DM with mass of 3.5 keV. \label{fig:pksub_fisher}}
\end{figure}

Thus far, measurement of the lensing power spectrum has been attempted by \cite{Bayer:2018vhy}, and an upper limit on its amplitude was derived using HST archival data. The currently known samples of galaxy-galaxy lenses numbers in the few dozens, and LSST is expected to increase this number several-fold as mentioned above. High-resolution follow up using either space-based or AO-enabled ground-based observatories will be required (\secref{SLcomplement}) to measure the power spectrum from these targets and thus probe small-scale structure in a new way. \citet{Cyr-Racine:2018htu} has performed detailed forecasts (\figref{pksub_fisher}) for the sensitivity of different observational scenarios to the perturber power spectrum for lenses of the type that LSST is expected to discover at optical wavelengths. It was found that images only a factor of a few deeper than what is currently typically available \citep[\eg, from the SLACS sample][]{Bolton2008} could be sufficient to detect the overall amplitude of the lensing power spectrum. On the other hand, constraining the slope at larger wavenumbers, which could help distinguish between WDM and CDM (\figref{pksub_fisher}), would require much deeper imaging.

\subsection{Halo Profiles} 

The standard CDM model predicts that dark matter halos should be cuspy, asymptoting to high central densities.
This results from the inability of collisionless dark matter to redistribute kinetic energy, and is born out in numerical simulations which give rise to a family of cuspy halo profiles \citep[\eg, the NFW profile,][]{Navarro:1998}.
If dark matter is able to interact through scattering or the exchange of some light mediator (see \secref{sidm}), then the density of halos could instead flatten out to produce dark matter ``cores''.
These interactions can also lead to an isotropization of dark matter velocity distribution, leading to more spherical halos.
Thus, measurements of the radial density profiles and shapes of dark matter halos are sensitive to the microphysics governing dark matter self-interactions.
Here we explore the contributions that LSST will make towards measuring the profiles of dark matter halos in isolated small galaxies and clusters of galaxies.
We highlight these systems because they reside at opposite extremes of the galaxy mass spectrum where dark matter dominates over baryonic processes that can also alter the shapes of halos.


\subsubsection{Dwarf Galaxies as Lenses \Contact{Yao}}
\Contributors{Yao, Annika, James, Tony, Manoj?, ...}
\label{sec:halo_profile_group}

\subsection{Dwarf Galaxies as Lenses \Contact{Yao}}
\label{sec:halo_profile_group}
\Contributors{Yao-Yuan Mao, M.\ James Jee, Alex Drlica-Wagner, J.\ Anthony Tyson, Annika H.\ G.\ Peter, Chihway Chang, Rachel Mandelbaum, Manoj Kaplinghat}

Dwarf galaxies ($M_\star \lesssim 10^{9} \Msun$) provide the best visible tracers of low-mass dark matter halos. 
The relatively low baryonic content makes dwarf galaxies sensitive probes of  dark matter physics through the shape of their dark matter halo profiles. 
In particular, the ``core-cusp'' problem in dwarf galaxies has been cited as one of the most significant challenges to CDM \citep[\eg,][]{2010AdAst2010E...5D,Bullock:2017xww}.
The standard CDM model predicts that dark matter halos should have steeply rising (``cuspy'') central densities in contrast to the shallower (``cored'') mass profiles that are observationally inferred for many dwarf galaxies.  
Evidence for cored profiles exists for Milky Way satellite galaxies from kinematic and theoretical studies \citep[\eg,][]{Walker:2009, 2012ApJ...759L..42P}, and is stronger when one studies the inner density profiles of dwarf galaxies based on high-resolution neutral hydrogen surveys \citep[\eg,][]{Begum:2008,Hunter:2012,Cannon:2011,Oh:2015}. 
Many of these observations show inferred central slopes of the dark matter density profile, $\rho(r) \sim r^{-\gamma}$, that are significantly shallower ($\gamma \approx 0$--$0.5$) than the CDM prediction $\gamma \approx 0.8$--1.4 \citep{Navarro:2010}.

A wide range of solutions to the core-cusp problem have been proposed including observational, astrophysical, and dark matter explanations.
From a dark matter perspective, SIDM can significantly suppress the the central density of halos.
A self-interaction cross-section of $\sigma / m_\chi \sim 1 \cmg$ can explain the diversity of rotation curves seen in low-mass spiral galaxies \citep[\eg,][]{1504.01437,2017PhRvL.119k1102K,Tulin:2017ara}.
In addition, ultra-light or fuzzy dark matter has also been suggested as a possible solution to the core-cusp problem through the formation of uniform density solitonic cores \citep[\eg,][]{1502.03456,Hui:2017}. 
However, baryonic feedback remains a major complication for interpreting central density profile measurements in a dark matter context \citep{1996MNRAS.283L..72N,2005MNRAS.356..107R,2008Sci...319..174M,2012MNRAS.421.3464P,Madau:2014,Read:2016}. 
If dwarf galaxies form enough stars, energy from SN explosions can flatten the profiles of dark matter and baryons; however, if too many stars are formed, the excess baryonic mass can have the opposite effect of steepening the slope of the central density profile \citep{Bullock:2017}.
Technical challenges in implementing multi-phase gas and baryonic physics make it difficult to directly address and calibrate baryonic predictions based on hydrodynamical simulations \citep{Tollet:2016,1611.02281,Sawala:2016}.
However, one key prediction is that the creation of cores will be sensitive to the exact star formation history \citep[\eg,][]{governato2012,dicintio2014,onorbe2015,Read:2016,read2018,1811.11768,2019MNRAS.tmp....3R}.
Thus, robust measurements of both the stellar and dark matter mass of dwarf galaxies is essential to investigate the effect of baryonic feedback on the central dark matter density.
In addition, it has been argued that significant observational and astrophysical systematics, such as beam smearing, center offsets, inclinations, and non-circular motions can bias central density measurements toward flatter profiles \citep[\eg,][]{astro-ph/0006048,2004ApJ...617.1059R,2008AJ....136.2761O,2016MNRAS.462.3628R}. 
Thus, accurate independent measurements of dwarf galaxy density profiles are critical.

LSST can provide joint statistical measurements of both the central density and stellar content of dwarf galaxies. 
The stacked gravitational weak lensing signal from a large sample of dwarf galaxies will provide the most direct measurement of the amount and distribution of dark matter.  
In this section we predict the sensitivity of LSST to a stacked weak lensing signal from dwarf galaxies.

\vspace{1em} \noindent {\bf Dwarf galaxy lenses}

We are interested in estimating the number of isolated dwarf galaxies accessible to LSST as a function of dark matter halo mass.
To predict the abundance of the dwarf galaxy sample, we assume the mass-to-light ratio derived from the subhalo abundance matching technique, which links the global galaxy luminosity function with (sub)halo mass function by their respective abundance \citep[\eg,][]{2004ApJ...609...35K,2013ApJ...771...30R}. We use \code{colossus} \citep{2018ApJS..239...35D} to obtain the halo mass function and adopt the global galaxy luminosity function measured by GAMA \citep{2015MNRAS.451.1540L}. We match galaxy luminosity to current halo mass with the definition of $M_{200c}$. We also assume the mass-to-light ratio does not evolve significantly in this low-redshfit regime. 
We use this predicted galaxy luminosity to estimate the limiting redshift for dwarf galaxy detection as a function of galaxy halo mass for two LSST limiting magnitudes: $r \sim 25$ and $r \sim 27$. 
\figref{dwarf_redshift} shows that to probe dark matter halos with mass $\lesssim 10^9 \Msun$, it will be necessary to select galaxies at $z < 0.01$. 
While selecting very low-$z$ galaxies with photometric data is challenging, current projects like the SAGA Survey \citep{Geha:2017} have shown that it is possible using data from SDSS. 
Future large, multi-object spectrographs will greatly expand the spectroscopic data for training these selections. 
It will also be possible to use morphological information to select nearby dwarf galaxies.
LSST will be able to distinguish a dwarf galaxy with $M_V=-14$ from background galaxies of the same apparent magnitude out to a distance of $\roughly 100 \Mpc$ \citep[Section 9 of][]{0912.0201}.

\begin{figure}
\centering
\includegraphics[width=0.6\columnwidth]{halo_mass_redshift_log}
\caption{\label{fig:dwarf_redshift} Limiting redshift for detecting a dwarf galaxy that lives in a dark matter halo of certain masses, assuming a luminosity--halo mass relation derived from the  subhalo abundance matching technique, which matches galaxy luminosity from the GAMA luminosity function to present-day halo mass ($M_{200c}$) by their respective abundance.}
\end{figure}

\vspace{1em} \noindent {\bf Source galaxies}

The conservative LSST 10-year ``gold'' sample for cosmic shear measurements of dark energy is expected to have a source galaxy density of $\roughly 27 \amin^{-2}$ \citep{Chang:2013,1809.01669}. 
However, we expect that the dwarf lensing analysis can retain significantly more source galaxies for the following reasons.
(1) Our measurement uncertainty is dominated by the low number of dwarf galaxy lenses, rather than the  multiplicative shear measurement bias that must be strictly controlled for dark energy measurements. This allows us to include fainter, smaller, and more blended sources.
(2) Unlike the lenses used for cosmic shear measurements, the dwarf galaxy lenses are at very low redshift. This means that most detected sources are background galaxies.
(3) We expect to be able to combine shape measurements from multiple filters, which could increase the source density by $\roughly 80\%$. 
Combining these factors, we estimate a source galaxy density of $50 \amin^{-2}$, which is consistent with the fiducial, multi-band estimate of \citet{Chang:2013}.
The primary focus of the source galaxy selection will be to avoid catastrophic \photoz outliers (low-$z$ galaxies reported at high-$z$), which typically occur for less than a few percent of galaxies in current surveys \citep{1406.4407}. 
%\Photoz algorithms incorporating machine learning currently achieve better performance, giving posterior p(z) estimating which enables cuts on suspect source galaxies. 

\vspace{1em} \noindent {\bf Sensitivity}

We calculate the expected strength of a lensing signal for three different bins in halo mass,  $M_{200c} = \{10^{10},\, 3\times10^9,\, 10^{9}\}\,h^{-1}\Msun$, each with a width of $0.5$\,dex in mass. 
These samples correspond to $N = \{1.2\times10^8,\, 7.8\times10^6,\, 1.6\times10^5\}$ dwarf galaxies out to a redshift of $z = \{0.35,\, 0.07,\, 0.014\}$, respectively.
Source galaxies are placed at $z = 1.2$ with a density of $50 \hbox{ arcmin}^{-2}$ and a shear uncertainty of $\sigma_\gamma = 0.25$.
We model the mass distribution in each dwarf galaxy with an NFW halo assuming the concentration--mass relation from \citet{1809.07326}.
We calculate the shear from the stacked dwarf galaxy lens sample using \code{colossus} \citep{2018ApJS..239...35D}, assuming that each lens is placed at the limiting detectable redshift.
The results are shown in \figref{dwarf_sn}, where we find that LSST has the potential to measure the lensing shear with ${\rm S/N} \gtrsim 10$ for halos with $M \gtrsim 3 \times 10^9\,h^{-1}\Msun$.
Note that some of our assumptions are clearly optimistic. In particular, the number density of the source galaxies we assumed is high, and the assumption of perfect lens galaxy selection is also unlikely to hold. Nevertheless, since the S/N ratio goes $\sim 1/\sqrt{N_\text{lens} N_\text{src}}$, and thus lowering these numbers by a factor of $\sim 2$ would still yield a very high S/N ratio.

\begin{figure}
\centering
\includegraphics[width=\columnwidth]{halo_mass_lensing_sn}
\caption{
\label{fig:dwarf_sn} Lensing signal (reduced tangential shear; \textit{left}) and signal-to-noise (\textit{right}) for stacked samples of dwarf galaxies in three different mass bins (shown by different shapes of markers), each with width of 0.5 dex in mass. Two different density profiles are used for this calculation: the NFW profile (blue) and a NFW profile with a core (orange). 
The calculation assumes perfect selection of dwarf galaxies within the redshift range over which they are detectable by LSST. 
Source galaxies are assumed to be at $z=1.2$, with a surface number density of $50\,\amin^{-2}$, and a shear uncertainty of $\sigma_\gamma = 0.25$ per component.}
\end{figure}

As mentioned earlier, a cored density profile is a signature of SIDM, hence we also calculate the shear signal for a modified NFW profile with a central core,
\begin{equation}
\rho_\text{core}(r) = \rho_\text{NFW}(r) \times (1 -  e^{-3r/r_s})\,,
\end{equation}
where $\rho_\text{NFW}(r)$ is a standard NFW profile and $r_s$ is the scale radius of the NFW profile. 
We show the predicted shear signal from the cored profile in \figref{dwarf_sn} to be compared with the signal from NFW profile. 
We see that the overall signal-to-noise does not change much with the profile.  
However, to statistically distinguish the different profiles, one needs to measure the shear at very small angular scales ($< 10\,h^{-1}\kpc$, which corresponds to 2.9\,arcsec at $z=0.35$ and 10\,arcsec at $z=0.07$). This small-scale regime is where the systematics due to PSF modeling and blending would dominate. 
In other words, while the numbers of source and lens galaxies that LSST can find will be high enough to distinguish the difference between the two profiles, shear measurement systematics may present the major obstacle. 

The median seeing of LSST is about 0.7\,arcsec \citep[LSST SRD,][]{LPM-17}. Since the dwarf galaxy lenses are at very low redshift ($z=0.07$ for the $M_\text{halo}=3\times10^9\,h^{-1}\Msun$ sample), the angular scale ($\sim$10\,arcsec) that we would use to distinguish the cored profile is still well above a few times the median seeing. However, the uncertainty in PSF models can affect the shape measurement up to the scale of 3\,arcmin \citep{2012MNRAS.427.2572C}. We believe that, with improved PSF models and marginalization over model uncertainty, it will still be feasible to  utilize dwarf galaxy lenses to distinguish different halo profiles at small scales. 
 % edit in dwarf-lensing.tex

%\subsubsection{Halo Profiles of Bright Galaxies}
%\Contributors{Manoj?, Haibo?, Drew Newman?, Sean Tulin}
%\label{sec:halo_profile_galaxy}
% 
%\vspace{1em} \noindent {\bf Galaxy-Galaxy Lensing}
%\Contributors{?}
% 
%\vspace{1em} \noindent {\bf Caustics Structures}
%\Contributors{?}

\subsubsection{Galaxy Clusters \Contact{Susmita}}
\label{sec:halo_profile_clusters}

\subsection{Galaxy Clusters \Contact{Susmita}}
\label{sec:halo_profile_clusters}
\Contributors{Susmita Adhikari, William A.\ Dawson, Nathan Golovich, David Wittman,  M.\ James Jee, Annika H.\ G.\ Peter, Daniel A.\ Polin, Robert Armstrong}

Galaxy clusters are the most massive gravitationally bound structures in the universe. The high matter density and high velocity dispersions of clusters make them ideal laboratories for testing dark matter self-interaction models in a very different regime from individual galaxies.
In the following section we discuss several probes that use galaxy clusters to constrain the nature of dark matter.  We show that current constraints from many different cluster-scale probes are of the order of $0.1$--$1\cmg$.  To understand why this is so, it is important to note that the average column density of a cluster-scale halo is of the order of $1 \g \cm^{-2}$.  Improved cross section constraints will come from a combination of the large statistical data sets that will be collected by LSST and other telescopes in the LSST era, and more sophisticated theoretical predictions for observables for specific SIDM models.

\vspace{1em} \noindent {\bf Distribution of matter and substructure}

As we describe below, the current best cluster-scale SIDM constraints come from the radial dark matter profiles of halos.  However, cluster-scale halos that consist of SIDM and CDM exhibit other differences, which may prove to be highly constraining given the vastly detailed LSST cluster data sets.  Significantly more theoretical work is required to project robust constraints in the LSST era for those probes.

\paragraph{Radial profile:} Interactions among dark matter particles allow for the exchange of energy between different parts of the halo. The high number of interactions near the dense central region of a dark matter halo increases the temperature, or the velocity dispersion, near the central region. This process can be thought of as a transfer of heat from the outer (hotter) parts of the halo to the inner (colder) region. The excess dispersion due to self interaction leads to flattening of the inner density of the halo, leading to the formation of a cored density profile. For cluster-scale halos, the high densities near the center make the timescales for thermalization shorter at a given cross section than they are for lower-mass objects (although it must be noted that low-mass halos are generally older and have a longer time to thermalize).  The short thermalization time is important because dark matter thus behaves as a fluid in the innermost part of cluster-scale halos, and can relax to a hydrostatic equilibrium configuration at the center of the halo, where baryons dominate the potential \citep{Kaplinghat:2015aga}.  Depending on the merger history, cluster-scale halos can be as cuspy as those in CDM-only simulations (for recent mergers), or relax to a hydrostatic equilibrium (for highly relaxed systems) in which the dark matter halo has a small but relatively dense core \citep{Robertson:2017mgj}.

Density profiles of massive galaxy clusters therefore serve as probes for SIDM. Clusters tend to be dark matter dominated outside the very central regions, and they are the only known systems where the matter distribution can be individually mapped to the virial radius using weak lensing. Strong lensing also provides a measure of cluster mass independent of the dynamical state. And stellar kinematics of the central galaxy can be used to measure the dark matter density profile in the innermost regions. LSST will produce an unrivaled catalog of strong and weak lensing measurement of cluster density profiles. This, in concert with X-ray mass estimates and stellar kinematics, will provide a strong test of the NFW dark matter density profile predicted by cold, collisionless dark matter \citep{Newman:2013,Kaplinghat:2015aga,Robertson:2018anx,Andrade:2019wzn}. Moreover, the strong lensing cross section is an additional probe of the density profile \citep{Robertson:2018anx}.  For hard-sphere scattering, cross section constraints are of the order of $0.1$--$1\cmg$, but without fully quantified systematic uncertainties.

\paragraph{Halo shape:} Apart from the density profile itself, in SIDM models, dark matter velocity distributions become more isotropic than in the CDM model, especially at the center of the halo.  Correspondingly,  the halo density profile becomes more spherical.  Historically, constraints from cluster and galaxy ellipticies \citep{Miralde-Escuda:2000} provided strong constraints on the cross section of SIDM; however, later investigations found these constraints to be somewhat optimistic \citep{Peter:2013}. 
Recent measurements of the shapes of cluster-scale dark matter halos include studies with: cluster members \citep{2018MNRAS.475.2421S},  X-rays \citep{Hashimoto:2007},  lensing \citep{Mandelbaum:2006, Evans:2009, Oguri:2010}, and combinations of observables \citep{Clampitt:2016, Sereno:2018}.  
Current constraints on the cross section are sensitive to the order of $\sigmam \sim 1 \cmg$.
Several groups have shown in $N$-body simulations that the effects of SIDM with a cross section of roughly $\text{(a few)}\times 0.1$--$1 \cmg$ are potentially observable, although baryons can alter the probability distribution function of halo shapes by an amount that is not yet robustly quantified \citep[\eg][]{Peter:2013, Robertson:2017mgj, Brinckmann:2018}.


\paragraph{Substructure:} Structures form hierarchically in the standard CDM scenario: small objects form first and merge to form larger mass structures such as galaxy clusters. These clusters continue to accrete smaller halos and some of these small structures survive as subhalos within the cluster. It is therefore interesting to study the distribution of substructures within larger halos, to understand how the distribution is affected by self interactions among dark matter particles. 

Subhalos can be affected by SIDM models in three different ways within a cluster. First, dark matter particles in subhalos can evaporate due to interactions with the particles in the host cluster. Subhalos lose mass when they enter a cluster. In the CDM scenario particles that are at larger radii and are loosely bound get stripped as the subhalo orbits within a cluster. In SIDM models, evaporation due to self-scattering leads to additional mass loss. Unlike tidal stripping, self-interactions can also affect the inner regions of the subhalos. Simulations show that evaporation is inefficient at increasing the subhalo disruption rate unless hard-sphere cross sections are of order $\sigmam \sim 10\cmg$, or subhalos are on nearly radial orbits through the cluster center \citep{2012MNRAS.423.3740V,Rocha:2012jg,Dooley:2016ajo}. 

While this generally means that the total subhalo mass function within the virial volume is largely unaffected relative to CDM, other effects of evaporation may be detectable. Measuring the mass and the profile around cluster satellites (especially as a function of orbit eccentricity) using galaxy--galaxy lensing to measure the mass-to-light ratio of subhalos can be a promising probe for dark matter physics \citep{Natarajan:2017sbo}. The lensing signal around subhalos is weak and will be contaminated by the cluster mass profile, so methods like subtracting the lensing signal from diametrically opposite points within the cluster can be used to extract the signal. Given the statistics of cluster galaxies in LSST, it is ideally suited for a study of the weak lensing signal of subhalos.  
Second, as subhalos are also tracers of the dark matter density field within the cluster, their orbits will be affected by the change in the potential of the cluster near the core relative to CDM.  This effect can lead to an imprint in the radial distribution of subhalos in clusters, generally by making the subhalos less concentrated toward the halo centers.

Third, non-expulsive interactions can lead to a drag force on subhalos.  This has several potentially interesting observable consequences.  The location of the splashback radius is sensitive to dynamics of subhalos within the cluster. The splashback radius is the boundary of the multistreaming region of a halo and is the largest apocenter of recently accreted objects \citep{Diemer:2014xya,Adhikari:2014lna}. The slope of the density profile of a halo falls off rapidly in a narrow localized region around this radius, and the splashback radius is observed as a minimum in the slope of the projected number density profile of galaxies \citep{More:2016vgs,Baxter:2017csy,Chang:2017hjt}.
The apocenter of the orbits of subhalos can change if there is extra drag beyond dynamical friction \citep{Kummer2018}.  
Therefore measuring the location of the splashback radius can help distinguish between different models of dark matter, although the difference between splashback locations in the CDM and SIDM scenarios has not yet been well quantified.

Similar to the situations discussed above and in merging clusters (Section~\ref{sec:merging_clusters}), the drag force due to non-expulsive interactions may also lead to offsets between the light distribution and dark matter distribution of individual satellites with respect to their subhalos. Small offsets between the subhalo and the galaxy within it may be detectable by indirect means: the potential gradient established by the dark matter at the position of the stellar centroid would induce a U-shaped warp in the stellar disk facing the direction of infall, and a longer-lasting disk thickening. Numerical simulations show these to be observable by current and next-generation photometric surveys under SIDM models with $0.5 \cmg \lesssim \sigmam \lesssim 1 \cmg$~\citep{Secco}. While S-shaped disks formed by tidal distortions of the stellar light profile are abundantly observed in cluster environments, indicating that they are readily induced by ``baryonic effects,'' these effects are not likely to generate prominent U-shaped warps. Such warps are only formed by a differential force on the disk and its halo, due to, for example, the SIDM drag.  The offset between a satellite galaxy and its subhalo may also be observed directly or statistically with strong lensing \citep{Massey2011,Massey:2017cwf}, but the magnitude of the effect is highly model-dependent (depending strongly on the angular and velocity dependence of the cross section).  Current limits are $\mathcal{O}(1\cmg)$ for specific non-hard-sphere models \citep{Harvey:2015hha}.  

\vspace{1em} \noindent {\bf Merging Galaxy Clusters \Contact{Nate?}}
\label{sec:merging_clusters}

% intro para
In the previous section, we considered subhalos to be minor merger events onto the main cluster.  Major cluster mergers can probe the nature of dark matter by serving as the biggest ``dark matter colliders'' on account of their high mass and large collision velocities. Dense halos falling together at thousands of km\,s$^{-1}$ provide an environment where the scattering of dark matter particles off each other would have observable effects.  The observable effects vary depending on the dark matter model and the configuration of the merger \citep{Kim:2016ujt}. 
Cluster mergers may also be able to distinguish between particle models that yield frequent scattering with low momentum transfer (as in a long-range force) and those that yield infrequent scattering with high momentum transfer (as with hard sphere or contact scattering) due to their differing phenomenology in the merger environment.  This is in contrast to the halo radial profile and shape constraints discussed in the previous section, for which the energy and momentum transfer rate matters most and for which there is no preferred direction in the problem.

% why LSST discovery is important
The best known example of a colliding cluster system is the Bullet Cluster, which has been frequently studied as a laboratory for SIDM \citep{Randall:2007ph,2017MNRAS.465..569R}. 
However, since a cluster merger is an eons-long process of which we have only a single snapshot, the measurement uncertainty is dominated by our very limited knowledge of the merger history. While it will remain critical to investigate individual clusters in great detail, the power of LSST lies in systematically analyzing a population of merging clusters with a consistent method, thereby constraining the properties of dark matter.
LSST will contribute to better and more robust constraints not only through the study of already known systems, but also by enabling the discovery of many more merging systems. Because mergers displace plasma from galaxies, they are best discovered by cross-correlation of LSST optically-detected clusters with radio and X-ray surveys \citep{Golovich:2018,Wilber2018}.

%offsets

The first SIDM constraints based on a merging galaxy cluster came from the Bullet Cluster, which was originally identified as an extremely hot X-ray cluster with two galaxy peaks. Higher resolution optical and X-ray imaging revealed a spectacular post-merger system with a clear X-ray cold front and shock. The spatial agreement of the galaxies and mass centroids obtained by weak lensing, and the disassociation of the intra-cluster medium (ICM) led to the constraint $\sigmam \lesssim 2 \cmg$ for hard-sphere scattering \citep{Markevitch2004,Randall:2007ph,2017MNRAS.465..569R,Robertson:2016qef}. Many other dissociative mergers have been found and studied, with roughly similar cross section limits \citep[but with greater systematic uncertainty, \eg][]{bradac2008}. 

After several ``dissociative'' mergers had been discovered, ensemble studies of the offsets between dark matter, galaxies, and gas were utilized to drive down the Poisson noise from inference on individual systems. \citet{Harvey:2015hha} modeled 72 subclusters within 30 merging systems to place the strongest constraint on SIDM ($\sigmam<0.47 \cmg$).
The study assumes a simplified drag force model where dark matter behaves similar to the ICM. However, \citet{Wittman:2017gxn} reanalyzed the sample including more comprehensive data. They identified several substantial errors that were driving the result and obtained a revised limit of $\sigmam \lesssim 2\cmg$.

The drag force model applies best to particle models with frequent interaction and low momentum transfer per interaction. In models with infrequent, high momentum transfer interactions (including hard-sphere scattering), dark matter particles may be scattered out of the cluster entirely. (Evaporation also occurs for small-angle scattering, though---see \citealt{Kahlhoefer:2013dca}.) This mass loss may be detected by comparing the mass-to-light ratio of merging clusters with those of non-merging clusters, on the assumption that the merger does not affect the galaxy light. This argument leads to a constraint of $\sigmam \lesssim 1 \cmg$, similar to current constraints on the drag model. However, the assumption that the galaxy light is unaffected is a source of uncertainty here. The LSST discovery of many more merging clusters, with six-band LSST photometry, will help us quantify this source of uncertainty. 

Several billion years post-pericenter, after a merging cluster has coalesced into a single cluster, SIDM will still create a cored dark matter distribution in the center of the cluster. For $\sigmam \sim 1 \cm^{2} \g^{-1}$, this core is $\roughly 100 \kpc$ (although the baryonic potential can alter the dark matter distribution). \citet{Kim:2016ujt} presented the effect of this on the brightest cluster galaxies (BCGs) up to 10 Gyr post-pericenter. They demonstrated a wobbling in the BCG as it is able to oscillate about the shallow potential for many oscillations. \citet{1703.07365} analyzed a small set of massive clusters and compared the BCG location with a strong lensing based estimate of the gravitational potential centroid. They compared these observations with hydro-CDM simulations to show that the observations suggest a cored dark matter halo in these clusters of $\roughly 10\kpc$. \citet{Harvey:2018uwf} recently studied cluster-scale halos in hydrodynamic simulations, and saw offsets that grew with cross section and halo mass, although with a smaller amplitude than the dark-matter-only simulations of \citet{Kim:2016ujt} implied. LSST will characterize thousands of relaxed clusters that invariably will have undergone a merger in their history. With deep and relatively high resolution imaging, LSST will allow for single snapshots of the BCG alignment in every massive cluster, and also for detection of faint strong lensing streaks in many of these systems.

% Esra added the text below
SIDM properties are sensitive to the separation between the centroid of the X-ray emitting hot plasma, i.e., intra-cluster medium (ICM), galaxies, and dark matter. Accurate measurements of the X-ray centroid of the X-ray emitting gas in clusters of galaxies requires sub-arcsec imaging with X-ray telescopes. The {\it Chandra} X-ray observatory with 0.5~arcsec FWHM PSF currently provides the most precise location of the ICM. Next-generation high spatial resolution X-ray observatories, \eg, {\textit Lynx} and {\textit AXIS} with much higher throughput, will provide accurate measurements of centers of high-redshift clusters ($z > 1$) in the 2030's and will enable tests of SIDM models over a much larger redshift range.
 % cluster people have organized subsubsections within clusters.tex


\subsection{Compact Object Abundance \Contact{Will}}
\Contributors{Will, Nate, Michael M., Bob A., ... }
\label{sec:compact_objects}


MAssive Compact Halo Objects \citep[MACHOs;][]{1991ApJ...366..412G} have long been considered a viable dark matter candidate \citep{1974ApJ...193L...1O, 1980ApJS...44...73B, 1981ApJ...243..140G, 1986ApJ...304....1P, Bird:2016}. 
Cosmological observations of the CMB, BAO, and deuterium abundances have shown that compact objects must be non-baryonic if they are to make up a majority of dark matter \citep[\eg][]{Ade:2015xua}. 
As described in \secref{machos}, this has led to the identification of primordial black holes (PBHs) as the most popular candidate for MACHO dark matter.
There are a number of astrophysical probes that constrain the PBH dark matter abundance over mass scales ranging from $10^{-17}-10^{15}\,M_\odot$ (\figref{macho_constraints}).
At the lowest masses ($M < 10^{-9}\Msun$), PBHs are constrained through the non-detection of PBH evaporation in the extragalactic gamma-ray background \citep[\eg,][]{2010PhRvD..81j4019C}, non-detection of fempto-lensing of gamma-ray bursts \citep[\eg,][]{2012PhRvD..86d3001B}, the rate of SN Type 1a \citep{1805.07381}, and neutron star capture \citep[\eg,][]{2013PhRvD..87l3524C}.
The landscape of intermediate-mass MACHOS ($10^{-11} \Msun < M < 10 \Msun$) is predominantly constrained by microlensing observations, which limit the monolithic MACHO dark matter fraction to be $\lesssim 10\%$ over this mass range \citep[\eg][]{Alcock:2001,Tisserand:2007,1701.02151}.
At the high-mass end ($M \gtrsim 10^3\Msun$), PBH dark matter is subject to constraints from dynamical stability of wide binary stars \citep[\eg][]{Yoo:2004}, star clusters \citep[\eg]{Brandt:2016}, and the Galactic disk \citep[\eg][]{Lacey:1985}.
Lyman-$\alpha$ observations suggest that PBHs with $M > 10^4\Msun$ do not exist based on an observed plateau in the Poisson term of the matter power spectrum \citep{astro-ph/0302035}.
In addition, observations of the CMB have been used to place strong constraints on the PBH fraction for PBHs with mass $\gtrsim 1 \Msun$ due to unobserved distortions in the CMB due to expected gas accretion around PBHs \citep{2008ApJ...680..829R}.
However, these constraints have been shown to be extremely model dependent and were relaxed substantially in subsequent studies (see e.g., \citealt{1612.05644}).
%Several indirect constraints have been published that rule out most of the mass scales above the sensitivity of these microlensing surveys; however, these constraints rely on complex astrophysical assumptions.
This, in addition to recent direct observations from LIGO \citep{1602.03837} and indirect evidence from observations of globular clusters \citep{???} and high-z quasars \citep{???}, has motivated a renewed interest in PBH dark matter with larger masses than have been so far constrained directly by microlensing.
%\WAD{the next sentence should be moved down after the signals are described}
%\ADW{Need citations}

In this section, we focus on the ability of LSST to directly detect signals of compact halo objects through precise, short ($\sim30\,$s) and long-duration ($\sim$ years) gravitational microlensing observations.
If scheduled optimally, the wide field-of-view, high cadence, and precise photometry of LSST have the potential to extend PBH sensitivity to $\roughly 0.03\%$ of the dark matter fraction for masses $10^{-12}\lesssim \Msun \lesssim 10^{-6}$ and $>0.1\Msun$, see Figure \ref{fig:macho_constraints}.
We briefly mention that LSST will also probe PBHs by determining the rate of SN Type 1a, identifying candidate wide-binary star systems at greater distance than is possible with \Gaia, and through devoted mini-surveys of high-stellar density fields (similar to that performed with HSC by \citealt{1701.02151}).

\begin{figure}[t]
\label{fig:macho_constraints}
\centering
\includegraphics[width=0.9\columnwidth]{macho_limits.pdf}
\caption{
    Constraints on the fraction of dark matter that can be accounted for in compact objects from existing probes (blue and gray) and projections for LSST's capability (gold).
    The existing constraints include: lack of extragalactic gamma-rays from PBH evaporation \citep[EGR][]{PhysRevD.81.104019, PhysRevD.94.044029}, gamma-ray femptolensing \citep[GF][]{PhysRevD.86.043001}, neutron star capture \citep[NS][]{PhysRevD.87.123524}, M31 microlensing \citep[M31ML][]{2017arXiv170102151N}, MACHO/EROS/OGLE Milky Way microlensing \citep[MWML][]{10.1051/0004-6361:20066017, 2001ApJ...550L.169A, 2009MNRAS.397.1228W}, lensed super nova \citep[LSN][]{PhysRevLett.121.141101, arxiv:1712.06574v1}, Eridanus II and dwarf galaxy constraints \citep[EII][]{2016ApJ...824L..31B, arxiv:1611.05052v2}, wide binary stars \citep[WB][]{2009MNRAS.396L..11Q, 2004ApJ...601..311Y}, cosmic microwave background \citep[CMB][]{2017PhRvD..95d3534A, 2008ApJ...680..829R}, and disk stability \citep[DS][]{1985ApJ...299..633L, 1994ApJ...437..184X}.
    To help with the clarity of the figure we have not shown some astrophysical constraints where they are less sensitive than a presented constraint; see \citet{2016PhRvD..94h3504C} for a more complete review.
    There are a range of constraints for most of astrophysical probes in the literature, sometimes due to varying assumptions within a single work (EGR, NS, and EII), and sometimes due to reanalysis/disagreements between groups (LSN, WB, CMB).
    We present the most conservative constraints in blue and the most aggressive constraints in gray.
    The LSST M31 microlensing projection is based on extrapolating the \citet{2017arXiv170102151N} by assuming a ten-day mini-survey of M31 with a 12 second cadence between exposures.
    Such a survey is approximately 10 times longer with an order of magnitude faster cadence than the survey of \citet{2017arXiv170102151N}.
    The projected LSST Milky Way (MW) microplensing and paralensing constraints are from a Monte Carlo analysis where various primordial black hole lenses were injected into light curves based on LSST OpSim cadence simulations (see \url{https://github.com/lsstdarkmatter/dark-matter-paper/issues/8} for details).
    The paralensing constraint comes from assuming that only the secondary microlensing parallax signal is used for discovery, and not the primary heliocentric microlensing signal.
}
\end{figure}

\subsubsection{Microlensing}
\Contributors{Will D., Nate G., Michael M., Bob A., ...}
\label{sec:microlensing}

Gravitational microlensing, the achromatic brightening and dimming of background stars due to the transit of a massive compact foreground object, can be used to directly detect and measure the properties of PBHs.
The idea of employing microlensing to search for compact objects in the Galactic halo was proposed by \citet{1986ApJ...304....1P}, and several photometric surveys commenced in the 1990's including MACHO \citep{1992ASPC...34..193A}, OGLE \citep{1992AcA....42..253U}, and EROS \citep{1993Msngr..72...20A}.
These collaborations provided the first \emph{direct} constraints on the compact nature of dark matter. 
However, these surveys were limited by image quality, analysis techniques, and computational resources.
These limitations, combined with the $\roughly10$-year duration of these surveys, led to a loss of sensitivity at $M \gtrsim 1 \Msun$.
LSST can surpass this limitation by directly detecting events based purely on their paralensing signal, see Figure \ref{fig:microlensing_cartoon}.
By combining parallactic and astrometric microlensing signals, LSST can break the lensing mass-geometry degeneracies and make precise measurements of individual black hole masses and therefore the black hole mass spectrum in the Milky Way Halo.
Thus, if PBHs make up a significant fraction of dark matter, LSST will effectively measure their ``particle'' properties. 
A precise measurement of the PBH mass spectrum will provide insight into the fundamental physics of the early universe during which time they were formed.

\begin{figure}
\centering
\includegraphics[width=0.6\columnwidth]{microlensing_cartoon.png}
\caption{\label{fig:microlensing_cartoon}
    \emph{[Left]} 
        The microlensing and paralensing signals for a $23^\mathrm{rd}$ magnitude source being lensed by a $50\,\mathrm{M}_\odot$ black hole. 
        For events with an Einstein crossing time much less than a year ($\lesssim 1 \Msun$), the microlensing magnification will appear symmetric in time (orange curve).
        For microlensing events lasting on the order of a year or more ($\gtrsim 1 \Msun$), the lensing geometry changes due to parallax as Earth orbits the Sun.
        This paralensing signal has a period of a year, with the phase determined by the coordinates of the source star, making it robust to other astrophysical systematics.
        %It is also possible to detect binary dark matter, and extend the mass range to planet mass compact dark matter, via the source passing through one of the gravitational lensing caustic curves formed by the binary lens.
        %The green curve on top of the heliocentric orange curve is representative of a typical planetary microlensing event caused by a caustic crossing.
        %While LSST can measure these events if lucky, we will rely on LSST to detect the heliocentric microlensing event and trigger targeted follow-up higher cadence observations to measure the planetary microlensing event.
        The black data points are representative of extending the LSST wide-fast-deep cadence into the Galactic plane. \WAD{Need to update this figure with the LSST WFD cadence.}
        \emph{[Right]} 
        A cartoon diagram of paralensing. 
        For microlensing events lasting on the order of a year or more the lensing geometry changes as Earth orbits the Sun, leading to a parallax effect.
        \Contributors{Will D., PALS Collaboration}
    }
\end{figure}

\noindent \textbf{The Microlensing Signal}
%Gravitational microlensing occurs when a massive lens passes between a background source and an observer, causing the light from that source to pass through a warped space-time acting as a lens. \WAD{cite Wambsganss for a detailed review}

%\WAD{Heliocentric microlensing}
Gravitational microlensing results in two potentially observable features: (1) photometric microlensing, a temporary achromatics amplification of the brightness of the background source, and (2) astrometric microlensing, an apparent shift in the centroid position of the source.
The characteristic photometric signal of a simple point-source point-lens (PSPL) model as observed from the center of the solar system is symmetric, achromatic, and has both a timescale and maximum amplification that depend on the mass of the lens.
LSST will observe billions of stellar sources in multiple filters over several years to enable the detection of thousands of microlensing events across a wide range of timescales and consequently a wide range of masses.
This simple PSPL model is complicated by astrophysical factors including the velocity distribution of sources and lenses, extinction due to Galactic dust, blending in dense stellar fields, and the shift in perspective resulting from viewing a microlensing event while the Earth revolves around the Sun.
Fortunately, these complications can be addressed and disentangled to arrive at the mass of the gravitational lens and a detection of dark matter via microlensing \citep{1405.3134,1509.04899}.

%\WAD{Paralensing}
One particularly powerful feature for long-duration microlensing events results from the change in the geometric configuration of the source-lens-observer system as the Earth orbits the Sun (\figref{microlensing_cartoon}).
The change in viewing angle and distance results in a parallax effect that imposes a 1-year periodicity on top of an otherwise symmetric microlensing light curve.
This additional signal exists irrespective of the mass of the lens, providing an independent measurements of the distance of the lens and breaking the mass-distance degeneracy of a microlensing signal \citep[\eg][]{1509.04899}.
This enables microlensing to directly constrain compact dark matter at much larger mass scales ($M \gtrsim 1\Msun$), where the duration of the event is $\gtrsim 1$ year.   
\ADW{What is the maximum distance for which the parallax signal is measureable?}

%\WAD{astrometric (LSST vs AO); I feel like this paragraph might belong better in the complementarity section, and should just be referenced in the paragraph above.}
Photometric microlensing timescale measurements result in lens mass measurements in units of the Einstein crossing radius, which is impossible to measure directly due to the unresolved nature of the microlensing event.
Astrometric microlensing makes measurements of the shift in the unresolved centroid position of the background source in units of the Einstein radius as well.
Triggering astrometric followup on instruments that have already proven capable of making these measurements (Keck Adaptive Optics, HST), as well as next generation telescopes (TMT, ELT) will prove a vital measurements to complement the lightcurves generated by LSST.

%\WAD{achromatic}
Gravitational lensing is achromatic, making the multiple filter observations of LSST a key advantage in making microlensing detections.
The benefit derived from confirmed suspected microlensing signals across different filters will depend strongly on the selected LSST cadence. 
Microlensing signals are made stronger by more frequent observations in fewer filters, as long as sources are observed in at least two colors. 

\noindent \textbf{Survey Specifications}

\emph{Where to survey:} Given the typical scale \WAD{give typical scale range for low and high mass} of the Einstein radius (the approximate region where a microlensing signal is detectable), the odds of any one source star experiencing a microlensing event is approximately $10^{-9}$. Due to the low probability of a microlensing event, most observable microlensing events will be in dense stellar fields (e.g., the Galactic plane, Magellanic Clouds, and M31), driving microlensing surveys to these dense stellar fields.

\emph{Difference imaging:} Then dense survey fields, coupled with LSST depth and ground based point spread function (PSF) of $\sim0.8''$ leads to significant ambiguous blending (i.e., multiple objects within a single PSF).
Towards the Galactic center there are $\sim50$ stars within an LSST PSF, similarly there are $\sim15$ and $\sim5$ in the bulge, and disk respectively \cite{arXiv:1806.06372}.
LSST can overcome much of the blending problem to detect variations in the brightness of stars, including the detection of microlensing lightcurves, through the process of difference imaging.
Reference images are built through the coaddition of multiple observation of the same fields, resulting in a deep image of the static sky. Reference images are then scaled and PSF-matched to individual science images as they observed and subtracted off, resulting in a difference image that only contains the signal that has difference from the static sky. This limits the locations in the sky where difference imaging can be performed to those locations where reference images have already been built, requiring reference building the first and foremost priority to enable LSST to measure sky variability.  We will also be able to leverage the difference imaging tools developed for the LSST Data Release Production pipeline.  This will benefit from improved calibrations and thus increased sensitivity.

\emph{Reference image building:} Different science cases require different strategies for building reference images. Short timescale events are agnostic to the cadence of those images which are used to build a reference image because any observations prior to the event can be assumed to be photometrically identical. Microlensing signals caused by massive lenses can be on the scale of years to decades. If the reference images are built by combining images across months or years, they will contain the microlensing signal within the reference and make it far more difficult to detect microlensing events from difference images. We therefore strongly prefer reference images built from images taken across a short time period to reduce static / signal confusion.

\emph{Dense field photometry:} While differential photometry is ideal for detecting variable objects in dense fields it does not provide the baseline photometry of individual stars. The baseline photometry of stars provide important information as to the stars location within the Milky Way which is important to constrain the lensing geometry then thereby the compact object mass and velocity properties. There are a number of pipelines for dense-field photometry---e.g., \code{DAOPhot} \citep{1987PASP...99..191S} or \code{crowdsource} \citep{1710.01309}---capable of providing this; however, no methods has yet been implemented in the LSST Data Management system.

\emph{Band choices:}
The microlensing signal is achromatic, modulo secondary ambiguous blending effects, which is atypical of most other astrophysical variables which have chromatic variability due to associated temperature changes.
To leverage the achromatic nature of lensing, to discriminate microlensing events from most other astrophysical variables, it is necessary to survey in at least two bands.
If surveying the Milky Way bulge or M31 it is optimal to use more red sensitive bands which are less sensitive to galactic dust between the observer and the stars.

\noindent \textbf{Microlensing Systematics}
As with any physics experiment, there are microlensing measurement systematics that must be accounted for.
We briefly summarize these, and mitigation strategies, here.

\emph{Variable stars} are a potential source of false detections at all timescales.
However this systematic can be mitigated through a number of independent ways.
Perhaps the most powerful means is by leveraging the achromatic nature of gravitational lensing.
Nominally, the microlensing signal will have the same amplitude in all photometric bands (although this is absolutely true only in the absense of blending and interstellar extinction, see below), while most astrophysical variables are associated with a temperature dependence and thus not achromatic.
For microlensing events with durations $\gtrsim 1$ year, there will be 365 day period parallax signal imprinted on the microlensing signal which is difficult for other astrophysical variables to mimic.
Also as the survey progresses repeating variable stars will be better characterized.

\emph{Blending:} Difference imaging should reduce many of systematic effects associated with blending; however, it will still be difficult characterize the intrinsic source star baseline photometric properties. Crowding can also lead to a secondary chromatic signal with exactly the same duration/shape as the microlensing event, although with different amplitude. Follow-up high resolution imaging from space or ground-based AO can mitigate most blending issues for detected microlensing events.

\emph{Galaxy Model:} While microlensing can be detected independent of any detailed knowledge of Galactic structure, properly incorporating uncertainty in the Galactic dust, stellar velocity distributions, and dark matter halo model is essential to interpret the microlensing signal in the context of dark matter.
Great improvements have been made on this front since the first microlensing surveys \citep[e.g.,][]{2018MNRAS.479.2889C}.

\emph{Binary vs isolated stars:} A potential systematic for the paralensing signal is binary star systems with approximately year-long periods. This is unlikely to be a significant systematic due to the low probability of having a binary system with a year long period and that remains achromatic. %\WAD{There was a researcher who did some of these studies for our microlensing group. Need to reference his work/arguments.} 


\noindent {\bf Projected Sensitivity}

We estimate the projected sensitivity of an LSST survey optimized for the detection of microlensing signals from PBHs and present the results in Figure \ref{fig:macho_constraints}.
The LSST M31 microlensing projection is based on extrapolating the \citet{2017arXiv170102151N} by assuming a ten-day mini-survey of M31 with a 12 second cadence between exposures.
Such a survey is approximately 10 times longer with an order of magnitude faster cadence than the survey of \citet{2017arXiv170102151N}.
The projected LSST Milky Way (MW) microplensing and paralensing constraints are from a Monte Carlo analysis where various primordial black hole lenses were injected into light curves based on LSST OpSim cadence simulations (see \url{https://github.com/lsstdarkmatter/dark-matter-paper/issues/8} for details).
The paralensing constraint comes from assuming that only the secondary microlensing parallax signal is used for discovery, and not the primary heliocentric microlensing signal.

\begin{comment}
    % Old figure caption
        The microlensing and paralensing signals for a $23^\mathrm{rd}$ magnitude sourcestar in the bulge being lensed by a $50\,\mathrm{M}_\odot$ black hole. 
        For events with an Einstein crossing time much less than a year (i.e., approximately solar mass and below), the microlensing magnification will appear symmetric in time (orange curve).
        For microlensing events lasting on the order of a year or more (i.e., approximately solar mass and above), the lensing geometry changes as Earth orbits the Sun, leading to a parallax effect.
        This paralensing signal has a period of a year, with the phase determined by the coordinates of the source star, making it robust to other astrophysical systematics.
        It is also possible to detect binary dark matter, and extend the mass range to planet mass compact dark matter, via the source passing through one of the gravitational lensing caustic curves formed by the binary lens.
        The green curve on top of the heliocentric orange curve is representative of a typical planetary microlensing event caused by a caustic crossing.
        While LSST can measure these events if lucky, we will rely on LSST to detect the heliocentric microlensing event and trigger targeted follow-up higher cadence observations to measure the planetary microlensing event.
        The black data points are representative of extending the LSST wide-fast-deep cadence into the Galactic plane. \WAD{Need to update this figure with the LSST WFD cadence.}
        \emph{[Right]}
        A cartoon diagram of paralensing. 
        For microlensing events lasting on the order of a year or more the lensing geometry changes as Earth orbits the Sun, leading to a parallax effect.
        \Contributors{Will D., PALS Collaboration}
\end{comment}



%\subsubsection{White Dwarf Explosions}
%\label{sec:wd_explosions}

%\WAD{Add some material here about the Type 1A SN rate constraint.}

%\ADW{I've commented out this section, since it seems unlikely that it will get written on a short timescale }

%\citep{1805.07381}


\begin{comment}
\ADW{I've cleaned up this text in the intro to this section. This is left here for future reference.}


\WAD{Describe these constraints better, or at least throw in some references; also give numbers for what this mass range is.}
\WAD{Better introduce and conclude the two following paragraphs that mention methods for which LSST can't contribute much to. It might even be better to exclude these paragraphs and just focus on the probes that LSST can improve. Perhaps we could just move this to the complementarity section.}
At the low mass MACHO end there are constraints on the mass spectrum due to extragalactic gamma-ray evaporation, fempto-lensing of gamma-ray bursts, and Neutron star capture \WAD{Describe these constraints better, or at least throw in some references; also give numbers for what this mass range is.}, for which LSST is unable to significantly improve.

At the low mass MACHO end there exist various constraints on the mass spectrum.
Extragalactic gamma-ray evaporation... (cite Carr, Kohri, Sendouda, Yokoyama).
PBHs are ruled out at $\sim 10^{-16}, M_\odot$ \WAD{what is the actual range} due to a lack of an interference pattern in the energy spectrum of gamma-ray bursts that would occur due to the similar scale of the Schwarzschild radii the wavelength of gamma rays known as fempto-lensing (cite Barnacka, Glicenstein, Moderski) \WAD{Fix wording}.
Neutron stars in globlular clusters, where a rich dark matter environment is known to exist, would be destroyed by PBHs at $10^{-15}-10^{-8}\,M_\odot$ due to capture of the BH by the NS and subsequent accretion of the NS onto the BH.
Thus the very existence of neutron stars refuse the possibility of compact dark matter in this mass range (cite Capela, Pshirkov, Tinyakov).
While these analyses are valuable constraints on the dark matter fraction at various mass scales, LSST is unable to significantly improve upon the already existing literature.

The abundance of white dwarf explosions place constraints on the mass scale of $10^{-15}-10^{-14}\,M_\odot$, which can be constrained by the rate of Type IA supernovae.
LSST will provide the best characterization of the Type IA SN rate, and, as discussed in \S\ref{sec:wd_explosions}, greatly improve the constraint over this mass range.
In conglomeration, various gravitational microlensing surveys constrain the MACHO dark matter fraction over the mass range $10^{-9}-10^{1}\,M_\odot$ \WAD{Add citations and check the low mass range for the latest Subaru HSC constraint.}.
With the correct survey footprint and cadence, LSST can improve these dark matter fraction constraints by several orders of magnitude and extend the mass range to $10^{-10}-10^{5}\,M_\odot$, see \S\ref{sec:microlensing}.
LSST holds the greatest potential of improving the microlensing MACHO constraints, potentially over the mass range of \WAD{Complete this.}


\WAD{Make sure that we say something about how LSST's astrometry and photometry enable wide binary candidate identification and characterization to greater distances than GIA, but note the sensitivity of this method to false detections, i.e. chance alignments.}

\end{comment}


\subsection{Anomalous Energy Loss Mechanisms\Contact{Maurizio}}
\Contributors{Maurizio, Oscar, (Alex), Zentner?, Sam McDermott}
\label{sec:cooling}

\section{Anomalous Energy Loss Mechanisms\Contact{Maurizio}}
\label{sec:cooling}
\Contributors{Maurizio Giannotti, Oscar Straniero, Samuel D.\ McDermott, Alex Drlica-Wagner}

Observations of stars provide a mechanism to probe temperatures, particle densities, and time scales that are inaccessible to laboratory experiments.
Since conventional astrophysics allows us to quantitatively model the evolution of stars, the detailed study of stellar populations can provide a powerful technique to probe new physics.
In particular, if new light particles exist and are coupled to Standard Model fields, their emission would provide an additional channel for energy loss. 
Such anomalous energy loss mechanisms would change the time that stars spend in specific stellar evolutionary phases.
Such deviations are a robust predictions of light, weakly coupled particles, and the general agreement between observations and Standard Model predictions has been used to constrain the properties of many types of new particles \citep{hep-ph/0611350, 1210.1271, 1302.3884, 1305.2920, 1611.03864, 1611.05852, 1803.00993}.

While the predictions of the Standard Model are broadly consistent with observations of stellar evolution, several independent observations have shown a systematic preference for an additional subdominant energy-loss mechanism (see \citealt{Giannotti:2017hny} for a recent review).
These observations include red giants branch (RGB) stars, in particular the luminosity of the tip of the branch~\citep{Viaux:2013lha,Viaux:2013hca}; 
horizontal branch stars (HB), specifically by comparing the number of HB and RGB stars~\citep{Ayala:2014,Straniero:2015nvc};
variable white dwarf (WD) stars, for which the cooling efficiency was extracted from the rate of the period change~\citep{KeplerEtAl,Isern:1992gia,BischoffKim:2007ve,Corsico:2012ki,Corsico:2012sh,Corsico:2014mpa,Corsico:2016okh,Battich:2016htm}; 
and the WD luminosity function (WDLF), which describes the distribution of WDs as a function of their luminosity~\citep{Isern:2008nt,Bertolami:2014wua,Isern:2018uce}.
Observed discrepancies between these stellar measurements and predictions from conventional models of stellar cooling can be interpreted as the need for additional energy loss (\figref{axions}).
\cite{Giannotti:2015kwo} provide a systematic analysis of the new-physics interpretation of stellar observations  where cooling anomalies have been reported, and they conclude that axions and ALPs are the best candidates to account for the observed discrepancies. 
While these `hints' of anomalous cooling represent subdominant deviations to broadly successful models of stellar evolution, it is imperative to explore possible signatures of new physics when they arise.

LSST will greatly improve our understanding of stellar evolution by providing unprecedented photometry, astrometry, and temporal sampling for a large sample of faint stars \citep{0912.0201}.
These observations will allow us to better assess the significance of claimed anomalies, and will further guide constraints on (or detection of) new physics.
A better understanding of astrophysical energy transport will ultimately help shed light on the physics of light, weakly-coupled particles and will offer an invaluable guide to future experimental searches for axions and ALPs~\citep{Irastorza:2018dyq}.

\begin{figure}[t]
\centering
\includegraphics[width=0.6\columnwidth]{axions.png}
\includegraphics[width=0.39\columnwidth]{alps.jpg}
\caption{Left: Existing experimental and observational constraints on the QCD axion \citep{Redino:2015}.  
Right: Constraints on ALP coupling to photons \citep{Ringwald:2012}.
Astrophysical constraints and hints include observations of white dwarfs (WD), globular clusters (GC), supernova (SN), and horizontal branch stars (HB).
Note the wide range of mass and coupling scales that are constrained by these observations.
\label{fig:axions}
}
\end{figure}

\subsection{White Dwarf Luminosity Function}

The white dwarf luminosity function (WDLF) plays a particularly significant role in our understanding of stellar cooling and offers a fundamental method to test new physics.
Measurements of the slope of the WDLF can probe additional energy loss mechanisms and the production rate of the novel particle responsible for the nonstandard cooling.
The general agreement between the observed WDLF and predictions from standard astrophysics has been used to place bounds on the axion-electron coupling \citep{Isern:2008nt,Bertolami:2014wua}, on the anomalous neutrino magnetic moment \citep{Bertolami:2014noa}, on the kinematic coupling of dark photons to standard photons \citep{Chang:2016qfl}, and on the variation of the gravitational constant \citep{Althaus:2011ca}.
However, several recent analyses of the WDLF have shown a preference for additional energy loss with respect to the Standard Model predictions.
In particular, \cite{Bertolami:2014wua} used data from the Sloan Digital Sky Survey (SDSS) and the SuperCOSMOS Sky Survey (SCSS) to show a $2 \sigma$ discrepancy from the Standard Model prediction, which could be explained by axions coupling to electrons with $g_{\phi e}\simeq 1.4\times 10^{-13}$.\footnote{The additional energy can also be accounted for by dark photons~\citep{Giannotti:2015kwo,Chang:2016qfl}, but not by anomalous neutrino electromagnetic form factors~\citep{Bertolami:2014noa}.}
These measurements of the WDLF have guided experimental searches for axions and ALPs, particularly the IAXO~\citep{Irastorza:2011gs,Armengaud:2014gea}, and ALPS II~\citep{Bahre:2013ywa,ALPSII} experiments.

Observations from the \Gaia satellite have already increased the catalog of WDs by an order of magnitude with respect to SDSS \citep{1805.01227,1807.02559,1807.03315}.
The growing sample of WDs with precisely measured distances will enable an improved measurement of the WDLF.  
However, the completeness of the \Gaia sample is limited to WDs within 100 pc~\citep{1807.03315}.
%Oscar. In the GAIA DR2 about 260.000 WDs have been identified. However the sample is complete up to G=20-21, only for those within 100 pc, which are about 11.000 stars. The major problem is that the completeness drops at low Galactic latitudes, and the magnitude limit of the catalogue varies significantly across the sky as a function of Gaia’s scanning law.  A larger and more complete sample will be certainly available with the final data release.
LSST is expected to detect WDs that are 5 to 6 magnitudes fainter than those detected by \Gaia, ultimately increasing the census of WDs to tens of millions~\citep{0912.0201}.
LSST will provide more complete and homogeneous samples of WDs, allowing for a significant reduction in both the statistical and systematic uncertainties in measurements of the WDLF. 
LSST is expected to measure hundreds of thousands of WDs in the Galactic halo, enabling the construction of a reliable luminosity function of halo WDs. 
By deriving independent WDLFs from different Galactic populations it will be possible to reduce uncertainties related to star formation histories, and to ultimately provide a more clear assessment of the physical origin of the cooling anomalies \citep{Isern:2018uce}. 

The growing sample of WDs will similarly increase the known population of variable WDs with pulsation periods of 100s--1500s. 
Measured changes in the pulsation periods of WDs can be used to directly constrain the rate of cooling \citep[\eg][]{1007.2659}.
Indeed, hints of anomalous cooling from axions have been claimed \citep[\eg][]{Corsico:2012ki,Corsico:2012sh}, though more recent analyses set upper limits at the level of $g_{ae} < 3.3 \times 10^{-13}$ \citep{Battich:2016htm}. 
LSST will greatly increase the sample of pulsating WDs, enabling high-cadence follow-up observations to precisely measure changes in pulsation period and probe anomalous cooling mechanisms.


\subsection{Globular Cluster Stars}

Massive stars, specifically those close to the helium burning phase, provide another excellent environment to study anomalous energy loss mechanisms. 
Stellar evolutionary codes such as MESA \citep{1009.1622} provide a good model for the evolution of massive stars, allowing constraints to be placed on novel particle production \citep[\eg,][]{1210.1271,1611.05852}.
Several recent analyses of giant branch stars have reported deviations from standard stellar model predictions that can be interpreted as a signature of anomalous energy loss.
For example, studies have shown a brighter-than-expected tip of the RGB (TRGB) in the M5 globular cluster~\citep{Viaux:2013lha,Viaux:2013hca}, indicating somewhat over-efficient cooling during the evolutionary phase preceding the helium flash.
The anomalous brightness, $\Delta M_{I,{\rm TRGB}}\simeq 0.2$ mag in absolute $I$-band magnitude, observed in M5 can be interpreted as an anomalous cooling of a few $10^{33}$ erg/s.
Such cooling could be accounted for by a neutrino magnetic moment or an axion-electron coupling of the order of that predicted from the WDLF~\citep{Viaux:2013lha}. 
These constraints can be improved using multi-band photometry of multiple globular clusters \citep[\eg,][]{Straniero:2018fbv}.

Advances in the analysis of globular cluster RGB stars are currently limited by modeling uncertainties on the stellar evolution of the giant branch.
Fundamental improvements should be expected in the near future. 
In particular, exquisite astrometry from the {\it Gaia} satellite will precisely determine cluster distances, currently the largest sources of observational uncertainty in the determination of the absolute luminosity of the TRGB.\footnote{The {\it Gaia} data relevant for GCs are expected in 2022~\citep{Gaia}.}
Moreover, the angular resolution of the next-generation space-based missions, such as JWST~\citep{Gardner:2006ky}, will enlarge the statistical sample of RGB members near the cores of GCs. 
The brightness of RGB stars in nearby GCs limits the contributions of LSST, which saturates at $g \sim 17$ mag and suffers from crowding near the cores of GCs.
However, LSST will enable independent measurements of GC distances, providing a valuable handle on systematic uncertainties of {\it Gaia} observations.\footnote{The parallaxes of bright ($G<14$ mag) sources can be derived with a median uncertainty of 0.04 mas in {\it Gaia} DR2. However, for fainter stars the parallaxes become sensitive to systematic errors.  Presently, these systematics hamper a precise determination of GC distances \citep{Chen:2018}.}
Moreover, it is likely that the homogeneity and precise photometry of LSST will improve the calibration of the bolometric corrections for RGB and HB stars and ultimately contribute to a more clear assessment of the cooling of GC stars.


\subsection{Massive stars and core-collapse supernovae}

The cores of massive stars are among the most powerful natural laboratories to investigate the possible production of weakly interacting light particles, particularly axions. 
The energy loss rate via axions is quite sensitive to temperature. 
For instance, the rate of the Primakoff process (the photon-axion conversion in the static electric field of ions and electrons) scales as $T^4$. 
\figref{massivestar} shows the evolution of the neutrino and axion luminosities in a $18\Msun$ star. 
After He burning, the central temperature rapidly increases, becoming larger than $10^9$ K. 
In standard stellar models (no axions or other non-standard cooling), the energy loss by neutrinos largely overcomes the energy loss by photons. This rapidly decreases the evolutionary time scale and determines the chemical and physical structure of the star at the onset of core collapse. 
In this context, an additional energy loss mechanism may significantly affect the pre-explosive stellar structure and, in turn, may determine the success or failure of a core collapse supernova (CCSN). 
Such an effect may be revealed by connecting CCSNe to their massive star progenitors, something that will be enabled by the wide area and high temporal cadence of LSST.

Another powerful strategy by which novel particles can be constrained is by considering the evolution of the neutrino cooling phase of nearby SNe  \citep[\ie, SN1987A][]{Burrows:1988, Raffelt:1988}.
The simplest and most robust method by which such constraints can be implemented is the so-called ``Raffelt criterion,'' which limits the luminosity of new particles to be below the luminosity of neutrinos during the neutrino-cooling phase \citep{hep-ph/0611350}.
Neutrino observations of SN1987A have been used to place limits on a wide variety of new particles \citep{hep-ph/0207098, 1611.03864, 1611.05852, 1803.00993, 1808.10136}.
Again, it is possible that a subdominant release of energy into new particles is responsible for resolving some lingering inconsistencies with the Standard Model-only picture of CCSNe explosions \citep{0806.4273, 1805.07381}, though such an effect is difficult to resolve analytically on top of other qualitative uncertainties of the Standard Model-only picture \citep{1809.05106, 1811.11178}.
LSST, in combination with upcoming neutrino experiments (i.e., DUNE), will help reduce Standard Model uncertainties and expand this analysis to future and more distant CCSNe \citep{1807.10334}.

\begin{figure}[t]
\centering
\includegraphics[width=0.5\columnwidth]{massivestar.png}
\caption{Evolution of the luminosities of neutrinos and axions in a $M=18 \Msun$ stellar model ($t$ in years), relative to the photon luminosity. The various evolutionary phases are indicated. 
%In the calculation of the axion rate we have assumed the presently available upper bounds, as obtained from astrophysical constraints, for the axion-photon and axion-electron coupling constants.
In the calculation of the axion rate, we have assumed the current upper bounds on the axion-photon and axion-electron coupling.
\ADW{Do we need a reference for this figure?}
}
\label{fig:massivestar}
\end{figure}

LSST is expected to discover $\roughly 3.5\times10^5$ CCSNe per year \citep{Lien:2009}. 
%\ADW{Better to update this from Goldstein et al. (2018).}\AHGP{I think the Goldstein ref is with respect to lensed SNe}\ADW{Goldstein starts from a full population analysis, which is what we would take.}
For nearby SNe, LSST will be able to resolve massive progenitor stars in pre-explosion imaging. 
So far, a clear identification of progenitor stars has been obtained only for about 20 type II SNe \citep{Smartt:2015}.  
The identification of a much larger number of massive stars before they explode is a mandatory step for understanding the CCSNe process and the possible activation of non-standard cooling processes during the late evolution of massive stars. 
Recent theoretical studies have investigated the conditions for which a massive star successfully bounces after core collapse, giving rise to a SN \citep[\eg,][and references therein]{OConnor:2011,Sukhbold:2016}.   
In particular, it was found that the ability to explode predominantly depends on the structure of the progenitor. 
%Therefore, the impact of LSST in this field is twofold: it will constrain the supernova engine and may provide hints for new physics beyond the standard model. 
Therefore, besides providing more clear insight into the SN engine, LSST will provide a solid framework to test the presence of novel cooling channels efficient during pre-SN evolution, constraining or hinting at the existence of axions or other weakly interacting particles.



\subsection{Large Scale Structure \Contact{Tony}}
\Contributors{Tony, Rogerio,...}

The physics of dark matter could be probed via LSST if dark energy were an aspect of dark matter at late times. 
The natural inhomogenieties in dark matter on large scales would then be reflected as spatial inhomogenieties in the developing late time acceleration. 
This could be observed as spatial variations in cosmic acceleration by LSST. 
It is possible that dark matter and dark energy are causally related. 
For a recent paper on emergent dark energy, see \cite{1801.09658}. 
A spatially complicated potential leads to a small cosmological constant from an energy difference between its global and local minima, and dark energy and dark matter are intertwined. 
If so, and if the universe on sub-horizon scales is not homogeneous, then spatial fluctuations in one should be correlated with spatial fluctuations in the other, particularly near the epoch of emergence.

There are other models where dark matter and dark energy are intertwined, such as models where they interact 
\citep{Amendola:1999er,Holden:1999hm}.
Interacting models can be described phenomenologically via two fluids that can exchange energy and momentum, 
described by energy-momentum tensors that are individually not conserved. One can parametrize the coupling of
dark matter and dark energy by writing the divergence of the individual energy-momentum tensors as 
\begin{eqnarray}
\nabla_{\mu} T^{(DE)}\,^{\mu}_{\nu} &=& C^{(DE)}_{\nu}, \label{cons_phi} \\
\nabla_{\mu} T^{(DM)}\,^{\mu}_{\nu} &=& C^{(DM)}_{\nu}, \label{cons_dm}
\end{eqnarray}
where the superscript $(DM)$ stands for the dark matter fluid and $(DE)$ for the dark energy.
The conservation of the total dark component energy-momentum tensor 
(we assume the separate conservation of the energy momentum of radiation and baryons)
\begin{equation}
\label{energyconservation}
\nabla_{\mu} \left[ T^{(DM)} \,^{\mu}_{\nu} + T^{(DE)} \,^{\mu}_{\nu} \right]= 0,
\end{equation}
implies that
\begin{equation}
C^{(DM)}_{\nu}=-C^{(DE)}_{\nu}.
\end{equation}

The coupling between dark matter and dark energy is determined by the function $C^{(DM)}_{\nu}$ which is usually
written as
\begin{equation}
C^{DM)}_{\nu} = (8\pi G)^{1/2} \,\beta\rho_{DM}\nabla_{\nu} \phi,
\end{equation}
where $\beta$ is a constant that expresses the coupling strength. In this model dark energy must be dynamical and here it is modelled by a scalar field  $\phi$, such as a quintessence field.
In this model $\beta$ is the only new parameter in addition to the usual description of the dark energy sector.
The standard uncoupled case is recovered for $\beta=0$.

There is a vast literature studying this class of models that can modify both the evolution of the 
background cosmology as well as the evolution of perturbations. For instance, it has been recently claimed
that such a model can ease the tension in the measurements of $\sigma_8$ from CMB and galaxy surveys 
\citep{Barros:2018efl}.

LSST can separately map dark matter and dark energy at a redshift where they have roughly comparable influences on the expansion rate of the universe. 
There are two complementary methods of reconstructing dark energy on the sky: SNe and 3$\times$2pt in 20 degree patches \citep[Figure 15.9 in ][]{0912.0201}.
The transition between a dark-matter-dominated universe to one with late-time acceleration (dark energy) may hint at some connection between these two components. 
An angular cross correlation between maps of dark energy and tomographic weak lens maps of dark matter could yield a non-zero signal.  
If so, the ratio of the cross correlation to the auto-correlations would be a diagnostic of the underlying physics. 
In this scenario measurements of dark energy anisotropy become a probe of the nature of dark matter. 
See \cite{1810.11007} for a Ginzburg-Landau phase transition model that results in correlated dark matter-dark energy anisotropy (\figref{DMDEmap}). 
Quadrupole and higher correlated anisotropies are generated around redshift $z=0.7$.  
This is accessible in LSST maps of dark energy and dark matter in a broad redshift shell.

\paragraph{Systematics and synergies:}

Systematics in the DE and DM maps on large angular scales must be reduced below the level of any DM-DE correlation signal.  
For example, systematics in apparent magnitude and photo-z due to uncorrected extinction from Galactic dust would be one focus. 
Happily the two measures of DE anisotropy are very differently dependent on the wavelength dependent extinction. 
A useful null test will be the cross correlation between any of the three DE and DM maps with dust maps.  
This would set the floor for residual extinction systematics, forming the basis for a forward simulation of the resulting DE-DM false correlation. 
As in analysis of CMB data, cuts on Galactic latitude can reveal the level of residual systematics. 
Finally, any dependence of the cross-correlations on redshift could discriminate between models as well as detect z-dependent systematics.

For detection of low multipole sky correlations, observations in the north as well as the south will be useful.
There is important synergy with WFIRST and EUCLID observations in the north.  
These complementary data could be calibrated and tested by joint null tests in overlap areas with the LSST survey.

\begin{figure}[t]
\centering
\includegraphics[width=0.9\columnwidth]{DMDE-anisotropy.png}
\caption{A schematic diagram of the emergence of dark energy anisotropy from an Ising model phase transition and a coupling with the anisotropic distribution of dark matter. Figure taken from \cite{1810.11007}.}
\label{fig:DMDEmap}
\end{figure}

% -----------------------------------------------------------------------
\section{Complementarity with Other Experiments}
\Contributors{Josh S., Ting L., Will, Andrew P. Manuel, Chanda, Alex, many others ...}
\label{sec:complementarity}
% Please see the complementarity.tex file.
\chapter{Complementarity with Other Experiments}
\Contributors{Josh S., Ting L., Will, Andrew P. Manuel, Chanda, Alex, many others ...}
\label{sec:complementarity}
\bigskip

The LSST data set will uniquely complement many other experimental studies of dark matter.
Below we summarize some of these complementary probes, with a specific focus on spectroscopic observations, high resolution imaging, indirect detection experiments, and direct detection experiments.
While LSST can substantial improve our understanding of dark matter in isolation, support of these experiments is essential to provide an comprehensive picture of dark matter physics.
This section is not intended to be comprehensive, but rather serves to demonstrate the influence that LSST will have on dark matter studies generally.

% Spectroscopy
\section{Spectroscopy \Contact{Ting}}
 \Contributors{Josh S., Ting L., Erik T., ...}
 \label{sec:spectroscopy}

%(TL: do we want to make a table to compare the specs of various spectroscopy facility. I do not think this is the scope of this paper...) ADW: I agree, this seems beyond our scope%\TL{need to check if ESO wide-field instrument has a name already or not}

While the photometric and astrometric measurements from LSST alone are quite powerful, their impact can be significantly augmented by additional spectroscopic observations. 
In particular, spectroscopic follow-up studies will provide kinematic and redshifts information for many of the objects studied by LSST.
Given the faintness and high density of targets that are expected from LSST, community access to multi-object spectrograph on large aperture telescopes is essential for these studies \citep{2016arXiv161001661N}. 
Due to LSST's location in the southern hemisphere, southern spectroscopic facilities are best, as this  maximizes the overlapping sky area.

Many next-generation telescopes and instruments are currently under preparation or construction. These instruments are broadly divide into two categories: massively multiplexed spectrographs on 8 to 10-meter telescopes, and giant segmented mirror telescopes (GSMTs, $\sim30$-meter class) with smaller field of view. The former category includes facilities on exisiting and future telescopes including the Southern Spectroscopic Survey Instrument (SSSI), a project recommended for consideration by the DOE’s Cosmic Visions panel \citep{1604.07626, 1604.07821}, the Primary Focus Spectrograph (PFS) instrument on the Subaru telescope \citep{2014PASJ...66R...1T}, the Maunakea Spectroscopic Explorer \citep[MSE;][]{MSEbook2018}, and a possible future ESO wide-field spectroscopic facility. 
The latter category is populated by new facilities such as Thirty Meter Telescope \citep[TMT;][]{1505.01195}, the Giant Magellan Telescope \citep[GMT;][]{GMT:2018}, and the European Extremely Large Telescope \citep[E-ELT;][]{EELT:2009}. 

In this section, we illustrate several examples of how complementary spectroscopy will improve the measurements on the dark matter properties with LSST.

\subsection{Milky Way Dwarfs \Contact{Josh}}
\Contributors{Josh, Ting, Erik, ...}
In Section~\ref{sec:smallest_galaxies} we discussed the derivation of an upper limit on the minimum dark matter halo mass based only on the observed luminosity function of satellites discovered by LSST. \ET{Should we mention the fact that confirmation of the LSST discoveries *also* may require spectroscopy at least for some of the ambiguous cases?}  An alternative approach is to obtain spectroscopy of individual stars in each satellite to measure its velocity dispersion, from which the central mass and density can be inferred.  Then one can compare either the densities or the circular velocity function directly with theoretical predictions without assumptions about the subhalo mass function or the stellar mass-halo mass relation.

Spectroscopy of individual stars in the faint Milky Way satellites that will be identified with LSST will require deep observations with multiplexed spectrographs on large telescopes.  Measurements of the stellar velocity dispersions of these systems can be obtained either with 8-10~m-class telescopes or with the next generation of 25-30~m telescopes.  As illustrated in Fig.~\ref{fig:specfollowup_distance}, spectroscopy of a nearly complete sample of satellites can be pushed $\sim2$~mag fainter in luminosity and a factor of $\sim2$ farther in distance with plausible investments of observing time on a GSMT than with existing facilities.

In additional to inferring the minimum dark matter halo mass, kinematics from stellar spectroscopy can also reveal the density profile of the dwarf galaxies at the lowest luminosities, in which the baryonic effects are minimum and therefore dark matter physics can be separated from the astrophysics of galaxy formation (cite). A direct measurement of the density profile in these dwarf galaxies will allow us to distinguish between collisionless CDM  which predicts a cusp NFW profile, and SIDM which predicts a core profile (cite). Moreover, the stellar kinematics will also reveal the integral of the dark matter density profile in the dwarf galaxy (or J-factor), which is an essential input for the constraints on the dark matter self-annihilation cross section for the indirect dark matter search in X-ray and gamma-ray experiments \citep[e.g.][]{1108.3546}.


\begin{comment}
\begin{figure}
\centering
\vspace{-2in}
\includegraphics[width=0.85\textwidth]{figures/dwarf_observability_barplot_distance.pdf}
\vspace{-2in}
\caption{Possibility of spectroscopic follow-up for the LSST satellite population as a function of distance. Current telescopes will be able to measure velocity dispersions for $\sim50\%$ of the expected satellites, while a GSMT can measure velocity dispersions for $\sim80\%$. }\label{fig:specfollowup_distance}
\end{figure}


\begin{figure}
\centering
\vspace{-2in}
\includegraphics[width=0.85\textwidth]{figures/dwarf_observability_barplot_luminosity.pdf}
\vspace{-2in}
\caption{Possibility of spectroscopic follow-up for the LSST satellite population as a function of magnitude. }\label{fig:specfollowup_distance}
\end{figure}
\end{comment}

\begin{figure}
  \centering
  \includegraphics[width=0.49\textwidth]{figures/dwarf_observability_barplot_distance.pdf}
  \includegraphics[width=0.50\textwidth]{figures/dwarf_observability_barplot_luminosity.pdf}
  \caption{Possibility of spectroscopic follow-up for the LSST satellite population as a function of distance (left) and magnitude (right). Current telescopes will be able to measure velocity dispersions for $\sim50\%$ of the expected satellites, while a GSMT can measure velocity dispersions for $\sim80\%$.}
  \label{fig:specfollowup_distance}
\end{figure}

\subsection{Stellar Streams \Contact{Ting}}
\Contributors{Ting, Denis ...}

As discussed in Section~\ref{sec:stream_gaps}, subhalo encounters with cold stellar streams will induce density perturbations that will be detectable by LSST, constraining the minimum dark matter halo mass and the mass function of dark matter halos from $\sim10^{5} - 10^{9}$~M$_{\odot}$. In addition, these flybys cause velocity perturbations that correlate with the density variations.  The velocity signal near stream gaps can be measured either via line-of-sight velocity measurements from spectroscopy or tangential velocity measurements from astrometry, improving the precision with which the perturber mass can be determined.
The velocity variation (peak to peak) from these flybys will be small. To estimate the amplitude of the perturbation, we consider a stream orbiting the Milky Way at a distance of 14 kpc and compute the typical maximum velocity kick expected over its lifetime of 5 Gyr using the formalism from \cite{erkal2016}.  The velocity change is $\sim$0.6, 0.3, 0.1 km/s for subhalos in the range $10^7-10^8 M_\odot$, $10^6-10^7 M_\odot$, $10^5 -10^6 M_\odot$ respectively. Due to the low density of the stream stars, a massively multiplexed, wide field-of-view spectroscopic facility such as PFS on Subaru or MSE is needed. Furthermore, given the expected small velocity kick amplitude, the velocity accuracy for each star determined from the spectroscopic observations should be at or better than 1 km/s to unambiguously detect the signal with an ensemble of stream stars.

\subsection{Galaxy Clusters \Contact{Will}}
\Contributors{Will, ...}
As noted in \S\ref{sec:merging_clusters} one of largest systematics associated with merging galaxy cluster constraints of SIDM is modeling the merger. The more complex the merger the more severe the systematics.
The best means of constraining merging galaxy cluster substructure is with spectroscopic measurement of as many galaxy cluster member galaxies as possible \cite[see e.g.,][]{2018arXiv180610619G}.
As noted in \cite{2016arXiv161001661N}, perhaps the best spectroscopic follow-up facilities are large telescopes with slitmask-like multi-object spectrometers, or fiber-based multiplex spectrometers with low ($\mathcal{O}(arcsec)$) fiber collision regions, due to the density of cluster members.


\subsection{Lyman-$\alpha$ Forest \Contact{Francis-Yan}}

\ADW{This is expected to be one paragraph from Francis-Yan.}

% High-resolution imaging
\section{High-Resolution Imaging \Contact{Will}}
\Contributors{Will, ...}
\label{sec:highres}

Since the LSST point spread function (PSF) is limited to an angular resolution of $\sim 0.5\arcsec$ by the atmospheric seeing, there are many dark matter science cases where higher resolution imaging from space or ground-based adaptive optics (AO) facilities, which can reach $\sim 0.01\arcsec$ in some cases, can be highly complementary. We briefly summarize some of these cases in this subsection and relate them to the dark matter science capabilities of LSST.
\WAD{This section currently focuses on high resolution optical imaging, however it is worth considering other wavelengths, especially radio.}

% Astrometric microlensing of compact dark matter
\subsection{Astrometric Microlensing of Compact Dark Matter \Contact{Will}}
\Contributors{Will, ...}
\label{sec:astrometric_microlens}
Related to photometric microlensing (\S\ref{sec:microlensing}), astrometric microlensing relies on the fact that the two images generated during a compact object lensing event will be of differing brightness, and the brightness ratio of these two images will vary throughout the duration of the lensing event.
The two images will be of most similar brightness when the projected lens-source separation is at its minimum.
By precisely measuring the astrometry of these blended images as a function of time and combining with the LSST photometric microlensing measurement one can break the lens mass-distance degeneracy and precisely measure the mass and location of individual black holes \cite{2015ApJ...814L..11Y}.

% Strong-microlensing
\subsection{Strong-Microlensing of Compact Dark Matter \Contributors{Will, ...}}
Strong-microlensing is related to astrometric microlensing (\S\ref{sec:astrometric_microlens}).
The Einstein radius of a given lens, which is approximately the separation of the multiple images in a compact object lensing scenario, scales as $\sqrt{M_\mathrm{lens}}$.
In the intermediate mass black hole range, the separation of the two images approaches that of the resolution of various optical ground and spaced-based telescopes, see Figure \ref{fig:strong_microlensing}.
If the multiple images can be resolved and their flux ratio measured it enables precise measurement of the mass and distance of the lens.

\begin{figure}
\label{fig:strong_microlensing}
\centering
\includegraphics[width=0.6\columnwidth]{figures/StrongMicrolensing.png}
\caption{The Einstein radius (i.e., 1/2 the separation of the multiple images) for microlensing lensing events as a function of IM MACHO mass and distance between us and the MW bulge. Any parameter space below a black curve indicates that the multiple lensed source images will be resolvable by that telescope. \Contributors{Will, PALS Collaboration, ...}}
\end{figure}

% Merging Galaxy Clusters
\subsection{Merging Galaxy Clusters and Cluster Subhalos}
\Contributors{Will D., Dave W., ...}

Most dark matter constraints from merging galaxy clusters (\secref{halo_profile_clusters}) and cluster subhalos (\secref{halo_profile_clusters}) rely on accurately measuring the distribution of dark matter in (sub)clusters via gravitational lensing.
Strong and weak gravitational lensing both benefit from high resolution imaging.
For strong lensing the high resolution imaging enables better detection and characterization of strongly lensed background images in the dense cluster environment.
Similarly high resolution imaging provides $\sim4$ times more lensed source galaxies per unit area than ground-based imaging at similar depths, which enable higher resolution weak gravitational lensing.
Historically Hubble Space Telescope (HST) has provided this higher resolution imaging, although in the era of LSST it appears that we will rely on space-based telescopes such as JWST, Euclid, and WFIRST.
There is also the potential to leverage future wide field AO instruments \WAD{Need to provide and example.}.

% Strong gravitational lensing
\subsection{Strong Gravitational Lensing}
\label{sec:SLcomplement}
\Contributors{Chris F.} 

All three approaches to use strong gravitational lens systems to make inferences on the nature of dark matter that were described in \secref{stronglens} utilize LSST as a lens-finding facility.
Once the lenses are found, the dark matter science requires follow-up observations with other facilities.
The flux-ratio anomaly approach requires imaging that spatially resolves the lensed images from each other at a wavelength at which microlensing does not affect the image fluxes.
These observations can be in optical/near-IR wavelengths, utilizing IFU spectrographs behind the adaptive optics systems on ELTs to isolate the emission from the narrow-line regions of the lensed AGN, at mid-IR wavelengths with JWST, or at radio wavelengths for the subset of LSST lenses that is radio-loud.
The gravitational imaging and power-spectrum approaches both require milliarcsecond-scale angular resolution imaging for best results.
These observations require either ELT adaptive optics imaging or VLBI radio imaging of the targets.
ALMA can also be used in its most extended configuration, although this will not achieve as high a resolution as the ELTs and VLBI in most cases.

% Indirect Detection
\section{Indirect Detection }

In regions of high dark matter density, dark matter particles could continue to annihilate or decay through the same process that set their relic abundance.
Of specific interest are energetic photons (i.e., X-rays and $\gamma$ rays), since photons are produced generically by the annihilation/decay of many dark matter models (either directly or as secondarily from the production of quarks or leptons). In addition, astrophysical phenomena in extreme environments could lead to conversion between Standard Model particles and the dark sector (e.g., ALPs), which could be observable through the emission energetic photons or alterations in astrophysical spectra.
By precisely mapping the distribution of dark matter and tracking extreme events (e.g. core-collapse SN) LSST will enable more sensitive searches for energetic particles originating from the dark sector.

Conventional indirect detection searches focus predominantly on WIMPs with masses between several \GeV and tens of \TeV. 
The annihilation or decay of these particles could produce energetic standard model particles detectable by current or future experiments.
The most sensitive and robust indirect searches for dark matter rely on a precise determination of the distribution of dark matter in the universe.
The integrated flux of energetic Standard Model particles $\phi_s$ (${\rm particles} \cm^{-2} \second^{-1}$), expected from dark matter annihilation in a density distribution, $\rho(\vect{r})$, is given by

\begin{equation}
   \phi_s(\Delta\Omega) =
    \underbrace{ \frac{1}{4\pi} \frac{\Gamma}{m_{\DM}^{a}}\int^{E_{\max}}_{E_{\min}}\frac{\text{d}N}{\text{d}E}\text{d}E}_{\rm particle\ physics}
    \cdot
    \underbrace{\vphantom{\int_{E_{\min}}} \int_{\Delta\Omega}\int_{\rm l.o.s.}\rho^{a}(\vect{r})\text{d}l\text{d}\Omega '}_{\rm astrophysics}\,.
    \label{eqn:indirect}
\end{equation}
%\Big\{\Big\}
\noindent Here, the ``particle physics'' term is strictly dependent on the particle physics properties---i.e., the particle mass, $m_\DM$,  the interaction rate, $\Gamma$, and the differential particle yield per interaction, $\text{d}N/\text{d}E$, integrated over the experimental energy range.
The second term, denoted ``astrophysics'', represents the line-of-sight integral through the dark matter distribution integrated over a solid angle, $\Delta\Omega$. 
For cases of dark matter annihilation, the interaction rate is set by the thermally averaged self-annihilation cross section, $\Gamma = \sigmav/2$, and the astrophysical integral is performed over the square of the dark matter density ($a=2$). 
The resulting astrophysical term is referred to as the ``\Jfactor'' \citep[\eg,][]{1998APh.....9..137B}. 
In cases of dark matter decay, the interaction rate is inversely proportional to the lifetime of the dark matter particle, $\Gamma = 1/\tau$, and the integral is performed over the dark matter density, $a=1$. 
The resulting term is known as the ``\Dfactor'' \citep[\eg][]{1408.0002}.
Qualitatively, the astrophysics term encapsulates the spatial distribution of the dark matter signal, while the particle physics term sets its spectral character. 
LSST will improve the sensitivity to dark matter particle physics by improving our understanding of the astrophysics term.
While these improvements will influence a wide range of indirect detection experiments, in this section we focus predominantly on $\gamma$-ray measurements.


\subsection{Milky Way Satellites \Contact{Andrew}}
\Contributors{Manuel M., Esra B., Andrew P., Ethan N., Alex}

\begin{figure}[t]
\centering
\includegraphics[width=0.75\columnwidth]{id_annih.pdf}
\caption{Constraints on dark matter annihiltion to $b\bar{b}$ from {\it Fermi-LAT} observations of Milky Way satellite galaxies \citep[LAT Dwarfs;][]{} and HESS observations of the Galactic Center \citep[HESS GC;][]{1607.08142}. 
A bracketing range of dark matter interpretations to the  Fermi-LAT Galactic Center Excess is shown in red \citep[GCE;][]{1402.6703, Gordon:2013, Abazajian:2014}.
Projected sensitivity to dark matter annihilation combining LSST discoveries of new Milky Way satellites, improved spectroscopy of these galaxies, and continued Fermi-LAT observations is shown in gold. This projection assumes 18 years of Fermi-LAT data, a factor of 3 increase in the integrated J-factor, and a factor of 2 improvement from improved spectroscopy. 
Projected sensitivity of 500h observations of the GC with CTA are shown in gray \citep[CTA GC;][]{Zaharijas:prep}.
\label{fig:indirect}
}
\end{figure}

Gamma-ray observations of Milky Way satellite galaxies currently provide the most robust and sensitive constraints on the dark matter self-annihilation cross section for GeV- to TeV-mass particles \citep[\eg][]{Ackermann:2014, Geringer-Sameth:2015, Ackermann:2015}.
The sensitivity of these searches will improve by combining new Milky Way satellite galaxies discovered by LSST, more precise $J$-factor measurements from novel spectroscopic observations, and additional Fermi-LAT data. \EN{What about CTA?}
We estimate each of these contributions to predict the improved sensitivity of dark matter annihilation searches in dwarf galaxies in the era of LSST.

To estimate the improvement in the integrated $J$-factor of the Milky Way satellite galaxy population, we combine cosmological zoom-in simulations of Milky Way dark matter substructure with a semi-analytic model to convert subhalo density profiles to $J$-factor estimates (this approach is is similar to that of \citealt{1309.4780}). 
Our simulation-based model accounts for modulations to dark matter-only subhalo populations due to baryonic physics, and we marginalize over the dependence of subhalo populations on host halo properties by sampling subhalo populations from a large number of hosts \citep{Nadler:2018}. 
To obtain an estimate for the increase to the integrated $J$-factor, we select a host halo with the largest number of nearby subhalos, consistent with recent observations of over-abundance of nearby satellites associated with the Milky Way \citep{Kim:2018, Graus:2018}. 
We exclude subhalos with heliocentric distances $< 20 \kpc$ to avoid anomalously large projections due to a single nearby satellite.
We follow the analytic formalism presented by \citet{1604.05599} and \citet{1802.06811} to convert the dark matter profiles of our simulated subhalos to \Jfactors.  
This approach estimates the \Jfactor of each subhalo based on $r_{\max}$, $V_{\max}$, and heliocentric distance. 
%We calculate the cumulative $J$-factor within 100 kpc we find for the most optimistic case 
%(i.e., selecting the host halo with the highest cumulative $J$-factor and ignoring the effects of subhalo disruption) 
We find that the cumulative \Jfactor within 100\kpc may increase by as much as a factor of 3 relative to the known dSphs with measured \Jfactors. 

Recent studies have suggested that an additional factor of 2 improvement in sensitivity may be possible through better spectroscopic measurements  of the stars in known satellite galaxies \citep{Albert:2017}, and we include this factor in our projections.
In addition, current constraints from the Fermi-LAT Collaboration used 6 years of data \citep{Ackermann:2015}; however, the Fermi-LAT has collected more than 10 years of data and could continue to collect data for another 10+ years.
These additional data will improve the statistical sensitivity of the gamma-ray search most drastically for large dark matter particle masses ($>500\GeV$).
We quantitatively evaluate the improvement from continued Fermi-LAT data taking using the results of \citep{Charles:2016}.
We combine the predicted improvements from new dwarfs, better determined $J$-factors into a projected sensitivity for future searches for dark matter annihilation in dwarf galaxies \figref{indirect}.


\subsection{Cross correlation with gamma rays \Contact{Horiuchi,Alessandro C}}
\Contributors{Horiuchi, Alessandro C}

%Wide-area weak-lensing measurements from LSST will help extract  potential dark matter contributions to the isotropic gamma-ray background \citep[IGRB;][]{1410.3696}. The IGRB is defined as the residual all-sky $\gamma$-ray emission after subtracting individually detected sources and the Galactic diffuse emission, and provides the distance frontier of indirect dark matter searches with $\gamma$ rays. Contributions to the IGRB include unresolved sources that are individually too faint to be detected---e.g., blazars \citep{1110.3787,1310.0006}, star-forming galaxies \citep{1206.1346}, and misaligned AGNs \citep{1304.0908}---as well as a potential contribution from dark matter annihilation \citep{1312.0608,1501.05464,1501.05301,1608.07289}. Analyses of the IGRB intensity spectrum, auto-correlation angular power spectrum, and photon count statistics show that a linear combination of astrophysical sources can explain the observed IGRB, but the uncertainties are still large \citep[e.g.,][]{1502.02866}.

%LSST will prove invaluable by mapping the distribution of matter on large scales via measurements of cosmic shear from weak gravitational lensing. 
%The small distortions in images of distant objects caused by gravitational lensing by the large-scale matter distribution along the line of sight is called cosmic shear. %ADW: I think this should be covered elsewhere
%Since cosmic shear and cosmological $\gamma$-ray emission from dark matter annihilation are sourced by the same underlying dark matter distribution, cross correlating them yields novel information on the composition of the IGRB \citep{1212.5018,1411.4651}. Cosmic shear unbiased tracer of dark matter distribution, which mitigates many of the systematics from using galaxies to trace dark matter---i.e., assumptions about the relationship between galaxy luminosity and halo mass, reliance on assumptions of hydrostatic equilibrium, and strong correlations with astrophysical $\gamma$-ray emission. At present, weak lensing surveys of several hundred square degrees allow studies of the IGRB to probe slightly above the thermal annihilation cross section \citep{1404.5503,1607.02187,1611.03554}. A simple forecast for LSST can be made by scaling the covariance matrix of the correlation estimator by the sky coverage. This shows that a combination of LSST lensing maps and all-sky Fermi-LAT data will reach a sensitivity where it is possible to \textit{detect} at $3\sigma$ WIMP annihilation to $b\bar{b}$ at the thermal cross cross section for up to 100 GeV masses \citep{1404.5503}. Compared to the IGRB intensity or auto-correlation, the cross correlation will yield more than $\sim 10$ times higher sensitivity to dark matter \citep{1411.4651}.  Combined cross correlations with other baryonic and gravitational tracers, e.g., galaxies and galaxy clusters \citep[\eg][]{1506.01030,Lisanti:2018,1709.01940}, will provide a better handle on the astrophysical contributors, thereby further improving sensitivity to the dark matter contribution. 
% Other references
% https://arxiv.org/abs/1411.4651

Wide-area weak-lensing measurements from LSST will help extract  potential dark matter contributions to the isotropic gamma-ray background \citep[IGRB;][]{1410.3696}. The IGRB is defined as the residual all-sky $\gamma$-ray emission after subtracting individually detected sources and the Galactic diffuse emission, and provides the distance frontier of indirect dark matter searches with $\gamma$ rays. Contributions to the IGRB include unresolved sources that are individually too faint to be detected---e.g., blazars \citep{1110.3787,1310.0006}, star-forming galaxies \citep{1206.1346}, and misaligned AGNs \citep{1304.0908}---as well as a potential contribution from dark matter annihilation \citep{1312.0608,1501.05464,1501.05301,1608.07289}. Analyses of the IGRB intensity spectrum, auto-correlation angular power spectrum, and photon count statistics show that a linear combination of astrophysical sources can explain the observed IGRB, but the uncertainties are still large \citep[e.g.,][]{1502.02866}.

LSST will prove invaluable by mapping the distribution of matter on large scales via measurements of galaxy clustering and of cosmic shear from weak gravitational lensing. 
%The small distortions in images of distant objects caused by gravitational lensing by the large-scale matter distribution along the line of sight is called cosmic shear. %ADW: I think this should be covered elsewhere
Since  cosmological $\gamma$-ray emission from dark matter annihilation also follows the same underlying dark matter distribution traced by cosmic shear and galaxies, cross correlating them yields novel information on the composition of the IGRB \citep{1212.5018,1411.4651,1506.01030,Lisanti:2018,1312.4403}. 
Compared to the IGRB intensity or auto-correlation, the cross correlation will yield more than $\sim 10$ times higher sensitivity to dark matter \citep{1411.4651,1503.05922}.
Cross-correlations with galaxy catalogs have been derived in \cite{1709.01940,1503.05918,1103.4861} up to  $z\sim 0.6$ 
which is the largest redshift where present  catalogs still have enough sky-coverage and galaxy density to robustly
detect the correlation.  On the other hand, the IGRB is expected to extend in redshift  up to $z\sim 2$--$3$ \citep{1502.02866}. 
LSST, with its large sky-coverage and galaxy density and broad redshift range, thus fills this gap to map the IGRB-LSS cross-correlation up to high redshift. 
A complete mapping of the IGRB up to $z\sim3$ will constitute a crucial tool to robustly separate the different 
astrophysical contributions, as well as to isolate the DM annihilation signal, breaking the degeneracies which
are present when only low redshift results are used \citep{1506.01030}.    

Complementary to galaxy catalogs, cosmic shear has the advantage of being an unbiased tracer of dark matter distribution, which mitigates many of the systematics from using galaxies to trace dark matter---i.e., assumptions about the relationship between galaxy luminosity and halo mass, reliance on assumptions of hydrostatic equilibrium, and strong correlations with astrophysical $\gamma$-ray emission. At present, weak lensing surveys of several hundred square degrees allow studies of the IGRB to probe slightly above the thermal annihilation cross section \citep{1404.5503,1607.02187,1611.03554}. A simple forecast for LSST can be made by scaling the covariance matrix of the correlation estimator by the sky coverage. This shows that a combination of LSST lensing maps and all-sky Fermi-LAT data will reach a sensitivity where it is possible to \textit{detect} at $3\sigma$ WIMP annihilation to $b\bar{b}$ at the thermal cross cross section for dark matter particle masses up to 100 GeV \citep{1404.5503}.   

%Combined cross correlations with other baryonic and gravitational tracers, e.g., galaxies and galaxy clusters \citep[\eg][]{1506.01030,Lisanti:2018,1709.01940}, will provide a better handle on the astrophysical contributors, thereby further improving sensitivity to the dark matter contribution. 
% Other references
% https://arxiv.org/abs/1411.4651


From the gamma-ray side, improvements in the mapping of the cross-correlation with galaxies and cosmic shear are expected with the foreseen new generation gamma-ray instruments AMEGO\footnote{\url{https://asd.gsfc.nasa.gov/amego/index.html}} and eASTROGAM\footnote{\url{http://eastrogam.iaps.inaf.it/}} \citep{1711.01265}.
At the present, the main limitation in detecting the cross-correlation at GeV and sub-GeV energies is the angular resolution, rather than the available statistics.  
eASTROGAM, in particular, will have in the energy range \mbox{100 MeV-1 GeV} an angular resolution 5-6 times better than Fermi-LAT~\citep{1711.01265}. This will translate in harmonic space into a multipole reach 5-6 times larger than presently achievable, and as a consequence stronger constraints from the cross-correlation.
Precise measurements of the cross-correlation at sub-GeVs will further improve the ability to separate the astrophysical IGRB sources from the DM signal, increasing the sensitivity to the latter. 


\subsection{Axion-like particle emission from supernovae \Contact{Manuel}}
\Contributors{Manuel, ...}

Axion-like particles (ALPs) might be produced during core-collapse supernova explosions through the conversion of thermal photons in the electro-static fields of protons and ions, i.e., through the Primakoff effect \citep{1996slfp.book.....R}.  
Similar to neutrinos, ALPs would quickly escape the core and, if they are sufficiently light ($m_\phi \lesssim 10^{-9}\,$eV), they could convert into $\gamma$~rays in the magnetic field of the Milky Way and/or the host galaxy of the core-collapse supernova. 
The resulting $\gamma$~rays would arrive in temporal coincidence with the neutrinos in a burst lasting tens of seconds with a 
thermal spectrum peaking 60\MeV, depending on the mass of the progenitor \citep{2015JCAP...02..006P}.
The non-observation of a $\gamma$-ray burst from the SN1987A, which occurred in the Large Magellanic Cloud, has been used to derive stringent constraints on the photon-ALP coupling $g_{\phi\gamma}<5.3\times10^{-12}\GeV^{-1}$ for $m_\phi < 4.4\times10^{10}\eV$ \citep{1996PhLB..383..439B, 1996PhRvL..77.2372G,2015JCAP...02..006P}.
In the case of a core-collapse supernova within the Milky Way, the \textit{Fermi} LAT could improve these limits by more than an order of magnitude \citep{2017PhRvL.118a1103M}. 
However, with a Galactic supernova rate of $\roughly 3$ per century \citep[e.g.,][]{2013ApJ...778..164A}, and the LAT field of view of 20\% of the sky, the chance to observe at least one such event in the next five years is $\sim 0.03 \cdot 0.2 \cdot 5 = 0.03$ assuming that the occurrence of supernovae is a Poisson process. This estimate is still optimistic since the supernova rate is calculated for the entire Galaxy, which is not inside the field of view at any given moment.\footnote{If the supernova is sufficiently close-by or the photon-ALP coupling is close to current limits, a signal could be detected with the BGO detectors (senisitive up to 40\,MeV) of the \emph{Fermi} Gamma-ray Burst Monitor, which observes the entire unocculted sky.}
Increasing the search volume to extragalactic supernova is the obvious way to overcome this low rate. 
However, for core collapse SN beyond the LMC and SMC, current-generation neutrino detectors lack the sensitivity to detect a signal \citep[e.g.,][]{2011PhRvD..83l3008K}, and hence no precise time stamp will be provided for ALP-induced $\gamma$-ray emission, however, well-sampled optical light curves can be used to estimate the explosion time on the time scale of hours \citep{2010APh....33...19C}. 
LSST will detect a plethora of core-collapse supernova light curves \citep{Lien:2009}. 
Estimates for the delay between the core collapse and the shock breakout range from minutes for massive Wolf-Rayet stars (supernova of type Ib/c) to days for red supergiants (type II supernovae) \citep{2013ApJ...778...81K}. 
Thus, type Ib/c supernova caught early after their shock breakout and with subsequently well sampled light curves are a prime target for the search of an ALP-induced $\gamma$-ray burst. 

Since the $\gamma$-ray flux scales as $g_{\phi\gamma}^4 / d^2$, where $d$ is the luminosity distance, the sensitivity for $g_{\phi\gamma}$ scales as $\sqrt{d}$. 
Limits of the order of $g_{\phi\gamma} \lesssim 2\times10^{-12}\,\mathrm{GeV}^{-1}$ should be possible for a single supernova in M31 ($d=778$\,kpc) \citep{2017PhRvL.118a1103M}. 
If one allows these limits to degrade by a factor of 10, constraints better than the ones from CAST should still be possible for $d\lesssim 80\,$Mpc ($z \lesssim 0.02$) for a single supernova assuming that the time of the core collapse is known. 
LSST is expected to detect tens of type Ib/c core-collapse supernovae every year with redshifts $z \lesssim 0.02$ \citep{Goldstein:2018} and could conduct such searches in conjunction with the \textit{Fermi} satellite or future $\gamma$-ray satellites like AMEGO, eASTROGAM, or Gamma-400 \citep[\eg,][]{2017ICRC...35..910C,1502.02976} 
%(EN Gamma-400?). \ADW{I think the high-energy field of view of Gamma-400 is probably too small to make this compelling, but the GRB system could be interesting up to 15 MeV.}
%LSST is expected to detect more than $10^4$  core-collapse supernovae every year with redshifts $z \lesssim 0.1$ \citep{2009JCAP...01..047L} and is therefore an excellent instrument to conduct such searches in conjunction with the \textit{Fermi} LAT or future $\gamma$-ray satellites like AMEGO or eASTROGAM \citep[see, e.g.,][]{2017ICRC...35..910C}. 
A stacking analysis of the $\gamma$-ray data with explosions times estimated from LSST light curves provides the exciting possibility to photon-ALP couplings in the regime where ALPs could make up the entirety of the dark matter. 


%\subsubsection{Transient Objects \Contact{???}}
%\Contributors{Renee?, Chanda, Esra, Manuel M., ... }

%From Chanda PW and Esra B: 
%\ADW{This is an X-ray cross-correlation paragraph written by Chanda and Esra. I think we would need a more quantitative estimate to include it in the discussion. Out to what distance would these KN be detected? How good would the timing need to be? etc.}
%The wide-fast-deep LSST survey will be uniquely positioned to detect up to 10 kilonovae per year (LSST Science Book 2009, Rosswog et al. 2016). These detections can be combined with timing data from those sources from ongoing X-ray experiments such as NICER (and proposed experiments such as STROBE-X and IXPE) can be used to search for proposed axion cooling tracks (Keller and Sedrakian 2013, Sedrakian 2016), leading to constraints on the axion parameter space. 

%LIGO/kilonova connection for optical follow-up?
%\TT{I could write something here re LSST followup}
% ADW: Probably a better estimate of the kilonova rate (3 - 6 per year) 
% ADW: Rough estimate for the kilonova rate (75 - 2200 like GW170817 within z ~ 0.25) from Section 7.1 of 1809.04295


% Direct Detection
\section{Direct Detection }
\ADW{Some introduction and context is needed here. Much of Lina's intro could be moved here.}

\subsection{Baryon Scattering \Contact{Vera}}
\Contributors{Vera, Kim, Lina N.}

The most sensitive low-energy searches for dark matter are looking to directly detect collisions of dark matter particles from the local galactic halo in underground detectors \citep{2013arXiv1310.8327C}. 
They have unprecedented sensitivity to WIMPs with masses well above a GeV, but the current generation of experiments is largely insensitive to lighter particles, for kinematic reasons. 
New technologies are necessary to open up sub-GeV models of dark matter to detailed exploration \citep{Battaglieri:2017aum}. 
Moreover, due to the extensive shielding of their targets, direct detection experiments have a ceiling on their sensitivity to large cross sections. 
The portion of dark matter parameter space excluded by current null results is shown in \figref{dd}. 
\begin{figure}
\centering
\includegraphics[width=0.6\columnwidth]{figures/planck_dd.png}
\caption{Currently-excluded regions of dark matter parameter space (mass versus cross section for scattering with protons through a velocity-independent spin-independent contact interaction) are shown as shaded regions. Gray region is excluded by various direct detection null results \citep{2018PhRvD..97l3013K} and red is excluded by CMB measurements \citep{Gluscevic:2017ywp}. We note that there are other limits in the same parameter space, but we choose to compare only these two, for illustration of complementarity between cosmological and low-energy laboratory searches.}
\label{fig:dd}
\end{figure}

Current null results from targeted laboratory searches motivate broad scans of parameter space that is inaccessible to underground experiments. Cosmological and astrophysical observables provide such a complementary search strategy. In particular, they are sensitive to scattering of sub-GeV particles with baryons at any point in cosmic history. Furthermore, there is no upper boundary on the interaction cross section they can probe. Finally, they are not subject to the uncertainty on local astrophysical properties of dark matter particles (their phase-space distribution), which affects the inferred limits on the particle properties of dark matter. 

If dark matter particles scatter with baryons, they transfer momentum between the two cosmological fluids, affecting density fluctuations and suppressing power at small scales; the power suppression can be captured by a variety of observables. The current limits come from the CMB \citep{Gluscevic:2017ywp}, cosmic-ray \citep{Cappiello:2018hsu}, and Lyman-$\alpha$ forest measurements \cite{Xu:2018efh}. For illustration, Figure \ref{fig:dd} compares currently excluded regions of dark matter parameter space, from analyses of Planck data, and from null results of various direct-detection searches.\footnote{We caution the reader that this is not a comprehensive list of current upper limits, but only serves to illustrate complementarity of cosmological and direct detection probes.} LSST will deliver state-of-the-art measurements of observables that trace matter fluctuations on a range of smaller scales, extending the sensitivity of astrophysical and cosmological searches far beyond the reach of Planck.

\subsection{Local Dark Matter Velocity Distribution \Contact{Lina}}
\Contributors{Lina N.}

%One way to detect dark matter (DM) is a process called direct detection, where DM particles scatter off heavy nuclei, emitting scintillation/ionization light that provide a direct signal of DM \citep{Goodman:1984dc}. 

The signal strength of dark matter (DM) scattering in direct detection experiments depends on both the local DM density and the DM velocity distribution. In this section we focus on the DM velocity distribution.

The differential rate with respect to the recoil energy $dR/dQ$ depends on the integral of the DM velocity distribution, $f(v)$, as
\begin{equation}
    \frac{dR}{dQ} \propto \int_{v_{\rm{min}}}^{v_{\rm{esc}}} \frac{f(v)}{v} dv, 
\end{equation}
where $v_{\rm{min}} = \sqrt{Q m_N/ (2 \mu^2)}$, with $Q$ the recoil energy, $m_N$ the mass of the nucleus against which DM is scattering, and $\mu = (m_N m_\chi / (m_N + m_\chi))$ the reduced mass of the nucleus $m_N$ and the DM mass $m_\chi$.

A novel method has recently been proposed to use the stars as tracers for the DM velocity \citep{Herzog-Arbeitman:2017fte,Necib:2018b}. These papers suggest that since accreted DM and stars have a comment origin, and are both collisionless, accreted stars are able to trace the velocity distribution of DM. This correlation holds for both the relaxed component of the DM, traced by older metal poor stars, and DM velocity substructure called debris flow traced by less metal poor stars from more recent mergers \citep{Lisanti:2011as,Kuhlen:2012fz,Lisanti:2014dva}. 

This method has already been applied on RAVE-TGAS data \citep{Herzog-Arbeitman:2017zbm}, and the second data release of Gaia in \cite{necib2018}. It has been found that the relaxed component of the DM although isotropic, has a mean speed lower than that of the assumed Maxwell Boltzmann distribution, reducing current limits by direct detection experiments \citep{Aprile:2018dbl}.

Another interesting aspect is the ability to reconstruct of more recent mergers. Using the second data release of Gaia, a new merger called the Gaia Sausage or the Gaia Enceleadus \citep{2018MNRAS.477.1472B,2018Natur.563...85H} has been found. Using the correlation observed in simulations, \cite{necib2018} extracted the new velocity distribution of DM brought in by the same merger, and studied its implications in current direct detection experiments. 

In order to do obtain the full empirical distribution of DM, one needs the 3-d velocities of the stars in the local neighborhood. Gaia provides proper motion and parallaxes for stars down to 20th magnitude. LSST will be able to extend this dataset to fainter stars, giving us a more accurate measurement of proper motions of stars in the solar neighborhood and beyond.

 Using proper motions of stars from LSST, coupled with radial velocity measurements from future telescopes like MSE, we will be able to obtain the most accurate 3-d velocity measurements of the local stars, and subsequently use this information to obtain a full empirical measurement of DM. Such detailed analysis will unveil new structures much smaller than the Gaia Sausage, but with equal importance in DM direct detection if it passes by the Solar neighborhood.



\begin{comment}
\bsection{Particle Accelerators \Contact{???}}
\Contributors{...}
\ADW{If no one steps up here, we should remove this section.}
\end{comment}

\begin{comment}
\section{Cosmic Microwave Background \Contributors{Vera,Kim,Francis-Yan,Cora}}

Consider removing this section for now. Points to potentially consider in the future:
\begin{itemize}
    \item Cross-correlation science.
    \item CMB limits on dark matter-baryon and dark matter-dark radiation interactions are robust, but probe large scales. Smaller-scale probes (dwarfs etc) have a better discovery potential.
    \item In a meta-analysis where many observables are fit with all parameters of interest, we'd want to vary all LCDM parameters. We'd use CMB data for this, in a joint likelihood analysis. But we're not there yet.
\end{itemize}
\end{comment}

\begin{comment}
\subsection{Gravitational Waves \Contact{???}}
\Contributors{...}
\ADW{If no one steps up here, we should remove this section.}
\end{comment}

\begin{comment}
\subsection{Radio \Contact{Will?}}
\Contributors{Will, Tony, ...}
\label{sec:radio} 

% Radio Relics: SKA, etc.
\WAD{Need to flesh out this section.}
\begin{itemize}
    \item Radio relics are...
    \item Because radio relics are only associated with post-major-cluster-mergers they can be a convenient way of identifying post-mergers.
    \item Given the spectral index ($\alpha\sim-2$) radio relics are easier to detect at frequencies of order 100 MHz.
    \item In the era of LSST, SKA will provide the best means of detecting mergers through their radio relics
    \item According to \cite{2012MNRAS.420.2006N} LOFAR will detect $\sim2500$ radio relics. Given that SKA will be XXX times more sensitive it can be expected to detect YYY radio relics in the southern hemisphere. \WAD{Need to scale the sensitivity to estimate the number of relics that will be detected.}
\end{itemize}
\end{comment}

\begin{comment}
\subsection{X-Ray \Contact{???}}
\Contributors{Will, Tony, Esra, ...}
\label{sec:xray} 
% LSST plus X-ray facilities: etc.
{\color{red} Esra will write a section on eROSITA and Athena complementary X-ray observations here. This section can also be moved to Galaxy clusters section}

\begin{itemize}
    \item Given Athena's all sky coverage and resolution it should provide the necessary course spatial and spectral characterization of gas and merging clusters.
\end{itemize}
\end{comment}

\begin{comment}
\subsection{Targeted Follow Up \Contact{???}}
\Contributors{Nate, Will, Michael...}
\subsubsection{Transients}
\end{comment}

%https://www.overleaf.com/10858269czpqjyyjnbyx


% ----------------------------------------------------------------------
\section{Discovery Potential \Contact{Francis-Yan}}
\Contributors{Tony, Vera, Cora, Francis-Yan, Alex, Keith ...}
\label{sec:discovery}

Cosmology has a long history of probing the fundamental properties of dark matter.
Neutrinos were long considered a viable dark matter candidate \citep[\eg,][]{Kolb:1988}, and it was through precise cosmological measurements that it became clear that the universe contains multiple invisible components.
\KB{Alternatively, the rest of this paragraph could go into a footnote.}
For a case study of the interplay between particle physics experiments and astrophysics observations, consider the 30~eV neutrino dark matter candidate.
\citet{Lyubimov:1980un} reported the discovery of a non-zero neutrino rest mass in the range $14 \unit{eV} < m_{\nu} < 46 \unit{eV}$ which was subsequently tested by several other tritium $\beta$-decay experiments over the next decade.
% See equation 19 of  https://arxiv.org/pdf/1212.6154.pdf
Neutrinos with this mass would provide a significant fraction of the critical energy density needed to close the universe, and would be relativistic at the time of decoupling (i.e., hot dark matter).
During the same period, the first stellar velocity dispersion results for dwarf spheroidal galaxies showed that these galaxies are highly dark matter dominated.
The inferred dark matter density within the compact region of stellar population of the dwarfs was used to place lower limits on the neutrino rest mass that were incompatible with the 30~eV neutrino dark matter candidate \citep{Aaronson:1983,Gerhard:1992}.
%In 1980, a $\beta$-decay experiment at ITEP reported the discovery of non-zero neutrino mass in the range  \citep{Lyubimov:1980un}.
%As a specific case study, the tritium $\beta$-decay experiment at ITEP reported a neutrino mass of $\approx30$~eV for much of the 1980s \citep{Lyubimov:1980un}. Neutrinos at this mass would provide a significant fraction of the critical energy density. However, the stellar velocity dispersion of dwarf spheroidal galaxies orbiting the Milky Way, e.g., Draco \citep{Aaronson:1983}, already  }

Similar stories can be told of heavy leptons \citep{Gunn:1978}, \FIXME{and other dark matter candidates}, all of which were excluded by cosmological measurements.
Cosmology has continually proven that it is impossible to separate the \emph{macroscopic distribution} of dark matter from the \emph{microscopic physics} governing dark matter.

% See discussion on "Outline for discovery section" 19 December 2018
% https://docs.google.com/document/d/1RaxmjYjRYaAAY-4zqg-v4bEtUa8rcgiVgnJlFJQZifc/edit#

Through much of this work, we have expressed sensitivity to dark matter microphysics in terms of upper limits in the case of non-detection of deviations from the baseline CDM paradigm.
In this Section, we consider two potential astrophysical discovery scenarios for non-minimal dark matter properties that could be realized in the LSST era.
In each scenario, a critical question is whether the systematic uncertainties associated with conventional astrophysical processes can be controlled at a level that would sufficiently compelling to guide non-gravitational dark matter searches with collider, direct, and indirect experiments.

%{\bf Compact Object Discovery}
\subsection{Compact Object Discovery}
% ADW: Need some help from Will et al.

While current constraints from the dynamics of stellar systems make it unlikely that all of dark matter is composed of compact objects, it is nearly certain that LSST will measure the mass spectrum of Galactic black holes (\figref{macho_discovery}).
The discovery of some excess component to the black hole population attributable to dark matter will require a fit of the underlying population of stellar remnants.
If such a population is discovered, it will be possible to measure not only the fraction of dark matter in compact objects, but the compact object mass spectrum, which will in turn set constraints on the spectrum of perturbations during and after inflation.
Knowing that some fraction of the dark matter exists as PBHs will cause a massive re-interpretation of limits from direct and indirect searches.
Preferred regions of WIMP parameter space that are excluded under the assumption that all the dark matter is particles will be reopened.
Ironically, the outlook for the WIMP may be stronger in a universe where PBHs make up some fraction of the dark matter density.

\begin{figure}[t]
\centering
\includegraphics[width=0.6\columnwidth]{nevents_vs_t.pdf}
\caption{
    \label{fig:macho_discovery}
    The expected number of $2\theta_\mathrm{E}$ microlensing events in a $10\times10\,\mathrm{deg}^2$ bulge field (blue histogram), with the fraction due to black holes resulting from stellar evolution shown by the red histogram (for these events our detection efficiency is $\sim0.1\%$ at low-$t_\mathrm{E}$ and $\sim1\%$ at high-$t_\mathrm{E}$).
    \WAD{Need to come up with better estimates for the number of expected event, i.e., how to scale the y-axis for LSST, especially considering different regions of the sky and potentially different cadences in each of these regions. Could just present a conservative estimate for $10\times10\,\mathrm{deg}^2$ bulge field.}
    Other colored histograms show the event rate assuming different fractions of dark matter composed of LIGO-mass black holes (an order of magnitude more massive than the stellar remnant population).
    The gray vertical dashed lines show the published sensitivity ranges of the MOA and OGLE microlensing surveys (insensitive to the high-mass/long-$t_\mathrm{E}$ tail).
    The shaded green regions shows the sensitivity range for LSST.
    %This includes the planetary microlensing sweet spot (purple), which will be probed via alerted follow up.
    It also importantly includes the peak of the distribution (yellow dot), and enables us to accurately calibrate the slope (yellow dot-dashed line), which is necessary to provide accurate IMBH constraints.
    \Contributors{Jessica Lu, Casey Lam, Michael M., Will D.}
    }
\end{figure}

%{\bf SIDM-WDM Discovery}
\subsection{WDM/SIDM Discovery}

We now turn to a more challenging scenario in which dark matter possesses a particle mass or self-interaction cross section that would partially account for observed small-scale structure anomalies.
There are currently several hints of non-minimal dark matter particle properties arising from comparisons between theoretical predictions and observed galaxy populations at the dwarf galaxy scale, i.e., distances below $1 \Mpc$ and mass scales below $10^{11} \Msun$ \citep[reviewed by][]{BuckleyPeter:2017,Bullock:2017xww}.
However, the interpretation of these discrepancies in terms of dark matter microphysics has been hindered to a large extent by our present uncertainty in the mapping between visible stellar populations and invisible dark matter halos, which involves both the physics of galaxy formation as well as the connection between observable and intrinsic galaxy properties (see \secref{smallest_galaxies} and \secref{halo_profile_group}).
In a regime where we are already limited by systematic uncertainty, it is therefore quite reasonable to ask how the increased statistical power of LSST will help to resolve our current small-scale structure quandary.

We argue here that the decisive advantage of LSST is the opportunity to combine an ensemble of astrophysical dark matter probes that offer complementary perspectives on dark matter halo abundances and profile shapes, and which are affected by different sources of systematic uncertainty.
For the purpose of illustration, we outline a possible ``roadmap to discovery'' for a dark matter model that produces as a cutoff in the matter power spectrum and a suppression of the central dark matter profile just below the current sensitivity limit---i.e., $M_{hm} = 10^{8.5}$.

The first indication of a problem might come shortly after the first public data release of LSST survey data when automated searches for additional Milky Way satellites reveal only a handful of new candidate ultra-faint galaxies. 
Using the framework described in \secref{smallest_galaxies}, these observations could be combined to derive contours on the parameter space of WDM mass vs SIDM cross-section.
As shown in \figref{sidm_wdm_discovery}, there is some degeneracy between WDM particle mass and SIDM cross section.
This degeneracy is reminiscent of similar cosmological contours---e.g., that between $\Omega_m$ and $\sigma_8$.

Using the same LSST data release, the combined depth and sky coverage of LSST will enable the study of dwarf galaxy satellite populations around several other hosts out to several Mpc, as well as the ``field'' population of isolated dwarf galaxies.
By generating a statistical sample of low-luminosity galaxies in a wide variety of environments, LSST will provide a wealth of input data to theoreticians developing galaxy formation simulations.
\FIXME{These observations would help to break degeneracies of reionization physics, stellar feedback internal to the dwarf, etc.}
\FIXME{Targets for spectroscopic follow-up}

In parallel with the study of visible galaxy populations, LSST is expected to reveal many new stellar streams and gravitational lens systems which would provide access to dark matter halos below the mass threshold of galaxy formation \FIXME{add secrefs}.
In our scenario with a halo mass cutoff at the ultra-faint galaxy scale, the search for stream gaps and lensing anomalies would be particularly well motivated.
We could expect a period of several years to collect and analyze follow-up observations of the most favorable streams and lens systems.

{\bf Semi-quantitative extraction of particle properties}

%As a quantitative aspect, we specifically constrain ourselves to the scenario of a dark matter model that manifests itself as a cutoff in the matter power spectrum and a suppression of the central dark matter profile. 

%We assume a maximal model where that cutoff occurs just below the current sensitivity limit---i.e., $M_{hm} = 10^{8.5}$.
%This model would manifest itself as an absence of low-mass dark matter halos detectable through measurements of Milky Way satellites, the lack of gaps in stellar streams, and an underabundance of perturbations in strongly lensed systems.

This is the contour figure from Francis-Yan.

{\bf Roadmap to measurement}

In the regime where a ``discovery'' occurs, it is important to examine what ``measurement'' looks like in the context of astrophysical probes with LSST.

%Using the framework described in \secref{smallest_galaxies}, these observations could be combined to derive contours on the parameter space of WDM mass vs SIDM cross-section.
%As shown in \figref{sidm_wdm_discovery}, there is some degeneracy between WDM particle mass and SIDM cross section.
%This degeneracy is reminiscent of similar cosmological contours---e.g., that between $\Omega_m$ and $\sigma_8$.

Similar to cosmological constraints on dark energy, degeneracies in the fundamental properties of dark matter can be broken through the combination of multiple probes.
In the simple WDM case of our example, combining measurements from galaxy cluster profiles will help to constrain the SIDM cross section at a scale that is impossible to probe with dwarf galaxies. 

In the LSST era, dark matter science needs to move from a disperate set of communities of 

\begin{comment}
%ADW: This comment block contains notes from our discussion at Livermore.
\begin{enumerate}
    \item We may start by alluding to the brief era where we thought that neutrinos could have been dark matter before it was proven that they would free stream
    \item This was orthogonal to DOE's mantra: "we don't care where the dark matter is, but what it is"
\end{enumerate}

\begin{enumerate}
    \item Individual analyses can proceed in parallel, just need to be able to create a likelihood
    \item This is the case with the CMB analyses
    \item The dwarf galaxy analysis could be used as a demonstration of one of these likelihoods
\end{enumerate}

\begin{enumerate}
    \item Do we need a specific example of a discovery?
    \item If there was a cutoff at a mass of $10^8 \Msun$, what would it look like?
    \item Can we make an example of the WDM-SIDM figure a closed contour plot?
    \item Make the ``optimistic'' assumption of detecting $10^5 \Msun$ halos.
    \item Assume a cutoff in the mass function at $10^7 \Msun$ (probably not sensitive currently)
    \item Once we have the detection, start expanding to a .
    \item Uncertainties could be wrong, but they will be narrowed down.
    \item In the future we can also include correlations with other parameters
\end{enumerate}

\begin{enumerate}
    \item It is likely premature to assume that are current tools are the only tools that we will have after LSST discoveries
    \item There will likely be new probes and new ways forward.
    \item The dimensionality could be much higher.
    \item Correlation between dark energy and dark matter might explore new models, hitherto unconsidered
    \item Discovery potential for compact object dark matter?
    \item Discovery potential emergent dark matter
\end{enumerate}
\end{comment}
% ----------------------------------------------------------------------

%\section{Discussion \Contributors{..}}
%\label{sec:discussion}

%\subsection{systematics?}
%For example, baryonic effect.

% ----------------------------------------------------------------------

\section{Conclusion \Contact{Alex}}
\Contributors{Alex, Yao, ...}
\label{sec:conclusion}



% ----------------------------------------------------------------------

\subsection*{Acknowledgments}

%%% Here is where you should add your specific acknowledgments, remembering that some standard thanks will be added via the \code{desc-tex/ack/*.tex} and \code{contributions.tex} files.

%This paper has undergone internal review in the LSST Dark Energy Science Collaboration. % REQUIRED if true

%\input{contributions} % Standard papers only: author contribution statements. For examples, see http://blogs.nature.com/nautilus/2007/11/post_12.html

% This work used TBD kindly provided by Not-A-DESC Member and benefitted from comments by Another Non-DESC person.

% Standard papers only: A.B.C. acknowledges support from grant 1234 from ...

%\input{desc-tex/ack/standard} % also available: key standard_short

% This work used some telescope which is operated/funded by some agency or consortium or foundation ...

% We acknowledge the use of An-External-Tool-like-NED-or-ADS.

%{\it Facilities:} \facility{LSST}

% Include both collaboration papers and external citations:
\bibliographystyle{yahapj}
\bibliography{main}

\end{document}

% ======================================================================
