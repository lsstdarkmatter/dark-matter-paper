\RequirePackage{docswitch}
% \flag is set by the user, through the makefile:
%    make note
%    make apj
% etc.
\setjournal{\flag}

\documentclass[\docopts]{\docclass}

% You could also define the document class directly
%\documentclass[]{emulateapj}

% Custom commands from LSST DESC, see texmf/styles/lsstdesc_macros.sty
\usepackage{lsstdesc_macros}

\usepackage{graphicx}
\graphicspath{{./}{./figures/}}
\bibliographystyle{apj}

% Add your own macros here:



% ======================================================================

\begin{document}

\title{Astrophysical Probes of Dark Matter with LSST}

\maketitlepre

\begin{abstract}

This document presents an overview of astrophysical probes of the fundamental nature of dark matter that can be explored with the Large Synoptic Survey Telescope.

\end{abstract}

% Keywords are ignored in the LSST DESC Note style:
\dockeys{}

\maketitlepost

% ----------------------------------------------------------------------
% 
\tableofcontents

\section{Introduction}
\label{sec:intro}

  The fundamental nature of dark matter, which constitutes 85\% of the matter density and 26\% of the energy density of the Universe, remains an outstanding problem in physics.
Over the past several decades, the search for dark matter has proceeded along several avenues.
Specifically, the search for new particles in experiments at colliders, e.g., ATLAS and CMS at the Large Hadron Collider
Direct detection experiments attempt to directly detect very rare interactions between dark matter and standard model particles.
Indirect dark matter searches seek to detect the secondary products from the annihilation or decay of dark matter in situ.
To date, the only robust, positive signature of the existence of dark matter is gravitational.

Atrophysical probes, ...., provide a complementary means to study the 


% ----------------------------------------------------------------------

\section{Methods}
\label{sec:methods}


% ----------------------------------------------------------------------

\section{Results}
\label{sec:results}



% ----------------------------------------------------------------------

\section{Discussion}
\label{sec:discussion}



% ----------------------------------------------------------------------

\section{Conclusion}
\label{sec:conclusion}



% ----------------------------------------------------------------------

\subsection*{Acknowledgments}

%%% Here is where you should add your specific acknowledgments, remembering that some standard thanks will be added via the \code{desc-tex/ack/*.tex} and \code{contributions.tex} files.

%This paper has undergone internal review in the LSST Dark Energy Science Collaboration. % REQUIRED if true

\input{contributions} % Standard papers only: author contribution statements. For examples, see http://blogs.nature.com/nautilus/2007/11/post_12.html

% This work used TBD kindly provided by Not-A-DESC Member and benefitted from comments by Another Non-DESC person.

% Standard papers only: A.B.C. acknowledges support from grant 1234 from ...

\input{desc-tex/ack/standard} % also available: key standard_short

% This work used some telescope which is operated/funded by some agency or consortium or foundation ...

% We acknowledge the use of An-External-Tool-like-NED-or-ADS.

%{\it Facilities:} \facility{LSST}

% Include both collaboration papers and external citations:
\bibliography{main,lsstdesc}

\end{document}

% ======================================================================
