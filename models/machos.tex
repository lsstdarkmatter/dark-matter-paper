\section{Compact Objects \Contact{Simeon}}
\label{sec:machos}
\Contributors{Simeon Bird, Juan Garc\'ia-Bellido, George Chapline, William A.\ Dawson, Nathan Golovich, Michael Medford}

Compact objects, particularly black holes, represent one of the oldest and most venerable models of dark matter. 
%Primordial black holes could originate from small-scale density fluctuations during the era of inflation. 
\NEW{Primordial black holes could form at early times from the direct gravitational collapse of large density perturbations that originated during inflation.}
The same fluctuations that lay down the seeds of galaxies, if boosted on small scales, can lead to some small areas having a Schwarzschild mass within the horizon, which spontaneously collapse to form black holes \citep{Carr:1974nx,Meszaros:1974,1975Natur.253..251C,Bellido:1996,2016PhRvD..94h3504C}. 
\NEW{Alternatively, some particle dark matter models may allow dark matter cooling and collapse, providing another mechanism for black hole formation \citep[\eg][]{1705.10341,1707.03419,1707.03829,1802.08206,1812.07000}.} 
%Because these black holes do not accrete or radiate strongly (at the time of formation there is no gas to form an accretion disc), they are a natural candidate for dark matter 
\NEW{Such black holes are unlikely to radiate strongly enough from accretion to leave a detectable signature and are thus a natural candidate for dark matter}. 
The abundance of compact objects tests dark matter through a purely gravitational channel and is thus sensitive to dark matter models that cannot be probed in the laboratory.}

Compact object dark matter is fundamentally different from particle models; primordial black holes cannot be studied in an accelerator and can only be detected through their gravitational force. Current constraints suggest that primordial black holes do not make up all of dark matter \citep[\eg][]{Sasaki:2018}. However, these constraints may be evaded if \NEW{primordial black holes} are spatially clustered \citep{Clesse:2015,Clesse:2017}. Moreover, primordial black holes are one possible source of the merging $30 \Msun$ black holes recently detected by LIGO \citep{Bird:2016,Clesse:2016}. This possibility has rekindled interest in these objects, both as a source of dark matter and in their own right.

Limits on the abundance and mass range of primordial black holes are wholly observational. The black hole mass is set by the mass enclosed within the horizon at the time of black hole collapse and thus ranges between $10^{-18} \Msun$ ($10^{15}\g$), below which the black hole would evaporate, and $10^9 \Msun$ ($10^{42}\g$), above which structure formation, Big Bang Nucleosynthesis and the formation of the microwave background would be severely affected \citep{Sasaki:2018}. 
For stellar mass black holes, the gold standard for detecting compact objects is microlensing. Current microlensing constraints set limits on the black hole abundance at the level of $10\%$ for black holes $0.01 - 10 \Msun$ \citep[however, see][]{Calcino:2018}. LSST will revolutionize the astrometric microlensing technique,  constraining the abundance of primordial black holes to a level of $10^{-4}$ of the dark matter over a wide range of masses (\secref{compact_objects}).

As primordial black holes form directly from the primordial density fluctuations, a measurement of their abundance would directly constrain the amplitude of density fluctuations \citep{Carr:1974nx, Meszaros:1974}. %1203.2681 
Although these constraints are several orders of magnitude weaker than, for example, those from the microwave background, they probe small scales between $k = 10^{7} - 10^{19}$ $h$/Mpc, much smaller than those measured by other current and future probes \citep{Bringmann:2012}. Because these scales are highly non-linear in the late-time universe, there is no other possible constraint; the information present at early times has been washed away by gravitational evolution. Primordial black holes are thus a probe of primordial density fluctuations in a range that is inaccessible to other techniques~\citep{Josan:2009,Bellido:2017,Bellido:2018}. These curvature fluctuations are imprinted on space-time hypersurfaces during inflation, at extremely high energies, beyond those currently accessible by terrestrial and cosmic accelerators. 
Our understanding of the universe at these high energies, of order $10^{15} \GeV$ and above, comes predominantly from extrapolations of known physics at the electroweak scale.
Measurements of the primordial density fluctuations via the abundance of primordial black holes would provide unique insights into physics at these ultra-high energies.

\NEW{In additon, dark matter interactions with the Standard Model may also generate new channels for black hole formation, by triggering collapse of astrophysical objects \citep[\eg][]{1989PhRvD..40.3221G,1004.0629,1012.2039,1405.1031,1804.06740}. LSST could be sensitive to transient events that could be triggered by these formation scenarios \citep[\eg][]{1706.00001} or as part of a muli-messanger campaign to measure sub-solar mass black hole mergers \citep[\eg][]{1808.04771,1808.04772}.}


Furthermore, it may be possible for LSST to constrain the existence of ultra-compact mini-halos using correlated microlensing signals \citep{erickcek2011,li2012}. These objects arise from initial overdensities that are too small to collapse into primordial black holes. These overdensities still collapse at high redshift to form low-mass halos; thus, since these objects form early and have few mergers \citep{Bringmann:2012,Delos:2018}, they have a high concentration and a steeper internal density profile than the standard Navarro-Frenk-White shape. In turn, this makes them easier to detect via lensing and harder to disrupt than standard CDM subhalos. Current constraints on these objects are highly model-dependent. In particular, they largely come from counting gamma-ray photons from astrophysical sources under the assumption of a WIMP dark matter annihilation cross-section. LSST will place new constraints on the existence of small halos via micro-lensing and thereby constrain the physics of the inflaton on scales of $k = 10 \textup{--} 10^7 h$/Mpc for the first time in a model-independent way.
