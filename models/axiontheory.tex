\section{Field Dark Matter \Contact{Chanda}}
\label{sec:axions}

While current observations of the matter power spectrum constrain the minimum mass of thermally produced dark matter, other mechanisms can  produce dark matter with significantly lower masses. The landscape of light dark matter candidates is vast, and in this section, we specifically focus on the class of axion-like particle (ALP) dark matter candidates.
ALP models span a wide range of viable parameter space (both in coupling strength and mass), and many of the observables described in this section can be generically applied to a broader class of light scalar particles.

The ALP paradigm was inspired by the QCD axion, which arises as a by-product of the most successful solution to the Strong CP Problem in the Standard Model \citep{PecceiQuinn:1977}. 
The cosmological abundance of axions is set by the Peccei-Quinn symmetry breaking scale, $f_\phi$, with a value
\begin{equation}
\Omega_\phi\sim\left(\frac{f_\phi}{10^{11-12}\,{\rm GeV}}\right)^{7/6}.
\end{equation}
This expression may be altered due to the temperature-dependence of the axion mass and ignorance about whether the Peccei-Quinn symmetry breaks before or after inflation. 
QCD theory gives no {\it a priori} prediction for the axion mass; however, in the context of dark matter composed of QCD axions, the axion mass is considered to be $m_\phi< 10^{-3}$ eV ($10^{-39}$ kg). 
If the initial misalignment angle is order unity, this yields a QCD axion mass of $m_\phi \sim 10^{-5}$ eV ($10^{-41}$ kg).
%\ADW{Please check Chanda!}
The broader category of ALPs possess QCD-axion-like potentials producing light scalar particles that obey a shift symmetry ($\phi \rightarrow \phi + 2\pi n$), but do not obey the same coupling between particle mass and symmetry breaking scale. 
ALPs can be motivated by string theory, where there are many moduli with axion-like potentials, and can produce a range of astrophysical phenomenology.
%ALPs can be sufficiently different from the QCD axion so as to produce notably different astrophysical phenomenology. 
ALPs may be non-thermally produced in the early universe and survive as a cold dark matter population until today \citep[\eg][]{Arias:2012az}.

%Fuzzy dark matter: In this scenario, dark matter is made of ultra light scalar particles with mass around $10^{-22}~{\rm eV}$. With such as a small mass, the de Broglie wavelength of the dark matter particle is $\mathcal{O}({\rm kpc})$, comparable to galaxy sizes, and the quantum effect becomes relevant in structure formation. Fuzzy dark matter predicts a solitonic core in the halo center and its size is set by the de Broglie wavelength of the dark matter particle. On larger scales, the structure predicted in fuzzy dark matter is less clumpy and less abundant than its CDM counterpart due to the quantum interference effect.

Astrophysical observations place the only known lower bound on the mass of ALPs and other non-thermally produced ultra-light particles, commonly described as ``fuzzy'' dark matter \citep[FDM; \eg,][]{Hu:2000,Hui:2017}. 
The de Broglie wavelength of these particles is constrained to be smaller than the size of the smallest galaxy, $\mathcal{O}(1\kpc)$, setting a lower limit on particle mass at $m_\phi \gtrsim 10^{-22}$ eV. 
%The formation of halos with mass $M_h \lesssim 10^{10} (m_\phi / 10^{-22} \eV)^{-4/3} \Msun$ will be significantly suppressed, and halos with $M_h \lesssim 10^7 (m_\phi / 10^{-22} \eV)^{-3/2} \Msun$ should not form at all \citep{Hui:2017}.
In addition, FDM is predicted to produce solitonic cores in the centers of halos, which would measurably effect the velocity profiles of dark-matter dominated galaxies \citep{Robles:2012uy,Robles:2018fur,Schive:2014hza,Du:2016aik}. 
On larger scales, the structure predicted in FDM is less abundant than its CDM counterpart due to quantum interference effects.
This is similar to the case of WDM, and again dark matter properties can be constrained through mesurements of the least massive dark matter halos.
Combining Equation 8 of \citet[][]{1703.09126} with \eqnref{Mhm} in \secref{wdm}, we can express constraints on the minimum FDM mass, $m_\phi$, as a function of the half-mode halo mass, $M_{\rm hm}$:
\begin{equation}
M_{\rm hm} = 1.2 \times 10^{11} \left( \frac{m_\phi}{10^{-22}\eV} \right)^{-1.4} \Msun,
\end{equation}
or expressed in terms of $m_\phi$, 
\begin{equation}
m_\phi = 3.1 \times 10^{-21} \left( \frac{M_{\rm hm}}{10^{9}\Msun} \right)^{-0.71} \eV.
\end{equation}
LSST will constrain the minumum mass of light bosonic dark matter to be greater than $m_\phi \sim 10^{-20} \eV$ by precisely constraing the half-mode mass to be less than $M_{\rm hm} \sim 10^{8} \Msun$.

%ADW: I think the paragraphs below have slightly too much much detail, for this report. It would be good to talk about axion stars, galaxy caustics, and anomalous energy loss in stellar populations and SN.
%CPW, 10/29/2018: I think some of it should be reintroduced, but I have shortened and edited to suggest these are different scenarios under consideration.
%Sikivie & Yang 2009

There has been significant debate in the literature about the astrophysical phenomenology of the QCD axion and ALPs.
\citet{Sikivie:2009} noted that because the axion is a scalar with high abundance in the early universe (circa matter-radiation equality), the axion could potentially settle into a Bose-Einstein condensate (BEC) state, whereby all particles can be described using one coherent ground state wave function. 
Furthermore, \citet{Sikivie:2009} argue that during the radiation-dominated era, axions will rethermalize into BECs with a Hubble-scale correlation length.
This could produce significant observational implications, such as several-kpc-scale caustic structures observable in the stellar distributions of the Milky Way and other low-redshift galaxies \citep[\eg,][]{Natarajan:2006,0805.4556,Rindler-Daller:2013zxa}.

On the other hand, \citet{1412.5930} argue that a particle such as the QCD axion, which has an attractive self-interaction, in an attractive gravitational potential will not sustain Hubble-scale correlations.
Instead, \citet{1412.5930} predict that axions will form coherent clumps that have been called ``Axion stars'' or ``Bose stars'' \citep[\eg][]{Kolb:1993}.\footnote{For the QCD axion it would be more appropriate to call these ``axteroids'' (a term coined by Anna Watts) due to their mass of $\roughly 10^{-11} \Msun$ \citep{Tkachev:1991ka,Braaten:2018nag}.} 
Looking beyond the QCD axion, for some ALP models, compact BEC ``miniclusters'' could form and grow to $\gtrsim 1\Msun$, at which point they may be detectable by LSST through mergers with other compact objects \citep{1808.04746} or through microlensing \citep{1707.03310}.

%What is distinct about this proposal is not so much the idea that the axion might begin as a BEC -- this seems likely (although making this statement formal is an open problem) -- but rather once the particles experience perturbations, do they remain in a BEC state? %These are distinct from the spherical topology predicted for WIMPs \citep{Bertschinger:2006nq}. 
%These massive compact structures could be detected through collisions with stellar remnants, which could be constrained by the transient event rate measured by LSST \citep{1808.04746}.
%Still other theories predict that axions could collapse 
%\ADW{I don't know if I believe either of these detection scenarios.}

%This proposal has a distinct phenomenology and LSST observations providing insights into caustics around nearby galaxies, may help distinguish between the two.

%Complicating LSST's capacity to distinguish between models is the possibility of degeneracy between the SY model and other axion-like particle phenomenologies. Both of the aforementioned scenarios, which have kicked off significant debate and renewed interest in Bose-Einstein condensed axion phenomenology, focus on the QCD axion in a mass range of around $10^{-5}$ eV. There is an extensive literature regarding axion phenomenology beyond the QCD axion and this mass range, e.g. ultralight axions (ULA) and fuzzy dark matter (FDM). In ULA/FDM scenarios, the De Broglie wavelength of the particle is such that the coherent wave can be halo-scale, which may give distinct stellar density distributions, which LSST will measure, than the SY proposal or the clump scenario.

Additional astrophysical constraints on ALPs generally come from proposed couplings with photons and/or electrons. 
For example, the Lagrangian can be expressed as,
\begin{equation}
    \mathcal{L} = -\frac{1}{2} \partial_\mu\phi\partial^\mu\phi + \frac{1}{2}m_\phi^2 \phi^2 - \frac{1}{4}g_{\phi\gamma}F_{\mu\nu}\tilde{F}^{\mu\nu}\phi - g_{\phi e}\frac{\partial_\mu\phi}{2m_e}\bar{\psi}_e \gamma^\mu\gamma_5\psi_e,
\end{equation}
where $g_{\phi\gamma}$ is the photon-axion coupling, $g_{\phi e}$ is the axion-electron coupling, $F^{\mu\nu}$ is the electromagnetic field tensor (and $\tilde{F}$ its dual), and $\psi_e$ is the electron field \citep[\eg][]{1302.6283,Redondo:2013wwa}.\footnote{Additional couplings to nucleons are allowed, but are not relevant for the LSST observations discussed here.}
For sufficiently large couplings to photons or electrons, the ALP can manifest as an additional anomalous energy loss mechanism, transporting energy out of the interiors of stars \citep[\eg,][]{Raffelt:1990}.
This energy loss could affect the evolution of stars, for example altering the lifetimes of giant stars \citep{Ayala:2014,Viaux:2013hca,Viaux:2013lha} or the cooling rate of white dwarf stars \citep{Isern:2008}.
The precise photometry of LSST will provide sensitive measurements of stellar populations to search for deviations from the predictions of standard stellar evolutionary models.

\begin{comment}
\TT{We should also mention the dark photon}
\ADW{Would we want to include some discussion of soliton cores from fuzzy DM?}

LSST will be able to probe axion dark matter in the following ways:
\begin{itemize}
    \item Caustics around Nearby Galaxies \CPW{11.25 Dealt with, now?}
    \item Anomalous Cooling in Stellar Populations 
    \begin{itemize}
        \item White dwarf luminosity function
        \item Globular cluster
        \item Cepheids / blue loop?
    \end{itemize}
    \item Supernova Observations
    
\end{itemize}
\end{comment}



