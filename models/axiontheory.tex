\section{Field Dark Matter \Contact{Chanda}}
\label{sec:axions}
\Contributors{Chanda Prescod-Weinstein, Samuel D.\  McDermott, Oscar Straniero,  Maurizio Giannotti, Alex Drlica-Wagner, Manuel Meyer}

While current observations of the matter power spectrum constrain the minimum mass of thermally produced dark matter, other mechanisms can  produce dark matter with significantly lower masses. The landscape of light dark matter candidates is vast, and in this section, we specifically focus on the class of axion-like particle (ALP) dark matter candidates.
ALP models span a wide range of viable parameter space (both in coupling strength and mass), and many of the observables described in this section can be generically applied to a broader class of light scalar particles.

The ALP paradigm was inspired by the QCD axion, which arises as a by-product of the most successful solution to the Strong CP Problem in the Standard Model \citep{PecceiQuinn:1977}. 
The cosmological abundance of axions is set by the Peccei-Quinn symmetry breaking scale, $f_\phi$, with a value
\begin{equation}
\Omega_\phi\sim\left(\frac{f_\phi}{10^{11-12}\,{\rm GeV}}\right)^{7/6}.
\end{equation}
This expression may be altered due to the temperature-dependence of the axion mass and ignorance about whether the Peccei-Quinn symmetry breaks before or after inflation. 
QCD theory gives no {\it a priori} prediction for the axion mass; however, in the context of dark matter composed of QCD axions, the axion mass is considered to be $m_\phi< 10^{-3} \eV$. %($10^{-39}$ kg) 
If the initial misalignment angle is order unity, this yields a QCD axion mass of $m_\phi \sim 10^{-5} \eV$. %($10^{-41}$ kg).
The broader category of ALPs possess QCD-axion-like potentials producing light scalar particles that obey a shift symmetry ($\phi \rightarrow \phi + 2\pi n$), but do not obey the same coupling between particle mass and symmetry breaking scale. 
ALPs can be motivated by string theory, where there are many moduli with axion-like potentials, and can produce a range of astrophysical phenomenology.
%ALPs can be sufficiently different from the QCD axion so as to produce notably different astrophysical phenomenology. 
ALPs may be non-thermally produced in the early universe and survive as a cold dark matter population until today \citep[\eg][]{Arias:2012az}.



There has been significant debate in the literature about the astrophysical phenomenology of the QCD axion and ALPs.
\citet{Sikivie:2009} noted that because the axion is a scalar with high abundance in the early universe (circa matter-radiation equality), the axion could potentially settle into a Bose-Einstein condensate (BEC) state, whereby all particles can be described using one coherent ground state wave function. 
Furthermore, \citet{Sikivie:2009} argue that during the radiation-dominated era, axions will rethermalize into BECs with a Hubble-scale correlation length.
This could produce significant observational implications, such as several-kpc-scale caustic structures observable in the stellar distributions of the Milky Way and other low-redshift galaxies \citep[\eg,][]{Natarajan:2006,0805.4556,Rindler-Daller:2013zxa}.

On the other hand, \citet{1412.5930} argue that a particle such as the QCD axion, which has an attractive self-interaction, will not sustain Hubble-scale correlations in an attractive potential.
Instead, \citet{1412.5930} predict that axions will form coherent clumps that have been called ``Axion stars'' or ``Bose stars'' \citep[\eg][]{Kolb:1993}.\footnote{For the QCD axion it would be more appropriate to call these ``axteroids'' (a term coined by Anna Watts) due to their mass of $\roughly 10^{-11} \Msun$ \citep{Tkachev:1991ka,Braaten:2018nag}.} 
Looking beyond the QCD axion, some ALP models suggest that compact BEC ``miniclusters'' could form and grow to $\gtrsim 1\Msun$, at which point they may be detectable by LSST through mergers with other compact objects \citep{1808.04746} or through microlensing \citep{1707.03310}, as discussed in \secref{compact_objects}.

Additional astrophysical constraints on ALPs generally come from proposed couplings with photons and/or electrons. 
For example, the Lagrangian can be expressed as
\begin{equation}
    \mathcal{L} = -\frac{1}{2} \partial_\mu\phi\partial^\mu\phi + \frac{1}{2}m_\phi^2 \phi^2 - \frac{1}{4}g_{\phi\gamma}F_{\mu\nu}\tilde{F}^{\mu\nu}\phi - g_{\phi e}\frac{\partial_\mu\phi}{2m_e}\bar{\psi}_e \gamma^\mu\gamma_5\psi_e,
\end{equation}
where $g_{\phi\gamma}$ is the photon-axion coupling, $g_{\phi e}$ is the axion-electron coupling, $F^{\mu\nu}$ is the electromagnetic field tensor (and $\tilde{F}$ its dual), and $\psi_e$ is the electron field \citep[\eg][]{1302.6283,Redondo:2013wwa}.\footnote{Additional couplings to nucleons are allowed, but are not relevant for the LSST observations discussed here.}
For sufficiently large couplings to photons or electrons, the ALP can manifest as an additional anomalous energy loss mechanism, transporting energy out of the interiors of stars \citep[\eg,][]{Raffelt:1990}.
This energy loss could affect the evolution of stars, for example altering the lifetimes of giant stars \citep{Ayala:2014,Viaux:2013hca,Viaux:2013lha} or the cooling rate of white dwarf stars \citep{Isern:2008nt}.
The precise photometry of LSST will provide sensitive measurements of stellar populations to search for deviations from the predictions of standard stellar evolutionary models (\secref{cooling}).

Astrophysical observations place the only known lower bound on the mass of ALPs and other non-thermally produced ultra-light particles, commonly described as ``fuzzy'' dark matter \citep[FDM; \eg,][]{Hu:2000,Hui:2017}. 
The de Broglie wavelength of these particles is constrained to be smaller than the size of the smallest galaxy, $\mathcal{O}(1\kpc)$, setting a lower limit on particle mass at $m_\phi \gtrsim 10^{-21}\eV$ \citep{1703.04683}. 
In addition, FDM is predicted to produce solitonic cores in the centers of halos, which would measurably affect the velocity profiles of dark-matter dominated galaxies \citep{Robles:2012uy,1807.06018,Schive:2014hza,Du:2016aik}. 
Dark matter substructure is predicted to be less abundant in FDM than its CDM counterpart due to quantum interference effects.
This is similar to the case of WDM, and again dark matter properties can be constrained through mesurements of the least massive dark matter halos.
Combining Equation 8 of \citet[][]{1703.09126} with \eqnref{Mhm} in \secref{wdm}, we can express constraints on the minimum FDM mass, $m_\phi$, as a function of the half-mode halo mass, $M_{\rm hm}$:
\begin{equation}
M_{\rm hm} = 1.2 \times 10^{11} \left( \frac{m_\phi}{10^{-22}\eV} \right)^{-1.4} \Msun,
\end{equation}
or expressed in terms of $m_\phi$, 
\begin{equation}
m_\phi = 3.1 \times 10^{-21} \left( \frac{M_{\rm hm}}{10^{9}\Msun} \right)^{-0.71} \eV.
\end{equation}
LSST will be sensitive to light bosonic dark matter with mass $m_\phi \sim 10^{-20} \eV$ by probing the power spectrum of dark matter halos with half-mode mass of $\Mhm \sim 10^{8} \Msun$. 
Sensitivity to heavier bosonic particles ($m_\phi > 10^{-19} \eV$) would be possible through the detection of even smaller halos ($\roughly 10^{6} \Msun$).




