\chapter{Complementarity with Other Programs}
\label{sec:complementarity}
\bigskip

\Contributors{Kimberly K.\ Boddy, Esra Bulbul, Johann Cohen-Tanugi, Alessandro Cuoco, William A.\ Dawson, Alex Drlica-Wagner, Cora Dvorkin, Christopher Eckner, Christopher D.\ Fassnacht, Vera Gluscevic, Shunsaku Horiuchi, Charles R.\ Keeton, Ting S.\ Li, Marc Moniez, Manuel Meyer, Lina Necib, Ethan O.\ Nadler, Jeffrey A.\ Newman, Eric Nuss, Andrew B.\ Pace, Justin I.\ Read, Joshua D.\ Simon, Erik Tollerud, David Wittman, Gabrijela Zaharijas}

LSST will uniquely complement several other experimental studies of dark matter.
Below we summarize some of these complementary probes, with a specific focus on spectroscopic observations, high-resolution imaging, indirect detection experiments, and direct detection experiments.
While LSST can substantially improve our understanding of dark matter in isolation, these experiments are essential to provide a holistic picture of dark matter physics.
This section is not intended to be comprehensive, but rather serves to demonstrate the influence that LSST will have on dark matter studies generally.

% Spectroscopy
\section{Spectroscopy \Contact{Ting}}
\label{sec:spectroscopy}
\Contributors{Joshua D.\ Simon, Ting S.\ Li, William A.\ Dawson, Denis Erkal, David Wittman, Erik Tollerud}

The power of photometric and astrometric measurements from LSST will be significantly augmented by additional spectroscopic observations.
In particular, spectroscopic follow-up studies will provide kinematic and redshift information for many of the objects studied by LSST.
Given the faintness and high density of targets that are expected from LSST, community access to multi-object spectrographs on large-aperture telescopes is essential for these studies \citep{2016arXiv161001661N}. 
Due to LSST's location in the southern hemisphere, southern spectroscopic facilities are best at maximizing observational overlap.

Many next-generation telescopes and instruments are currently under development or construction. 
These facilities can be broadly divided into two categories: wide-field, massively multiplexed spectroscopy on medium- to large-aperture telescopes ($\roughly$8--10-meter class), and giant segmented mirror telescopes (GSMTs, $\roughly30$-meter class) with relatively smaller fields of view. 
%These instruments can be broadly divided into two categories: massively multiplexed spectrographs on 8- to 10-meter telescopes, and giant segmented mirror telescopes (GSMTs, $\roughly 30$-meter class) with smaller fields of view. 
The former category includes facilities on existing and future telescopes including the Prime Focus Spectrograph (PFS) instrument on the Subaru telescope \citep{2014PASJ...66R...1T} that is currently under construction; the Maunakea Spectroscopic Explorer \citep[MSE;][]{MSEbook2018}, a planned 11.25m, wide-field, optical and near-infrared facility completely dedicated to multi-object spectroscopy; the Southern Spectroscopic Survey Instrument (SSSI), a project concept recommended for consideration by the DOE’s Cosmic Visions panel \citep{1604.07626, 1604.07821}, and a possible future ESO wide-field spectroscopic facility. 
The latter category is populated by new facilities such as the Thirty Meter Telescope \citep[TMT;][]{1505.01195}, the Giant Magellan Telescope \citep[GMT;][]{GMT:2018}, and the Extremely Large Telescope \citep[ELT;][]{EELT:2009}.

In this section, we illustrate several examples of how complementary spectroscopy will improve the measurement of dark matter properties with LSST.

\subsection{Milky Way Satellite Galaxies \Contact{Josh}}\label{sec:MW_sats_spec}
%\Contributors{Josh, Ting, Erik, ...}
In \secref{smallest_galaxies} we discussed the derivation of an upper limit on the minimum dark matter halo mass based on the observed luminosity function of satellites discovered by LSST. 
%\ET{Should we mention the fact that confirmation of the LSST discoveries *also* may require spectroscopy at least for some of the ambiguous cases?}  
An alternative approach is to obtain spectroscopy of individual stars in each satellite to measure its velocity dispersion, from which the central mass and density can be inferred.  Then one can compare either the densities or the circular velocity function directly with theoretical predictions without assumptions about the subhalo mass function or the stellar mass-halo mass relation.


Spectroscopy of individual stars in the faint Milky Way satellites that will be identified with LSST will require deep observations with multiplexed spectrographs on large telescopes.  Measurements of the stellar velocity dispersions of these systems can be obtained either with 8--10-meter-class telescopes or with the next generation of 25--30-meter-class telescopes.  To assess the feasibility of spectroscopy of LSST satellites, we create a mock sample of dwarfs based on projections of the sensitivity of LSST satellite searches. We estimate the limiting magnitude set by targeting the brightest 20 stars in each simulated dwarf and apply exposure time calculators for Keck/DEIMOS and GMT/GMACS to determine the integration time needed for each satellite, with the additional constraint that no object would be observed for more than 30 hours ($\sim3$~nights).
As illustrated in \figref{specfollowup_distance}, spectroscopy of a nearly complete sample of satellites can be pushed $\roughly 2$~mag fainter in luminosity and a factor of $\roughly 2$ farther in distance with plausible investments of observing time on a GSMT than with existing facilities.

In addition to inferring the minimum dark matter halo mass, kinematics from stellar spectroscopy can also reveal the inner densities of the lowest-luminosity dwarf galaxies, for which baryonic effects are minimal and dark matter physics can be separated from the astrophysics of galaxy formation \citep{governato2012,read2017}. \AHGP{Problem: Justin read shows that density profile measurement requires about a thousand stars.  It's (relatively) easy to get a central density, hard to do a profile.} A direct measurement of the inner density in these dwarf galaxies will allow us to distinguish between collisionless CDM,  which predicts a cuspy NFW profile, and SIDM, which predicts lower central densities \citep[][though see \citealt{Nishikawa:2019lsc}]{2012MNRAS.423.3740V,Rocha:2012jg}. Moreover, the stellar kinematics will also reveal the integral of the dark matter density profile in dwarf galaxies (i.e., the J-factor), which is an essential input for constraints on the dark matter self-annihilation cross section for indirect dark matter searches in X-ray and $\gamma$-ray experiments \citep[\eg,][]{1108.3546}.

\begin{figure}
  \centering
  \includegraphics[width=0.49\textwidth]{figures/dwarf_observability_barplot_distance.pdf}
  \includegraphics[width=0.50\textwidth]{figures/dwarf_observability_barplot_luminosity.pdf}
  \caption{Possibility of spectroscopic follow-up for the LSST satellite population as a function of distance (left) and magnitude (right). Current telescopes will be able to measure velocity dispersions for $\roughly50\%$ of the expected satellites, while a GSMT can measure velocity dispersions for $\roughly80\%$.}
  \label{fig:specfollowup_distance}
\end{figure}

\subsection{Stellar Streams \Contact{Ting}}
%\Contributors{Ting, Denis ...}

As discussed in \secref{stream_gaps}, subhalo encounters with cold stellar streams will induce density perturbations that will be detectable by LSST, constraining the minimum dark matter halo mass and the mass function of dark matter halos from $\roughly10^5$--$10^9 \Msun$. In addition, these flybys cause velocity perturbations that correlate with the density variations.  The velocity signal near stream gaps can be measured either via line-of-sight velocity measurements from spectroscopy or tangential velocity measurements from astrometry, improving the precision with which the perturber mass can be determined.
The velocity variation (peak to peak) from these flybys will be small. To estimate the amplitude of the perturbation, we consider a stream orbiting the Milky Way at a distance of 14 kpc and compute the typical maximum velocity kick expected over its lifetime of 5 Gyr using the formalism from \citet{erkal2016}.  The velocity change is $\roughly \{0.6, 0.3, 0.1\} \kms$ for subhalos in the range $\{ 10^{7.5}, 10^{6.5}, 10^{5.5} \} \Msun$, respectively. %$\{ 10^7-10^8, 10^6-10^7, 10^5-10^6\} \Msun$

Due to the low density of the stream stars, a massively multiplexed, wide field-of-view spectroscopic facility such as PFS on Subaru or MSE is needed. 
Furthermore, given the expected small velocity kick amplitude, the velocity for each star determined from the spectroscopic observation need to be extremely precise.
%Furthermore, given the expected small velocity kick amplitude, the velocity accuracy for each star determined from the spectroscopic observations should be at or better than $1 \kms$ to unambiguously detect the signal with an ensemble of stream stars. 
Considering that the typical gap size for a $10^6 M_\odot$ subhalo is a few degrees, and the typical known streams has 10--100 stars/degree at $r\lesssim23$ \citep[\eg,][]{erkal2017, DeBoer:2018}, a velocity precision of $\sim1\,\mathrm{km\,s}^{-1}$ per star would allow an unambiguously detection on the signal with an ensemble of stream stars.  


\subsection{Galaxy Clusters \Contact{Will}}
%\Contributors{Will, ...}

%As noted in \secref{merging_clusters}, one of largest systematics associated with merging galaxy cluster constraints on SIDM is modeling the merger. The more complex the merger, the more severe the systematics will be.
%The best means of constraining merging galaxy cluster substructure is with spectroscopic measurement of as many 
%galaxy cluster member galaxies as possible \citep[\eg,][]{Golovich:2018}.
%As noted in \citet{2016arXiv161001661N}, perhaps the best spectroscopic follow-up facilities are large telescopes with slitmask-like multi-object spectrographs, or fiber-based multiplex spectrometers with low ($\mathcal{O}(\unit{arcsec})$) fiber collision regions, due to the density of cluster members.

As noted in \secref{merging_clusters} one of largest systematics associated with merging galaxy cluster constraints on SIDM is modeling the merger dynamics, most importantly the relative speed at the time of
pericenter. This can be inferred from the observed line-of-sight speed by correcting for the viewing angle and evolving the equations of motion back to the time of pericenter. Therefore, a minimal requirement is a good measurement of the systemic line-of-sight speed of each subcluster, which implies spectroscopic redshifts for hundreds of members of each subcluster \citep[\eg,][]{Golovich:2018}. In addition, a thorough spectroscopic survey of the surrounding $\sim$10 arcmin region is required to test for the presence of additional substructures that could affect reconstruction of the merger scenario. 10-m class collecting area is sufficient for most known mergers, but because member galaxies can be packed quite closely in their respective subclusters, fiber collision avoidance is a limiting factor for many instruments designed for wide-field highly multiplexed spectroscopy.  As noted in \citet{2016arXiv161001661N}, perhaps the best spectroscopic follow-up facilities are large telescopes with slitmask-like multi-object spectrographs, or fiber-based multiplex spectrographs with low ($\mathcal{O}(\unit{arcsec})$) fiber collision regions.

% High-resolution imaging
\section{High-Resolution Imaging}
\label{sec:highres}

\Contributors{William A.\ Dawson, Christopher D.\ Fassnacht, Charles R.\ Keeton, David Wittman, Marc Moniez}

Since the LSST point spread function (PSF) is limited to a median angular resolution of $\roughly 0.7\arcsec$ by a combination of instrumental and  atmospheric effects \citep{0805.2366}, there are many dark matter science cases where higher resolution imaging from space or ground-based adaptive optics (AO) facilities, which can reach $\roughly 0.01\arcsec$ in some cases, can be highly complementary. We briefly summarize some of these cases in this subsection and relate them to the dark matter science capabilities of LSST.
\WAD{This section currently focuses on high resolution optical imaging, however it is worth considering other wavelengths, especially radio.}

% Astrometric microlensing of compact dark matter
\subsection{Astrometric Microlensing}
%\subsection{Astrometric Microlensing of Compact Dark Matter \Contact{Will}}
\label{sec:astrometric_microlens}
%\Contributors{Will, ...}

Related to photometric microlensing (\secref{microlensing}), astrometric microlensing relies on the fact that the two images generated during a compact object lensing event will be of differing brightness, and the brightness ratio of these two images will vary throughout the duration of the lensing event.
The two images will be of most similar brightness when the projected lens-source separation is at its minimum.
By precisely measuring the astrometry of these blended images as a function of time and combining with the LSST photometric microlensing measurement one can break the lens mass-distance degeneracy and precisely measure the mass and location of individual black holes \citep{2015ApJ...814L..11Y}.
Astrometric microlensing will require precise astrometry from ground-based optical/NIR systems, space-based observatories, or longer microwave or radio wavelength observations.
Astrometric microlensing may even give access to less compact dark matter substructures \citep{1804.01991}.

In addition, high cadence, around the clock monitoring of microlensing events will be possible from organized teams of small- to medium-sized telescopes (\eg, MiNDSTEp\footnote{http://www.mindstep-science.org/}, RoboNet\footnote{https://robonet.lco.global/}, MicroFUN\footnote{http://www.astronomy.ohio-state.edu/~microfun/}, and PLANET\footnote{http://planet.iap.fr/}). 
These follow-up observations will be sensitive to distorsions with respect to the point-source, point-lens rectilinear approximation.
It will be critical that these collaborations receive automated alerts when LSST detects probable microlensing, or generically non-standard, transient events. 
Such efforts are planned as part of the LSST Alert Stream \citep{0805.2366}.

%These teams are currently fed by the MOA and OGLE-IV alert systems, and could be in perfect synergy with the LSST microlensing detections, providing extra information on individual events that cannot be extracted from the LSST data alone. 
%Tsapras, Y. et al., 2009, AN, 330, 4T: RoboNet-II: Follow-up observations of microlensing events with a robotic network of telescopes
%http://www.planet-legacy.org/

% Strong-microlensing
\subsection{Strong Microlensing \Contact{Will Dawson}}
%\subsection{Strong-Microlensing of Compact Dark Matter \Contact{Will Dawson}}
Strong microlensing is related to astrometric microlensing.
The Einstein radius of a given lens, which is approximately the separation of the multiple images in a compact object lensing scenario, scales as $\sqrt{M_\mathrm{lens}}$.
In the intermediate mass black hole range, the separation of the two images approaches that of the resolution of various optical ground and spaced-based telescopes (\figref{strong_microlensing}).
If the multiple images can be resolved and their flux ratio measured, precise measurements of the mass and distance of the lens are achievable.

\begin{figure}
\label{fig:strong_microlensing}
\centering
\includegraphics[width=0.5\columnwidth]{StrongMicrolensing.png}
\caption{The Einstein radius (i.e., 1/2 the separation of the multiple images) for microlensing lensing events as a function of compact object mass and distance (assuming a source distance of $8\kpc$). Parameter space below a black curve indicates that the multiple lensed source images will be resolvable by that telescope.}
\end{figure}

% Merging Galaxy Clusters
\subsection{Merging Galaxy Clusters and Cluster Subhalos}
%\Contributors{Will D., Dave W., ...}

Most dark matter constraints from merging galaxy clusters (\secref{halo_profile_clusters}) and cluster subhalos (\secref{halo_profile_clusters}) rely on accurately measuring the distribution of dark matter in (sub)clusters via gravitational lensing.
Strong and weak gravitational lensing both benefit from high-resolution imaging.
For strong lensing, high-resolution imaging enables better detection and characterization of strongly lensed background images in the dense cluster environment.
Similarly high-resolution imaging provides $\roughly4$ times more lensed source galaxies per unit area than ground-based imaging at similar depths, which enable higher-resolution weak gravitational lensing.
Historically, the Hubble Space Telescope (HST) has provided this higher-resolution imaging, although space-based telescopes such as JWST, Euclid, and WFIRST may take on most of this burden in the era of LSST.

% Strong gravitational lensing
\subsection{Strong Gravitational Lensing}
\label{sec:SLcomplement}
%\Contributors{Chris F.} 

All three approaches to use strong gravitational lens systems to make inferences on the nature of dark matter  described in \secref{stronglens} utilize LSST as a lens-finding facility.
Once the lenses are found, the dark matter science requires follow-up observations with other facilities.
The flux-ratio anomaly approach requires imaging that spatially resolves the lensed images from each other at a wavelength at which microlensing does not affect the image fluxes.
These observations can be in optical/near-IR wavelengths, utilizing IFU spectrographs behind the adaptive optics systems on ELTs to isolate the emission from the narrow-line regions of the lensed AGN, at mid-IR wavelengths with JWST, or at radio wavelengths for the subset of LSST lenses that are radio-loud.
The gravitational imaging and power-spectrum approaches both require milliarcsecond-scale angular resolution imaging for best results.
These observations require either ELT adaptive optics imaging or VLBI radio imaging of the targets.
ALMA can also be used in its most extended configuration, although this will not achieve as high a resolution as the ELTs and VLBI in most cases.
\NEW{SKA is expected to greatly increase the number of radio-detected strong lenses by several orders of magnitude \citep[\eg][]{2015aska.confE..84M}.}

% Indirect Detection
\section{Indirect Detection }
\Contributors{Esra Bulbul, Johann Cohen-Tanugi, Alessandro Cuoco, Alex Drlica-Wagner, Shunsaku Horiuchi, Manuel Meyer, Andrew B.\ Pace, Ethan O.\ Nadler, Eric Nuss, Gabrijela Zaharijas}
  

In regions of high dark matter density, dark matter particles could continue to annihilate or decay through the same process that set their relic abundance.
Of specific interest are energetic photons (i.e., X-rays and $\gamma$-rays), since photons are generically produced by the annihilation/decay of dark matter in many models (either directly or as secondary products following the production of quarks or leptons). In addition, astrophysical phenomena in extreme environments could lead to conversion between Standard Model particles and the dark sector (e.g., ALPs), which could be observable through the emission of energetic photons or via alterations in astrophysical spectra.
By precisely mapping the distribution of dark matter and tracking extreme events (\eg, CCSNe) LSST will enable more sensitive searches for energetic particles originating from the dark sector.

Conventional indirect detection searches focus predominantly on WIMPs with masses between several \GeV and tens of \TeV. 
The annihilation or decay of these particles could produce energetic Standard Model particles detectable by current or future experiments.
The most sensitive and robust indirect searches for dark matter rely on a precise determination of the distribution of dark matter in the universe.
The integrated flux of energetic Standard Model particles, $\phi_s$ (${\rm particles} \cm^{-2} \second^{-1}$), expected from dark matter annihilation in a density distribution, $\rho(\vect{r})$, is given by

\begin{equation}
   \phi_s(\Delta\Omega) =
    \underbrace{ \frac{1}{4\pi} \frac{\Gamma}{m_{\chi}^{a}}\int^{E_{\max}}_{E_{\min}}\frac{\text{d}N}{\text{d}E}\text{d}E}_{\rm particle\ physics}
    \cdot
    \underbrace{\vphantom{\int_{E_{\min}}} \int_{\Delta\Omega}\int_{\rm l.o.s.}\rho^{a}(\vect{r})\text{d}l\text{d}\Omega '}_{\rm astrophysics}\,.
    \label{eqn:indirect}
\end{equation}
%\Big\{\Big\}
\noindent Here, the ``particle physics'' term is strictly dependent on the dark matter particle physics properties---i.e., the particle mass, $m_\chi$,  the interaction rate, $\Gamma$, and the differential particle yield per interaction, $\text{d}N/\text{d}E$, integrated over the experimental energy range.
The second term, denoted ``astrophysics,'' represents the line-of-sight integral through the dark matter distribution integrated over a solid angle $\Delta\Omega$. 
For cases of dark matter annihilation, the interaction rate is set by the thermally averaged self-annihilation cross section, $\Gamma = \sigmav/2$, and the astrophysical integral is performed over the square of the dark matter density ($a=2$). 
The resulting astrophysical term is referred to as the ``\Jfactor'' \citep[\eg,][]{1998APh.....9..137B}. 
In cases of dark matter decay, the interaction rate is inversely proportional to the lifetime of the dark matter particle, $\Gamma = 1/\tau$, and the integral is performed over the dark matter density, $a=1$. 
The resulting term is known as the ``\Dfactor'' \citep[\eg][]{1408.0002}.
Qualitatively, the astrophysics term encapsulates the spatial distribution of the dark matter signal, while the particle physics term sets its spectral character. 
LSST will improve the sensitivity to dark matter particle physics by improving our understanding of the astrophysics term.
While these improvements will influence a wide range of indirect detection experiments, in this section we focus predominantly on $\gamma$-ray measurements.


\subsection{Milky Way Satellites \Contact{Andrew}}
\label{sec:dwarfs_id}
\Contributors{Andrew B.\ Pace, Alex Drlica-Wagner, Ethan O.\ Nadler, Manuel Meyer}

\begin{figure}[t]
\centering
\includegraphics[width=0.75\columnwidth]{annih_limits.pdf}
\caption{Constraints on dark matter annihilation to $b\bar{b}$ from {\it Fermi-LAT} observations of Milky Way satellite galaxies \citep[LAT Dwarfs;][]{Ackermann:2015} and HESS observations of the Galactic Center \citep[HESS GC;][]{1607.08142}. 
A bracketing range of dark matter interpretations to the  Fermi-LAT Galactic Center Excess is shown in red \citep[GCE;][]{1402.6703, Gordon:2013, Abazajian:2014}.
Projected sensitivity to dark matter annihilation combining LSST discoveries of new Milky Way satellites, improved spectroscopy of these galaxies, and continued Fermi-LAT observations is shown in gold. This projection assumes 18 years of Fermi-LAT data, a factor of 3 increase in the integrated J-factor, and a factor of 2 improvement from improved spectroscopy. 
Projected sensitivity of 500h observations of the GC with CTA are shown in gray \citep[CTA GC;][]{Zaharijas:prep}.
\label{fig:indirect}
}
\end{figure}

Gamma-ray observations of Milky Way satellite galaxies currently provide the most robust and sensitive constraints on the dark matter self-annihilation cross section for GeV- to TeV-mass particles \citep[\eg][]{Ackermann:2014, Geringer-Sameth:2015, Ackermann:2015, 1812.06986}.
The sensitivity of these searches will improve by combining new Milky Way satellite galaxies discovered by LSST, more precise \Jfactor measurements from novel spectroscopic observations, and additional Fermi-LAT data. 
We estimate each of these contributions to predict the improved sensitivity of dark matter annihilation searches in dwarf galaxies in the era of LSST.

To estimate the improvement in the integrated \Jfactor of the Milky Way satellite galaxy population, we combine cosmological zoom-in simulations of Milky Way dark matter substructure with a semi-analytic model to convert subhalo density profiles to \Jfactor estimates (this approach is is similar to that of \citealt{1309.4780}). 
Our simulation-based model accounts for modulations to dark matter-only subhalo populations due to baryonic physics, and we marginalize over the dependence of subhalo populations on host halo properties by sampling subhalo populations from a large number of hosts \citep{Nadler:2018}. 
To obtain an estimate for the increase in the integrated \Jfactor, we select a host halo with the largest number of nearby subhalos, consistent with recent observations of an overabundance of nearby satellites associated with the Milky Way \citep{Kim:2017iwr, Graus:2018}. 
We exclude subhalos with heliocentric distances $< 20 \kpc$ to avoid anomalously large projections due to a single nearby satellite.
We follow the analytic formalism presented by \citet{1604.05599} and \citet{1802.06811} to convert the dark matter profiles of our simulated subhalos to \Jfactors.  
This approach estimates the \Jfactor of each subhalo based on $r_{\max}$, $V_{\max}$, and heliocentric distance. 
%We calculate the cumulative $J$-factor within 100 kpc we find for the most optimistic case 
%(i.e., selecting the host halo with the highest cumulative $J$-factor and ignoring the effects of subhalo disruption) 
We find that the cumulative \Jfactor within 100\kpc may increase by as much as a factor of 3 relative to the known dSphs with measured \Jfactors. 

Recent studies have suggested that an additional factor of 2 improvement in sensitivity may be possible through better spectroscopic measurements  of the stars in known satellite galaxies \citep{Albert:2017}, and we include this factor in our projections.
In addition, current constraints from the Fermi-LAT Collaboration used 6 years of data \citep{Ackermann:2015}; however, the Fermi-LAT has collected more than 10 years of data and could continue to collect data for another 10+ years.
These additional data will improve the statistical sensitivity of the $\gamma$-ray search most drastically for large dark matter particle masses ($>500\GeV$).
We quantitatively evaluate the improvement from continued Fermi-LAT data taking using the results of \cite{Charles:2016}.
We combine the predicted improvements from new dwarfs, better determined \Jfactors, and more Fermi-LAT data into a projected sensitivity for future searches for dark matter annihilation in dwarf galaxies in \figref{indirect}.


\subsection{Cross correlation with gamma rays \Contact{Horiuchi,Alessandro C}}
\label{sec:igrb_id}
\Contributors{Shunsaku Horiuchi, Alessandro Cuoco}

Wide-area weak-lensing measurements from LSST will help extract  potential dark matter contributions to the isotropic gamma-ray background \citep[IGRB;][]{1410.3696}. The IGRB is defined as the residual all-sky $\gamma$-ray emission after subtracting individually detected sources and the Galactic diffuse emission, and provides the distance frontier of indirect dark matter searches with $\gamma$-rays. Contributions to the IGRB include unresolved sources that are individually too faint to be detected---e.g., blazars \citep{1110.3787,1310.0006}, star-forming galaxies \citep{1206.1346}, and misaligned AGNs \citep{1304.0908}---as well as a potential contribution from dark matter annihilation \citep{1312.0608,1501.05464,1501.05301,1608.07289}. Analyses of the IGRB intensity spectrum, auto-correlation angular power spectrum, and photon count statistics show that a linear combination of astrophysical sources can explain the observed IGRB, but the uncertainties are still large \citep[e.g.,][]{1502.02866}.

LSST will prove invaluable by mapping the distribution of matter on large scales via measurements of galaxy clustering and of cosmic shear from weak gravitational lensing. 
Since  cosmological $\gamma$-ray emission from dark matter annihilation follows the same underlying dark matter distribution traced by cosmic shear and galaxies, cross correlating them yields novel information on the composition of the IGRB \citep{1212.5018,1312.4403,1411.4651,1506.01030,Lisanti:2018}. 
Compared to the IGRB intensity or auto-correlation, the cross correlation will yield more than $\roughly 10$ times higher sensitivity to dark matter \citep{1411.4651,1503.05922}.
Cross-correlations with galaxy catalogs have been derived in \cite{1103.4861,1503.05918,1709.01940} up to $z \sim 0.6$, 
which is the largest redshift where current catalogs have enough sky-coverage and galaxy density to robustly
detect the correlation.  On the other hand, the IGRB is expected to extend up to $z \sim 2$--$3$ \citep{1502.02866}. 
LSST, with its large sky-coverage and galaxy density and broad redshift range, thus fills this gap to map the IGRB-LSS cross-correlation up to high redshift. 
A complete mapping of the IGRB up to $z \sim 3$ will constitute a crucial tool to robustly separate the different 
astrophysical contributions, as well as to isolate the DM annihilation signal, breaking the degeneracies which
are present when only low-redshift results are used \citep{1506.01030}.    

In contrast to galaxy catalogs, cosmic shear has the advantage of being an unbiased tracer of the dark matter distribution, which mitigates many of the systematics from using galaxies to trace dark matter---i.e., assumptions about the relationship between galaxy luminosity and halo mass, reliance on assumptions of hydrostatic equilibrium, and strong correlations with astrophysical $\gamma$-ray emission. At present, weak lensing surveys of several hundred square degrees allow studies of the IGRB to probe slightly above the thermal annihilation cross section \citep{1404.5503,1607.02187,1611.03554}. A simple forecast for LSST can be made by scaling the covariance matrix of the correlation estimator by the sky coverage. This shows that a combination of LSST lensing maps and all-sky Fermi-LAT data will reach a sensitivity where it is possible to \textit{detect} at $3\sigma$ WIMP annihilation to $b\bar{b}$ at the thermal cross cross section for dark matter particle masses up to 100 GeV \citep{1404.5503}.   


From the $\gamma$-ray side, improvements in the mapping of the cross-correlation with galaxies and cosmic shear are expected with the next generation of $\gamma$-ray instruments: AMEGO\footnote{\url{https://asd.gsfc.nasa.gov/amego/index.html}} and eASTROGAM\footnote{\url{http://eastrogam.iaps.inaf.it/}} \citep{1711.01265}.
At the present, the main limitation in detecting the cross-correlation at GeV and sub-GeV energies is instrumental angular resolution.  
eASTROGAM will have an angular resolution 5--6 times better than the Fermi-LAT in the energy range from $100 \MeV$ to $1 \GeV$~\citep{1711.01265}. This will translate in harmonic space into a multipole reach 5-6 times larger than presently achievable, leading to stronger constraints from cross correlations.
Precise measurements of the cross-correlation at sub-GeV energies will further improve the ability to separate the astrophysical IGRB sources from the DM signal, increasing the sensitivity to the latter. 

\subsection{Axion-like particle emission from supernovae \Contact{Manuel}}
\label{sec:alp_id}
\Contributors{Manuel Meyer}

Axion-like particles (ALPs) might be produced during CCSNe explosions through the conversion of thermal photons in the electro-static fields of protons and ions, \ie, through the Primakoff effect \citep{1996slfp.book.....R}.  
Similar to neutrinos, ALPs would quickly escape the core and, if they are sufficiently light ($m_\phi \lesssim 10^{-9} \eV$), they could convert into $\gamma$-rays in the magnetic field of the Milky Way and/or the host galaxy of the CCSN. 
The resulting $\gamma$-rays would arrive in temporal coincidence with the neutrinos in a burst lasting tens of seconds with a 
thermal spectrum peaking at 60\MeV, depending on the mass of the progenitor \citep{2015JCAP...02..006P}.
The non-observation of a $\gamma$-ray burst from SN1987A, which occurred in the LMC, has been used to derive stringent constraints on the photon-ALP coupling $g_{\phi\gamma}<5.3\times10^{-12}\GeV^{-1}$ for $m_\phi < 4.4\times10^{-10}\eV$ \citep{1996PhLB..383..439B, 1996PhRvL..77.2372G,2015JCAP...02..006P}.
In the case of a CCSN within the Milky Way, the \textit{Fermi} LAT could improve these limits by more than an order of magnitude \citep{2017PhRvL.118a1103M}. 
However, with a Galactic supernova rate of $\roughly 3$ per century \citep[\eg,][]{2013ApJ...778..164A}, and the LAT field of view of 20\% of the sky, the chance to observe at least one such event in the next five years is $\sim 0.03 \cdot 0.2 \cdot 5 = 0.03$ (assuming that the occurrence of SNe is a Poisson process). This estimate is still optimistic since the supernova rate is calculated for the entire Galaxy, which is not inside the field of view at any given moment.\footnote{If the supernova is sufficiently nearby or the photon-ALP coupling is close to current limits, a signal could be detected with the BGO detectors (senisitive up to 40\MeV) of the \emph{Fermi} Gamma-ray Burst Monitor, which observes the entire unocculted sky.}
Increasing the search volume to extragalactic SNe is the obvious way to overcome this low rate. 
However, for CCSNe beyond the LMC and SMC, current-generation neutrino detectors lack the sensitivity to detect a signal \citep[\eg,][]{2011PhRvD..83l3008K}, and hence no precise time stamp will be provided for ALP-induced $\gamma$-ray emission; however, well-sampled optical light curves can be used to estimate the explosion time on the time scale of hours \citep{2010APh....33...19C}. 
LSST will detect a plethora of CCSNe light curves \citep{Lien:2009}. 
Estimates for the delay between the core collapse and the shock breakout range from minutes for massive Wolf-Rayet stars (type Ib/c) to days for red supergiants (type II) \citep{2013ApJ...778...81K}. 
Thus, SN type Ib/c caught early after their shock breakout and with subsequently well-sampled light curves are a prime target for the search of an ALP-induced $\gamma$-ray burst. 

Since the $\gamma$-ray flux scales as $g_{\phi\gamma}^4 / d^2$, where $d$ is the luminosity distance, the sensitivity for $g_{\phi\gamma}$ scales as $\sqrt{d}$. 
Limits of the order of $g_{\phi\gamma} \lesssim 2\times10^{-12}\,\mathrm{GeV}^{-1}$ should be possible for a single supernova in M31 ($d=778\kpc$) \citep{2017PhRvL.118a1103M}. 
If one allows these limits to degrade by a factor of 10, constraints better than those from CAST should still be possible for $d\lesssim 80\Mpc$ ($z \lesssim 0.02$) for a single SN assuming that the time of the core collapse is precisely known. 
LSST is expected to detect tens of type Ib/c CCSNe each year with redshifts $z \lesssim 0.02$ \citep{Goldstein:2018} and could conduct such searches in conjunction with the \textit{Fermi} satellite or future $\gamma$-ray satellites like AMEGO, eASTROGAM, or Gamma-400 \citep[\eg,][]{2017ICRC...35..910C,1502.02976} 
A stacking analysis of the $\gamma$-ray data with explosions times estimated from LSST light curves provides the exciting possibility to probe photon-ALP couplings in the regime where ALPs make up the entirety of the dark matter.


% Direct Detection
\section{Direct Detection }
\label{sec:direct}
\Contributors{Kimberly K.\ Boddy, Alex Drlica-Wagner, Cora Dvorkin, Vera Gluscevic, Ethan O.\ Nadler, Lina Necib, Justin I.\ Read}

Direct detection experiments seek to directly detect interactions between particles in the Galactic dark matter halo and an experimental apparatus. In the case of WIMP dark matter, these experiments are sensitive to the scattering between dark matter particles and atomic nuclei \citep[\eg,][]{1509.08767}.
Interpreting the results of direct detection experiments in the context of dark matter necessarily relies on astrophysical measurements of the distribution of dark matter.
In addition, limitations from the energy thresholds and shielding of direct detection experiments can limit the ranges of dark matter particle masses and cross sections that can be probed.
In this section, we describe how LSST will complement direct detection experiments by improving measurements of the local phase-space density of dark matter, and how cosmological measurements with LSST can help probe dark matter  masses and cross sections outside the range accessible to direct detection experiments.


\subsection{Local Dark Matter Distribution \Contact{Lina}}
\Contributors{Justin I.\ Read, Lina Necib}

\def\rhodm{\rho_\mathrm{\chi}}
\def\rhodmlab{\tilde{\rho}_\mathrm{\chi}}
\def\rhodmext{\rho_\mathrm{\chi,ext}}

The signal strength of dark matter scattering in direct detection experiments depends on the local dark matter density and its velocity distribution. 
For spin-independent scattering of dark matter particles off atomic nuclei, the recoil rate (per unit mass, nuclear recoil energy $E$, and time) in such experiments is given by \citep[\eg,][]{1996APh.....6...87L}: 

\begin{equation}
\frac{dR}{dE} = \frac{\rhodmlab \sigma_\chi |F(E)|^2}{2 m_\chi \mu^2} \int_{v>\sqrt{m_N E/2\mu^2}}^{v_\mathrm{max}} \frac{f({\bf v},t)}{v}d^3 {\bf v} 
\label{eqn:recoilrate} 
\end{equation} 
where $\sigma_\chi$ and $m_\chi$ are the interaction cross section and mass of the dark matter particle, $|F(E)|$ is a nuclear form factor that depends on the choice of detector material, $m_N$ is the mass of the target nucleus, $\mu$ is the reduced mass of the dark matter-nucleus system, $v = |{\bf v}|$ is the speed of the dark matter particles, $f({\bf v},t)$ is the velocity distribution function of the dark matter particles, $v_\mathrm{max} = 533^{+54}_{-41}\kms$ (at 90\% confidence) is the Galactic escape speed \citep{2014AA...562A..91P}; and $\rhodmlab$ is the dark matter density within the detector. 
 
From equation \ref{eqn:recoilrate}, we can see that $\rhodmlab$ is trivially degenerate with the properties of the dark matter particle, $\sigma_\chi/m_\chi$. For this reason, significant effort has gone into estimating the amount of dark matter within a few hundred parsecs of the Sun, $\rhodm$, from which we can extrapolate $\rhodmlab$ \citep[see][for a review]{1404.1938}. The latest values favor $\rhodm \sim 0.5\GeV \cm^{-3}$, with an uncertainty of order $20-30$\% \citep[\eg,][]{2014A&A...571A..92B,2018MNRAS.478.1677S}. With the advent of unprecedented data from the \Gaia satellite, the systematic and random errors on $\rhodm$ will continue to fall \citep{1404.1938}. However, equally important in equation \ref{eqn:recoilrate} is the velocity distribution function of dark matter, $f({\bf v},t)$, which is much more challenging to measure.\footnote{The time dependence of $f({\bf v},t)$ owes primarily to the motion of the Earth around the Sun and is, therefore, straightforward to calculate \citep[\eg,][]{1986PhRvD..33.3495D}.}

The shape of $f({\bf v})$ has been constrained primarily by numerical simulations of structure formation in the standard cosmological model \citep[\eg,][]{2009MNRAS.395..797V,1210.2721}. Such simulations include treatments to model the effects of unresolved substructure and debris, and the impact of dark matter particles scattering within the solar system \citep[\eg,][]{2009PhRvD..79j3531P}. However, such effects can only be treated statistically and might not apply to the real $f({\bf v})$ in our Galaxy.

With the advent of LSST, we will be able to {\it empirically} probe $f({\bf v})$ with unprecedented precision. The key idea is to use the oldest and most metal poor stars, which were accreted onto the Milky Way as it formed, as luminous tracers of the underlying dark matter halo \citep{Lisanti:2011as,Kuhlen:2012fz,2014MNRAS.445L..21T,Lisanti:2014dva,2018PhRvL.120d1102H,Necib:2018b}. Such accreted stars also trace the presence of a `dark disk' formed from the late accretion of massive and more metal rich satellites \citep{1989AJ.....98.1554L,2008MNRAS.389.1041R,2009MNRAS.397...44R,2014MNRAS.444..515R,2015MNRAS.450.2874R}, and structures that are not yet fully phase mixed like `debris flows' \citep[\eg,][]{Lisanti:2011as,2018MNRAS.477.1472B,2018Natur.563...85H,necib2018} and tidal streams \citep[\eg,][]{2005PhRvD..71d3516F,1807.09004}. All of these structures imprint features on $f({\bf v})$ that alter the expected flux at a given recoil energy in dark matter detection experiments, and the expected annual modulation signal \citep[\eg,][]{2005PhRvD..71d3516F,2009ApJ...696..920B,2018arXiv181011468E}.

The wide sky coverage and depth of LSST will allow us to select metal poor halo star candidates in statistically significant quantities \citep[\eg,][]{2017MNRAS.471.2587S}, with proper motion data available for the brighter stars. Combined with follow-up spectroscopy, this will provide a direct probe of the velocity distribution of the Milky Way's smooth phase-mixed component \citep{2018PhRvL.120d1102H}. In addition, LSST will find a slew of new structures and streams, allowing us to probe also the non-phased mixed component \citep{2005PhRvD..71d3516F,2018arXiv181011468E}. Finally, combining LSST with spectroscopic surveys of the disk will allow us to place ever tighter constraints on the possible presence of a dark disk \citep{2015MNRAS.450.2874R}.


\begin{figure}
\centering
\includegraphics[width=0.75\columnwidth]{bsdm_limits.pdf}
\caption{
Constraints on dark matter-baryon scattering through a velocity-independent, spin-independent contact interaction with protons. 
Existing constraints (shown in blue) include measurements of the CMB power spectrum \citep[CMB;][]{Gluscevic:2017ywp} and constraints from the X-ray Quantum Calorimeter experiment \citep[XQC;][]{0704.0794}. Direct detection constraints include results from CRESST-III \citep{1711.07692}, the CRESST 2017 surface run \citep{1707.06749}, and XENON1T \citep{1705.06655}, as interpreted by \citet[][]{1802.04764}. %\citep{2018PhRvD..97l3013K}.
Additional constraints that include the effects of cosmic-ray heating of dark matter are shown in gray \citep[][]{1810.10543}.
The projected sensitivity of LSST to dark matter-baryon scattering through observations of Milky Way satellite dwarf galaxies is shown in gold.
}
\label{fig:dd}
\end{figure}

\subsection{Cosmic Baryon Scattering \Contact{Vera}}
\Contributors{Vera Gluscevic, Kimberly K.\ Boddy, Ethan O.\ Nadler, Alex Drlica-Wagner, Cora Dvorkin}

The most sensitive direct searches for dark matter seek to detect the scattering of dark matter particles from the local Galactic halo in underground detectors \citep[\eg,][]{1509.08767}. 
They have unprecedented sensitivity to WIMPs with masses above a GeV, but are limited by kinematics when searching for lighter particles. 
New experimental techniques are being explored to directly search for sub-GeV models of dark matter \citep{Battaglieri:2017aum}. 
However, due to atmospheric and terrestrial shielding, most direct dark matter experiments are largely insensitive to dark matter particles with large scattering cross sections. 
Current null results from conventional direct detection experiments motivate broad searches in regions of parameter space that are largely inaccessible to underground experiments. 

%This has motivated a number of experimental tests using astrophysical and cosmological measurements including the X-ray Quantum Calorimeter experiment \citep[XQC;][]{0704.0794} and studies of dark matter-cosmic-ray interactions \citep{Cappiello:2018hsu,1810.10543}.

Cosmological and astrophysical observables are sensitive to the scattering of sub-GeV particles with baryons at any point in cosmic history. 
These observations can constrain the interaction cross section to arbitrarily high values and are not subject to uncertainties in the local astrophysical properties of dark matter particles \citep[\eg,][]{1210.2721,1404.1938}. 
If dark matter particles scatter with baryons, they will transfer momentum between the two cosmological fluids, affecting density fluctuations and suppressing power at small scales. 
This power suppression can be captured by a variety of observables including measurements of the CMB \citep{Dvorkin:2013cea,Gluscevic:2017ywp} and the Lyman-$\alpha$ forest \citep{Xu:2018efh}.
Assuming a velocity-independent, spin-independent contact interaction, cosmological constraints can be directly compared against those from direct detection experiments \citep[\eg,][]{Boddy:2018kfv}.
In \figref{dd}, we compare existing constraints on dark matter-baryon scattering from analyses of the CMB and direct-detection searches.\footnote{We caution the reader that this figure does not include a comprehensive list of current constraints, but rather serves to illustrate the complementarity of cosmological and direct detection probes.} 
To estimate the future sensitivity of LSST, we map the projected WDM constraints presented in \secref{smallest_galaxies} to dark matter-baryon scattering constraints by matching the characteristic cutoff scale in the matter power spectrum probed by the lowest-mass subhalos LSST can detect via observations of Milky Way satellite galaxies. 
LSST will deliver measurements of observables that trace matter fluctuations on even smaller scales (\eg, stellar stream gaps), which will potentially extend the sensitivity of these astrophysical and cosmological searches even farther beyond the reach of Planck.

