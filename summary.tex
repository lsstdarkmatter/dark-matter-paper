\begin{comment}

\begin{center}
  {\Large \bf Abstract}
\end{center}
Astrophysical and cosmological observations currently provide the only robust, empirical measurements of dark matter. Future observations with Large Synoptic Survey Telescope (LSST) will provide necessary guidance for the experimental dark matter program. This white paper represents a community effort to summarize the science case for studying the fundamental physics of dark matter with LSST. We discuss how LSST will inform our understanding of the fundamental properties of dark matter, such as particle mass, self-interaction strength, non-gravitational couplings to the Standard Model, and compact object abundances. Additionally, we discuss the ways that LSST will complement other experiments to strengthen our understanding of the fundamental characteristics of dark matter. More information on the LSST dark matter effort can be found at https://lsstdarkmatter.github.io/ .

\clearpage
\end{comment}


\begin{center}
  {\Large \bf Executive Summary}
\end{center}

More than 85 years after its astrophysical discovery, the fundamental nature of dark matter remains one of the foremost open questions in physics.
Over the last several decades, an extensive experimental program has sought to determine the cosmological origin, fundamental constituents, and interaction mechanisms of dark matter. 
While the existing experimental program has largely focused on weakly-interacting massive particles, there is strong theoretical motivation to explore a broader set of dark matter candidates.
As the high-energy physics program expands to ``search for dark matter along every feasible avenue'' \citep{P5Report}, it is essential to keep in mind that the only direct, empirical measurements of dark matter properties to date come from astrophysical and cosmological observations.


The Large Synoptic Survey Telescope (LSST), a major joint experimental effort between NSF and DOE, provides a unique and impressive platform to study dark sector physics.
LSST was originally envisioned as the ``Dark Matter Telescope'' \citep{Tyson:2001}, though in recent years, studies of fundamental physics with LSST have been more focused on dark energy.
Dark matter is an essential component of the standard \LCDM model, and a detailed understanding of dark energy cannot be achieved without a detailed understanding of dark matter.
In the precision era of LSST, studies of dark matter and dark energy are \emph{extremely complementary} from both a technical and scientific standpoint.
In addition, cosmology has consistently shown that it is impossible to separate the \emph{macroscopic distribution} of dark matter from the \emph{microscopic physics} governing dark matter.
In this document, we reaffirm LSST's ability to test well-motivated theoretical models of dark matter: \ie, self-interacting dark matter, warm dark matter, dark matter-baryon scattering, ultra-light dark matter, axion-like particles, and primordial black holes. 

Studies with LSST will use observations of Milky Way satellite galaxies, stellar streams, and strong lens systems to detect and characterize the smallest dark matter halos, thereby probing the minimum mass of ultra-light dark matter and thermal warm dark matter.
Precise measurements of the density and shapes of dark matter halos in dwarf galaxies and galaxy clusters will be sensitive to dark matter self-interactions probing hidden sector and dark photon models.
Microlensing measurements will directly probe primordial black holes and the compact object fraction of dark matter at the sub-percent level over a wide range of masses.
Precise measurements of stellar populations will be sensitive to anomalous energy loss mechanisms and will constrain the coupling of axion-like particles to photons and electrons.
Unprecedented measurements of large-scale structure will spatially resolve the influence of both dark matter and dark energy, enabling searches for correlations between the only empirically confirmed components of the dark sector.
In addition, the complementarity between LSST, direct detection, and other indirect detection dark matter experiments will help constrain dark matter-baryon scattering, dark matter self-annihilation, and dark matter decay.

The study of dark matter with LSST presents a small experimental program with a short timescale and low cost that is guaranteed to provide critical information about the fundamental nature of dark matter over the next decade.
LSST will rapidly produce high-impact science on fundamental dark matter physics by exploiting a soon-to-exist facility. 
The study of dark matter with LSST will explore parameter space beyond the high-energy physics program's current sensitivity, while being highly complementary to other experimental searches. % BRN text.
This has been recognized in Astro2010 \citep{Astro2010}, during the Snowmass Cosmic Frontier planning process \citep[\eg,][]{1310.8642, 1310.5662, 1305.1605}, in the P5 Report \citep[]{P5Report}, and in a series of more recent Cosmic Visions reports \citep[\eg,][]{1604.07626,1802.07216}, including the ``New Ideas in Dark Matter 2017:\ Community Report'' \citep{Battaglieri:2017aum}.
It is worth remembering that astrophysical probes provide the only constraints on the minimum and maximum mass scale of dark matter, and astrophysical observations will likely continue to guide the experimental particle physics program for years to come.

\clearpage

\begin{center}
  {\Large \bf Preface}
\end{center}

This white paper is the product of a large community of scientists who are united in support of probing the fundamental nature of dark matter with LSST.
%Although LSST was originally proposed as the ``Dark Matter Telescope'' \citep{Tyson:2001}, none of the eight existing LSST Science Collaborations is specifically focused on exploring the microscopic identity of dark matter.
The study of dark matter is currently distributed across several of the LSST Science Collaborations, making it difficult to combine results and build a cohesive physical picture of dark matter.
It was recognized that the existing situation could hamper research on dark matter physics with LSST, and the current effort was started to coordinate dark matter studies among various LSST Science Collaborations, to enlarge the dark matter community, and to strengthen connections between theory and experiment.
The concept for this white paper emerged from a series of meetings and regular telecons organized in 2017--2018 around the topic of astrophysical probes of dark matter in the era of LSST.
Sessions were held at the LSST Project and Community Workshop in 2017 and 2018, multiple LSST Dark Energy Science Collaboration (DESC) meetings, and two dedicated multi-day workshops at the University of Pittsburgh and Lawrence Livermore National Laboratory.
Funding was provided by individual institutions and through a grant from the LSST Corporation (LSSTC) Enabling Science Program.
Through participation in the workshops, numerous telecons, sensitivity analyses, writing, editing, and reviewing, roughly \CHECK{100} scientists have directly contributed to this white paper.
We encourage interested scientists to \NEW{learn more about} this effort at: \url{https://lsstdarkmatter.github.io/}.
