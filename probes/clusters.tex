\subsection{Galaxy Clusters \Contact{Susmita}}
\label{sec:halo_profile_clusters}
\Contributors{Susmita Adhikari, William A.\ Dawson, Nathan Golovich, David Wittman,  M.\ James Jee, Annika H.\ G.\ Peter, Daniel A.\ Polin, Robert Armstrong}

Galaxy clusters are the most massive gravitationally bound structures in the universe. The high matter density and high velocity dispersions of clusters make them ideal laboratories for testing dark matter self-interaction models in a very different regime from individual galaxies.
In the following section we discuss several probes that use galaxy clusters to constrain the nature of dark matter.  We show that current constraints from many different cluster-scale probes are of the order of $0.1$--$1\cmg$.  To understand why this is so, it is important to note that the average column density of a cluster-scale halo is of the order of $1 \g \cm^{-2}$.  Improved cross section constraints will come from a combination of the large statistical data sets that will be collected by LSST and other telescopes in the LSST era, and more sophisticated theoretical predictions for observables for specific SIDM models.

\vspace{1em} \noindent {\bf Distribution of matter and substructure}

As we describe below, the current best cluster-scale SIDM constraints come from the radial dark matter profiles of halos.  However, other differences between SIDM and CDM cluster halos exist, and may prove highly constraining with the vastly detailed LSST cluster data sets.  Significantly more theoretical work is required to project robust constraints in the LSST era for those probes.

\paragraph{Radial profile:} Interactions among halo particles allow for the exchange of energy between different parts of the halo. The high number of interactions near the dense central regions of dark matter halos increases the temperature or the velocity dispersion near the central regions, a process that can be thought of as a transfer of heat from the outer (hotter) parts of the halo to the inner (colder) region. The excess dispersion due to self-interaction leads to flattening of the inner density of the halo, leading to the formation of a cored density profile. For cluster mass halos, the high densities near the center make the timescales for thermalization shorter at a given cross-section than they are for lower-mass objects. (Although it must be noted that low-mass halos are generally older and have a longer time to thermalize).  The short thermalization time is important because dark matter thus behaves as a fluid in the innermost parts of cluster-scale halos, and can relax to a hydrostatic equilibrium configuration at the center of the halo, where baryons dominate the potential \citep{Kaplinghat:2015aga}.  Depending on the merger history, cluster halos can be as cored as indicated in dark-matter-only simulations (for recent mergers), or relax to a hydrostatic equilibrium (for highly relaxed systems) in which the dark matter halo has a small but relatively dense core \citep{Robertson:2017mgj}.

Density profiles of massive galaxy clusters therefore serve as probes for SIDM. Clusters tend to be dark matter dominated outside the very central regions, and they are the only known system where the matter distribution can be individually mapped to the virial radius using weak lensing. Strong lensing also provides a measure of cluster mass independent of dynamical state, and stellar kinematics of the central galaxy can be used to measure the dark matter density profile in the innermost regions. LSST will produce an unrivaled catalog of strong and weak lensing measurement of cluster density profiles. This, in concert with X-ray mass estimates and stellar kinematics, will provide a strong test of the NFW dark matter density profile predicted by cold, collisionless dark matter \citep{Newman:2013,Kaplinghat:2015aga,Robertson:2018anx,Andrade:2019wzn}. Moreover, the strong lensing cross section is an additional probe of the density profile \citep{Robertson:2018anx}.  For hard-sphere scattering, cross section constraints are of order $0.1\cmg-1\cmg$, but without fully quantified systematic uncertainties.

\paragraph{Halo shape:} Apart from the density profile itself, in SIDM models, dark matter velocity distributions become more isotropic than in CDM, especially in the center of the halo.  Correspondingly,  the halo density profile becomes more spherical.  Historically, constraints from cluster and galaxy ellipticies \citep{Miralde-Escuda:2000} provided strong constraints on the cross-section of SIDM; however, later investigations found these constraints to be a bit too optimistic \citep{Peter:2013}. 
Recent measurements of the shapes of cluster dark matter halos includes studies with: cluster members \citep{2018MNRAS.475.2421S},  X-rays \citep{Hashimoto:2007},  lensing \citep{Mandelbaum:2006, Evans:2009, Oguri:2010}, and a combinations of observables \citep{Clampitt:2016, Sereno:2018}.  
Current constraints are sensitive to cross sections of order $\sigmam \sim 1 \cmg$.
Several groups have shown in N-body simulations that the effects of SIDM with a cross section of roughly (a few)$\times 0.1 \cmg \hbox{ to } 1 \cmg$ are potentially observable, although baryons can alter the probability distribution function of halo shapes by an amount that is not robustly quantified \citep[\eg][]{Peter:2013, Robertson:2017mgj, Brinckmann:2018}.


\paragraph{Substructure:} Structure forms hierarchically in the standard CDM scenario: small objects form first and merge to form larger mass structures such as galaxy clusters. These clusters continue to accrete smaller halos and some of these small structures survive as subhalos within the cluster. It is therefore interesting to study the distribution of substructure within halos and how it is affected by self-interactions among dark matter particles. 

Subhalos can be affected in three different ways within a cluster. Firstly, subhalo particles can evaporate due to interactions with the dark matter in the host cluster. Subhalos lose mass when they enter a cluster. In the CDM scenario particles that are at larger radii and are loosely bound get stripped as the subhalo orbits within a cluster. In SIDM evaporation due to self-scattering leads to additional mass loss. Unlike tidal stripping, self-interactions can also affect the inner regions of the subhalos. Simulations show that evaporation is inefficient at increasing the subhalo disruption rate unless hard-sphere cross sections are of order $\sigmam \sim 10\cmg$, or subhalos are on nearly radial orbits through the cluster center \citep{2012MNRAS.423.3740V,Rocha:2012jg,Dooley:2016ajo}. 

While this generally means that the total subhalo mass function within the virial volume is largely unaffected relative to CDM, other effects of evaporation may be detectable. Measuring the mass and the profile around cluster satellites (especially as a function of orbit eccentricity) using galaxy-galaxy lensing to measure the mass-to-light ratio of subhalos can be a promising probe for dark matter physics \citep{Natarajan:2017sbo}. The lensing signal around subhalos is weak and will be contaminated by the cluster mass profile, so methods like subtracting the lensing signal from diametrically opposite points within the cluster can be used to extract the signal. Given the statistics of cluster galaxies in LSST, it is ideally suited for a study of the weak lensing signal of subhalos.  

Secondly, as subhalos are also tracers of the dark matter density field within the cluster, their orbits will be affected by the change in the potential of the cluster near the core relative to CDM.  This effect can lead to an imprint in the radial distribution of subhalos in clusters, generally by making the subhalos less concentrated toward the halo centers.

Thirdly, non-expulsive interactions can lead to a drag-like force on subhalos.  This has several potentially interesting observable consequences.  First, the location of the splashback radius is sensitive to dynamics of subhalos within the cluster. The splashback radius is the boundary of the multistreaming region of a halo and is the largest apocenter of recently accreted objects \citep{Diemer:2014xya,Adhikari:2014lna}. The slope of the density profile of a halo falls off rapidly in a narrow localized region around this radius, and the splashback radius is observed as a minimum in the slope of the projected number density profile of galaxies \citep{More:2016vgs,Baxter:2017csy,Chang:2017hjt}.
The apocenter of the orbits of subhalos can change if there is extra drag beyond dynamical friction \citep{Kummer2018}.  
%In general the location of splashback for massive subhalos is effected by dynamical friction, high mass subhalos have smaller splashback radius, so a trend in its location with galaxy magnitude can probe dynamical friction. The presence of a core in the cluster halo can reduce the effects of friction and also in general allow orbits to have larger apocenters compared to CDM. 
Therefore measuring the location of the splashback radius can help distinguish between different models of dark matter, although the differences between splashback in the CDM and SIDM scenarios have not been well quantified.

%The lower potential in the core can lead to subhalos having larger orbits and massive subhalos will feel reduced forces from dynamical friction due to the core in the center. Besides, subhalos on radial orbits, that have small pericenters have more interactions than those on larger angular momentum orbits, that may lead to differences in distriution of these satellites as well.  Therefore both the abundance of subhalos within the cluster and its distribution can be used as a probe for self interacting dark matter.  

Similar to the situation in merging clusters (see the next subsection), the non-expulsive interactions can lead to a drag force, which may lead to offsets between the light distribution and dark matter distribution of individual satellites with respect to their subhalos. Small offsets between the subhalo and the galaxy within it may be detectable by indirect means: the potential gradient established by the DM at the position of the stellar centroid would induce a U-shaped warp in the stellar disk facing the direction of infall, and a longer-lasting disk thickening. Numerical simulations show these to be observable by current and next-generation photometric surveys under SIDM models with $0.5 \cmg \lesssim \sigmam \lesssim 1 \cmg$~\citep{Secco}. While S-shaped disks formed by tidal distortions of the stellar light profile are abundantly observed in cluster environments, indicating that they are readily induced by ``baryonic effects,'' these effects are not likely to generate prominent U-shaped warps. Such warps are only formed by a differential force on the disk and its halo, due for example to SIDM drag.  The offset between a satellite galaxy and its subhalo may also be observed directly or statistically with strong lensing \citep{Massey2011,Massey:2017cwf}, but the magnitude of the effect is highly model-dependent (depending strongly on the angular  and velocity dependence of the cross section).  Current limits are $\mathcal{O}(1\cmg)$ for specific non-hard-sphere models \citep{Harvey:2015hha}.  

\vspace{1em} \noindent {\bf Merging Galaxy Clusters \Contact{Nate?}}
\label{sec:merging_clusters}

% intro para
In the previous section, we considered subhalos to be minor merger events onto the main cluster.  Major cluster mergers can probe the nature of dark matter by serving as the biggest ``dark matter colliders'' on account of their high mass and large collision velocities. Dense halos falling together at thousands of km/s provide an environment where the scattering of dark matter particles off each other would have observable effects.  The observable effects vary depending on the dark matter model and the configuration of the merger \citep{Kim:2016ujt}.  %\AHGP{Commented this out only because of redundancy with the previous section} Of course, there are other environments where such scattering effects would be observable. For example, in galaxy cores we expect dark matter particles to experience frequent scattering events. However, cluster mergers are complementary in that baryonic effects such as AGN feedback is degenerate with dark matter scattering effect in isolated cluster cores. In addition, their scattering events would occur at much higher relative velocity (thousands of km/s versus tens). Therefore, together these two probes have a great potential to constrain velocity-dependent models.  
Cluster mergers may also be able to distinguish between particle models that yield frequent scattering with low momentum transfer (as in a long-range force) and those that yield infrequent scattering with high momentum transfer (as with hard sphere or contact scattering) due to their differing phenomenology in the merger environment.  This is in contrast to the halo radial profile and shape constraints discussed in the previous section, for which the energy and momentum transfer rate matters most and for which there is no preferred direction in the problem.

% why LSST discovery is important
The best known example of a colliding cluster system is the Bullet Cluster, which has been frequently studied as a laboratory for SIDM \citep{Randall:2007ph,2017MNRAS.465..569R}. 
%Currently, observations of the Bullet cluster set an upper limit of $\sigmam \lesssim 2 \cmg$.
However, since a cluster merger is an eons-long process of which we have only a single snapshot, the measurement uncertainty is dominated by the scatter in our knowledge of the merger history. While it will remain critical to investigate individual clusters in great detail, the power of LSST lies in systematically analyzing a population of merging clusters with a consistent method, thereby constraining the properties of dark matter.
LSST will contribute to better and more robust constraints not only through the study of already known systems, but also by enabling the discovery of many more merging systems. Because mergers displace plasma from galaxies, they are best discovered by cross-correlation of LSST optically-detected clusters with radio and X-ray surveys \citep{Golovich:2018,Wilber2018}.

%offsets

The first SIDM constraints based on a merging galaxy cluster came from the Bullet Cluster, which was originally identified as an extremely hot X-ray cluster with two galaxy peaks. Higher resolution optical and X-ray imaging revealed a spectacular post-merger system with a clear X-ray cold front and shock. The spatial agreement of the galaxies and mass centroids obtained by weak lensing, and the disassociation of the intra-cluster medium (ICM) led to the constraint $\sigmam \lesssim 2 \cmg$ for hard-sphere scattering \citep{Markevitch2004,Randall:2007ph,2017MNRAS.465..569R,Robertson:2016qef}. Many other dissociative mergers have been found and studied, with roughly similar cross section limits \citep[but with greater systematic uncertainty, \eg][]{bradac2008}. 

After several ``dissociative" mergers had been discovered, ensemble studies of the offsets between dark matter, galaxies, and gas were utilized to drive down the Poisson noise from inference on individual systems. \citet{Harvey:2015hha} modeled 72 subclusters within 30 merging systems to place the strongest constraint on SIDM ($\sigmam<0.47 \cmg$).
The study assumes a simplified drag force model where dark matter behaves similar to the ICM. However, \citet{Wittman:2017gxn} reanalyzed the sample including more comprehensive data. They identified several substantial errors that were driving the result and obtained a revised limit of $\sigmam \lesssim 2\cmg$.

The drag force model applies best to particle models with frequent interaction and low momentum transfer per interaction. In models with infrequent, high momentum transfer interactions (including hard-sphere scattering), dark matter particles may be scattered out of the cluster entirely. (Evaporation also occurs for small-angle scattering, though---see \citealt{Kahlhoefer:2013dca}.) This mass loss may be detected by comparing the mass-to-light ratio $M/L$ of merging clusters with those of non-merging clusters, on the assumption that the merger does not affect the galaxy light. This argument leads to a constraint of $\sigmam \lesssim 1 \cmg$, similar to current constraints on the drag model. However, the assumption that the galaxy light is unaffected is a source of uncertainty here. The LSST discovery of many more merging clusters, with six-band LSST photometry, will help us quantify this source of uncertainty. 

Several Gyr post-pericenter, after a merging cluster has coalesced into a single cluster, SIDM will still create a cored dark matter distribution in the center of the cluster. For $\sigmam \sim 1 \cm^{2} \g^{-1}$, this core is $\roughly 100 \kpc$ (although the baryonic potential can alter the dark matter distribution). \citet{Kim:2016ujt} presented the effect of this on the cluster BCGs up to 10 Gyr post-pericenter. They demonstrated a wobbling in the BCG as it is able to oscillate about the shallow potential for many oscillations. \citet{1703.07365} analyzed a small set of massive clusters and compared the BCG location with a strong lensing based estimate of the gravitational potential centroid. They compared these observations with hydro-CDM simulations to show that the observations suggest a cored dark matter halo in these clusters of $\roughly 10\kpc$. \citet{Harvey:2018uwf} recently studied cluster-scale halos in hydrodynamic simulations, and saw offsets that grew with cross section and halo mass, although with a smaller amplitude than the dark-matter-only simulations of \citet{Kim:2016ujt} implied. LSST will characterize thousands of relaxed clusters that invariably will have undergone a merger in their history. The deep and relatively high resolution imaging LSST will allow for single snapshots of the BCG alignment in every massive cluster and will also detect faint strong lensing streaks in many of these systems.

% Esra added the text below
SIDM properties are sensitive to the separation between the  centroid of the X-ray emitting hot plasma, i.e. intra-cluster medium (ICM), galaxies, and dark matter. Accurate measurements of the X-ray centroid of the X-ray emitting gas in clusters of galaxies requires sub-arcsec imaging with X-ray telescopes. The {\it Chandra} X-ray observatory with 0.5 arcsec FWHM PSF currently provides the most precise location of the ICM. Next-generation high spatial resolution X-ray observatories, \eg, {\textit Lynx} and {\textit AXIS} with much higher throughput, will provide accurate measurements of centers of high-redshift clusters ($z > 1$) in the 2030's and will enable tests of SIDM models over a much larger redshift range.
