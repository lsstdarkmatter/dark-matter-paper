\chapter{Dark Matter Probes}
\label{sec:probes}
\bigskip

Each of the theoretical models described in \secref{theory} produce one or more deviations from the predictions of cold, collisionless, non-interacting dark matter.
These ``probes'' of dark matter physics include: a minimum dark matter halo mass, alterations to halo density profiles, an over-abundance of compact objects, anomalous energy loss, and unexpected correlations in large-scale structure.
In some cases, several distinct physical models of dark matter can be probed by the same (or very similar) observables (\eg, keV-mass thermal dark matter and ultra-light fuzzy dark matter). 
On the other hand, a single probe can manifest itself in a wide range of astrophysical systems (\eg, changes to dark matter profile shape could be observable in the least massive galaxies and the most massive clusters of galaxies).
In this section we do not attempt to provide a comprehensive discussion of all possible astrophysical probes of dark matter physics.
Rather, we focus on specific probes and observables where LSST will have a major impact.
%ADW: should we note something like, "major impact alone or in combination with other observation"?

\section{Minimum Halo Mass}
\label{sec:halo_mass}

The cold, collisionless model of dark matter makes a strong prediction that dark matter halos should exist down to Earth-mass scales (or below) in WIMP and non-thermal axion models \citep{Green:2003un,2005Natur.433..389D,1412.5930}.
Several modifications to the cold, collisionless dark matter paradigm can suppress the formation of dark matter halos on these small scales.
Current observations provide a robust measurement of the dark matter halo mass spectrum for halos with mass $> 10^{10}\Msun$, and the smallest known galaxies provide an existence proof for halos of mass $\roughly 10^8 \Msun - 10^9 \Msun$ \citep{2017MNRAS.467.2019R,behroozi2018,Jethwa:2018,Kim:2017iwr,Nadler:2018,1807.07093}. 
Extending below this halo mass threshold is challenging due to our limited observational sensitivity to the faintest galaxies.
In addition, halos with mass $\lesssim 10^8 \Msun$ are generally expected to host few (if any) stars \citep{1102.4638,1505.06209}, necessitating novel detection techniques that do not rely on the baryonic content of halos.
Here we explore improvements that LSST will make in measuring the faintest galaxies and in probing dark matter halos below the threshold of galaxy formation with stellar streams and strongly lensed systems. We then use these improvements to forecast constraints on specific dark matter models.

\subsection{Constraints from the Smallest Galaxies \Contact{Ethan}} 
\Contributors{Alex, Keith, Mitch, Andrew Pace, Ethan, Yao, Arka, Risa, Mei-Yu, Francis-Yan, Kim, Erik, ...}
\label{sec:smallest_galaxies}

% Some slides from KITP Dark Matter Program in May 2018 with potential ideas:
% https://drive.google.com/open?id=1BhuwyNE7vClIeVV6FhM99QtictTvpQ0p

\vspace{1em} \noindent \textbf{The Threshold of Galaxy Formation}

Galaxies are born and grow within dark matter halos.
To first approximation, galaxies with the largest stellar masses reside within the highest-mass dark matter halos, while fainter galaxies---which are much more numerous---occupy dark matter halos with progressively smaller typical masses; however, the scatter between stellar mass and halo mass is likely large in the low-mass regime (see \citealt{Wechsler:2018} for a recent review).
Therefore, the smallest and faintest galaxies offer a natural place to search for extremely low-mass dark matter halos, which are in turn sensitive probes of dark matter microphysics. Another advantage of probing low-mass dark matter halos using faint galaxies is that we can study their properties in detail, \eg via follow-up spectroscopy (\secref{complementarity}).

The challenge in interpreting observations of faint galaxies is the complex relationship between baryons and halos at this extreme mass scale and the effects of baryonic physics both within subhalos and on subhalo populations as a whole \citep[\eg,][]{DOnghia:2009xhq,Brooks:2012ah,errani2017,Garrison-Kimmel:2017zes,1811.11791,brooks2018}. Nevertheless, probing the extreme faint end of the galaxy luminosity function is valuable both astrophysically and in terms of constraining dark matter models. For example, a driving question for near-field cosmology with LSST is how well we can use the population of Milky Way satellites to constrain the minimum dark matter halo mass necessary for galaxy formation. 
This ``minimum halo mass" depends on the details of reionization and other forms of baryonic feedback that prevent gas from accreting and cooling in low-mass subhalos; however, it might also reflect a cutoff in the subhalo mass function determined by the particle nature of dark matter (\eg, WDM or FDM). In particular, models that produce a cutoff in the matter power spectrum generally suppress the number of subhalos below a characteristic mass threshold (\eqnref{Mhm}). Thus, the existence, abundance, and properties of the smallest galaxies generically lead to constraints on dark matter models that reduce small-scale power.

\vspace{1em} \noindent {\bf Minimum Subhalo Mass Inferred from Milky Way Satellites}

\begin{figure}[t]
\centering
\includegraphics[width=0.775\textwidth]{figures/LSST_Mmin.pdf}
\caption{Forecast for the minimum dark matter subhalo mass probed by LSST via observations of Milky Way satellites. The red band shows the $95\%$ confidence interval from our MCMC fits to mock satellite populations as a function of the true peak subhalo mass necessary for galaxy formation. Note that we marginalize over the relevant nuisance parameters associated with the galaxy--halo connection---including the effects of baryons using a model calibrated on subhalo disruption in hydrodynamic simulations \citep{2018ApJ...859..129N}---in our sampling. We indicate the corresponding constraints on the warm dark matter mass assuming $M_{\rm hm} = \mathcal{M}_{\rm{min}}$ (see \secref{wdm})}\label{fig:satellite_mmin}
\end{figure}

The least luminous galaxies currently known contain only a few hundred stars and have been found exclusively in the inner regions of the Milky Way due to observational selection effects. Although the census of Milky Way dwarf galaxies has grown from $\roughly 25$ to more than 50 in recent years \citep[\eg, with DES;][]{Bechtol:2015, Koposov:2015, Drlica-Wagner:2015}, our current census is certainly incomplete.
For example, the HSC-SSP Collaboration has detected two ultra-faint galaxy candidates in the first $300 \deg^2$ of the survey \citep{1609.04346,1704.05977}; these galaxies are faint and distant enough to have been undetectable in previous optical imaging surveys. HSC is representative of the depth that will be achieved by LSST over half the sky---an area 60 times larger than the current HSC-SSP footprint. Thus, based on the results of SDSS, HSC, DES, etc., several groups have predicted that LSST could detect %at least 20 --- and as many as 50 \EON{check numbers} \ADW{Seems a bit small. I thought Hargis predicted $\roughly250$.} \AHGP{Check out the numbers from Table 1 of Kim, Peter, \& Hargis: we predict 50-600 depending mostly on the radial distribution of satellites in the halo.  The lower number is the most conservative choice, for a population of UFDs concentrated at the center of the halo, and is consistent with Newton et al.  The largest number is what we predict if the hydro sims of satellite disruption are correct.}--- 
tens to hundreds of new low-luminosity Milky Way satellites, mainly at larger distances and fainter luminosities than those accessible with current-generation surveys \citep{Koposov:2008,Tollerud:2008,Hargis:2014,Newton:2018,Jethwa:2018,Nadler:2018,Kim:2017iwr}. 
In addition, novel techniques, such as the use of the correlated phase space motions of stars \citep{1507.04353,1805.02588} or clustering of variable stars \citep{1507.00734} could further expand the sample of ultra-faint galaxies.
LSST observations of Milky Way satellites therefore offer an exciting testing ground for dark matter models; for example, the measured abundance, luminosity function, and radial distribution of Milky Way satellites \emph{already} place competitive constraints on warm dark matter particle mass at the level of 3--4\keV \citep[\eg,][]{Jethwa:2018,Kim:2017iwr}.%, and these constraints will improve as observations continue to probe the low-mass end of the subhalo mass function.

To relate these questions to LSST observations, we have analyzed simulated ultra-faint galaxies as they would appear in LSST WFD coadd object catalogs to quantify LSST's ability to detect nearby satellite galaxies. %, and we have developed a theoretical framework to connect observations of the Milky Way satellite population to the underlying dark matter subhalo population. 
We detect ultra-faint galaxies as arcminute-scale statistical overdensities of individually resolved stars; in ground-based optical imaging surveys, it is often challenging to classify low signal-to-noise catalog objects near the detection threshold as either foreground stars or unresolved background galaxies. LSST will reach depths at which the galaxy counts far outnumber stellar counts, so the search sensitivity for ultra-faint galaxies will largely be determined by our ability to accurately perform star-galaxy separation at magnitudes $24 < r < 27.5$; importantly, our sensitivity analyses include these effects. 
%In detail, we inject many simulated stellar populations into the center of the LSST DESC DC2 simulated data set and 
We find that Milky Way satellites within $300\kpc$ are well-detected with a surface brightness detection threshold of $\mu = 32\ \rm{mag\ arcsec}^{-2}$ \citep{0912.0201} and an absolute magnitude cutoff of $M_V = 0\magn$.

\figref{satellite_mmin} shows the minimum subhalo mass that LSST can probe via observations of Milky Way satellites, obtained by folding our search sensitivity estimates through a cosmological model of the MW satellite population that predicts satellite luminosity functions, radial distributions, and size distributions that agree well with current observations. In particular, we generate many mock MW satellite populations using the model presented in \cite{Nadler:2018} given a ``true" value of the minimum peak subhalo virial mass necessary for galaxy formation, $\mathcal{M}_{\rm{min,true}}$, marginalizing over the remaining galaxy--halo connection parameters. We then perform mock observations of these generated satellite populations using the LSST selection function, and we compare these to the true satellite populations by MCMC sampling $\mathcal{M}_{\rm{min}}$ and the remaining galaxy--halo connection parameters assuming that satellite number counts are Poisson distributed in bins of absolute magnitude (see \citealt{Nadler:2018} for details on the fitting procedure). For each value of $\mathcal{M}_{\rm{min,true}}$, this procedure yields a posterior distribution over the minimum halo mass inferred by LSST observations. The red band in \figref{satellite_mmin} illustrates the recovered $95\%$ confidence interval as a function of $\mathcal{M}_{\rm{min,true}}$, and the blue dot-dashed line indicates the minimum halo mass inferred from known classical and SDSS-detected MW satellites. 
%LSST observations recover the true minimum halo mass at large values of $\mathcal{M}_{\rm{min,true}}$, since all of the predicted satellites are observable in this regime, while smaller values of $\mathcal{M}_{\rm{min,true}}$ yield satellites that do not pass our detection criteria, which prevents the lowest-mass subhalos that host satellites to be detected. 
For small $\mathcal{M}_{\rm{min,true}}$, the $95\%$ confidence level upper bound on the lowest detectable subhalo mass improves by a factor of $\sim 5$ with LSST, from $\sim 5 \times 10^{8}\ \Msun$ to $\sim 10^{8}\ \Msun$; this translates to a lower bound of $\sim 7\ \rm{keV}$ on WDM particle mass (see \secref{combine_probes} for details).

Although we have presented a ``population-based'' forecast for dark matter constraints from LSST-detected ultra-faint satellites, we note that kinematic data obtained by follow-up spectroscopy of newly discovered satellites also offers a powerful probe of dark matter microphysics. We estimate the number of LSST-detected MW satellites that can be spectroscopically confirmed in \secref{spectroscopy}, and we forecast the constraints offered by these stellar velocity dispersion measurements for WDM and SIDM in \secref{combine_probes}.

Further extending the sensitivity of LSST to a power spectrum cut-off on scales smaller than the mass threshold for galaxy formation requires techniques that are independent of satellite luminosity and that can detect subhalos purely through their gravitational signatures. Two examples of such probes are described in the following subsections.


\subsection{Gaps in Stellar Streams \Contact{David H.}}
\Contributors{Nora Shipp, Ting Li, David Hendel, Ana Bonaca, Andrew Pace, Jo Bovy, Sergey Koposov, Nilanjan Banik}
\label{sec:stream_gaps}

Stellar streams, in particular the tidally disrupting remnants of globular clusters, are fragile, dynamically cold systems and are sensitive tracers of gravitational perturbations \citep[][]{2002MNRAS.332..915I,2002ApJ...570..656J,2011ApJ...731...58Y,Carlberg:2012}.
The main track of a stream in 6D phase space is shaped primarily by the Milky Way's global matter distribution while the detailed structure of the stream contains information about small-scale perturbations. 
In particular, a dark matter subhalo passing by the stream will provide a net velocity kick, altering the orbits of the closest stream stars.
The main observable consequence of this interaction is the formation of a gap in the density of stars along the stream; the relative depth and size of the underdensity can be used to infer the time since the encounter and the properties of the perturber \citep{Carlberg:2012, Erkal:2015}. The mass required to produce an observable gap \citep[$10^5-10^6 \Msun$,][]{erkal2016,bovy:2017} is well below the limit where dark matter subhalos are expected to host galaxies. Thus, stellar streams provide one of the most exciting near-field tests of the minimum subhalo mass.

Current constraints on the minimum subhalo mass from stream gaps are limited by the small number of streams that are bright enough that observations can detect density variations at a useful signal-to-noise ratio. Deep and precise LSST photometry is expected to increase the contrast between streams and the contaminating Milky Way field stars, to have improved star-galaxy separation, and to extend much farther down the color-magnitude diagram for known streams, dramatically increasing our ability to detect density variations and thus leading to the identification of less prominent gaps created by low-mass perturbers. Critically, with LSST we move from examining individual gaps into the regime where we can ask questions about subhalo population statistics and their (in)consistency with cold dark matter.
Here we estimate the least massive subhalo that can be detected with LSST observations of gaps in stellar streams.

We consider a mock-stream observed at a Galactic latitude of $b=-60^\circ$ in the $g$- and $r$-band. We assume the stream is old (12\,Gyr), metal-poor ($Z = 0.0002$), thin (1$\sigma$ stream width of 20\,pc), and cold (velocity dispersion of 1\,km\,s$^{-1}$) We generate synthetic photometry of the stream at a given mean surface brightness (within the $1\sigma$ width) and over a range of heliocentric distances from 10 to $40\kpc$.  Simulated stream stars are drawn from a Chabrier IMF \citep{2003PASP..115..763C}, while a synthetic background of Milky Way stars is generated from the \code{Galaxia} model \citep{sharma2011}. 
We add noise to the simulated photometric measurements for both the stream and Milky Way stars in accordance with expectations for LSST \citep{0805.2366}. 
We then select stars in the color-magnitude diagram that are within $2\sigma$ of the theoretical isochrone of the stream's age and metallicity, where $\sigma$ is the magnitude-dependent photometric uncertainty using the same error model. 
% we assume $\sigma$ is no less then 0.02 mag (i.e. set to 0.02 if it's smaller).
We also assume a limiting magnitude to set the depth of the survey, choosing the point where the photometric uncertainty in either band exceeds 0.1 mag. We apply this color-magnitude selection to determine the density of stream stars and background stars.
The depth of the gap from a given subhalo mass is calculated using the theoretical relation derived by \citet{erkal2016}, assuming that the subhalo impact occurred within the past $0.5\Gyr$, moving at $150\kms$, with an impact parameter equal to the perturber's scale radius. Finally, we define a detection as a gap depth that is $5\,\sigma$ above the noise background (the effects of star-galaxy separation are not considered in this calculation).

\begin{figure}[t]
\centering
\includegraphics[width=0.85\textwidth]{figures/streamgap_constraints_3.png}
\caption{\label{fig:streamsurveys} Detection limits for gaps formed from subhalos of different masses using photometry from SDSS (blue) or the 10-year LSST stack (green) as a function of the stream surface brightness.
Shaded regions correspond to a 10-40 kpc distance range, with the lines representing 20 kpc. For streams with surface brightnesses similar to those found in DES, 32--$33 \magn \asec^{-2}$, LSST is expected to probe halo masses two to three orders of magnitude smaller than SDSS and substantially improve the current constraints from Milky Way satellites \citep{Nadler:2018, Jethwa:2018,Kim:2017iwr} and the Lyman-$\alpha$ forest \citep{2017PhRvD..96b3522I}. 
We connect the detected halos to the mass of the warm dark matter particle that would produce a minimum halo of that mass using the relationship determined by \cite{Bullock:2017xww}. Note that the halo mass definition used here is the $z=0$ virial mass; to relate this quantity to the peak subhalo mass used in our warm dark matter constraints, we have assumed the best-case scenario of no tidal mass loss.
}
\end{figure}


\figref{streamsurveys} shows the lowest-mass dark matter subhalo detectable using the 10-year LSST data as a function of stream surface brightness and heliocentric distance. For a stream with a surface brightness of 33 (31.5) $\mathrm{mag}\,\mathrm{arcsec}^{-2}$, LSST is able to detect subhalos with $M_{\rm vir}(z=0) \sim 2 \times 10^7 \Msun$ ($1 \times 10^6 \Msun$) at 20 kpc.
%\footnote{Note that these correspond to $z=0$ virial masses; to convert these into the peak mass that enters our half-mode mass estimate, we measure the mean $M_{\rm{peak}}$--$M_{\rm{vir}}$ relation from the highest-resolution simulation presented in \cite{Mao2015}.} 
As a comparison, we used the same model to calculate the gap detectability using SDSS DR9 photometry. LSST provides $\roughly 3$ orders of magnitude improvement at low surface brightnesses, where most known (and anticipated) streams lie. Crucially, this pushes the minimum detectable halo mass below current constraints from Milky Way satellites \citep[\eg,][]{Nadler:2018,Jethwa:2018,Kim:2017iwr} or the Lyman-$\alpha$ forest \citep[\eg,][]{2017PhRvD..96b3522I}.
To set constraints on \Mhm and \mWDM, we convert $M_{\rm vir}(z=0)$ to $M_{\rm peak}$ using a relation derived from the highest-resolution simulation presented in \citet{Mao2015}.\footnote{We find a mean relationship of: $\log_{10}(M_{\rm peak}) = 0.88 \log_{10}(M_{\rm vir}) + 1.28$.}
We then follow the same formalism as in \secref{smallest_galaxies} to convert from $M_{\rm peak}$ to $\Mhm$ and calculate \mWDM from \eqnref{mWDM}.


Given a gap density detection threshold and a subhalo population, this formalism can be used to predict the number of gaps in a given stream \citep{erkal2016}. Typically, this predicts $\roughly 1$ gap per stream, making it difficult to interpret well-studied individual streams (i.e. Palomar 5 and GD-1). LSST is expected to measure dozens of streams as precisely as Palomar 5 and GD-1 have currently been mapped and will therefore provide a much stronger constraint: at the 10-year LSST depth, \LCDM predicts that we should observe 17 gaps total in the 13 DES streams reported by \cite{2018ApJ...862..114S}. Observing fewer than 6 gaps would be inconsistent with \LCDM at a 99.9\,\% level. 

A single stellar stream is expected to experience several subhalo encounters over its dynamical lifetime. 
Recent strong impacts will result in the observable gaps described above, while weaker encounters will cause less prominent density and track variations.
The effects of ancient impacts will be gradually erased due to the internal velocity dispersion of the stream member stars; however, it is possible to extract statistical information about the impact history of the stream by studying the linear density and track power spectra, both of which are sensitive to the subhalo mass functions.
Impacts from higher-mass subhalos introduce power on large scales, while lower-mass subhalo impacts introduce power on small scales. 
Statistical analyses of the stream density power spectrum have been used to constrain the number of subhalos within the stream radius and the properties of dark matter \citep{bovy:2017, 2018JCAP...07..061B}. 
LSST will allow us to measure the stream density and stream track power spectra at small angular scales that were previously dominated by noise, and \citet{bovy:2017} project that the power spectrum method will be sensitive to subhalos down to mass $10^5 \Msun$. 
In addition, precise measurement of the densities and tracks of multiple streams can be combine to increase statistical power and mitigate systematics from any individual stream.


Depending on their orbits, stellar streams can also be perturbed by the baryonic structures such as the Galactic bar \citep[\eg,][]{erkal2017,pearson2017}, spiral arms \citep{Banik2018}, or giant molecular clouds \citep{amorisco2016}. The resulting gaps may be difficult to distinguish from gaps induced by dark matter subhalos and can bias measurements toward overestimating the number of subhalo impacts on a stellar stream. The only recourse is to carefully examine the streams' orbits to assess these possible confounding factors. Streams with pericenters of $\gtrsim 14\kpc$ should be relatively unaffected by these baryonic factors, and streams on retrograde orbits even less so. In addition, subhalos may also experience extra tidal shocks from the disk, which can alter the number of expected impacts in a given cosmological model \citep[\eg,][]{DOnghia:2009xhq,Garrison-Kimmel2017}. LSST will mitigate both of these issues by examining streams farther from the center of the Galaxy where these effects are lessened.

In this summary we have only considered the density structure of the stream. However, the perturbation that creates the gap necessarily affects the other phase space dimensions as well. The inclusion of these phase space dimensions allows for an almost unique determination of both the subhalo's internal and impact properties for each gap \citep{erkal2015b}. Furthermore, the perturber's effect produces a correlated signal across observables, improving the precision with which the statistical properties of the stream (\eg, power spectrum and cross-correlation of observables) can be used to measure subhalo properties \citep{bovy:2017}. This provides an exciting opportunity for synergy with current and future spectroscopic and astrometric surveys in addition to precise photometric distances and proper motions from LSST itself. Such efforts will greatly aid in the removal of foreground and background contamination, and they will tighten constraints on the stream progenitor's orbit and provide better measurements of the perturber's mass and size. See \secref{complementarity} for a discussion of some complementary science programs.



\subsection{Strong Lensing: Substructure and Line-of-Sight Halos\Contact{Chris F.} }
\label{sec:stronglens} 
\Contributors{Cora, Chris, Francis-Yan}

Strong gravitational lensing is one of the most powerful probes of dark matter halos beyond the Local Goup. 
Gravitational lensing directly probes the total mass distribution that a light ray encounters and does not require that mass to be luminous or baryonic.
Therefore, an analysis of lensing signals can be used to measure the presence, quantity, and mass of subhalos in massive galaxies and small isolated halos along the line-of-sight.  
The discovery of low-mass dark matter halos is possible even at cosmological distances, where the flux of any luminous material associated with the halos would fall below the detection limits of typical observations.  
Thus, the gravitational lensing approach is highly complementary to Local Group observations.

The (sub)halo-detection techniques described below utilize strong gravitational lensing, in which a massive foreground object bends the light from a background galaxy to produce multiple images of the background object.  
If the emission from the background object is dominated by a single point-like component, such as a quasar or other AGN, the lens system will contain multiple images of that component (\eg, \figref{stronglens_examples}a).
Typically these quasar lens systems consist of two or four images, creating ``doubles'' and ''quads'' respectively. 
If, on the other hand, the background object is dominated by stellar emission, then the lensed emission is in the form of tangentially stretched arcs or a full Einstein ring that surrounds the lensing galaxy (\eg, \figref{stronglens_examples}b).  
In both cases, substructure in the main lensing galaxy and small line-of-sight halos create small perturbations to the lensed images.

As will be described in detail below, there are three main techniques for detecting the presence of dark (sub)halos using strongly lensed systems: analysis of flux-ratio anomalies in lensed quasar systems,
gravitational imaging for lensed galaxy systems, and power spectrum approaches. 
Improved constraints on dark matter properties via these measurements will require: (1) a much larger samples of lens systems, and (2) follow-up observations with high-resolution imaging and spectroscopy.
LSST will play a critical role by increasing the number of lensed systems from the current sample of hundreds to an expected samples of thousands of lensed quasars \citep{O+M10} and tens of thousands of lensed galaxies \citep{Collett2015}.
The vast increases in sample sizes will provide much stronger statistical constraints on dark matter models than are currently possible (\eg, \figref{lensing_wdmlim_vs_nlens}).
The study of lensed systems will also require coordination with other facilities, namely space-based observatories, large ground-based telescopes with adaptive optics systems, ALMA, and very-long-baseline radio interferometry (see \secref{SLcomplement}). These facilities provide the milliarcsecond-scale angular resolution that is required to push the (sub)halo detection sensitivity into unexplored mass regimes.

\begin{figure}
    \centering
    \includegraphics[width=0.4\textwidth]{2045_vla_dec96_x.png}    
    \includegraphics[width=0.44\textwidth]{Clone_labeled.png}
    \caption{Examples of two gravitational lens systems that exhibit perturbations due to (potentially unseen) halos.  {\bf (a)} Radio-wavelength imaging of a quasar lens system, B2045, that has one of the strongest flux-ratio anomalies known.  Component B should be the brightest of the three close images and instead it is the faintest. Figure from \citet{Fassnacht++99}
    {\bf (b)} HST imaging of the ``Clone'' \citep{2009ApJ...699.1242L}, showing that the long lensed arc is split by the presence of a perturber, in this case galaxy G4.  Note that the location and mass of G4 could have been determined {\em even if G4 were purely dark}.  Figure from \citet{Vegetti_2010_1}.}
    \label{fig:stronglens_examples}
\end{figure}

\begin{figure}
    \centering
    \includegraphics[width=0.75\textwidth]{wdm_constraints_yh.png}
    \caption{ \label{fig:lensing_wdmlim_vs_nlens} Projected $2\sigma$ constraints on the WDM particle mass as a function of the number of strong lens systems that achieve a given (sub)halo mass detection threshold, under the assumption that CDM is correct. Exisiting Lyman-$\alpha$ forest constraints are shown with a dashed horizontal line \citep{2017PhRvD..96b3522I}. Figure courtesy of Y. Hezaveh.
%\ADW{These LyA constraints are different from the ones in the dwarf and stream section.} \AHGP{In the LyA paper, the limit is 3.3keV with the loosest prior on the thermal evolution of the IGM, more like 5.3keV with standard priors. The difference in the two sections almost certainly comes from prior preferences.}
}
\end{figure}


\vspace{1em} \noindent {\bf Flux-ratio Anomalies}

The presence of clumpy (dark) matter, whether within the main halo of the primary lens or along the line of sight, will perturb the gravitational potential of a strong lens system.
One of the effects of these perturbations is to change the magnification of the lensed images of a background AGN.
The angular scales over which the perturbations are important depend on the mass the perturber, so the presence of a small (sub)halo will typically affect only one of the lensed images and, thus, will change the relative fluxes of the images.
Furthermore, because the image magnification depends on the second derivatives of the gravitational potential, this method is, in theory, sensitive to smaller-mass structures than the gravitational imaging approach described below.\footnote{Indeed, flux-ratio anomalies can be produced by stars in the lens galaxy, through the microlensing phenomenon discussed below.}

The utility of this effect was first presented in \citet{Mao:1998aa}, which considered the effects of spiral arms in the lensing galaxy on the flux ratios of the lensed images, and for many years this was the only lensing technique used to investigate the presence of substructure in massive galaxies.
The approach is to describe the lensing galaxy with a relatively simple smooth single halo model.
These simple models are nearly always capable of fitting the observed positions of the lensed images to within the observational errors.
At that point, any deviation between the model-predicted image fluxes and the observed fluxes could be ascribed to some type of non-smooth / clumpy mass, either in the lensing galaxy or along the line of sight.
At optical and near-IR wavelengths, there are often significant differences between the predicted and observed image fluxes.
However, these perturbations are most likely to be produced by stars in the lensing galaxies, a process known as microlensing, and thus optical and near-IR fluxes are not informative in terms of the statistics of dark matter halos.
What is required is to observe at wavelengths at which the angular size of the emitting region in the background source is large compared to the micro-arcsecond scales at which stars produce their effects.  
Until recently, this meant observing lensed quasar systems at radio or mid-IR wavelengths, which vastly reduced the available sample sizes.

A seminal paper by \cite{Dalal:2002aa} used the statistics of observed flux-ratios in a sample of seven lens systems to place limits on the substructure fraction in the lensing galaxies, i.e., the percentage of the lens mass that is composed of clumpy structures, in the $10^6 - 10^9 \Msun$ range.
The small sample size was set by the number of radio-loud systems that was known at the time and the one lens system with a usable mid-IR data set.  
Because lensed radio-loud AGN are rare, and ground-based high-resolution mid-IR observations are extremely difficult, the sample size only increased by a few lenses over the next decade.  
In contrast, forecasts based on forward modeling simulations indicate that $\gtrsim$100 well-constrained flux-ratio systems are needed to provide 2$\sigma$ constraints of $10^{7.2} - 10^{7.5} \Msun$ for the half-mode mass in a WDM scenario, corresponding to a $\sim$5--6~keV thermal relic mass \citep{Gilman++18}.
Therefore, large increases in sample sizes are required.
The two most promising paths forward are to obtain large lensed quasar samples with LSST and then follow up with either high-resolution mid-IR imaging with JWST, or IFU spectrographs on ELTs or JWST.  The second technique takes advantage of the fact that in lensed AGN, the narrow-line region surrounding the central AGN is larger than the microlensing scale, even though the broad-line region and the source of the continuum emission are not.  Therefore, with high-resolution IFU observations, the narrow-line emission from each lensed image can be spatially resolved, thus providing the required microlensing-free flux ratios \citep{MoustakasMetcalf03, Nierenberg++14, Nierenberg:2017vlg}.

Deep high-resolution imaging in the optical or infrared is also necessary to address possible systematics in the flux-ratio technique.  
Investigations using Keck adaptive optics imaging of radio loud lenses have shown that, in some cases, the observed flux-ratio anomalies can be explained by baryonic structures in the lensing galaxy, namely edge-on stellar disks rather than dark matter halos \citep{Hsueh++2016, Hsueh++2017}.
These baryonic effects were also seen in simulated data \citep{Gilman++2017, Hsueh++2018}.
These studies indicate that a lack of knowledge about the baryonic structure of the lensing galaxy may lead to an overestimate of the amount of clumpy dark matter in the lens or along the line of sight.
With a sample of thousands of quasar lenses expected from LSST, it will be possible to select systems where baryonic effects are minimized.

\vspace{1em} \noindent {\bf Gravitational Imaging}

The presence of a massive peturber along the line of sight can change the shape of lensed emission. 
This effect can be utilized in strong lens system in which the background object is a galaxy that is lensed into long arcs or a complete Einstein ring.
Small (sub)halos that are close in projection to the lensed emission can distort arc shape to a degree that can be detected by high-resolution imaging observations.
This ``gravitational imaging'' technique was proposed by \cite{Koopmans:aa} and further refined by \citet{Vegetti:2008aa,Vegetti:2009aa}.  The size of this effect depends on the mass of the perturber and its projected distance from the lensed arcs, with more massive and closer perturbers having larger effects.
 
The first application of the gravitational imaging technique to real data was for the ``Clone,'' a system for which the primary lensing halo is a compact galaxy group \citep[\figref{stronglens_examples},][]{2009ApJ...699.1242L, Vegetti_2010_1}.
 In this system, the long lensed arc is broken and split at the location of the peturber, which in this case is a satellite galaxy in the group with a mass of $\roughly 10^{10} \Msun$ \citep[][]{Vegetti_2010_1}.  This massive galaxy located right on the arc produced an effect that could be seen by eye in high-resolution HST imaging.  Lower-mass detections were subsequently made using HST \citep[$\roughly 10^9 \Msun$;][]{Vegetti_2010_2}, Keck adaptive optics \citep[$\roughly 10^8 \Msun$;][]{Vegetti_2012}, and ALMA mm-wave interferometry \citep[$\roughly 10^8 \Msun$;][]{Hezaveh_2016ltk}.  
<<<<<<< HEAD
 Note that the masses reported in these papers usually assume a truncated mass distribution (\eg, a pseudo-Jaffe) or are explicitly given as mass contained within radii of, \eg, 600\pc, to better match dwarf galaxy measurements made within the Local Group.  Multiplying these values by a factor of 10 gives roughly the expected virial mass of their host halos.
=======
 Note that the masses reported in these papers usually assume a truncated mass distribution (e.g., a pseudo-Jaffe profile) or are explicitly given as mass contained within radii of, e.g., 600\pc, to better match dwarf galaxy measurements made within the Local Group.  Multiplying these values by a factor of 10 gives roughly the expected virial mass of their host halos.
>>>>>>> 35e0501e7109ff1d38838de6112c78652ef1a253
 
The implications for the nature of dark matter from the gravitational imaging technique come from comparing the number of detected halos to those predicted by various dark-matter models.  
For this reason, one of the strengths of the technique is that {\em non-detections} are as valuable as detections, and can be especially powerful at low masses where CDM models predict a large number of halos.
This analysis relies on an understanding of the lowest mass that can be detected at each location in the lens system \citep[\eg,][]{Vegetti2014, Hezaveh_2016ltk, Ritondale++18}.
 
Nearly all previous inferences on dark matter from gravitational imaging have considered solely the expected and measured effects of subhalos within the main halo of the primary lensing galaxy \citep[\eg,][]{Vegetti:2009aa, Vegetti_2012, Vegetti2014, Hezaveh_2016ltk}.
However, an additional perturbation signal is provided by the presence of halos along the line of sight.
An analysis of simulated data has shown that the signal from line-of-sight structures is significant even for lower redshift lenses and is the dominant contribution to any lensing signal for higher redshifts \citep{Keeton:2002ug,Despali++18}.
The line-of-sight structures may very well be a cleaner probe of dark-matter properties than substructures in the lensing galaxies.
This is because the line-of-sight halos are unlikely to have been tidally stripped and thus their measured masses reflect their true masses.
The techniques for including the line-of-sight signal have been developed and applied to recent analyses \citep{Ritondale++18}. 
 
For the relatively high (sub)halo masses that have been probed so far, $\gtrsim 10^9\Msun$, there is little difference between the predictions of CDM and models with a mass cutoff (\eg, WDM).
Therefore, even analyses of $\roughly10$-lens samples have not achieved the statistical precision to distinguish between dark matter models \citep{Vegetti2014, Ritondale++18}.
What is urgently needed is both to increase the sample sizes and, more importantly, to probe further down the mass function.
The mass-detection limit for gravitational imaging is set by three properties of the observations: (1) the signal-to-noise ratio, (2) the angular resolution of the imaging data, and (3) the surface-brightness structure of the lensed background galaxy.  This last point arises because it is easier to detect small astrometric shifts if there are strong gradients in the surface brightness, as opposed to a smooth light distribution.
These properties lead to the need for sensitive high resolution observations of the large samples of appropriate lenses that LSST will provide.
The high-resolutions observations can come from ELTs, which should provide milliarcsecond-scale angular resolution currently only available from VLBI radio observations.
For the subset of LSST lenses that are radio loud, VLBI and ALMA observations will provide excellent complementarity.

\vspace{1em} \noindent {\bf Small-scale Structure Power Spectrum}
\begin{figure}[t]
\centering
\includegraphics[width=0.6\textwidth]{Fisher_space_Pk_SIDM_rev.pdf}
\caption{Fisher forecast for the substructure convergence power spectrum in three logarithmic wavenumber bins. We consider here observations with the wide-field camera 3 (WFC3) aboard the Hubble Space Telescope (HST) using the F555W filter, resulting in a point-spread function FWHM of $0.07$ arcsec. The source is placed at $z_{\rm src}=0.6$ with an unlensed magnitude $m_{\rm AB}=24$. The error bars show the $1$-$\sigma$ regions, while the green rectangles display the sample variance contribution within each bin. We conservatively assume that only half of each orbit is available for observation. The blue solid line shows the fiducial substructure power spectrum model used in the forecast, which corresponds to a CDM population of subhalos modeled with truncated NFW profiles. The dotted magenta line shows the power spectrum for SIDM, assuming a subhalo core size equals to $70\%$ of the scale radius. For comparison, the orange dashed line shows the substructure power spectrum for a thermal relic warm DM with mass of 3.5 keV \citep{Viel:2013}. Figure adapted from \cite{Cyr-Racine:2018htu}. \label{fig:pksub_fisher}}
\end{figure}

While gravitational imaging can detect highly significant and well-localized perturbers along lensed arcs and Einstein rings, less massive perturbers or those located farther away from lensed images typically lead to observational signatures that are too subtle to be detected individually. However, the large number of such perturbers, both as subhalos within the lens galaxy and as field halos along the line of sight, means that their collective effect might be detectable at the statistical level \citep[\eg,][]{Birrer2017}. The power spectrum of the lensed deflection field is a particularly powerful quantity for capturing the aggregate behavior of lensing perturbers. This approach was proposed in \cite{Hezaveh_2014}, and further expanded in \cite{Rivero:2017mao}, \cite{Chatterjee_2017}, and \cite{Cyr-Racine:2018htu}. 
Furthermore, \cite{Daylan:2017kfh} proposed a technique to constrain the statistical properties of dark matter subhalos in the lens galaxies by studying the joint perturbations of unresolved subhalos.

A key advantage of this power spectrum approach is that it describe the effect of perturbers in terms of a \emph{spatial fluctuation} basis instead of the more traditionally used \emph{mass} basis. The power spectrum directly captures the spatial scales on which perturbers influence the lensed images without having to invoke the notion of (sub)halo density profile, the latter of which is usually required to map from the perturber mass space to the resulting spatial deflection field. As such, the power spectrum is a natural language to describe the collective effects of small lensing perturbers. 

To develop intuition about which dark matter properties could be probed from measurement of this new lensing statistic, \cite{Rivero:2017mao} developed a general formalism to compute from first principles the convergence power spectrum for different populations of subhalos (not yet including line-of-sight perturbers). The authors pointed out that this power spectrum can be mainly described by three quantities: a low-wavenumber amplitude, that depends on the subhalo abundance and on specific statistical moments of the subhalo mass function; on a turnover scale, that probes the truncation radius of the largest subhalos in the system; and on a higher-wavenumber ($k\gtrsim 1 \kpc^{-1}$) slope, that probes a combination of the subhalo inner density profiles and of a possible cutoff in the primordial matter power spectrum. These theoretical findings were then confirmed numerically in \cite{Brennan:2018jhq} using a semi-analytic galaxy formation model, and in \cite{Rivero:2018bcd} using high-resolution $N$-body simulations. Measurements of the power spectrum promise a wealth of information about the behavior of dark matter on small scales.

Several challenges need to be addressed to fully enable the constraining abilities of power spectrum measurements. Most importantly, the degeneracy between the possibly complex brightness profile of the source and the statistical effects of the lensing perturbers needs to be accurately explored. Also, the importance of line-of-sight structure remains to be properly quantified, and the effect of lens galaxy light and other luminous foregrounds on the power spectrum inference needs to be better understood. Finally, since instrumental artifacts such as a mismodeled point-spread function or camera sensitivity could potentially mimic a power spectrum signal, it is likely that these effects would have to be reconstructed at a higher precision than what is typically done for gravitational imaging. 

Thus far, measurement of the lensing power spectrum has been attempted by \cite{Bayer:2018vhy}, and an upper limit on its amplitude was derived using HST archival data. The currently known samples of galaxy-galaxy lenses numbers in the few dozens, and LSST is expected to increase this number several-fold as mentioned above. High-resolution follow up using either space-based or AO-enabled ground-based observatories will be required to measure the power spectrum from these targets and thus probe small-scale structure in a new way (\secref{SLcomplement}). \citet{Cyr-Racine:2018htu} has performed detailed forecasts for the sensitivity of different observational scenarios to the perturber power spectrum for lenses of the type that LSST is expected to discover at optical wavelengths (\figref{pksub_fisher}). It was found that images only a factor of a few deeper than what is currently typically available \citep[\eg, from the SLACS sample][]{Bolton2008} could be sufficient to detect the overall amplitude of the lensing power spectrum. On the other hand, constraining the slope at larger wavenumbers, which could help distinguish between WDM and CDM (\figref{pksub_fisher}), would require much deeper imaging.

\subsection{Minimum Halo Mass and Density Profile Measurements: Joint Analysis  \Contact{Francis-Yan} }
\label{sec:combine_probes} 
\Contributors{Francis-Yan, Alex, Ethan, ...}
 
As stated above, a cut-off in the matter power spectrum is related to early universe kinematics of the dark matter particle or the interactions of dark matter particles with a relativistic species. This cut-off directly feeds into both the subhalo mass function and the luminosity function of Milky Way satellites. Importantly, it also feeds into the internal properties of subhalos with masses near this threshold. Indeed, the power spectrum cut-off delays halo formation on mass scales near and smaller than the cut-off, hence lowering the concentration of these objects at fixed halo mass \citep[\eg,][]{Dunstan:2011bq}. Since dark matter self-interaction (\secref{sidm}) can also have a large impact on the central densities of small halos, combining information from minimum halo mass and density profile measurements (\secref{profiles}) could jointly constrain the presence of a cut-off and of a non-vanishing $\sigma_{\rm SIDM}$, hence simultaneously probing the cold and collisionless tenets of the CDM paradigm. To illustrate this, we estimate in this section the potential sensitivity of an analysis combining kinematic measurements of LSST-discovered Milky Way satellites with the minimum halo mass forecasts from \figref{satellite_mmin}. While not discussed here, we note that both stellar stream gaps and strong lensing could also be used for such joint analysis since they in principle have some sensitivity to internal properties of small halos as well. 

\vspace{1em} \noindent {\bf Joint impact of a cut-off and self-interaction on the central densities of MW dwarfs}

As a demonstration of the power of LSST to probe both a power spectrum cut-off and dark matter self-interaction, we focus here on a simplified summary statistic which captures the main essence of LSST's constraining power. The reader should keep in mind that a more detailed analysis using the full complexity of the LSST data set (and its spectroscopic follow-up) could unveil even more information about dark matter physics. Specifically, we consider here the impact of self-interaction or a cut-off on the cumulative number of satellites above a given luminosity threshold that have stellar velocity dispersion within their half-light radius above a minimum value $\sigma_{\star,\rm lim}$, $N_{\rm sat}(L_\star>L_{\rm lim}, \sigma_\star> \sigma_{\star,\rm lim})$. Here, we use the stellar velocity dispersion, $\sigma_\star$, as a probe of the subhalo's central density.

The connection between the central density of subhalos and the self-interaction cross section or cut-off in the power spectrum can be modeled in the following way to probe fundamental physics. For simplicity, we parameterize the cut-off in the power spectrum using the thermal WDM mass, $\mWDM$ \citep[\eg,][]{Bode:2000gq}, but note that other physics such as interactions with a relativistic species \citep[\eg,][]{Boehm:2004th,Cyr-Racine:2015ihg} could also cause a small-scale suppression of power. It is well-known that dark matter self-interaction creates constant density cores in the subhalos \citep{Spergel:1999mh}, which usually lowers the central density as compared to NFW halos. In the limit of large cross sections or significant subhalo mass loss, the self-interactions could also lead to core collapse and an increase in the subhalo central density \citep{Balberg:2002ue,Ahn:2004xt,Nishikawa:2019lsc}. 

In the absence of significant self-interaction, the power spectrum cut-off affects the central density through a modification of the subhalo concentration-mass relation \citep{Dunstan:2011bq,schneider2012,Lovell:2013ola,Bose:2016irl}. We adopt the following form for this modification  
\begin{equation}
c(M_{\rm vir}; \mWDM) = c_{\rm cdm}(M_{\rm vir})\left(\frac{M_{\rm vir}}{10^{12} \Msun} \right)^{\Delta\alpha(\mWDM)}\,,
\end{equation}
where $M_{\rm vir}$ is the subhalo virial mass, $c_{\rm cdm}$ is the subhalo concentration in the standard CDM case \citep{Moline:2016pbm}, and $\Delta\alpha$ is a power law index that depends on the power spectrum cut-off. In this small self-interaction cross section limit, we assume that the core size is negligible and the density profile is of the NFW form. Note that stellar feedback can change this statement, so there is the possibility of degeneracy between feedback and self-interactions. 

In the opposite limit of large cores created by self-interactions, the central density $\rho_0$ of the subhalo can be written as \citep{Nishikawa:2019lsc}
\begin{equation}
\rho_0=\rho_{\rm s} f(t/t_0)\,
\end{equation}
where $t$ is time elapsed since infall, $t_0 = a(\sigma_{\rm SIDM}/m)\rho_{\rm s} v_0$ with $v_0^2=4\pi G \rho_{\rm s} r_{\rm s}^2$ and $a=\sqrt{16/\pi}$ for a hard-sphere interaction~\citep{Balberg:2002ue}. Here, $\rho_{\rm s}$ and $r_{\rm s}$ are the NFW density and scale radius parameters, respectively. The function $f$ encodes the evolution in time of the subhalo's central density in the presence of self-interactions, which includes its initial suppression due to core formation, and its subsequent increase due to the onset of the gravo-thermal instability \citep{Ahn:2004xt}. Importantly, the onset of this latter phase depends on the satellite's orbital history, with highly tidally stripped subhalos reaching it on a much shorter timescale than field halos \citep{Nishikawa:2019lsc}. To incorporate these results we make the further assumption that the tidal evolution of subhalos is not highly sensitive to dark matter particle physics. Current simulations tend to support this point of view, although it is likely that the stripped mass fraction will be somewhat larger for subhalos that have lower central densities due to either a cut-off in the power spectrum or self-interactions \citep{Lovell:2013ola,Dooley:2016ajo}. 

We adopt the following simple model for relating the mean stellar dispersion $\bar{\sigma}_\star$ to the central density $\rho_0$ \citep{Wolf:2009tu}
\begin{equation}
\bar{\sigma}_\star = 1 \kms \sqrt{\frac{\rho_0}{0.1 \Msun/{\rm pc}^3}} \frac{R_{\rm h}}{50 {\rm pc}}\,,
\end{equation}
which is valid as long as the core radius $r_{\rm c}$ is much larger than the project half-light radius $R_{\rm h}$. In the opposite case where $r_{\rm c} < R_{\rm h}$, we modify this relation by putting an upper bound on $f(t/t_0)$ at a value given by $1/(x_{\rm h}(1+x_{\rm h})^2)$, where $x_{\rm h} = R_{\rm h}/r_{\rm s}$, and where the core radius is defined by the equation $\rho_{\rm NFW}(r_{\rm c}) = \rho_0$.   

To map the present-day mass of our subhalos to their luminosities, we combine the zoom-in simulations presented in \cite{Mao2015} with the subhalo--satellite galaxy model presented in \cite{Nadler:2018} to obtain the probability that a subhalo of present-day virial mass $M_{\rm{vir}}$ hosts a satellite of luminosity $L_\star$ and stellar dispersion $\sigma_\star$. The mapping from subhalos to satellites includes a prescription for hydrodynamic effects such as enhanced subhalo disruption due to a central galactic disk, the galaxy formation threshold due to reionization, and a flexible model for the relationship between luminosity and subhalo peak circular velocity. To characterize $P(L_\star|M_{\rm{vir}})$, we sample from the posterior distribution of model parameters from the fit to classical and SDSS-detected satellites in \cite{Nadler:2018}, generate a large number of satellite population realizations for each MW host halo, and fit the resulting $P(L_\star|M_{\rm{vir}})$ relation with a log-normal distribution. We also use these simulated satellite populations to perform a large number of mock LSST observations to obtain the probability distribution of our summary statistic $N_{\rm sat}(L_\star>L_{\rm lim}, \sigma_\star> \sigma_{\star,\rm lim})$.

\begin{figure}[t]
\centering
\includegraphics[width=0.6\columnwidth]{figures/SIDM_WDM_figw_coll.pdf}
%\includegraphics[width=0.7\columnwidth]{figures/wdm_sidm.pdf}
\caption{\label{fig:sidm_wdm} Projected joint sensitivity to the WDM mass and SIDM cross section from LSST observations of dark matter substructure. 
The red region is already ruled out at 95\% confidence level by current observations of the Milky Way satellite population, while the left-most dashed vertical line corresponds to current constraints from the \Lya forest \citep{2017PhRvD..96b3522I}.
LSST will be sensitive to deviations from the standard CDM scenario through several different probes.
Constraints on the minimum WDM mass are shown with dashed vertical lines for observations of LSST-discovered strong lenses and stellar streams.
The discovery additional Milky Way satellites and their subsequent spectroscopic follow-up will probe the region in blue.
The lighter blue region with large SIDM cross section may not be probed by Milky Way satellite galaxies due to halo core collapse in this regime \citep{Nishikawa:2019lsc}. 
We caution that the exact shape of this latter region will depend on the amount of tidal disruption that subhalos experience.  
The top axis displays the corresponding half-mode mass as per Eq.~\eqref{eqn:Mhm}. 
It is understood that $\sigma_{\rm SIDM}$ stands for the self-interaction cross section evaluated at velocities relevant for Milky Way satellite galaxies ($v_{\rm rel}\sim5-50 \kms$). }
\end{figure}

Using the mapping from $L_\star$ and halo mass to $M_{\rm vir}$, our summary statistic can be computed using the expression:
\begin{equation}\label{eq:Nsat_above_thresh}
 N_{\rm sat}(L_\star>L_{\rm lim}, \sigma_\star> \sigma_{\star,\rm lim}) =  \int_{L_{\rm lim}} dL_\star \int_{\sigma_{\star,\rm lim}} d\sigma_\star \int dM_{\rm vir} \frac{dn}{d M_{\rm vir}} P(L_\star|M_{\rm vir})
 \delta\left(\sigma_\star-\bar{\sigma}_\star(\rho_0,c(M_{\rm vir};\mWDM))\right),
\end{equation}
where $dn/dM_{\rm vir}$ is the redshift zero subahlo mass function (which depends on $m_{\rm WDM}$) and $\delta$ is the Dirac delta function. Given values of $m_{\rm WDM}$ and $\sigma_{\rm SIDM}/m$, we can use Eq.~\eqref{eq:Nsat_above_thresh} to compute the number of MW satellites observable with LSST that have stellar dispersion above our chosen threshold. As in \secref{smallest_galaxies},
we take $M_V=0$ mag and $\mu=32$ mag/arcsec$^2$ as our detection threshold for LSST, and set $\sigma_{\star,\rm lim}=2.6 \kms$. This latter choice is driven by the minimum stellar dispersion value obtained in our mock observation of satellite galaxies passing the LSST detection threshold as described above, assuming standard CDM. Since measuring stellar dispersion will require spectroscopic follow-up of LSST-discovered satellite galaxies, we fold in our analysis the probability that a given target can be followed-up with 30-meter class telescopes given its luminosity and heliocentric distance, as provided in \secref{spectroscopy}. 

The resulting projected joint sensitivity to the WDM particle mass and the SIDM cross section are shown in \figref{sidm_wdm}. 
The red region to the left of the figure is already excluded by observations of known classical and SDSS-discovered Milky Way satellites. 
The dashed vertical lines show current constraints from the \Lya forest \citep{2017PhRvD..96b3522I} and the projected sensitivity of strongly lensed systems and stellar streams discovered by LSST.
In blue, we show the region of SIDM-WDM parameter space that would be probed by LSST+spectroscopic measurements of the Milky Way satellite population. 
In the white region at low SIDM cross sections, the central core caused by self-interaction is too small to significantly affect the dynamic of the satellites, while the light blue region at high cross sections corresponds to tidally-stripped subhalos that undergo core collapse and have similar stellar dispersion to their CDM counterparts. We note that this region is more uncertain since it depends on the amount of tidal disruption that subhalos experience. This latter region does not extend to arbitrarily high cross sections as complete gravothermal collapse and subhalo evaporation would prevent these models from being viable.  We thus see that spectroscopic follow-up of LSST-discovered satellites could significantly improve our knowledge of dark matter physics in the prime parameter space corresponding to $\mWDM \sim 5-10 \keV$ and $\sigma_{\rm SIDM}/m \sim 0.1-2 \cmg$.

\subsection{Halo Profiles} 
\label{sec:profiles}

The standard CDM model predicts that dark matter halos should be cuspy, asymptoting to high central densities.
This results from the inability of collisionless dark matter to redistribute kinetic energy, and is born out in numerical simulations which give rise to a family of cuspy halo profiles \citep[\eg, the NFW profile,][]{Navarro:1996gj}.
If dark matter is able to interact through scattering or the exchange of some light mediator (see \secref{sidm}), then the density of halos could instead flatten out to produce dark matter ``cores'' \citep{Spergel:1999mh}.
These interactions can also lead to an isotropization of dark matter velocity distribution, leading to more spherical halos \citep{Peter:2013}.
Thus, measurements of the radial density profiles and shapes of dark matter halos are sensitive to the microphysics governing dark matter self-interactions.
Here we explore the contributions that LSST will make towards measuring the profiles of dark matter halos in isolated small galaxies and clusters of galaxies.
We highlight these systems because they reside at opposite extremes of the galaxy mass spectrum where dark matter dominates over baryonic processes that can also alter the shapes of halos.

\subsection{Dwarf Galaxies as Lenses \Contact{Yao}}
\label{sec:halo_profile_group}
\Contributors{Yao-Yuan Mao, M.\ James Jee, Alex Drlica-Wagner, J.\ Anthony Tyson, Annika H.\ G.\ Peter, Chihway Chang, Rachel Mandelbaum, Manoj Kaplinghat}

Dwarf galaxies ($M_\star \lesssim 10^{9} \Msun$) provide the best visible tracers of low-mass dark matter halos. 
The relatively low baryonic content makes dwarf galaxies sensitive probes of  dark matter physics through the shape of their dark matter halo profiles. 
In particular, the ``core-cusp'' problem in dwarf galaxies has been cited as one of the most significant challenges to CDM \citep[\eg,][]{2010AdAst2010E...5D,Bullock:2017xww}.
The standard CDM model predicts that dark matter halos should have steeply rising (``cuspy'') central densities in contrast to the shallower (``cored'') mass profiles that are observationally inferred for many dwarf galaxies.  
Evidence for cored profiles exists for Milky Way satellite galaxies from kinematic and theoretical studies \citep[\eg,][]{Walker:2009, 2012ApJ...759L..42P}, and is stronger when one studies the inner density profiles of dwarf galaxies based on high-resolution neutral hydrogen surveys \citep[\eg,][]{Begum:2008,Hunter:2012,Cannon:2011,Oh:2015}. 
Many of these observations show inferred central slopes of the dark matter density profile, $\rho(r) \sim r^{-\gamma}$, that are significantly shallower ($\gamma \approx 0$--$0.5$) than the CDM prediction $\gamma \approx 0.8$--1.4 \citep{Navarro:2010}.

A wide range of solutions to the core-cusp problem have been proposed including observational, astrophysical, and dark matter explanations.
From a dark matter perspective, SIDM can significantly suppress the the central density of halos.
A self-interaction cross-section of $\sigma / m_\chi \sim 1 \cmg$ can explain the diversity of rotation curves seen in low-mass spiral galaxies \citep[\eg,][]{1504.01437,2017PhRvL.119k1102K,Tulin:2017ara}.
In addition, ultra-light or fuzzy dark matter has also been suggested as a possible solution to the core-cusp problem through the formation of uniform density solitonic cores \citep[\eg,][]{1502.03456,Hui:2017}. 
However, baryonic feedback remains a major complication for interpreting central density profile measurements in a dark matter context \citep{1996MNRAS.283L..72N,2005MNRAS.356..107R,2008Sci...319..174M,2012MNRAS.421.3464P,Madau:2014,Read:2016}. 
If dwarf galaxies form enough stars, energy from SN explosions can flatten the profiles of dark matter and baryons; however, if too many stars are formed, the excess baryonic mass can have the opposite effect of steepening the slope of the central density profile \citep{Bullock:2017}.
Technical challenges in implementing multi-phase gas and baryonic physics make it difficult to directly address and calibrate baryonic predictions based on hydrodynamical simulations \citep{Tollet:2016,1611.02281,Sawala:2016}.
However, one key prediction is that the creation of cores will be sensitive to the exact star formation history \citep[\eg,][]{governato2012,dicintio2014,onorbe2015,Read:2016,read2018,1811.11768,2019MNRAS.tmp....3R}.
Thus, robust measurements of both the stellar and dark matter mass of dwarf galaxies is essential to investigate the effect of baryonic feedback on the central dark matter density.
In addition, it has been argued that significant observational and astrophysical systematics, such as beam smearing, center offsets, inclinations, and non-circular motions can bias central density measurements toward flatter profiles \citep[\eg,][]{astro-ph/0006048,2004ApJ...617.1059R,2008AJ....136.2761O,2016MNRAS.462.3628R}. 
Thus, accurate independent measurements of dwarf galaxy density profiles are critical.

LSST can provide joint statistical measurements of both the central density and stellar content of dwarf galaxies. 
The stacked gravitational weak lensing signal from a large sample of dwarf galaxies will provide the most direct measurement of the amount and distribution of dark matter.  
In this section we predict the sensitivity of LSST to a stacked weak lensing signal from dwarf galaxies.

\vspace{1em} \noindent {\bf Dwarf galaxy lenses}

We are interested in estimating the number of isolated dwarf galaxies accessible to LSST as a function of dark matter halo mass.
To predict the abundance of the dwarf galaxy sample, we assume the mass-to-light ratio derived from the subhalo abundance matching technique, which links the global galaxy luminosity function with (sub)halo mass function by their respective abundance \citep[\eg,][]{2004ApJ...609...35K,2013ApJ...771...30R}. We use \code{colossus} \citep{2018ApJS..239...35D} to obtain the halo mass function and adopt the global galaxy luminosity function measured by GAMA \citep{2015MNRAS.451.1540L}. We match galaxy luminosity to current halo mass with the definition of $M_{200c}$. We also assume the mass-to-light ratio does not evolve significantly in this low-redshfit regime. 
We use this predicted galaxy luminosity to estimate the limiting redshift for dwarf galaxy detection as a function of galaxy halo mass for two LSST limiting magnitudes: $r \sim 25$ and $r \sim 27$. 
\figref{dwarf_redshift} shows that to probe dark matter halos with mass $\lesssim 10^9 \Msun$, it will be necessary to select galaxies at $z < 0.01$. 
While selecting very low-$z$ galaxies with photometric data is challenging, current projects like the SAGA Survey \citep{Geha:2017} have shown that it is possible using data from SDSS. 
Future large, multi-object spectrographs will greatly expand the spectroscopic data for training these selections. 
It will also be possible to use morphological information to select nearby dwarf galaxies.
LSST will be able to distinguish a dwarf galaxy with $M_V=-14$ from background galaxies of the same apparent magnitude out to a distance of $\roughly 100 \Mpc$ \citep[Section 9 of][]{0912.0201}.

\begin{figure}
\centering
\includegraphics[width=0.6\columnwidth]{halo_mass_redshift_log}
\caption{\label{fig:dwarf_redshift} Limiting redshift for detecting a dwarf galaxy that lives in a dark matter halo of certain masses, assuming a luminosity--halo mass relation derived from the  subhalo abundance matching technique, which matches galaxy luminosity from the GAMA luminosity function to present-day halo mass ($M_{200c}$) by their respective abundance.}
\end{figure}

\vspace{1em} \noindent {\bf Source galaxies}

The conservative LSST 10-year ``gold'' sample for cosmic shear measurements of dark energy is expected to have a source galaxy density of $\roughly 27 \amin^{-2}$ \citep{Chang:2013,1809.01669}. 
However, we expect that the dwarf lensing analysis can retain significantly more source galaxies for the following reasons.
(1) Our measurement uncertainty is dominated by the low number of dwarf galaxy lenses, rather than the  multiplicative shear measurement bias that must be strictly controlled for dark energy measurements. This allows us to include fainter, smaller, and more blended sources.
(2) Unlike the lenses used for cosmic shear measurements, the dwarf galaxy lenses are at very low redshift. This means that most detected sources are background galaxies.
(3) We expect to be able to combine shape measurements from multiple filters, which could increase the source density by $\roughly 80\%$. 
Combining these factors, we estimate a source galaxy density of $50 \amin^{-2}$, which is consistent with the fiducial, multi-band estimate of \citet{Chang:2013}.
The primary focus of the source galaxy selection will be to avoid catastrophic \photoz outliers (low-$z$ galaxies reported at high-$z$), which typically occur for less than a few percent of galaxies in current surveys \citep{1406.4407}. 
%\Photoz algorithms incorporating machine learning currently achieve better performance, giving posterior p(z) estimating which enables cuts on suspect source galaxies. 

\vspace{1em} \noindent {\bf Sensitivity}

We calculate the expected strength of a lensing signal for three different bins in halo mass,  $M_{200c} = \{10^{10},\, 3\times10^9,\, 10^{9}\}\,h^{-1}\Msun$, each with a width of $0.5$\,dex in mass. 
These samples correspond to $N = \{1.2\times10^8,\, 7.8\times10^6,\, 1.6\times10^5\}$ dwarf galaxies out to a redshift of $z = \{0.35,\, 0.07,\, 0.014\}$, respectively.
Source galaxies are placed at $z = 1.2$ with a density of $50 \hbox{ arcmin}^{-2}$ and a shear uncertainty of $\sigma_\gamma = 0.25$.
We model the mass distribution in each dwarf galaxy with an NFW halo assuming the concentration--mass relation from \citet{1809.07326}.
We calculate the shear from the stacked dwarf galaxy lens sample using \code{colossus} \citep{2018ApJS..239...35D}, assuming that each lens is placed at the limiting detectable redshift.
The results are shown in \figref{dwarf_sn}, where we find that LSST has the potential to measure the lensing shear with ${\rm S/N} \gtrsim 10$ for halos with $M \gtrsim 3 \times 10^9\,h^{-1}\Msun$.
Note that some of our assumptions are clearly optimistic. In particular, the number density of the source galaxies we assumed is high, and the assumption of perfect lens galaxy selection is also unlikely to hold. Nevertheless, since the S/N ratio goes $\sim 1/\sqrt{N_\text{lens} N_\text{src}}$, and thus lowering these numbers by a factor of $\sim 2$ would still yield a very high S/N ratio.

\begin{figure}
\centering
\includegraphics[width=\columnwidth]{halo_mass_lensing_sn}
\caption{
\label{fig:dwarf_sn} Lensing signal (reduced tangential shear; \textit{left}) and signal-to-noise (\textit{right}) for stacked samples of dwarf galaxies in three different mass bins (shown by different shapes of markers), each with width of 0.5 dex in mass. Two different density profiles are used for this calculation: the NFW profile (blue) and a NFW profile with a core (orange). 
The calculation assumes perfect selection of dwarf galaxies within the redshift range over which they are detectable by LSST. 
Source galaxies are assumed to be at $z=1.2$, with a surface number density of $50\,\amin^{-2}$, and a shear uncertainty of $\sigma_\gamma = 0.25$ per component.}
\end{figure}

As mentioned earlier, a cored density profile is a signature of SIDM, hence we also calculate the shear signal for a modified NFW profile with a central core,
\begin{equation}
\rho_\text{core}(r) = \rho_\text{NFW}(r) \times (1 -  e^{-3r/r_s})\,,
\end{equation}
where $\rho_\text{NFW}(r)$ is a standard NFW profile and $r_s$ is the scale radius of the NFW profile. 
We show the predicted shear signal from the cored profile in \figref{dwarf_sn} to be compared with the signal from NFW profile. 
We see that the overall signal-to-noise does not change much with the profile.  
However, to statistically distinguish the different profiles, one needs to measure the shear at very small angular scales ($< 10\,h^{-1}\kpc$, which corresponds to 2.9\,arcsec at $z=0.35$ and 10\,arcsec at $z=0.07$). This small-scale regime is where the systematics due to PSF modeling and blending would dominate. 
In other words, while the numbers of source and lens galaxies that LSST can find will be high enough to distinguish the difference between the two profiles, shear measurement systematics may present the major obstacle. 

The median seeing of LSST is about 0.7\,arcsec \citep[LSST SRD,][]{LPM-17}. Since the dwarf galaxy lenses are at very low redshift ($z=0.07$ for the $M_\text{halo}=3\times10^9\,h^{-1}\Msun$ sample), the angular scale ($\sim$10\,arcsec) that we would use to distinguish the cored profile is still well above a few times the median seeing. However, the uncertainty in PSF models can affect the shape measurement up to the scale of 3\,arcmin \citep{2012MNRAS.427.2572C}. We believe that, with improved PSF models and marginalization over model uncertainty, it will still be feasible to  utilize dwarf galaxy lenses to distinguish different halo profiles at small scales. 
 % edit in dwarf-lensing.tex

\subsection{Galaxy Clusters \Contact{Susmita}}
\label{sec:halo_profile_clusters}
\Contributors{Susmita Adhikari, William A.\ Dawson, Nathan Golovich, David Wittman,  M.\ James Jee, Annika H.\ G.\ Peter, Daniel A.\ Polin, Robert Armstrong}

Galaxy clusters are the most massive gravitationally bound structures in the universe. The high matter density and high velocity dispersions of clusters make them ideal laboratories for testing dark matter self-interaction models in a very different regime from individual galaxies.
In the following section we discuss several probes that use galaxy clusters to constrain the nature of dark matter.  We show that current constraints from many different cluster-scale probes are of the order of $0.1$--$1\cmg$.  To understand why this is so, it is important to note that the average column density of a cluster-scale halo is of the order of $1 \g \cm^{-2}$.  Improved cross section constraints will come from a combination of the large statistical data sets that will be collected by LSST and other telescopes in the LSST era, and more sophisticated theoretical predictions for observables for specific SIDM models.

\vspace{1em} \noindent {\bf Distribution of matter and substructure}

As we describe below, the current best cluster-scale SIDM constraints come from the radial dark matter profiles of halos.  However, cluster-scale halos that consist of SIDM and CDM exhibit other differences, which may prove to be highly constraining given the vastly detailed LSST cluster data sets.  Significantly more theoretical work is required to project robust constraints in the LSST era for those probes.

\paragraph{Radial profile:} Interactions among dark matter particles allow for the exchange of energy between different parts of the halo. The high number of interactions near the dense central region of a dark matter halo increases the temperature, or the velocity dispersion, near the central region. This process can be thought of as a transfer of heat from the outer (hotter) parts of the halo to the inner (colder) region. The excess dispersion due to self interaction leads to flattening of the inner density of the halo, leading to the formation of a cored density profile. For cluster-scale halos, the high densities near the center make the timescales for thermalization shorter at a given cross section than they are for lower-mass objects (although it must be noted that low-mass halos are generally older and have a longer time to thermalize).  The short thermalization time is important because dark matter thus behaves as a fluid in the innermost part of cluster-scale halos, and can relax to a hydrostatic equilibrium configuration at the center of the halo, where baryons dominate the potential \citep{Kaplinghat:2015aga}.  Depending on the merger history, cluster-scale halos can be as cuspy as those in CDM-only simulations (for recent mergers), or relax to a hydrostatic equilibrium (for highly relaxed systems) in which the dark matter halo has a small but relatively dense core \citep{Robertson:2017mgj}.

Density profiles of massive galaxy clusters therefore serve as probes for SIDM. Clusters tend to be dark matter dominated outside the very central regions, and they are the only known systems where the matter distribution can be individually mapped to the virial radius using weak lensing. Strong lensing also provides a measure of cluster mass independent of the dynamical state. And stellar kinematics of the central galaxy can be used to measure the dark matter density profile in the innermost regions. LSST will produce an unrivaled catalog of strong and weak lensing measurement of cluster density profiles. This, in concert with X-ray mass estimates and stellar kinematics, will provide a strong test of the NFW dark matter density profile predicted by cold, collisionless dark matter \citep{Newman:2013,Kaplinghat:2015aga,Robertson:2018anx,Andrade:2019wzn}. Moreover, the strong lensing cross section is an additional probe of the density profile \citep{Robertson:2018anx}.  For hard-sphere scattering, cross section constraints are of the order of $0.1$--$1\cmg$, but without fully quantified systematic uncertainties.

\paragraph{Halo shape:} Apart from the density profile itself, in SIDM models, dark matter velocity distributions become more isotropic than in the CDM model, especially at the center of the halo.  Correspondingly,  the halo density profile becomes more spherical.  Historically, constraints from cluster and galaxy ellipticies \citep{Miralde-Escuda:2000} provided strong constraints on the cross section of SIDM; however, later investigations found these constraints to be somewhat optimistic \citep{Peter:2013}. 
Recent measurements of the shapes of cluster-scale dark matter halos include studies with: cluster members \citep{2018MNRAS.475.2421S},  X-rays \citep{Hashimoto:2007},  lensing \citep{Mandelbaum:2006, Evans:2009, Oguri:2010}, and combinations of observables \citep{Clampitt:2016, Sereno:2018}.  
Current constraints on the cross section are sensitive to the order of $\sigmam \sim 1 \cmg$.
Several groups have shown in $N$-body simulations that the effects of SIDM with a cross section of roughly $\text{(a few)}\times 0.1$--$1 \cmg$ are potentially observable, although baryons can alter the probability distribution function of halo shapes by an amount that is not yet robustly quantified \citep[\eg][]{Peter:2013, Robertson:2017mgj, Brinckmann:2018}.


\paragraph{Substructure:} Structures form hierarchically in the standard CDM scenario: small objects form first and merge to form larger mass structures such as galaxy clusters. These clusters continue to accrete smaller halos and some of these small structures survive as subhalos within the cluster. It is therefore interesting to study the distribution of substructures within larger halos, to understand how the distribution is affected by self interactions among dark matter particles. 

Subhalos can be affected by SIDM models in three different ways within a cluster. First, dark matter particles in subhalos can evaporate due to interactions with the particles in the host cluster. Subhalos lose mass when they enter a cluster. In the CDM scenario particles that are at larger radii and are loosely bound get stripped as the subhalo orbits within a cluster. In SIDM models, evaporation due to self-scattering leads to additional mass loss. Unlike tidal stripping, self-interactions can also affect the inner regions of the subhalos. Simulations show that evaporation is inefficient at increasing the subhalo disruption rate unless hard-sphere cross sections are of order $\sigmam \sim 10\cmg$, or subhalos are on nearly radial orbits through the cluster center \citep{2012MNRAS.423.3740V,Rocha:2012jg,Dooley:2016ajo}. 

While this generally means that the total subhalo mass function within the virial volume is largely unaffected relative to CDM, other effects of evaporation may be detectable. Measuring the mass and the profile around cluster satellites (especially as a function of orbit eccentricity) using galaxy--galaxy lensing to measure the mass-to-light ratio of subhalos can be a promising probe for dark matter physics \citep{Natarajan:2017sbo}. The lensing signal around subhalos is weak and will be contaminated by the cluster mass profile, so methods like subtracting the lensing signal from diametrically opposite points within the cluster can be used to extract the signal. Given the statistics of cluster galaxies in LSST, it is ideally suited for a study of the weak lensing signal of subhalos.  
Second, as subhalos are also tracers of the dark matter density field within the cluster, their orbits will be affected by the change in the potential of the cluster near the core relative to CDM.  This effect can lead to an imprint in the radial distribution of subhalos in clusters, generally by making the subhalos less concentrated toward the halo centers.

Third, non-expulsive interactions can lead to a drag force on subhalos.  This has several potentially interesting observable consequences.  The location of the splashback radius is sensitive to dynamics of subhalos within the cluster. The splashback radius is the boundary of the multistreaming region of a halo and is the largest apocenter of recently accreted objects \citep{Diemer:2014xya,Adhikari:2014lna}. The slope of the density profile of a halo falls off rapidly in a narrow localized region around this radius, and the splashback radius is observed as a minimum in the slope of the projected number density profile of galaxies \citep{More:2016vgs,Baxter:2017csy,Chang:2017hjt}.
The apocenter of the orbits of subhalos can change if there is extra drag beyond dynamical friction \citep{Kummer2018}.  
Therefore measuring the location of the splashback radius can help distinguish between different models of dark matter, although the difference between splashback locations in the CDM and SIDM scenarios has not yet been well quantified.

Similar to the situations discussed above and in merging clusters (Section~\ref{sec:merging_clusters}), the drag force due to non-expulsive interactions may also lead to offsets between the light distribution and dark matter distribution of individual satellites with respect to their subhalos. Small offsets between the subhalo and the galaxy within it may be detectable by indirect means: the potential gradient established by the dark matter at the position of the stellar centroid would induce a U-shaped warp in the stellar disk facing the direction of infall, and a longer-lasting disk thickening. Numerical simulations show these to be observable by current and next-generation photometric surveys under SIDM models with $0.5 \cmg \lesssim \sigmam \lesssim 1 \cmg$~\citep{Secco}. While S-shaped disks formed by tidal distortions of the stellar light profile are abundantly observed in cluster environments, indicating that they are readily induced by ``baryonic effects,'' these effects are not likely to generate prominent U-shaped warps. Such warps are only formed by a differential force on the disk and its halo, due to, for example, the SIDM drag.  The offset between a satellite galaxy and its subhalo may also be observed directly or statistically with strong lensing \citep{Massey2011,Massey:2017cwf}, but the magnitude of the effect is highly model-dependent (depending strongly on the angular and velocity dependence of the cross section).  Current limits are $\mathcal{O}(1\cmg)$ for specific non-hard-sphere models \citep{Harvey:2015hha}.  

\vspace{1em} \noindent {\bf Merging Galaxy Clusters \Contact{Nate?}}
\label{sec:merging_clusters}

% intro para
In the previous section, we considered subhalos to be minor merger events onto the main cluster.  Major cluster mergers can probe the nature of dark matter by serving as the biggest ``dark matter colliders'' on account of their high mass and large collision velocities. Dense halos falling together at thousands of km\,s$^{-1}$ provide an environment where the scattering of dark matter particles off each other would have observable effects.  The observable effects vary depending on the dark matter model and the configuration of the merger \citep{Kim:2016ujt}. 
Cluster mergers may also be able to distinguish between particle models that yield frequent scattering with low momentum transfer (as in a long-range force) and those that yield infrequent scattering with high momentum transfer (as with hard sphere or contact scattering) due to their differing phenomenology in the merger environment.  This is in contrast to the halo radial profile and shape constraints discussed in the previous section, for which the energy and momentum transfer rate matters most and for which there is no preferred direction in the problem.

% why LSST discovery is important
The best known example of a colliding cluster system is the Bullet Cluster, which has been frequently studied as a laboratory for SIDM \citep{Randall:2007ph,2017MNRAS.465..569R}. 
However, since a cluster merger is an eons-long process of which we have only a single snapshot, the measurement uncertainty is dominated by our very limited knowledge of the merger history. While it will remain critical to investigate individual clusters in great detail, the power of LSST lies in systematically analyzing a population of merging clusters with a consistent method, thereby constraining the properties of dark matter.
LSST will contribute to better and more robust constraints not only through the study of already known systems, but also by enabling the discovery of many more merging systems. Because mergers displace plasma from galaxies, they are best discovered by cross-correlation of LSST optically-detected clusters with radio and X-ray surveys \citep{Golovich:2018,Wilber2018}.

%offsets

The first SIDM constraints based on a merging galaxy cluster came from the Bullet Cluster, which was originally identified as an extremely hot X-ray cluster with two galaxy peaks. Higher resolution optical and X-ray imaging revealed a spectacular post-merger system with a clear X-ray cold front and shock. The spatial agreement of the galaxies and mass centroids obtained by weak lensing, and the disassociation of the intra-cluster medium (ICM) led to the constraint $\sigmam \lesssim 2 \cmg$ for hard-sphere scattering \citep{Markevitch2004,Randall:2007ph,2017MNRAS.465..569R,Robertson:2016qef}. Many other dissociative mergers have been found and studied, with roughly similar cross section limits \citep[but with greater systematic uncertainty, \eg][]{bradac2008}. 

After several ``dissociative'' mergers had been discovered, ensemble studies of the offsets between dark matter, galaxies, and gas were utilized to drive down the Poisson noise from inference on individual systems. \citet{Harvey:2015hha} modeled 72 subclusters within 30 merging systems to place the strongest constraint on SIDM ($\sigmam<0.47 \cmg$).
The study assumes a simplified drag force model where dark matter behaves similar to the ICM. However, \citet{Wittman:2017gxn} reanalyzed the sample including more comprehensive data. They identified several substantial errors that were driving the result and obtained a revised limit of $\sigmam \lesssim 2\cmg$.

The drag force model applies best to particle models with frequent interaction and low momentum transfer per interaction. In models with infrequent, high momentum transfer interactions (including hard-sphere scattering), dark matter particles may be scattered out of the cluster entirely. (Evaporation also occurs for small-angle scattering, though---see \citealt{Kahlhoefer:2013dca}.) This mass loss may be detected by comparing the mass-to-light ratio of merging clusters with those of non-merging clusters, on the assumption that the merger does not affect the galaxy light. This argument leads to a constraint of $\sigmam \lesssim 1 \cmg$, similar to current constraints on the drag model. However, the assumption that the galaxy light is unaffected is a source of uncertainty here. The LSST discovery of many more merging clusters, with six-band LSST photometry, will help us quantify this source of uncertainty. 

Several billion years post-pericenter, after a merging cluster has coalesced into a single cluster, SIDM will still create a cored dark matter distribution in the center of the cluster. For $\sigmam \sim 1 \cm^{2} \g^{-1}$, this core is $\roughly 100 \kpc$ (although the baryonic potential can alter the dark matter distribution). \citet{Kim:2016ujt} presented the effect of this on the brightest cluster galaxies (BCGs) up to 10 Gyr post-pericenter. They demonstrated a wobbling in the BCG as it is able to oscillate about the shallow potential for many oscillations. \citet{1703.07365} analyzed a small set of massive clusters and compared the BCG location with a strong lensing based estimate of the gravitational potential centroid. They compared these observations with hydro-CDM simulations to show that the observations suggest a cored dark matter halo in these clusters of $\roughly 10\kpc$. \citet{Harvey:2018uwf} recently studied cluster-scale halos in hydrodynamic simulations, and saw offsets that grew with cross section and halo mass, although with a smaller amplitude than the dark-matter-only simulations of \citet{Kim:2016ujt} implied. LSST will characterize thousands of relaxed clusters that invariably will have undergone a merger in their history. With deep and relatively high resolution imaging, LSST will allow for single snapshots of the BCG alignment in every massive cluster, and also for detection of faint strong lensing streaks in many of these systems.

% Esra added the text below
SIDM properties are sensitive to the separation between the centroid of the X-ray emitting hot plasma, i.e., intra-cluster medium (ICM), galaxies, and dark matter. Accurate measurements of the X-ray centroid of the X-ray emitting gas in clusters of galaxies requires sub-arcsec imaging with X-ray telescopes. The {\it Chandra} X-ray observatory with 0.5~arcsec FWHM PSF currently provides the most precise location of the ICM. Next-generation high spatial resolution X-ray observatories, \eg, {\textit Lynx} and {\textit AXIS} with much higher throughput, will provide accurate measurements of centers of high-redshift clusters ($z > 1$) in the 2030's and will enable tests of SIDM models over a much larger redshift range.
 % cluster people have organized subsubsections within clusters.tex

%\subsubsection{Halo Profiles of Bright Galaxies}
%\Contributors{Manoj?, Haibo?, Drew Newman?, Sean Tulin}
%\label{sec:halo_profile_galaxy}
% 
%\vspace{1em} \noindent {\bf Galaxy-Galaxy Lensing}
%\Contributors{?}
% 
%\vspace{1em} \noindent {\bf Caustics Structures}
%\Contributors{?}


\section{Compact Object Abundance \Contact{Will}}
\label{sec:compact_objects}
\Contributors{William A.\ Dawson, Nathan Golovich, Simeon Bird, Yacine Ali-Ha\"imoud, Juan Garc\'ia-Bellido, Marc Moniez, Michael Medford, Robert Armstrong, Jessica Lu, Casey Lam}

MAssive Compact Halo Objects \citep[MACHOs;][]{1991ApJ...366..412G} have a long history as dark matter candidates \citep{1974ApJ...193L...1O, 1980ApJS...44...73B, 1981ApJ...243..140G, 1986ApJ...304....1P, Bellido:1996, Clesse:2015, Bird:2016, Clesse:2016}. 
Cosmological observations of the CMB, BAO, and deuterium abundances have shown that compact objects must be non-baryonic if they are to make up a majority of dark matter \citep[\eg][]{Ade:2015xua}. 
As described in \secref{machos}, this has led to the identification of primordial black holes (PBHs) as the most popular candidate for MACHO dark matter \citep{Bellido:1996}.
There are a number of astrophysical probes that constrain the PBH dark matter abundance over mass scales ranging from $10^{-17}-10^{15}\,M_\odot$ (\figref{macho_constraints}).
At the lowest masses ($M < 10^{-9}\Msun$), PBHs are constrained through the non-detection of PBH evaporation in the extragalactic gamma-ray background \citep[\eg,][]{0912.5297, 1604.05349}, non-detection of femto-lensing of gamma-ray bursts \citep[\eg,][]{1204.2056}, the rate of SN Type 1a \citep{1805.07381}, and neutron star capture \citep[\eg,][]{1301.4984}.
The landscape of intermediate-mass MACHOS ($10^{-11} \Msun < M < 10 \Msun$) is predominantly constrained by microlensing observations, which limit the monochromatic compact dark matter fraction to be $\lesssim 10\%$ over this mass range \citep[\eg,][]{2001ApJ...550L.169A, 2007A&A...469..387T, 2009MNRAS.397.1228W, 1509.04899, 1701.02151, Calcino:2018}.
At the high-mass end ($M \gtrsim 10^3\Msun$), PBH dark matter is subject to constraints from dynamical stability of wide binary stars \citep[\eg,][]{2009MNRAS.396L..11Q, 2004ApJ...601..311Y}, star clusters \citep[\eg,][]{2016ApJ...824L..31B, 1611.05052}, dwarf galaxies \citep{1704.01668}, and the Galactic disk \citep[\eg,][]{1985ApJ...299..633L, 1994ApJ...437..184X}.
Lyman-$\alpha$ observations disfavor PBHs with $M > 10^4\Msun$ based on an observed plateau in the Poisson term of the matter power spectrum \citep{astro-ph/0302035}.
Strong constraints have also been placed on the abundance of PBHs with mass $\gtrsim 1 \Msun$ using CMB anisotropies \citep{2008ApJ...680..829R}.
However, these constraints have been shown to be extremely model dependent and were relaxed substantially in subsequent studies \citep{2017PhRvD..95d3534A}.
%Several indirect constraints have been published that rule out most of the mass scales above the sensitivity of these microlensing surveys; however, these constraints rely on complex astrophysical assumptions.
This, in addition to recent direct observations by LIGO of mergers of $10-50 \Msun$ black holes, potentially with less spin than expected \citep{1602.03837, LIGOScientific:2018b, LIGOScientific:2018a},
has motivated a renewed interest in PBH dark matter with larger masses than have been so far constrained directly by microlensing.

\begin{figure}[t]
\centering
\includegraphics[width=0.75\textwidth]{macho_limits.pdf}
\caption{\label{fig:macho_constraints}
    Constraints on the maximal fraction of dark matter in compact objects from existing probes (blue and gray) and projections for LSST (gold).
    Existing constraints include: lack of extragalactic gamma-rays from PBH evaporation \citep[EGR;][]{0912.5297, 1604.05349}, gamma-ray femtolensing \citep[GF;][]{1204.2056}, neutron star capture \citep[NS][]{1301.4984}, M31 microlensing \citep[M31ML][]{1701.02151}, Milky Way microlensing \citep[MWML;][]{2007A&A...469..387T, 2001ApJ...550L.169A, 2009MNRAS.397.1228W}, lensing of supernovae \citep[LSN;][]{1712.02240,1712.06574}, Eridanus II and other dwarf-galaxy constraints \citep[EII;][]{2016ApJ...824L..31B, 1611.05052}, wide binary stars \citep[WB;][]{2009MNRAS.396L..11Q, 2004ApJ...601..311Y}, cosmic microwave background \citep[CMB;][]{2017PhRvD..95d3534A, 2008ApJ...680..829R}, and disk stability \citep[DS;][]{1985ApJ...299..633L, 1994ApJ...437..184X}.
    %To improve figure clarity we have not shown some astrophysical constraints where they are less sensitive than a presented constraint; see \citet{2016PhRvD..94h3504C} for a more complete review.
    There are a range of constraints for most astrophysical probes in the literature due to varying assumptions within a single work (EGR, NS, and EII) and reanalysis/disagreements between groups (WB, CMB).
    We present \NEW{more} conservative constraints in blue and \NEW{more} aggressive constraints in gray.
    %The LSST M31 microlensing projection is based on extrapolating HSC constraints \citep{1701.02151} assuming a 10-day mini-survey of M31 with a $12\second$ cadence between exposures.
    %% Such a survey is approximately 10 times longer with an order of magnitude faster cadence than the existing HSC survey.
    The projected LSST Milky Way (MW) microlensing and paralensing constraints come from a Monte Carlo analysis where lenses were injected into light curves based on LSST OpSim cadence simulations. 
    %% (see \url{https://github.com/lsstdarkmatter/dark-matter-paper/issues/8} for details).
    The paralensing constraint \NEW{assumes} that the secondary microlensing parallax signal is used for discovery without incorporating the primary heliocentric microlensing signal.
    \NEW{The LSST projected sensitivity does not include contamination from astrophysical microlensing events.}
}
\end{figure}

In this section, we focus on the ability of LSST to directly detect signals of compact halo objects through precise, short ($\sim30\,$s) and long-duration ($\sim$ years) gravitational microlensing observations.
If scheduled optimally, the wide field-of-view, high cadence, and precise photometry of LSST \NEW{will provide sensitivity to the fraction of dark matter in compact objects down to} $\roughly 0.03\%$ for masses $>0.1\Msun$ (\figref{macho_constraints}).
We briefly mention that LSST will also probe PBHs by determining the rate of SN Type 1a, identifying candidate wide-binary star systems at greater distance than is possible with \Gaia, and through dedicated mini-surveys of high stellar density fields (similar to that performed with HSC by \citealt{1701.02151}).


\subsection{Microlensing}
\label{sec:microlensing}

Gravitational microlensing, the achromatic brightening and dimming of background stars due to the transit of a massive compact foreground object, can be used to directly detect and measure the properties of PBHs.
The idea of employing microlensing to search for compact objects in the Galactic halo was proposed by \citet{1986ApJ...304....1P}, and several photometric surveys commenced in the 1990's including MACHO \citep{1992ASPC...34..193A}, OGLE \citep{1992AcA....42..253U}, and EROS \citep{1993Msngr..72...20A}.
These collaborations provided the first \emph{direct} constraints on the compact nature of dark matter; however, they were limited by image quality, analysis techniques, and computational resources.
These limitations, combined with the $\roughly10$-year duration of these surveys, led to a loss of sensitivity at $M \gtrsim 1 \Msun$.
LSST can surpass this limitation by directly detecting events based purely on the parallactic component of the lensing signal (\figref{microlensing_cartoon}).
Furthermore, by supplementing the LSST survey with astrometric microlensing surveys using other telescopes (HST, Keck AO), LSST can break the lensing mass-geometry degeneracies and make precise measurements of individual black hole masses, thereby measuring the black hole mass spectrum in the Milky Way halo.
\NEW{If compact objects make up a significant fraction of dark matter, LSST will provide insight into the primordial perturbations and early universe equation of state, in the case of PBHs, or provide evidence that dark matter particle physics is complex enough to allow significant cooling channels. 
As dark matter with sufficient self-interactions to cool will also affect halo profiles, LSST will be well-placed to distinguish different models for the formation of novel compact objects.}
%Thus, if PBHs make up a significant fraction of dark matter, LSST will effectively measure their ``particle'' properties. 
%A precise measurement of the PBH mass spectrum will provide insight into the fundamental physics of the early universe.

\begin{figure}
\centering
\includegraphics[width=0.7\columnwidth]{microlensing_cartoon.png}
\vspace{1em}
\caption{\label{fig:microlensing_cartoon}
    \emph{Left:} 
        The microlensing and paralensing signals for a $23 \magn$ source being lensed by a $50 \Msun$ black hole. 
        For events with an Einstein crossing time much less than a year ($\lesssim 1 \Msun$), the microlensing magnification will appear symmetric in time (orange curve).
        For microlensing events lasting on the order of a year or more ($\gtrsim 1 \Msun$), the lensing geometry changes due to parallax as Earth orbits the Sun.
        This paralensing signal has a period of a year, with the phase determined by the coordinates of the source star, making it robust to astrophysical systematics.
        %It is also possible to detect binary dark matter, and extend the mass range to planet mass compact dark matter, via the source passing through one of the gravitational lensing caustic curves formed by the binary lens.
        %The green curve on top of the heliocentric orange curve is representative of a typical planetary microlensing event caused by a caustic crossing.
        %While LSST can measure these events if lucky, we will rely on LSST to detect the heliocentric microlensing event and trigger targeted follow-up higher cadence observations to measure the planetary microlensing event.
        The black data points are representative of extending the LSST wide-fast-deep cadence into the Galactic plane. \WAD{Need to update this figure with the LSST WFD cadence.}
        \emph{Right:} 
        A cartoon diagram of paralensing. 
        For microlensing events lasting on the order of a year or more, the lensing geometry changes as Earth orbits the Sun, leading to a parallax effect.
        %\Contributors{Will D., PALS Collaboration}
    }
\end{figure}

\noindent \textbf{The Microlensing Signal}
%Gravitational microlensing occurs when a massive lens passes between a background source and an observer, causing the light from that source to pass through a warped space-time acting as a lens. \WAD{cite Wambsganss for a detailed review}

Gravitational microlensing results in two potentially observable features: (1) photometric microlensing, a temporary achromatic amplification of the brightness of the background source, and (2) astrometric microlensing, an apparent shift in the centroid position of the source.
The characteristic photometric signal of a simple point-source, point-lens (PSPL) model as observed from the center of the solar system is symmetric, achromatic, and has both a timescale and maximum amplification that depend on the mass of the lens.
LSST will observe billions of stellar sources in multiple filters over several years to enable the detection of thousands of microlensing events across a wide range of timescales and consequently a wide range of masses.
This simple PSPL model is complicated by astrophysical factors including the velocity distribution of sources and lenses, extinction due to Galactic dust, blending in dense stellar fields, and the shift in perspective resulting from viewing a microlensing event while the Earth revolves around the Sun.
Fortunately, these complications can be addressed and disentangled to arrive at the mass of the gravitational lens and a detection of dark matter via microlensing \citep{1405.3134,1509.04899}.

One particularly powerful feature for long-duration microlensing events results from the change in the geometric configuration of the source-lens-observer system as the Earth orbits the Sun (\figref{microlensing_cartoon}).
The change in viewing angle and distance results in a parallax effect that imposes a 1-year periodicity on top of an otherwise symmetric microlensing light curve.
This additional signal exists irrespective of the mass of the lens, providing an independent measurements of the distance of the lens and breaking the mass-distance degeneracy of a microlensing signal \citep[\eg,][]{1509.04899}.
This enables microlensing to directly constrain compact dark matter at much larger mass scales ($M \gtrsim 1\Msun$), where the duration of the event is $\gtrsim 1$ year. 
Based on Figure 3 from \citet{1509.04899}, we expect that LSST will be able to use paralensing to detect $10\Msun$ lenses out to a distance of $\roughly5 \kpc$.

Moreover, if primordial black holes are significantly clustered in the halo of our galaxy, forming pockets of hundreds of massive PBH, then one should expect to detect a few very-long-duration microlensing events, of order decades~\citep{Bellido:2017}. If one of the constituent PBH in the cluster happen to be directly along the line of sight of the magnified star then one would expect, on top of the smooth Paczynski curve, a sharp caustic associated with the Einstein-ring crossing time of the individual PBH within the cluster~\citep{Bellido:2018}.

%\ADW{What is the maximum distance for which the parallax signal is measureable?}


%\WAD{achromatic}
Gravitational lensing is achromatic, making the multiple filter observations of LSST a key advantage for distinguishing a microlensing signal from other astrophysical transient and variable objects.
The benefit derived from multi-filter observations will depend strongly on the selected LSST cadence. 
Microlensing signals are more easily extracted from frequent observations in fewer filters, as long as sources are observed in at least two colors. 


\noindent \textbf{Microlensing Systematics}

As with any empirical observation, microlensing measurements are subject to systematics that must be accounted for.
We briefly summarize these systematics and strategies for mitigating their effect.

\paragraph{Variable and binary stars:} Temporally variable objects are a potential source of false detections at all timescales. However, the microlensing signal can be distinguished because, modulo secondary ambiguous blending effects, it is achromatic. Most astrophysical variable and binary stars, by contrast, are associated with a temperature dependence and thus have chromatic variability. To leverage this fact, it is necessary for LSST to survey high-stellar-density fields in at least two bands. Furthermore, for microlensing events with durations $\gtrsim 1$ year, mimicking the annual parallax signal imprinted on the microlensing signal would require a binary or variable star with a period of exactly 1 year.

\paragraph{Blending:} Given the typical scale \WAD{give typical scale range for low and high mass} of the Einstein radius (the approximate region where a microlensing signal is detectable), the odds of a microlensing event along an arbitrary line of sight are $\roughly 10^{-7}$--$10^{-5.5}$ \citep[\eg,][]{2000ApJ...541..734A,2006ApJ...636..240S}.
Due to the low probability of a microlensing event, most observable microlensing events will be in dense stellar fields (e.g., the Galactic plane, Magellanic Clouds, M31), driving microlensing surveys to these regions.

These dense survey fields, coupled with LSST depth and ground based PSF with FWHM $\sim 0.7 \arcsec$ \citep{0805.2366}, lead to significant ambiguous blending (i.e., multiple objects within a single PSF).
Towards the Galactic center there are $\roughly 50$ stars within an LSST PSF, similarly there are $\roughly 15$ and $\roughly 5$ stars per PSF in the Galactic bulge and disk, respectively \citep{1806.06372}.
LSST can overcome much of the blending problem to detect variations in the brightness of stars, including the detection of microlensing lightcurves, through the process of difference imaging.
With difference imaging, reference images are built through the coaddition of multiple observation of the same fields, resulting in a deep image of the static sky. Reference images are then scaled and PSF-matched to individual observed science images and subtracted off, resulting in a difference image that only contains the signal that differs from the static sky. 

Difference imaging should reduce many of the systematic effects associated with blending. Follow-up high resolution imaging from space or gronud-based adaptive optics can mitigate most blending issues for detected microlensing events.
%; however, it will still be difficult to characterize the intrinsic source star baseline photometric properties. Crowding can also lead to a secondary chromatic signal with exactly the same duration/shape as the microlensing event, although with different amplitude. 
Blending systematics could also be mitigated through spectroscopic follow-up of microlensing events before and after crossing. Lensing by low-mass stars will alter the spectrum of the source, while PBH lensing will leave the spectrum unchanged.

\paragraph{Cadence:} The temporal cadence of LSST observations will be important for optimizing sensitivity to microlensing events. 
While high-mass \NEW{black holes} should be accessible through relatively sparse observations distributed over the course of the year, smaller black holes require higher cadence observations.
\citep{1812.03137} suggest an observation strategy that includes a survey of the Galactic Bulge, Galactic Plane, and Magellanic Clouds with a reduced filter set at a cadence of 2 – 3 days. 
Observing the same region in at least two filters within the same night will allow tests of achromaticity.
\citet{1812.03139} suggest very high-cadence observations of the Magellanic Clouds (continuous 15s exposures), which should be sensitive to microlensing events from low-mass PBHs and scintillation light from invisible baryons \citep[\eg,][]{2003A&A...412..105M}.
The short readout time of the LSST Camera (2\,s) should allow for high-cadence observations of very short duration, subsolar mass PBHs, similar to the HSC observations of M31 \citep{1701.02151}.

\paragraph{Galaxy Model:} While microlensing can be detected independent of any detailed knowledge of Galactic structure, properly incorporating uncertainty in the Galactic dust, stellar velocity distributions, and dark matter halo model is essential to interpret the microlensing signal in the context of dark matter.
Significant improvements in our understanding of the Milky Way's dark matter halo have been made on this front since the first microlensing surveys \citep[\eg,][]{Calcino:2018}, and LSST will further improve these estimates (\secref{direct}).
In addition, systematic  microlensing measurements, especially when extended to the Galactic spiral arms, have the potential to strongly constrain the baryonic structure of the Galaxy in terms of mass density distribution and kinematical structure \citep[\eg,][]{Moniez:2017}.

%\paragraph{Binary vs isolated stars:} A potential systematic for the paralensing signal is binary star systems with approximately year-long periods. This is unlikely to be a significant systematic due to the low probability of having an achromatic binary system with a year long period. %\WAD{There was a researcher who did some of these studies for our microlensing group. Need to reference his work/arguments.} 

\noindent {\bf Projected Sensitivity}

%We estimate the projected sensitivity of an LSST survey optimized for the detection of microlensing signals from PBHs and present the results in \figref{macho_constraints}.
%The LSST M31 microlensing projection is based on extrapolating the \citet{1701.02151} by assuming a ten-day mini-survey of M31 with a 12 second cadence between exposures.
%Such a survey is approximately 10 times longer than that of \citet{1701.02151}, with an order of magnitude faster cadence.

We project the LSST Milky Way (MW) microlensing and paralensing sensitivity  from a Monte Carlo analysis of injected light curves based on LSST OpSim cadence simulations.
%\footnote{Details available in: \url{https://github.com/lsstdarkmatter/dark-matter-paper/issues/8}.}
We inject simulated microlensing signals onto the OpSim lightcurves for 200k PBH lenses in the mass range from $10^{-2}\Msun$ to $10^{8}\Msun$ distributed using a triangular density distribution, where $p(r = 0 \kpc) = 0$ and $p(r = 8\kpc) = 0.25$.
The sources are assumed to be $23\magn$ stars located at $8\kpc$ with measurement noise generated from the standard LSST photometric error model \citep{0805.2366}.
LSST is expected to detect $\roughly 10^{9}$ such stars, with an optical depth of $4.48 \times 10^{-6}$ \citep{2006ApJ...636..240S}.
The paralensing constraint comes from assuming that only the secondary microlensing parallax signal is used for discovery, and not the primary heliocentric microlensing signal.
\NEW{These sensitivity projection do not include contamination from microlensing events caused by stars and compact objects resulting from stellar evolution. They are thus an estimate of the limiting sensitivity of LSST in a ``background free'' regime (see \secref{pbh_discovery} for more details).}


\section{Anomalous Energy Loss Mechanisms\Contact{Maurizio}}
\label{sec:cooling}
\Contributors{Maurizio Giannotti, Oscar Straniero, Samuel D.\ McDermott, Alex Drlica-Wagner}

Observations of stars provide a mechanism to probe temperatures, particle densities, and time scales that are inaccessible to laboratory experiments.
Since conventional astrophysics allows us to quantitatively model the evolution of stars, the detailed study of stellar populations can provide a powerful technique to probe new physics.
In particular, if new light particles exist and are coupled to Standard Model fields, their emission would provide an additional channel for energy loss. 
Such anomalous energy loss mechanisms would change the time that stars spend in specific stellar evolutionary phases.
Such deviations are a robust predictions of light, weakly coupled particles, and the general agreement between observations and Standard Model predictions has been used to constrain the properties of many types of new particles \citep{hep-ph/0611350, 1210.1271, 1302.3884, 1305.2920, 1611.03864, 1611.05852, 1803.00993}.

While the predictions of the Standard Model are broadly consistent with observations of stellar evolution, several independent observations have shown a systematic preference for an additional subdominant energy-loss mechanism (see \citealt{Giannotti:2017hny} for a recent review).
These observations include red giants branch (RGB) stars, in particular the luminosity of the tip of the branch~\citep{Viaux:2013lha,Viaux:2013hca}; 
horizontal branch stars (HB), specifically by comparing the number of HB and RGB stars~\citep{Ayala:2014,Straniero:2015nvc};
variable white dwarf (WD) stars, for which the cooling efficiency was extracted from the rate of the period change~\citep{KeplerEtAl,Isern:1992gia,BischoffKim:2007ve,Corsico:2012ki,Corsico:2012sh,Corsico:2014mpa,Corsico:2016okh,Battich:2016htm}; 
and the WD luminosity function (WDLF), which describes the distribution of WDs as a function of their luminosity~\citep{Isern:2008nt,Bertolami:2014wua,Isern:2018uce}.
Observed discrepancies between these stellar measurements and predictions from conventional models of stellar cooling can be interpreted as the need for additional energy loss (\figref{axions}).
\cite{Giannotti:2015kwo} provide a systematic analysis of the new-physics interpretation of stellar observations  where cooling anomalies have been reported, and they conclude that axions and ALPs are the best candidates to account for the observed discrepancies. 
While these `hints' of anomalous cooling represent subdominant deviations to broadly successful models of stellar evolution, it is imperative to explore possible signatures of new physics when they arise.

LSST will greatly improve our understanding of stellar evolution by providing unprecedented photometry, astrometry, and temporal sampling for a large sample of faint stars \citep{0912.0201}.
These observations will allow us to better assess the significance of claimed anomalies, and will further guide constraints on (or detection of) new physics.
A better understanding of astrophysical energy transport will ultimately help shed light on the physics of light, weakly-coupled particles and will offer an invaluable guide to future experimental searches for axions and ALPs~\citep{Irastorza:2018dyq}.

\begin{figure}[t]
\centering
\includegraphics[width=0.6\columnwidth]{axions.png}
\includegraphics[width=0.39\columnwidth]{alps.jpg}
\caption{Left: Existing experimental and observational constraints on the QCD axion \citep{Redino:2015}.  
Right: Constraints on ALP coupling to photons \citep{Ringwald:2012}.
Astrophysical constraints and hints include observations of white dwarfs (WD), globular clusters (GC), supernova (SN), and horizontal branch stars (HB).
Note the wide range of mass and coupling scales that are constrained by these observations.
\label{fig:axions}
}
\end{figure}

\subsection{White Dwarf Luminosity Function}

The white dwarf luminosity function (WDLF) plays a particularly significant role in our understanding of stellar cooling and offers a fundamental method to test new physics.
Measurements of the slope of the WDLF can probe additional energy loss mechanisms and the production rate of the novel particle responsible for the nonstandard cooling.
The general agreement between the observed WDLF and predictions from standard astrophysics has been used to place bounds on the axion-electron coupling \citep{Isern:2008nt,Bertolami:2014wua}, on the anomalous neutrino magnetic moment \citep{Bertolami:2014noa}, on the kinematic coupling of dark photons to standard photons \citep{Chang:2016qfl}, and on the variation of the gravitational constant \citep{Althaus:2011ca}.
However, several recent analyses of the WDLF have shown a preference for additional energy loss with respect to the Standard Model predictions.
In particular, \cite{Bertolami:2014wua} used data from the Sloan Digital Sky Survey (SDSS) and the SuperCOSMOS Sky Survey (SCSS) to show a $2 \sigma$ discrepancy from the Standard Model prediction, which could be explained by axions coupling to electrons with $g_{\phi e}\simeq 1.4\times 10^{-13}$.\footnote{The additional energy can also be accounted for by dark photons~\citep{Giannotti:2015kwo,Chang:2016qfl}, but not by anomalous neutrino electromagnetic form factors~\citep{Bertolami:2014noa}.}
These measurements of the WDLF have guided experimental searches for axions and ALPs, particularly the IAXO~\citep{Irastorza:2011gs,Armengaud:2014gea}, and ALPS II~\citep{Bahre:2013ywa,ALPSII} experiments.

Observations from the \Gaia satellite have already increased the catalog of WDs by an order of magnitude with respect to SDSS \citep{1805.01227,1807.02559,1807.03315}.
The growing sample of WDs with precisely measured distances will enable an improved measurement of the WDLF.  
However, the completeness of the \Gaia sample is limited to WDs within 100 pc~\citep{1807.03315}.
%Oscar. In the GAIA DR2 about 260.000 WDs have been identified. However the sample is complete up to G=20-21, only for those within 100 pc, which are about 11.000 stars. The major problem is that the completeness drops at low Galactic latitudes, and the magnitude limit of the catalogue varies significantly across the sky as a function of Gaia’s scanning law.  A larger and more complete sample will be certainly available with the final data release.
LSST is expected to detect WDs that are 5 to 6 magnitudes fainter than those detected by \Gaia, ultimately increasing the census of WDs to tens of millions~\citep{0912.0201}.
LSST will provide more complete and homogeneous samples of WDs, allowing for a significant reduction in both the statistical and systematic uncertainties in measurements of the WDLF. 
LSST is expected to measure hundreds of thousands of WDs in the Galactic halo, enabling the construction of a reliable luminosity function of halo WDs. 
By deriving independent WDLFs from different Galactic populations it will be possible to reduce uncertainties related to star formation histories, and to ultimately provide a more clear assessment of the physical origin of the cooling anomalies \citep{Isern:2018uce}. 

The growing sample of WDs will similarly increase the known population of variable WDs with pulsation periods of 100s--1500s. 
Measured changes in the pulsation periods of WDs can be used to directly constrain the rate of cooling \citep[\eg][]{1007.2659}.
Indeed, hints of anomalous cooling from axions have been claimed \citep[\eg][]{Corsico:2012ki,Corsico:2012sh}, though more recent analyses set upper limits at the level of $g_{ae} < 3.3 \times 10^{-13}$ \citep{Battich:2016htm}. 
LSST will greatly increase the sample of pulsating WDs, enabling high-cadence follow-up observations to precisely measure changes in pulsation period and probe anomalous cooling mechanisms.


\subsection{Globular Cluster Stars}

Massive stars, specifically those close to the helium burning phase, provide another excellent environment to study anomalous energy loss mechanisms. 
Stellar evolutionary codes such as MESA \citep{1009.1622} provide a good model for the evolution of massive stars, allowing constraints to be placed on novel particle production \citep[\eg,][]{1210.1271,1611.05852}.
Several recent analyses of giant branch stars have reported deviations from standard stellar model predictions that can be interpreted as a signature of anomalous energy loss.
For example, studies have shown a brighter-than-expected tip of the RGB (TRGB) in the M5 globular cluster~\citep{Viaux:2013lha,Viaux:2013hca}, indicating somewhat over-efficient cooling during the evolutionary phase preceding the helium flash.
The anomalous brightness, $\Delta M_{I,{\rm TRGB}}\simeq 0.2$ mag in absolute $I$-band magnitude, observed in M5 can be interpreted as an anomalous cooling of a few $10^{33}$ erg/s.
Such cooling could be accounted for by a neutrino magnetic moment or an axion-electron coupling of the order of that predicted from the WDLF~\citep{Viaux:2013lha}. 
These constraints can be improved using multi-band photometry of multiple globular clusters \citep[\eg,][]{Straniero:2018fbv}.

Advances in the analysis of globular cluster RGB stars are currently limited by modeling uncertainties on the stellar evolution of the giant branch.
Fundamental improvements should be expected in the near future. 
In particular, exquisite astrometry from the {\it Gaia} satellite will precisely determine cluster distances, currently the largest sources of observational uncertainty in the determination of the absolute luminosity of the TRGB.\footnote{The {\it Gaia} data relevant for GCs are expected in 2022~\citep{Gaia}.}
Moreover, the angular resolution of the next-generation space-based missions, such as JWST~\citep{Gardner:2006ky}, will enlarge the statistical sample of RGB members near the cores of GCs. 
The brightness of RGB stars in nearby GCs limits the contributions of LSST, which saturates at $g \sim 17$ mag and suffers from crowding near the cores of GCs.
However, LSST will enable independent measurements of GC distances, providing a valuable handle on systematic uncertainties of {\it Gaia} observations.\footnote{The parallaxes of bright ($G<14$ mag) sources can be derived with a median uncertainty of 0.04 mas in {\it Gaia} DR2. However, for fainter stars the parallaxes become sensitive to systematic errors.  Presently, these systematics hamper a precise determination of GC distances \citep{Chen:2018}.}
Moreover, it is likely that the homogeneity and precise photometry of LSST will improve the calibration of the bolometric corrections for RGB and HB stars and ultimately contribute to a more clear assessment of the cooling of GC stars.


\subsection{Massive stars and core-collapse supernovae}

The cores of massive stars are among the most powerful natural laboratories to investigate the possible production of weakly interacting light particles, particularly axions. 
The energy loss rate via axions is quite sensitive to temperature. 
For instance, the rate of the Primakoff process (the photon-axion conversion in the static electric field of ions and electrons) scales as $T^4$. 
\figref{massivestar} shows the evolution of the neutrino and axion luminosities in a $18\Msun$ star. 
After He burning, the central temperature rapidly increases, becoming larger than $10^9$ K. 
In standard stellar models (no axions or other non-standard cooling), the energy loss by neutrinos largely overcomes the energy loss by photons. This rapidly decreases the evolutionary time scale and determines the chemical and physical structure of the star at the onset of core collapse. 
In this context, an additional energy loss mechanism may significantly affect the pre-explosive stellar structure and, in turn, may determine the success or failure of a core collapse supernova (CCSN). 
Such an effect may be revealed by connecting CCSNe to their massive star progenitors, something that will be enabled by the wide area and high temporal cadence of LSST.

Another powerful strategy by which novel particles can be constrained is by considering the evolution of the neutrino cooling phase of nearby SNe  \citep[\ie, SN1987A][]{Burrows:1988, Raffelt:1988}.
The simplest and most robust method by which such constraints can be implemented is the so-called ``Raffelt criterion,'' which limits the luminosity of new particles to be below the luminosity of neutrinos during the neutrino-cooling phase \citep{hep-ph/0611350}.
Neutrino observations of SN1987A have been used to place limits on a wide variety of new particles \citep{hep-ph/0207098, 1611.03864, 1611.05852, 1803.00993, 1808.10136}.
Again, it is possible that a subdominant release of energy into new particles is responsible for resolving some lingering inconsistencies with the Standard Model-only picture of CCSNe explosions \citep{0806.4273, 1805.07381}, though such an effect is difficult to resolve analytically on top of other qualitative uncertainties of the Standard Model-only picture \citep{1809.05106, 1811.11178}.
LSST, in combination with upcoming neutrino experiments (i.e., DUNE), will help reduce Standard Model uncertainties and expand this analysis to future and more distant CCSNe \citep{1807.10334}.

\begin{figure}[t]
\centering
\includegraphics[width=0.5\columnwidth]{massivestar.png}
\caption{Evolution of the luminosities of neutrinos and axions in a $M=18 \Msun$ stellar model ($t$ in years), relative to the photon luminosity. The various evolutionary phases are indicated. 
%In the calculation of the axion rate we have assumed the presently available upper bounds, as obtained from astrophysical constraints, for the axion-photon and axion-electron coupling constants.
In the calculation of the axion rate, we have assumed the current upper bounds on the axion-photon and axion-electron coupling.
\ADW{Do we need a reference for this figure?}
}
\label{fig:massivestar}
\end{figure}

LSST is expected to discover $\roughly 3.5\times10^5$ CCSNe per year \citep{Lien:2009}. 
%\ADW{Better to update this from Goldstein et al. (2018).}\AHGP{I think the Goldstein ref is with respect to lensed SNe}\ADW{Goldstein starts from a full population analysis, which is what we would take.}
For nearby SNe, LSST will be able to resolve massive progenitor stars in pre-explosion imaging. 
So far, a clear identification of progenitor stars has been obtained only for about 20 type II SNe \citep{Smartt:2015}.  
The identification of a much larger number of massive stars before they explode is a mandatory step for understanding the CCSNe process and the possible activation of non-standard cooling processes during the late evolution of massive stars. 
Recent theoretical studies have investigated the conditions for which a massive star successfully bounces after core collapse, giving rise to a SN \citep[\eg,][and references therein]{OConnor:2011,Sukhbold:2016}.   
In particular, it was found that the ability to explode predominantly depends on the structure of the progenitor. 
%Therefore, the impact of LSST in this field is twofold: it will constrain the supernova engine and may provide hints for new physics beyond the standard model. 
Therefore, besides providing more clear insight into the SN engine, LSST will provide a solid framework to test the presence of novel cooling channels efficient during pre-SN evolution, constraining or hinting at the existence of axions or other weakly interacting particles.

\section{Large-Scale Structure \Contact{Tony}}
\Contributors{Anthony Tyson, Rogerio Rosenfeld, Keith Bechtol, Francis-Yan Cyr-Racine}
\label{sec:lss}

LSST is anticipated to produce the largest and most detailed map of the distribution of matter and the growth of cosmic structure over the past 10 billion years. The large-scale clustering of matter and luminous tracers in the late-time universe is sensitive to the total amount of dark matter, the fraction of dark matter in light relics that behave as radiation at early times, and fundamental interactions in the dark sector. For example, the high densities of matter and energy in the early universe imply that even extremely weakly coupled dark matter particles can leave detectable imprints on the galaxy distribution today. Dark matter that couples to photons, neutrinos, or a light scalar field could produce dark acoustic oscillations analogous to baryon acoustic oscillations \citep{Cyr-Racine:2014}. Large-scale structure probes would be particularly competitive for multi-component dark matter models in which only a fraction of the dark matter couples to dark radiation. Meanwhile, LSST will enable tests of dark matter with non-standard gravitational interactions, such as violations of the equivalence principle \citep{Bonvin:2018} and interactions between dark matter and dark energy. The nature of the dark matter can be broadly described by a so-called Generalized Dark Matter model \citep{Hu:1998}, which has been recently explored in a study of the dark matter equation of state through cosmic history \citep{Kopp:2018}. 
Dark matter probes involving large-scale structure highlight the interconnectedness of dark matter and dark energy research, both in terms of testing the validity of the standard cosmological paradigm, and overlap in the specific analysis methods employed. The studies described in this section are illustrative examples of how the nature of dark matter can be probed employing the same galaxy clustering and weak lensing techniques as used in dark energy constraints. 

As one specific example, measurements of large-scale structure with LSST will enhance constraints on massive neutrinos and other light relics from the early universe that could compose a fraction of the dark matter. The combination of galaxy imaging and redshift surveys together with CMB experiments in the coming decade will make it possible to measure the density of relativistic particles in the early Universe at the percent level. The existence of the cosmic neutrino background can already be inferred from temperature fluctuations of the CMB \citep{Planck:2018_cosmo_params} and Big Bang nucleosynthesis \citep{Cooke:2018}, and a major goal of upcoming cosmology experiments is to measure the sum of neutrino masses at the few milli-eV level using weak gravitational lensing and galaxy clustering \citep[e.g.,][]{CMB-S4:2016,DESI:2016,Mishra-Sharma:2018}.\footnote{The LSST Dark Energy Science Collaboration ``Science Roadmap'' is available at \url{http://lsstdesc.org/sites/default/files/DESC_SRM_V1_4.pdf}.} In the standard cosmological model with three neutrinos, the effective number of relativistic free-streaming species is $N_{\rm eff}$ = 3.046. Several classes of dark matter models, such as axions, axion-like particles, dark photons, and light sterile neutrinos, predict deviations in $N_{\rm eff}$ that could be measurable with LSST and CMB experiments, even if the light species decouple before the QCD phase transition \citep{Font-Ribera:2014,Baumann:2018}. Galaxy surveys contribute to constraints on $N_{\rm eff}$ by independently constraining the Hubble constant, which is partially degenerate with $N_{\rm eff}$, and importantly, by measuring the broadband shape and phase of the galaxy power spectrum, which are sensitive to the gravitational influence of free-streaming light relics. The broadband galaxy power spectrum can also add robustness to the CMB results on $N_{\rm eff}$ because it is less dependent on the primordial helium abundance.

\begin{figure}[t]
\centering
\includegraphics[width=0.9\columnwidth]{DMDE-anisotropy.png}
\caption{A schematic diagram of the emergence of dark energy anisotropy from an Ising model phase transition and a coupling with the anisotropic distribution of dark matter. Figure taken from \citet{1810.11007}.}
\label{fig:DMDEmap}
\end{figure}

Another example demonstrating the potential of large-scale structure to study dark matter involves probing a possible coupling to dark energy, e.g., if dark energy were an aspect of dark matter at late times.
%The physics of dark matter could be probed via LSST 
The natural inhomogeneities in dark matter on large scales would then be reflected as spatial inhomogeneities in the developing late-time acceleration. 
This could be observed as spatial variations in cosmic acceleration by LSST correlated with the large scale dark matter weak lensing maps. 
Such a spatial correlation is a generic prediction if dark matter and dark energy are causally related or if dark energy is emergent \citep{1801.09658}. 
A spatially complicated potential leads to a small cosmological constant from an energy difference between its global and local minima, and dark energy and dark matter are thereby intertwined. 
If so, and if the universe on sub-horizon scales is not homogeneous, then spatial fluctuations in one component should be correlated with spatial fluctuations in the other, particularly near the epoch of emergence.

There are other models where dark matter and dark energy are intertwined, such as models where they interact 
\citep{Amendola:1999er,Holden:1999hm}.
Interacting models can be described phenomenologically via two fluids that can exchange energy and momentum, 
described by energy-momentum tensors that are not individually conserved. One can parameterize the coupling of
dark matter and dark energy by writing the divergence of the individual energy-momentum tensors as 
\begin{eqnarray}
\nabla_{\mu} T^{(DE)}\,^{\mu}_{\nu} &=& C^{(DE)}_{\nu}, \label{cons_phi} \\
\nabla_{\mu} T^{(DM)}\,^{\mu}_{\nu} &=& C^{(DM)}_{\nu}, \label{cons_dm}
\end{eqnarray}
where the superscript $(DM)$ stands for the dark matter fluid and $(DE)$ for the dark energy.
The conservation of the total dark component energy-momentum tensor 
(we assume the separate conservation of the energy momentum of radiation and baryons),
\begin{equation}
\label{energyconservation}
\nabla_{\mu} \left[ T^{(DM)} \,^{\mu}_{\nu} + T^{(DE)} \,^{\mu}_{\nu} \right]= 0,
\end{equation}
implies that
\begin{equation}
C^{(DM)}_{\nu}=-C^{(DE)}_{\nu}.
\end{equation}

The coupling between dark matter and dark energy is determined by the function $C^{(DM)}_{\nu}$ which is usually
written as
\begin{equation}
C^{(DM)}_{\nu} = (8\pi G)^{1/2} \,\beta\rho_{DM}\nabla_{\nu} \phi,
\end{equation}
where $\beta$ is a constant that expresses the coupling strength. In this model, dark energy must be dynamical and here it is modeled by a scalar field  $\phi$, such as a quintessence field.
In this model, $\beta$ is the only new parameter in addition to the usual description of the dark energy sector.
The standard uncoupled case is recovered for $\beta=0$.

There is a vast literature studying this class of models that can modify both the evolution of the 
background cosmology as well as the evolution of perturbations. For instance, it has been recently claimed
that such a model can ease the tension in the measurements of $\sigma_8$ from CMB and galaxy surveys 
\citep{Barros:2018efl}.

LSST can separately map dark matter and dark energy at a redshift where they have roughly comparable influences on the expansion rate of the universe. 
There are two complementary methods of reconstructing dark energy on the sky: SNe and 3$\times$2pt in independent patches within the survey footprint \citep[\eg][]{0902.2590}.
%\citep[Figure 15.9 in ][]{0912.0201}.
The transition between a dark matter-dominated universe to one with late-time acceleration (dark energy) may hint at some connection between these two components. 
An angular cross correlation between maps of dark energy and tomographic weak lens maps of dark matter could yield a non-zero signal.  
If so, the ratio of the cross correlation to the auto-correlations would be a diagnostic of the underlying physics. 
In this scenario, measurements of dark energy anisotropy become a probe of the nature of dark matter; \figref{DMDEmap}, reproduced from \cite{1810.11007}, illustrates a Ginzburg-Landau phase transition model that results in correlated dark matter-dark energy anisotropy. 
Quadrupole and higher-order correlated anisotropies are generated around redshift $z=0.7$.  
This is accessible in LSST maps of dark energy and dark matter in a broad redshift shell.

\paragraph{Systematics and synergies:}
Systematics in the dark energy and dark matter maps on large angular scales must be reduced below the level of any dark matter-dark energy correlation signal.  
For example, systematics in apparent magnitude and \photoz due to uncorrected extinction from Galactic dust would be one focus. 
Encouragingly, the two measures of dark energy anisotropy depend differently on wavelength-dependent extinction. 
A useful null test will be the cross correlation between dark energy and dark matter maps with dust maps.  
This would set the floor for residual extinction systematics, forming the basis for a forward simulation of the resulting dark energy-dark matter false correlation. 
As in analysis of CMB data, cuts on Galactic latitude can reveal the level of residual systematics. 
Finally, any dependence of the cross-correlations on redshift could discriminate between models as well as detect redshift-dependent systematics.

For detection of low multipole sky correlations, observations in the north as well as the south will be useful.
There is important synergy with WFIRST and EUCLID observations in the north.  
These complementary data could be calibrated and tested by joint null tests in overlap areas with the LSST survey.

