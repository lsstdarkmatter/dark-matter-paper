\subsection{Dwarf Galaxies as Lenses \Contact{Yao}}
\label{sec:halo_profile_group}
\Contributors{Yao-Yuan Mao, M.\ James Jee, Alex Drlica-Wagner, J.\ Anthony Tyson, Annika H.\ G.\ Peter, Chihway Chang, Rachel Mandelbaum, Manoj Kaplinghat}

Dwarf galaxies ($M_\star \lesssim 10^{9} \Msun$) provide the best visible tracers of low-mass dark matter halos. 
The relatively low baryonic content makes dwarf galaxies sensitive probes of  dark matter physics through the shape of their dark matter halo profiles. 
In particular, the ``core-cusp'' problem in dwarf galaxies has been cited as one of the most significant challenges to CDM \citep[\eg,][]{2010AdAst2010E...5D,Bullock:2017xww}.
The standard CDM model predicts that dark matter halos should have steeply rising (``cuspy'') central densities in contrast to the shallower (``cored'') mass profiles that are observationally inferred for many dwarf galaxies.  
Evidence for cored profiles exists for Milky Way satellite galaxies from kinematic and theoretical studies \citep[\eg,][]{Walker:2009, 2012ApJ...759L..42P}, and is stronger when one studies the inner density profiles of dwarf galaxies based on high-resolution neutral hydrogen surveys \citep[\eg,][]{Begum:2008,Hunter:2012,Cannon:2011,Oh:2015}. 
Many of these observations show inferred central slopes of the dark matter density profile, $\rho(r) \sim r^{-\gamma}$, that are significantly shallower ($\gamma \approx 0$--$0.5$) than the CDM prediction $\gamma \approx 0.8$--1.4 \citep{Navarro:2010}.

A wide range of solutions to the core-cusp problem have been proposed including observational, astrophysical, and dark matter explanations.
From a dark matter perspective, SIDM can significantly suppress the the central density of halos.
A self-interaction cross-section of $\sigma / m_\chi \sim 1 \cmg$ can explain the diversity of rotation curves seen in low-mass spiral galaxies \citep[\eg,][]{1504.01437,2017PhRvL.119k1102K,Tulin:2017ara}.
In addition, ultra-light or fuzzy dark matter has also been suggested as a possible solution to the core-cusp problem through the formation of uniform density solitonic cores \citep[\eg,][]{1502.03456,Hui:2017}. 
However, baryonic feedback remains a major complication for interpreting central density profile measurements in a dark matter context \citep{1996MNRAS.283L..72N,2005MNRAS.356..107R,2008Sci...319..174M,2012MNRAS.421.3464P,Madau:2014,Read:2016}. 
If dwarf galaxies form enough stars, energy from SN explosions can flatten the profiles of dark matter and baryons; however, if too many stars are formed, the excess baryonic mass can have the opposite effect of steepening the slope of the central density profile \citep{Bullock:2017}.
Technical challenges in implementing multi-phase gas and baryonic physics make it difficult to directly address and calibrate baryonic predictions based on hydrodynamical simulations \citep{Tollet:2016,1611.02281,Sawala:2016}.
However, one key prediction is that the creation of cores will be sensitive to the exact star formation history \citep[\eg,][]{governato2012,dicintio2014,onorbe2015,Read:2016,read2018,1811.11768,2019MNRAS.tmp....3R}.
Thus, robust measurements of both the stellar and dark matter mass of dwarf galaxies is essential to investigate the effect of baryonic feedback on the central dark matter density.
In addition, it has been argued that significant observational and astrophysical systematics, such as beam smearing, center offsets, inclinations, and non-circular motions can bias central density measurements toward flatter profiles \citep[\eg,][]{astro-ph/0006048,2004ApJ...617.1059R,2008AJ....136.2761O,2016MNRAS.462.3628R}. 
Thus, accurate independent measurements of dwarf galaxy density profiles are critical.

LSST can provide joint statistical measurements of both the central density and stellar content of dwarf galaxies. 
The stacked gravitational weak lensing signal from a large sample of dwarf galaxies will provide the most direct measurement of the amount and distribution of dark matter.  
In this section we predict the sensitivity of LSST to a stacked weak lensing signal from dwarf galaxies.

\vspace{1em} \noindent {\bf Dwarf galaxy lenses}

We are interested in estimating the number of isolated dwarf galaxies accessible to LSST as a function of dark matter halo mass.
To predict the abundance of the dwarf galaxy sample, we assume the mass-to-light ratio derived from the subhalo abundance matching technique, which links the global galaxy luminosity function with (sub)halo mass function by their respective abundance \citep[\eg,][]{2004ApJ...609...35K,2013ApJ...771...30R}. We use \code{colossus} \citep{2018ApJS..239...35D} to obtain the halo mass function and adopt the global galaxy luminosity function measured by GAMA \citep{2015MNRAS.451.1540L}. We match galaxy luminosity to current halo mass with the definition of $M_{200c}$. We also assume the mass-to-light ratio does not evolve significantly in this low-redshfit regime. 
We use this predicted galaxy luminosity to estimate the limiting redshift for dwarf galaxy detection as a function of galaxy halo mass for two LSST limiting magnitudes: $r \sim 25$ and $r \sim 27$. 
\figref{dwarf_redshift} shows that to probe dark matter halos with mass $\lesssim 10^9 \Msun$, it will be necessary to select galaxies at $z < 0.01$. 
While selecting very low-$z$ galaxies with photometric data is challenging, current projects like the SAGA Survey \citep{Geha:2017} have shown that it is possible using data from SDSS. 
Future large, multi-object spectrographs will greatly expand the spectroscopic data for training these selections. 
It will also be possible to use morphological information to select nearby dwarf galaxies.
LSST will be able to distinguish a dwarf galaxy with $M_V=-14$ from background galaxies of the same apparent magnitude out to a distance of $\roughly 100 \Mpc$ \citep[Section 9 of][]{0912.0201}.

\begin{figure}
\centering
\includegraphics[width=0.6\columnwidth]{halo_mass_redshift_log}
\caption{\label{fig:dwarf_redshift} Limiting redshift for detecting a dwarf galaxy that lives in a dark matter halo of certain masses, assuming a luminosity--halo mass relation derived from the  subhalo abundance matching technique, which matches galaxy luminosity from the GAMA luminosity function to present-day halo mass ($M_{200c}$) by their respective abundance.}
\end{figure}

\vspace{1em} \noindent {\bf Source galaxies}

The conservative LSST 10-year ``gold'' sample for cosmic shear measurements of dark energy is expected to have a source galaxy density of $\roughly 27 \amin^{-2}$ \citep{Chang:2013,1809.01669}. 
However, we expect that the dwarf lensing analysis can retain significantly more source galaxies for the following reasons.
(1) Our measurement uncertainty is dominated by the low number of dwarf galaxy lenses, rather than the  multiplicative shear measurement bias that must be strictly controlled for dark energy measurements. This allows us to include fainter, smaller, and more blended sources.
(2) Unlike the lenses used for cosmic shear measurements, the dwarf galaxy lenses are at very low redshift. This means that most detected sources are background galaxies.
(3) We expect to be able to combine shape measurements from multiple filters, which could increase the source density by $\roughly 80\%$. 
Combining these factors, we estimate a source galaxy density of $50 \amin^{-2}$, which is consistent with the fiducial, multi-band estimate of \citet{Chang:2013}.
The primary focus of the source galaxy selection will be to avoid catastrophic \photoz outliers (low-$z$ galaxies reported at high-$z$), which typically occur for less than a few percent of galaxies in current surveys \citep{1406.4407}. 
%\Photoz algorithms incorporating machine learning currently achieve better performance, giving posterior p(z) estimating which enables cuts on suspect source galaxies. 

\vspace{1em} \noindent {\bf Sensitivity}

We calculate the expected strength of a lensing signal for three different bins in halo mass,  $M_{200c} = \{10^{10},\, 3\times10^9,\, 10^{9}\}\,h^{-1}\Msun$, each with a width of $0.5$\,dex in mass. 
These samples correspond to $N = \{1.2\times10^8,\, 7.8\times10^6,\, 1.6\times10^5\}$ dwarf galaxies out to a redshift of $z = \{0.35,\, 0.07,\, 0.014\}$, respectively.
Source galaxies are placed at $z = 1.2$ with a density of $50 \hbox{ arcmin}^{-2}$ and a shear uncertainty of $\sigma_\gamma = 0.25$.
We model the mass distribution in each dwarf galaxy with an NFW halo assuming the concentration--mass relation from \citet{1809.07326}.
We calculate the shear from the stacked dwarf galaxy lens sample using \code{colossus} \citep{2018ApJS..239...35D}, assuming that each lens is placed at the limiting detectable redshift.
The results are shown in \figref{dwarf_sn}, where we find that LSST has the potential to measure the lensing shear with ${\rm S/N} \gtrsim 10$ for halos with $M \gtrsim 3 \times 10^9\,h^{-1}\Msun$.
Note that some of our assumptions are clearly optimistic. In particular, the number density of the source galaxies we assumed is high, and the assumption of perfect lens galaxy selection is also unlikely to hold. Nevertheless, since the S/N ratio goes $\sim 1/\sqrt{N_\text{lens} N_\text{src}}$, and thus lowering these numbers by a factor of $\sim 2$ would still yield a very high S/N ratio.

\begin{figure}
\centering
\includegraphics[width=\columnwidth]{halo_mass_lensing_sn}
\caption{
\label{fig:dwarf_sn} Lensing signal (reduced tangential shear; \textit{left}) and signal-to-noise (\textit{right}) for stacked samples of dwarf galaxies in three different mass bins (shown by different shapes of markers), each with width of 0.5 dex in mass. Two different density profiles are used for this calculation: the NFW profile (blue) and a NFW profile with a core (orange). 
The calculation assumes perfect selection of dwarf galaxies within the redshift range over which they are detectable by LSST. 
Source galaxies are assumed to be at $z=1.2$, with a surface number density of $50\,\amin^{-2}$, and a shear uncertainty of $\sigma_\gamma = 0.25$ per component.}
\end{figure}

As mentioned earlier, a cored density profile is a signature of SIDM, hence we also calculate the shear signal for a modified NFW profile with a central core,
\begin{equation}
\rho_\text{core}(r) = \rho_\text{NFW}(r) \times (1 -  e^{-3r/r_s})\,,
\end{equation}
where $\rho_\text{NFW}(r)$ is a standard NFW profile and $r_s$ is the scale radius of the NFW profile. 
We show the predicted shear signal from the cored profile in \figref{dwarf_sn} to be compared with the signal from NFW profile. 
We see that the overall signal-to-noise does not change much with the profile.  
However, to statistically distinguish the different profiles, one needs to measure the shear at very small angular scales ($< 10\,h^{-1}\kpc$, which corresponds to 2.9\,arcsec at $z=0.35$ and 10\,arcsec at $z=0.07$). This small-scale regime is where the systematics due to PSF modeling and blending would dominate. 
In other words, while the numbers of source and lens galaxies that LSST can find will be high enough to distinguish the difference between the two profiles, shear measurement systematics may present the major obstacle. 

The median seeing of LSST is about 0.7\,arcsec \citep[LSST SRD,][]{LPM-17}. Since the dwarf galaxy lenses are at very low redshift ($z=0.07$ for the $M_\text{halo}=3\times10^9\,h^{-1}\Msun$ sample), the angular scale ($\sim$10\,arcsec) that we would use to distinguish the cored profile is still well above a few times the median seeing. However, the uncertainty in PSF models can affect the shape measurement up to the scale of 3\,arcmin \citep{2012MNRAS.427.2572C}. We believe that, with improved PSF models and marginalization over model uncertainty, it will still be feasible to  utilize dwarf galaxy lenses to distinguish different halo profiles at small scales. 
