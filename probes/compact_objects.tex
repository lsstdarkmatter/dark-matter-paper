\section{Compact Object Abundance \Contact{Will}}
\Contributors{Will, Nate, Michael M., Bob A., ... }
\label{sec:compact_objects}

MAssive Compact Halo Objects \citep[MACHOs;][]{1991ApJ...366..412G} have a long history as dark matter candidates \citep{1974ApJ...193L...1O, 1980ApJS...44...73B, 1981ApJ...243..140G, 1986ApJ...304....1P, Bellido:1996, Clesse:2015, Bird:2016, Clesse:2016}. 
Cosmological observations of the CMB, BAO, and deuterium abundances have shown that compact objects must be non-baryonic if they are to make up a majority of dark matter \citep[\eg][]{Ade:2015xua}. 
As described in \secref{machos}, this has led to the identification of primordial black holes (PBHs) as the most popular candidate for MACHO dark matter \citep{Bellido:1996}.
There are a number of astrophysical probes that constrain the PBH dark matter abundance over mass scales ranging from $10^{-17}-10^{15}\,M_\odot$ (\figref{macho_constraints}).
At the lowest masses ($M < 10^{-9}\Msun$), PBHs are constrained through the non-detection of PBH evaporation in the extragalactic gamma-ray background \citep[\eg,][]{0912.5297, 1604.05349}, non-detection of femto-lensing of gamma-ray bursts \citep[\eg,][]{1204.2056}, the rate of SN Type 1a \citep{1805.07381}, and neutron star capture \citep[\eg,][]{2013PhRvD..87l3524C}.
The landscape of intermediate-mass MACHOS ($10^{-11} \Msun < M < 10 \Msun$) is predominantly constrained by microlensing observations, which limit the monochromatic MACHO dark matter fraction to be $\lesssim 10\%$ over this mass range \citep[\eg][]{2001ApJ...550L.169A, 2007A&A...469..387T, 2009MNRAS.397.1228W, 1701.02151}.
At the high-mass end ($M \gtrsim 10^3\Msun$), PBH dark matter is subject to constraints from dynamical stability of wide binary stars \citep[\eg][]{2009MNRAS.396L..11Q, 2004ApJ...601..311Y}, star clusters \citep[\eg][]{2016ApJ...824L..31B, 1611.05052}, dwarf galaxies \citep{1704.01668}, and the Galactic disk \citep[\eg][]{1985ApJ...299..633L, 1994ApJ...437..184X}.
Lyman-$\alpha$ observations suggest that PBHs with $M > 10^4\Msun$ do not exist based on an observed plateau in the Poisson term of the matter power spectrum \citep{astro-ph/0302035}.
Strong constraints have also been placed on the abundance of PBHs with mass $\gtrsim 1 \Msun$ using CMB anisotropies \citep{2008ApJ...680..829R}.
However, these constraints have been shown to be extremely model dependent and were relaxed substantially in subsequent studies \citep{2017PhRvD..95d3534A}.
%Several indirect constraints have been published that rule out most of the mass scales above the sensitivity of these microlensing surveys; however, these constraints rely on complex astrophysical assumptions.
This, in addition to recent direct observations of mergers of $10-45 \Msun$ black hole by LIGO \citep{1602.03837},
%and indirect evidence from observations of globular clusters \citep{???} and high-z quasars \citep{???}, 
has motivated a renewed interest in PBH dark matter with larger masses than have been so far constrained directly by microlensing.
%\WAD{the next sentence should be moved down after the signals are described}
%\ADW{Need citations}

\begin{figure}[t]
\centering
\includegraphics[width=0.8\textwidth]{macho_limits.pdf}
\caption{\label{fig:macho_constraints}
    Constraints on the maximal fraction of dark matter in compact objects from existing probes (blue and gray) and projections for LSST (gold).
    Existing constraints include: lack of extragalactic gamma-rays from PBH evaporation \citep[EGR;][]{0912.5297, 1604.05349}, gamma-ray femtolensing \citep[GF;][]{1204.2056}, neutron star capture \citep[NS][]{1301.4984}, M31 microlensing \citep[M31ML][]{1701.02151}, Milky Way microlensing \citep[MWML;][]{2007A&A...469..387T, 2001ApJ...550L.169A, 2009MNRAS.397.1228W}, lensing of supernovae \citep[LSN;][]{1712.02240,1712.06574}, Eridanus II and other dwarf-galaxy constraints \citep[EII;][]{2016ApJ...824L..31B, 1611.05052}, wide binary stars \citep[WB;][]{2009MNRAS.396L..11Q, 2004ApJ...601..311Y}, cosmic microwave background \citep[CMB;][]{2017PhRvD..95d3534A, 2008ApJ...680..829R}, and disk stability \citep[DS;][]{1985ApJ...299..633L, 1994ApJ...437..184X}.
    %To improve figure clarity we have not shown some astrophysical constraints where they are less sensitive than a presented constraint; see \citet{2016PhRvD..94h3504C} for a more complete review.
    There are a range of constraints for most astrophysical probes in the literature due to varying assumptions within a single work (EGR, NS, and EII) and reanalysis/disagreements between groups (WB, CMB).
    We present the most conservative constraints in blue and the most aggressive constraints in gray.
    %% ADW: I've reduced the caption length so the figure can appear closer to the text.
    The LSST M31 microlensing projection is based on extrapolating HSC constraints \citep{1701.02151} assuming a 10-day mini-survey of M31 with a $12\second$ cadence between exposures.
    %% Such a survey is approximately 10 times longer with an order of magnitude faster cadence than the existing HSC survey.
    The projected LSST Milky Way (MW) microlensing and paralensing constraints are from a Monte Carlo analysis where lenses were injected into light curves based on LSST OpSim cadence simulations. 
    %% (see \url{https://github.com/lsstdarkmatter/dark-matter-paper/issues/8} for details).
    The paralensing constraint comes from assuming that only the secondary microlensing parallax signal is used for discovery, and not the primary heliocentric microlensing signal.
}
\end{figure}

In this section, we focus on the ability of LSST to directly detect signals of compact halo objects through precise, short ($\sim30\,$s) and long-duration ($\sim$ years) gravitational microlensing observations.
If scheduled optimally, the wide field-of-view, high cadence, and precise photometry of LSST have the potential to extend PBH sensitivity to $\roughly 0.03\%$ of the dark matter fraction for masses $10^{-12}\lesssim \Msun \lesssim 10^{-6}$ and $>0.1\Msun$ (\figref{macho_constraints}).
We briefly mention that LSST will also probe PBHs by determining the rate of SN Type 1a, identifying candidate wide-binary star systems at greater distance than is possible with \Gaia, and through dedicated mini-surveys of high stellar density fields (similar to that performed with HSC by \citealt{1701.02151}).


\subsection{Microlensing}
\Contributors{Will D., Nate G., Michael M., Bob A., ...}
\label{sec:microlensing}

Gravitational microlensing, the achromatic brightening and dimming of background stars due to the transit of a massive compact foreground object, can be used to directly detect and measure the properties of PBHs.
The idea of employing microlensing to search for compact objects in the Galactic halo was proposed by \citet{1986ApJ...304....1P}, and several photometric surveys commenced in the 1990's including MACHO \citep{1992ASPC...34..193A}, OGLE \citep{1992AcA....42..253U}, and EROS \citep{1993Msngr..72...20A}.
These collaborations provided the first \emph{direct} constraints on the compact nature of dark matter; however, they were limited by image quality, analysis techniques, and computational resources.
These limitations, combined with the $\roughly10$-year duration of these surveys, led to a loss of sensitivity at $M \gtrsim 1 \Msun$.
LSST can surpass this limitation by directly detecting events based purely on the parallactic component of the lensing signal (\figref{microlensing_cartoon}).
Furthermore, by supplementing the LSST survey with astrometric microlensing surveys using other telescopes (HST, Keck AO), LSST can break the lensing mass-geometry degeneracies and make precise measurements of individual black hole masses, thereby measuring the black hole mass spectrum in the Milky Way halo.
Thus, if PBHs make up a significant fraction of dark matter, LSST will effectively measure their ``particle'' properties. 
A precise measurement of the PBH mass spectrum will provide insight into the fundamental physics of the early universe.

\begin{figure}
\centering
\includegraphics[width=0.6\columnwidth]{microlensing_cartoon.png}
\vspace{1em}
\caption{\label{fig:microlensing_cartoon}
    \emph{[Left]} 
        The microlensing and paralensing signals for a $23 \magn$ source being lensed by a $50 \Msun$ black hole. 
        For events with an Einstein crossing time much less than a year ($\lesssim 1 \Msun$), the microlensing magnification will appear symmetric in time (orange curve).
        For microlensing events lasting on the order of a year or more ($\gtrsim 1 \Msun$), the lensing geometry changes due to parallax as Earth orbits the Sun.
        This paralensing signal has a period of a year, with the phase determined by the coordinates of the source star, making it robust to astrophysical systematics.
        %It is also possible to detect binary dark matter, and extend the mass range to planet mass compact dark matter, via the source passing through one of the gravitational lensing caustic curves formed by the binary lens.
        %The green curve on top of the heliocentric orange curve is representative of a typical planetary microlensing event caused by a caustic crossing.
        %While LSST can measure these events if lucky, we will rely on LSST to detect the heliocentric microlensing event and trigger targeted follow-up higher cadence observations to measure the planetary microlensing event.
        The black data points are representative of extending the LSST wide-fast-deep cadence into the Galactic plane. \WAD{Need to update this figure with the LSST WFD cadence.}
        \emph{[Right]} 
        A cartoon diagram of paralensing. 
        For microlensing events lasting on the order of a year or more, the lensing geometry changes as Earth orbits the Sun, leading to a parallax effect.
        \Contributors{Will D., PALS Collaboration}
    }
\end{figure}

\noindent \textbf{The Microlensing Signal}
%Gravitational microlensing occurs when a massive lens passes between a background source and an observer, causing the light from that source to pass through a warped space-time acting as a lens. \WAD{cite Wambsganss for a detailed review}

%\WAD{Heliocentric microlensing}
Gravitational microlensing results in two potentially observable features: (1) photometric microlensing, a temporary achromatic amplification of the brightness of the background source, and (2) astrometric microlensing, an apparent shift in the centroid position of the source.
The characteristic photometric signal of a simple point-source point-lens (PSPL) model as observed from the center of the solar system is symmetric, achromatic, and has both a timescale and maximum amplification that depend on the mass of the lens.
LSST will observe billions of stellar sources in multiple filters over several years to enable the detection of thousands of microlensing events across a wide range of timescales and consequently a wide range of masses.
This simple PSPL model is complicated by astrophysical factors including the velocity distribution of sources and lenses, extinction due to Galactic dust, blending in dense stellar fields, and the shift in perspective resulting from viewing a microlensing event while the Earth revolves around the Sun.
Fortunately, these complications can be addressed and disentangled to arrive at the mass of the gravitational lens and a detection of dark matter via microlensing \citep{1405.3134,1509.04899}.

%\WAD{Paralensing}
One particularly powerful feature for long-duration microlensing events results from the change in the geometric configuration of the source-lens-observer system as the Earth orbits the Sun (\figref{microlensing_cartoon}).
The change in viewing angle and distance results in a parallax effect that imposes a 1-year periodicity on top of an otherwise symmetric microlensing light curve.
This additional signal exists irrespective of the mass of the lens, providing an independent measurements of the distance of the lens and breaking the mass-distance degeneracy of a microlensing signal \citep[\eg][]{1509.04899}.
This enables microlensing to directly constrain compact dark matter at much larger mass scales ($M \gtrsim 1\Msun$), where the duration of the event is $\gtrsim 1$ year.   
\ADW{What is the maximum distance for which the parallax signal is measureable?}

%\WAD{astrometric (LSST vs AO); I feel like this paragraph might belong better in the complementarity section, and should just be referenced in the paragraph above.}
%Photometric microlensing timescale measurements result in lens mass measurements in units of the Einstein crossing radius, which is impossible to measure directly due to the unresolved nature of the microlensing event.
%Astrometric microlensing makes measurements of the shift in the unresolved centroid position of the background source in units of the Einstein radius as well.
%Triggering astrometric followup on instruments that have already proven capable of making these measurements (Keck Adaptive Optics, HST), as well as next generation telescopes (TMT, ELT) will prove a vital measurements to complement the lightcurves generated by LSST.

%\WAD{achromatic}
Gravitational lensing is achromatic, making the multiple filter observations of LSST a key advantage for distinguishing a microlensing signal from other astrophysical transient and variable objects.
The benefit derived from multi-filter observations will depend strongly on the selected LSST cadence. 
Microlensing signals are more easily extracted from frequent observations in fewer filters, as long as sources are observed in at least two colors. 

%\noindent \textbf{Survey Specifications}

%\emph{Where to survey:} Given the typical scale \WAD{give typical scale range for low and high mass} of the Einstein radius (the approximate region where a microlensing signal is detectable), the odds of any one source star experiencing a microlensing event is approximately $10^{-9}$. Due to the low probability of a microlensing event, most observable microlensing events will be in dense stellar fields (e.g., the Galactic plane, Magellanic Clouds, and M31), driving microlensing surveys to these dense stellar fields.

%\emph{Difference imaging:} These dense survey fields, coupled with LSST depth and ground based point spread function (PSF) of $\sim0.8''$, lead to significant ambiguous blending (i.e., multiple objects within a single PSF).
%Towards the Galactic center there are $\sim50$ stars within an LSST PSF, similarly there are $\sim15$ and $\sim5$ in the bulge, and disk respectively \citep{1806.06372}.
%LSST can overcome much of the blending problem to detect variations in the brightness of stars, including the detection of microlensing lightcurves, through the process of difference imaging.
%Reference images are built through the coaddition of multiple observation of the same fields, resulting in a deep image of the static sky. Reference images are then scaled and PSF-matched to individual observed science images and subtracted off, resulting in a difference image that only contains the signal that differs from the static sky. %This limits the locations in the sky where difference imaging can be performed to those locations where reference images have already been built. Reference building is thus the first and foremost priority to enable LSST to measure sky variability.  We will be able to leverage the difference imaging tools developed for the LSST Data Release Production pipeline.  This in turn will benefit from improved calibrations and thus increased sensitivity.

%\emph{Reference image building:} Different science cases require different strategies for building reference images. Short timescale events are agnostic to the cadence of those images which are used to build a reference image because any observations prior to the event can be assumed to be photometrically identical. Microlensing signals caused by massive lenses can be on the scale of years to decades. If the reference images are built by combining images across months or years, they will contain the microlensing signal within the reference and make it far more difficult to detect microlensing events from difference images. We therefore strongly prefer reference images built from images taken across a short time period to reduce static / signal confusion.

%\emph{Dense field photometry:} While differential photometry is ideal for detecting variable objects in dense fields it does not provide baseline photometry for individual stars. Such baseline photometry constrains stellar locations within the Milky Way, which is needed for the lensing geometry and thus the mass of the compact object. There are a number of pipelines for dense-field photometry---e.g., \code{DAOPhot} \citep{1987PASP...99..191S} or \code{crowdsource} \citep{1710.01309}---capable of providing this; however, none have yet been implemented in the LSST Data Management system.

%\emph{Band choices:}
%The microlensing signal is achromatic, modulo secondary ambiguous blending effects. By contrast, most other astrophysical variables have chromatic variability due to associated temperature changes.
%To leverage this fact, it is necessary to survey in at least two bands.
%If surveying the Milky Way bulge or M31 it is optimal to use more red sensitive bands which are less sensitive to galactic dust between the observer and the stars.

\noindent \textbf{Microlensing Systematics}

As with any empirical observation, microlensing measurements are subject to systematics that must be accounted for.
We briefly summarize these systematics and strategies for mitigating their effect.

\emph{Variable and binary stars:} Temporally variable objects are a potential source of false detections at all timescales. However, the microlensing signal can be distinguished because, modulo secondary ambiguous blending effects, it is achromatic. Most astrophysical variable and binary stars, by contrast, are associated with a temperature dependence and thus have chromatic variability. To leverage this fact, it is necessary for LSST to survey high-stellar-density fields in at least two bands. Furthermore, for microlensing events with durations $\gtrsim 1$ year, mimicing the annual parallax signal imprinted on the microlensing signal would require a binary or variable star with a period of exactly $1$ year.

%However this systematic can be mitigated through a number of independent ways.
%Perhaps the most powerful means is by leveraging the achromatic nature of gravitational lensing.
%Nominally, the microlensing signal will have the same amplitude in all photometric bands (although this is absolutely true only in the absence of blending and interstellar extinction), while most astrophysical variables are associated with a temperature dependence and thus not achromatic.
%For microlensing events with durations $\gtrsim 1$ year, there will be a 365 day period parallax signal imprinted on the microlensing signal which is difficult for other astrophysical variables to mimic.
%Also as the survey progresses repeating variable stars will be better characterized.

\emph{Blending:} Given the typical scale \WAD{give typical scale range for low and high mass} of the Einstein radius (the approximate region where a microlensing signal is detectable), the odds of any one source star experiencing a microlensing event is approximately $10^{-9}$. Due to the low probability of a microlensing event, most observable microlensing events will be in dense stellar fields (e.g., the Galactic plane, Magellanic Clouds, and M31), driving microlensing surveys to these dense stellar fields.

These dense survey fields, coupled with LSST depth and ground based PSF with FWHM $\sim 0.8''$ \citep{0805.2366}, lead to significant ambiguous blending (i.e., multiple objects within a single PSF).
Towards the Galactic center there are $\roughly 50$ stars within an LSST PSF, similarly there are $\roughly 15$ and $\roughly 5$ in the bulge, and disk respectively \citep{1806.06372}.
LSST can overcome much of the blending problem to detect variations in the brightness of stars, including the detection of microlensing lightcurves, through the process of difference imaging.
With difference imaging, reference images are built through the coaddition of multiple observation of the same fields, resulting in a deep image of the static sky. Reference images are then scaled and PSF-matched to individual observed science images and subtracted off, resulting in a difference image that only contains the signal that differs from the static sky. 

Difference imaging should reduce many of systematic effects associated with blending. Follow-up high resolution imaging from space or gronud-based adaptive optics can mitigate most blending issues for detected microlensing events.
%; however, it will still be difficult to characterize the intrinsic source star baseline photometric properties. Crowding can also lead to a secondary chromatic signal with exactly the same duration/shape as the microlensing event, although with different amplitude. 

\emph{Galaxy Model:} While microlensing can be detected independent of any detailed knowledge of Galactic structure, properly incorporating uncertainty in the Galactic dust, stellar velocity distributions, and dark matter halo model is essential to interpret the microlensing signal in the context of dark matter.
Great improvements have been made on this front since the first microlensing surveys \citep[e.g.,][]{2018MNRAS.479.2889C}.

%\emph{Binary vs isolated stars:} A potential systematic for the paralensing signal is binary star systems with approximately year-long periods. This is unlikely to be a significant systematic due to the low probability of having an achromatic binary system with a year long period. %\WAD{There was a researcher who did some of these studies for our microlensing group. Need to reference his work/arguments.} 


\noindent {\bf Projected Sensitivity}

We estimate the projected sensitivity of an LSST survey optimized for the detection of microlensing signals from PBHs and present the results in \figref{macho_constraints}.
The LSST M31 microlensing projection is based on extrapolating the \citet{1701.02151} by assuming a ten-day mini-survey of M31 with a 12 second cadence between exposures.
Such a survey is approximately 10 times longer with an order of magnitude faster cadence than the survey of \citet{1701.02151}.

The projected LSST Milky Way (MW) microlensing and paralensing constraints come from a Monte Carlo analysis of injected light curves based on LSST OpSim cadence simulations.\footnote{Details available in: \url{https://github.com/lsstdarkmatter/dark-matter-paper/issues/8}.}
We inject simulated microlensing signals onto the OpSim lightcurves for 200k PBH lenses in the mass range from $10^{-2}\Msun$ to $10^{8}\Msun$ distributed using a triangular density distribution, where $p(r = 0 \kpc) = 0$ and $p(r = 8\kpc) = 0.25$.
The sources are assumed to be a $23\magn$ stars located at $8\kpc$ with measurement noise generated from the standard LSST photometric error model \citep{0805.2366}.
LSST is expected to detect $\roughly 10^{9}$ such stars, with an optical depth of $4.48 \times 10^{-6}$ \citep{2006ApJ...636..240S}.
The paralensing constraint comes from assuming that only the secondary microlensing parallax signal is used for discovery, and not the primary heliocentric microlensing signal.

\begin{comment}
    % Old figure caption
        The microlensing and paralensing signals for a $23^\mathrm{rd}$ magnitude sourcestar in the bulge being lensed by a $50\,\mathrm{M}_\odot$ black hole. 
        For events with an Einstein crossing time much less than a year (i.e., approximately solar mass and below), the microlensing magnification will appear symmetric in time (orange curve).
        For microlensing events lasting on the order of a year or more (i.e., approximately solar mass and above), the lensing geometry changes as Earth orbits the Sun, leading to a parallax effect.
        This paralensing signal has a period of a year, with the phase determined by the coordinates of the source star, making it robust to other astrophysical systematics.
        It is also possible to detect binary dark matter, and extend the mass range to planet mass compact dark matter, via the source passing through one of the gravitational lensing caustic curves formed by the binary lens.
        The green curve on top of the heliocentric orange curve is representative of a typical planetary microlensing event caused by a caustic crossing.
        While LSST can measure these events if lucky, we will rely on LSST to detect the heliocentric microlensing event and trigger targeted follow-up higher cadence observations to measure the planetary microlensing event.
        The black data points are representative of extending the LSST wide-fast-deep cadence into the Galactic plane. \WAD{Need to update this figure with the LSST WFD cadence.}
        \emph{[Right]}
        A cartoon diagram of paralensing. 
        For microlensing events lasting on the order of a year or more the lensing geometry changes as Earth orbits the Sun, leading to a parallax effect.
        \Contributors{Will D., PALS Collaboration}
\end{comment}



%\subsubsection{White Dwarf Explosions}
%\label{sec:wd_explosions}

%\WAD{Add some material here about the Type 1A SN rate constraint.}

%\ADW{I've commented out this section, since it seems unlikely that it will get written on a short timescale }

%\citep{1805.07381}


\begin{comment}
\ADW{I've cleaned up this text in the intro to this section. This is left here for future reference.}


\WAD{Describe these constraints better, or at least throw in some references; also give numbers for what this mass range is.}
\WAD{Better introduce and conclude the two following paragraphs that mention methods for which LSST can't contribute much to. It might even be better to exclude these paragraphs and just focus on the probes that LSST can improve. Perhaps we could just move this to the complementarity section.}
At the low mass MACHO end there are constraints on the mass spectrum due to extragalactic gamma-ray evaporation, fempto-lensing of gamma-ray bursts, and Neutron star capture \WAD{Describe these constraints better, or at least throw in some references; also give numbers for what this mass range is.}, for which LSST is unable to significantly improve.

At the low mass MACHO end there exist various constraints on the mass spectrum.
Extragalactic gamma-ray evaporation... (cite Carr, Kohri, Sendouda, Yokoyama).
PBHs are ruled out at $\sim 10^{-16}, M_\odot$ \WAD{what is the actual range} due to a lack of an interference pattern in the energy spectrum of gamma-ray bursts that would occur due to the similar scale of the Schwarzschild radii the wavelength of gamma rays known as fempto-lensing (cite Barnacka, Glicenstein, Moderski) \WAD{Fix wording}.
Neutron stars in globlular clusters, where a rich dark matter environment is known to exist, would be destroyed by PBHs at $10^{-15}-10^{-8}\,M_\odot$ due to capture of the BH by the NS and subsequent accretion of the NS onto the BH.
Thus the very existence of neutron stars refuse the possibility of compact dark matter in this mass range (cite Capela, Pshirkov, Tinyakov).
While these analyses are valuable constraints on the dark matter fraction at various mass scales, LSST is unable to significantly improve upon the already existing literature.

The abundance of white dwarf explosions place constraints on the mass scale of $10^{-15}-10^{-14}\,M_\odot$, which can be constrained by the rate of Type IA supernovae.
LSST will provide the best characterization of the Type IA SN rate, and, as discussed in \S\ref{sec:wd_explosions}, greatly improve the constraint over this mass range.
In conglomeration, various gravitational microlensing surveys constrain the MACHO dark matter fraction over the mass range $10^{-9}-10^{1}\,M_\odot$ \WAD{Add citations and check the low mass range for the latest Subaru HSC constraint.}.
With the correct survey footprint and cadence, LSST can improve these dark matter fraction constraints by several orders of magnitude and extend the mass range to $10^{-10}-10^{5}\,M_\odot$, see \S\ref{sec:microlensing}.
LSST holds the greatest potential of improving the microlensing MACHO constraints, potentially over the mass range of \WAD{Complete this.}


\WAD{Make sure that we say something about how LSST's astrometry and photometry enable wide binary candidate identification and characterization to greater distances than GIA, but note the sensitivity of this method to false detections, i.e. chance alignments.}

\end{comment}
